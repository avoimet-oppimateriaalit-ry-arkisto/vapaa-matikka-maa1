\providecommand{\lukufilter}[2]{#2} % ylikirjoitetaan kaanna_luku.sh -skriptistä.
\newcommand{\osa}[1]{\chapter{#1}} % osa
\newcommand{\nosa}[1]{\chapter*{#1} \addcontentsline{toc}{chapter}{#1}} % numeroimaton osa
\newcommand{\luku}[2]{\section{#2} \lukufilter{#1}{\input{content/TEORIA_#1} \input{content/TEHT_#1}}} % luku
\newcommand{\nluku}[2]{\section*{#2} \addcontentsline{toc}{section}{#2} \lukufilter{#1}{\input{content/#1}}} % numeroimaton luku
\newcommand{\vast}{\section*{Vastaukset} \addcontentsline{toc}{section}{Vastaukset} \begin{vastaussivu} \input{content/LIITE_vastaukset} \end{vastaussivu}}

\Opensolutionfile{ans}[content/LIITE_vastaukset] % kirjoittaa vastaukset tiedostoon content/LIITE_vastaukset.tex

\newpage
\nluku{LIITE_lahtotasotesti}{Lähtötasotesti}

\osa{Luvut ja laskutoimitukset}
    \luku{luvut_ja_laskutoimitukset}{Lukualueet}
    \luku{sieventaminen}{Lausekkeiden sieventäminen}
    \luku{rationaaliluvut}{Murtoluvuilla laskeminen}
    \luku{potenssi}{Potenssi}
    \luku{desimaaliluvut}{Luvun desimaaliesitys}
    \luku{yksikot_ja_pyoristaminen}{Yksiköt ja vastaustarkkuus}
    \luku{juuret}{Juuret}
    \luku{murtopotenssi}{Murtopotenssi}
    \luku{reaaliluvut}{Reaaliluvut lukusuoralla}
%    \nluku{Testaa tietosi!}


\osa{Yhtälöt}
    \luku{yhtalo}{Yhtälö}
    \luku{ensimmainen}{Ensimmäisen asteen yhtälö}
    \luku{prosenttilaskenta}{Prosenttilaskenta}
    \luku{potenssiyhtalot}{Potenssiyhtälöt}
    \luku{verrannollisuus}{Suoraan ja kääntäen verrannollisuus}
    %    \nluku{Testaa tietosi!}

\osa{Funktiot}
    \luku{funktio}{Funktio ja sen esitystavat}
	\luku{potenssifunktio}{Potenssifunktio}
	\luku{eksponenttifunktio}{Eksponenttifunktio}
	%    \nluku{Testaa tietosi!}
	
\nosa{Kertaustehtäviä}
	\nluku{LIITE_testaatietosi}{Testaa tietosi}
	\nluku{LIITE_kertausteht}{Kertaustehtäviä}
    \nluku{TEHT_harjoituskokeet}{Harjoituskokeita}
    \nluku{TEHT_ylioppilaskokeet}{Ylioppilaskoetehtäviä}
	\nluku{TEHT_paasykokeet}{Pääsykoetehtäviä} %vai VALINTAkoe?

\nosa{Lisämateriaalia}
	\nluku{LIITE_saannot}{Laskusääntöjen todistuksia}
	\nluku{LIITE_lukujarjestelmat}{Lukujärjestelmät}
	%\nluku{LIITE_lukuteoria}{Lukuteorian alkeet}
	%\nluku{LIITE_joukot}{Joukko-opin alkeet}
	%\nluku{LIITE_aksioomat}{Reaalilukujen aksioomat}
%    \nluku{LIITE_sanasto}{Hakemisto ja suomi–ruotsi–englanti-sanasto}
	%\nluku{LIITE_symbolit}{Symbolit}

\Closesolutionfile{ans}
\vast
