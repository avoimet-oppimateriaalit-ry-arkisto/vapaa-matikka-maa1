\begin{tehtavasivu}

\begin{tehtava}%Laati Henri Ruoho 10-11-2013
Sijoita lukusuoralle luvut
\[
\mbox{$-3$, $1$, $4$, $\sqrt{2}$, $\sqrt{3}$, $\pi$ ja $\sqrt{\pi}$.}
\]
\begin{vastaus}
Järjestys on \mbox{$-3$,$1$,$\sqrt{2}$,$\sqrt{3}$,$\sqrt{\pi}$,$\pi,4$}
\end{vastaus}
\end{tehtava}

\begin{tehtava} %Laati Henri Ruoho 10-11-2013
Piirrä samankokoiset kuvat lukusuoran väleistä, joiden päätepisteet ovat 
\begin{alakohdat}
\alakohta{$0$ ja $1$}
\alakohta{$0$ ja $0,001$}
\alakohta{$0$ ja $0,00001$.}
\end{alakohdat}
Miten b-kohdan väli sijoittuu a-kohdan välille? Entä c-kohdan väli b-kohdan välille? Mikä olisi samaa logiikkaa käyttäen seuraava väli ja miten se sijoittuisi c-kohdan välille? Havainnollista piirtämällä.
% \begin{vastaus}
% \begin{alakohdat}
% \alakohta{<kuva puuttuu!>} %FIXME kuva puuttuu
% \alakohta{<kuva puuttuu!>}
% \alakohta{<kuva puuttuu!>}
% \end{alakohdat}
% \end{vastaus}
\end{tehtava}


\begin{tehtava}
Mihin joukoista $\nn$, $\zz$, $\qq$, $\rr$ seuraavat luvut kuuluvat

a) $-5$ \qquad b) $\frac82$ \qquad c) $\pi$

d) $0,888...$ \qquad e) $2,995$ \qquad f) $\sqrt{2}$?

\begin{vastaus}
a) $\zz$, $\qq$, $\rr$ \qquad b) $\nn$, $\zz$, $\qq$, $\rr$ \qquad c) $\rr$

d) $\qq$, $\rr$ \qquad e) $\qq$, $\rr$ \qquad f) $\rr$
\end{vastaus}
\end{tehtava}

\begin{tehtava}
Ovatko seuraavat luvut rationaalilukuja vai irrationaalilukuja? Kunkin desimaalit
noudattavat yksinkertaista sääntöä.
\begin{alakohdat}
\alakohta{$0,123456789101112131415...$}
\alakohta{$2,415115115115115115115...$}
\alakohta{$1,010010001000010000010...$}
\end{alakohdat}
\begin{vastaus}
\begin{alakohdat}
\alakohta{irrationaaliluku}
\alakohta{rationaaliluku (jakso on 151)}
\alakohta{irrationaaliluku}
\end{alakohdat}
\end{vastaus}
\end{tehtava}

\begin{tehtava}%Laati Henri Ruoho 10-11-2013
\begin{alakohdat}
\alakohta{Luku $144$ on luonnollinen luku ja siten reaaliluku. Onko sen vastaluku luonnollinen luku?}
\alakohta{Luku $144$ on kokonaisluku ja siten reaaliluku. Onko sen käänteisluku kokonaisluku?}
\alakohta{Luku $144$ on rationaaliluku ja siten reaaliluku. Onko sen käänteisluku rationaaliluku?}
\end{alakohdat}
\begin{vastaus}
\begin{alakohdat}
\alakohta{Ei ole}
\alakohta{Ei ole}
\alakohta{On}
\end{alakohdat}
\end{vastaus}
\end{tehtava}


\begin{tehtava}
Mikä on pienin lukua $-3$ suurempi luku \\
a) kokonaislukujen b) luonnollisten lukujen c) reaalilukujen joukossa?
\begin{vastaus}
a) $-2$ b) $0$ c) Sellaista ei ole. Jos nimittäin $a > -3$, niin keskiarvo
$\frac{-3+a}{2}$ on vielä lähempänä lukua $-3$. 
\end{vastaus}
\end{tehtava}

\begin{tehtava}
Kuinka monta 
\begin{alakohdat}
\alakohta{luonnollista lukua}
\alakohta{kokonaislukua}
\alakohta{reaalilukua}
\end{alakohdat}
on reaalilukujen \(-\pi\) ja \(\pi\) välillä?
\begin{vastaus}
\begin{alakohdat}
\alakohta{4}
\alakohta{7}
\alakohta{äärettömän monta}
\end{alakohdat}
\end{vastaus}
\end{tehtava}

\begin{tehtava}
$\star$ Osoita, että \\
a) jokaisen kahden rationaaliluvun välissä on rationaaliluku \\
b) jokaisen kahden rationaaliluvun välillä on irrationaaliluku \\
c) jokaisen kahden irrationaaliluvun välissä on rationaaliluku \\
d) jokaisen kahden irrationaaliluvun välissä on irrationaaliluku. \\
e) Perustele edellisten kohtien avulla, että minkä tahansa kahden luvun
välissä on äärettömän monta rationaali- ja irrationaalilukua.
\begin{vastaus}
Vihjeet: a) keskiarvo \ b) $\sqrt{2}$ on irrationaaliluku. Käytä
painotettua keskiarvoa. \ c) pyöristäminen \ d) hyödynnä b-kohtaa
e) keskiarvot %painotettua keskiarvoa ei ole opetettu!
\end{vastaus}
\end{tehtava}

\end{tehtavasivu}
