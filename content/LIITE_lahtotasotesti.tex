Oheinen lähtötasotesti auttaa opettajaa selvittämään, mihin asioihin kurssilla kannattaa erityisesti perehtyä. Kirjoita ratkaisut välivaiheineen ja palauta testi opettajalle ensimmäisellä tunnilla arviointia varten. Testin tehtävien tekemiseen pitäisi kulua aikaa yhteensä alle varttitunti. Ei hätää, jos et osaa joitain näistä tehtävistä -- jokaiseen tehtävätyyppiin paneudutaan kyllä kurssin aikana.

%oma numerointi näille?

\begin{tehtava}
	Sievennä. 
	\begin{alakohdat}
        \alakohta{$-2^4-(-2)^3$}
		\alakohta{$4\cdot \frac{2}{5} + \frac{2}{3}\cdot \frac{3}{5}$}
		\alakohta{$\frac{2}{5} : \frac{3}{2}$}
		\alakohta{$-2(4x^2-x)-x$}
		\alakohta{$\frac{ab^2}{a^3b}$}
	\end{alakohdat}
	\begin{vastaus}
		\begin{alakohdat}
			\alakohta{$-16-(-8)=-16+8=-8$}
			\alakohta{$\frac{8}{5} + \frac{2}{5}=\frac{10}{5} = 2$}
			\alakohta{$\frac{2}{5} \cdot \frac{2}{3}=\frac{4}{15}$}
			\alakohta{$-8x^2+x$}
			\alakohta{$\frac{b}{a^2}$}
		\end{alakohdat}
	\end{vastaus}
\end{tehtava}

\begin{tehtava}
	Mitä on 20 kuutiodesimetriä
	\begin{alakohdat}
        \alakohta{litroina}
		\alakohta{kuutiometreinä?}
	\end{alakohdat}
	\begin{vastaus}
		\begin{alakohdat}
			\alakohta{$20$, sillä $1\,\text{l}=1\,\text{dm}^3$}
			\alakohta{$20\,\text{dm}^3=20\cdot(\text{dm})^3=20\cdot\text{d}^3\cdot\text{m}^3=20\cdot (\frac{1}{10})^3\,\text{m}^3=20\cdot\frac{1}{1000}\, \text{m}^3=\frac{20}{1000}\,\text{m}^3=\frac{2}{100}\,\text{m}^3=0,02\,\text{m}^3$}
		\end{alakohdat}
	\end{vastaus}
\end{tehtava}

\begin{tehtava}
	Ratkaise yhtälöt.
	\begin{alakohdat}
		\alakohta{$2x+5 = -\frac{1}{2}$}
		\alakohta{$x^2 = 9$}
	\end{alakohdat}
	\begin{vastaus}
		\begin{alakohdat}
			\alakohta{$x=-\frac{11}{4}$}
			\alakohta{$x=-3$ tai $x=3$}
		\end{alakohdat}
	\end{vastaus}
\end{tehtava}
%funktion nimi pystyyn?!?!?!

\begin{tehtava}
	Funktiot $f$ ja $g$ ovat määritelty seuraavasti: $f(x)= x^2+3x$ ja $g(x)=2x-8$.
	\begin{alakohdat}
		\alakohta{Laske $f(-3)$.}
%		\alakohta{Piirrä funktion $g$ kuvaaja.}
		\alakohta{Määritä funktion $g$ nollakohta.}
	\end{alakohdat}	
	\begin{vastaus}
		\begin{alakohdat}
			\alakohta{$(-3)^2+3\cdot(-3)=0$}
% 			\alakohta{	\begin{kuvaajapohja}{0.7}{-4}{4}{-4}{4}
% 		\kuvaaja{2x-8}{\qquad $g(x)=2x-8$}{black}
% 			      \end{kuvaajapohja}}
			\alakohta{$x=4$}
		\end{alakohdat}
	\end{vastaus}
\end{tehtava}

\begin{tehtava}
	Matka ja aika ovat suoraan verrannollisia. Aika ja nopeus ovat kääntäen verrannollisia. 
	\begin{alakohdat}
		\alakohta{Jos matka kaksinkertaistuu, niin mitä käy ajalle?}
		\alakohta{Jos nopeus kaksinkertaistuu, niin mitä käy ajalle?}
	\end{alakohdat}
	\begin{vastaus}
		\begin{alakohdat}
		\alakohta{Aika kaksinkertaistuu.}
		\alakohta{Aika puolittuu.}
		\end{alakohdat}
	\end{vastaus}
\end{tehtava}

\begin{tehtava}
Kaupassa on 15 prosentin alennusmyynti.
	\begin{alakohdat}
		\alakohta{Kuinka paljon 25,95 euron paita maksaa alennuksessa? Ilmoita vastaus pyöristettynä sentin tarkkuuteen.}
		\alakohta{Tuotteen alennettu hinta on 21,25 euroa. Mikä oli tuotteen alkuperäinen hinta?}
	\end{alakohdat}
	\begin{vastaus}
		\begin{alakohdat}
			\alakohta{22,06 euroa}
			\alakohta{25,00 euroa}
		\end{alakohdat}
	\end{vastaus}
\end{tehtava}

