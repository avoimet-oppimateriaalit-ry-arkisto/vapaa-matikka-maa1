Oheinen lähtötasotesti auttaa selvittämään, mihin asioihin kurssilla kannattaa erityisesti perehtyä. Testin tehtävien tekemiseen pitäisi kulua aikaa yhteensä alle varttitunti. Kirjoita ratkaisut välivaiheineen ja palauta testi opettajalle ensimmäisellä tunnilla arviointia varten. Itsearviointia varten näiden tehtävien ratkaisut löytyvät kirjan lopusta. Ei hätää, jos et osaa joitain näistä tehtävistä -- jokaiseen tehtävätyyppiin paneudutaan kyllä kurssin aikana, eikä varsinaisia ennakkovaatimuksia ole.

%ratkaisut!
\begin{multicols}{2}
\begin{tehtava}
	Sievennä. 
	\alakohdat{
        § $-2^4-(-2)^3$
		§ $4\cdot \frac{2}{5} + \frac{2}{3}\cdot \frac{3}{5}$
		§ $\frac{2}{5} : \frac{3}{2}$
		§ $-2(4x^2-x)-x$
		§ $\frac{ab^2}{a^3b}$
	}
	\begin{vastaus}
		\alakohdat{
			§ $-16-(-8)=-16+8=-8$
			§ $\frac{8}{5} + \frac{2}{5}=\frac{10}{5} = 2$
			§ $\frac{2}{5} \cdot \frac{2}{3}=\frac{4}{15}$
			§ $-8x^2+x$
			§ $\frac{b}{a^2}$
		}
	\end{vastaus}
\end{tehtava}

\begin{tehtava}
	Mitä on $20$ kuutiodesimetriä
	\alakohdat{
        § litroina
		§ kuutiometreinä?
	}
	\begin{vastaus}
		\alakohdat{
			§ $20$, sillä $1\,\text{l}=1\,\text{dm}^3$
			§ $20\,\text{dm}^3=20\cdot(\text{dm})^3=20\cdot\text{d}^3\cdot\text{m}^3=20\cdot (\frac{1}{10})^3\,\text{m}^3=20\cdot\frac{1}{1\,000}\, \text{m}^3=\frac{20}{1\,000}\,\text{m}^3=\frac{2}{100}\,\text{m}^3=0,02\,\text{m}^3$
		}
	\end{vastaus}
\end{tehtava}

\begin{tehtava}
	Ratkaise yhtälöt.
	\alakohdat{
		§ $2x+5=-\frac{1}{2}$
		§ $x^2=9$
		§ $3x^3=-1$
	}
	\begin{vastaus}
		\alakohdat{
			§ $x=-\frac{11}{4}$
			§ $x=-3$ tai $x=3$
			§ $x=-\frac{1}{\sqrt[3]{3}}$
		}
	\end{vastaus}
\end{tehtava}

\begin{tehtava}
	Funktioiden $f$ ja $g$ arvot on määritelty seuraavasti: $f(x)= x^2+3x$, $g(x)=2x-8$, missä $x$ on reaaliluku.
	\alakohdat{
		§ Laske $f(-3)$.
	%	§ Piirrä funktion $g$ kuvaaja.
		§ Määritä funktion $g$ nollakohta.
	}	
	\begin{vastaus}
		\alakohdat{
			§ $(-3)^2+3\cdot(-3)=0$
 	%		§ \begin{kuvaajapohja}{0.7}{-4}{4}{-4}{4
%		\kuvaaja{2x-8}{\qquad $g(x)=2x-8$}{black}
 %			      \end{kuvaajapohja}}
			§ $x=4$
		}
	\end{vastaus}
\end{tehtava}

\begin{tehtava}
	Matka $s$ ja aika $t$ (eli kuljetun matkan kesto) ovat suoraan verrannollisia toisiinsa. Aika $t$ ja nopeus $v$ ovat kääntäen verrannollisia toisiinsa.
	\alakohdat{
		§ Jos matka kaksinkertaistuu, niin mitä käy ajalle?
		§ Jos nopeus kaksinkertaistuu, niin mitä käy ajalle?
	}
	\begin{vastaus}
		\alakohdat{
		§ Aika kaksinkertaistuu.
		§ Aika puolittuu. %RATKAISUT
		}
	\end{vastaus}
\end{tehtava}

\begin{tehtava}
Kaupassa on $15$ prosentin alennusmyynti.
	\alakohdat{
		§ Kuinka paljon $25,95$ euron paita maksaa alennuksessa? Ilmoita vastaus pyöristettynä viiden sentin tarkkuuteen.
		§ Tuotteen alennettu hinta on $21,25$ euroa. Mikä oli tuotteen alkuperäinen hinta?
	}
	\begin{vastaus}
		\alakohdat{
			§ Alennettu hinta on $25,95\,€-25,95\cdot0,15=25,95\,€\cdot(1-0,15)=25,95\,€\cdot0,75=22,06\,€\approx 22,05$ euroa
			§ Merkataan tuotteen alkuperäistä hintaa (esimerkiksi) $x$:llä, jolloin saadaan yhtälö $x\cdot 0,75=21,25\,€$. Tästä saadaan ratkaistua jakolaskulla alkuperäiseksi hinnaksi $x=\frac{21,25\,€}{0,75}=25,00\,€$
		}
	\end{vastaus}
\end{tehtava}

\end{multicols}