\begin{tehtavasivu}
%tehtäviä päättelystä ja syntaksin ymmärtämisestä

%\subsubsection*{Opi perusteet}

%\begin{tehtava}
%Mitkä seuraavista esimerkeistä ovat lukuja?
%\alakohdat{
%
%}
%	\begin{vastaus}
%	\alakohdat{
%	§
%	§
%	}
%	\end{vastaus}
%\end{tehtava}

\begin{tehtava}
Onko relaatio refleksiivinen, symmetrinen tai transitiivinen?
\alakohdat{
§ Kahden luvun välinen relaatio $\leq$
§ Kahden luvun välinen relaatio $>$
§ ''$a$ on $b$:n lapsi'', missä $a$ ja $b$ edustavat ihmisiä 
§ "$a$ on $b$:n työkaveri'', missä $a$ ja $b$ edustavat ihmisiä
}
	\begin{vastaus}
	\alakohdat{
	§ refleksiivinen ja transitiivinen
	§ transitiivinen
	§ ei mikään vaihtoehdoista
	§ symmetrinen (ei transitiivinen, koska oma työkaveri voi olla töissä lisäksi jossain muualla)
	}
	\end{vastaus}
\end{tehtava}

\begin{tehtava}
Olkoot $l_1$ ja $l_2$ tason suoria. Määritellään niiden välinen relaatio $l_1||l_2$, joka luetaan ''suora $l_1$ on yhdensuuntainen suoran $l_2$ kanssa''. Onko relaatio refleksiivinen, symmetrinen tai transitiivinen?
%§ Miksi suorien nimeämisessä käytettiin alain
	\begin{vastaus}
	Relaatio on symmetrinen, symmetrinen ja transitiivinen.
	\end{vastaus}
\end{tehtava}
\end{tehtavasivu}