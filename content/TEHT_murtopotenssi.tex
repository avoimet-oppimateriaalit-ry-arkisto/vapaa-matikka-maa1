\begin{tehtavasivu}
\subsubsection*{Opi perusteet}

Muuta lausekkeet murtopotenssimuotoon.

\begin{tehtava}
	\alakohdatm{
	§ $\sqrt[3]{a}$ 
	§ $\sqrt[6]{a}$ 
	§ $\sqrt[n]{a}$
	}
	\begin{vastaus}
		\alakohdatm{
			§ $a^\frac{1}{3}$ 
			§ $a^\frac{1}{6}$
			§ $a^\frac{1}{n}$
		}
	\end{vastaus}
\end{tehtava}

\begin{tehtava}
	\alakohdatm{
		§ $(\sqrt[3]{z})^6$
		§ $(\sqrt[6]{y^3})$
		§ $(\sqrt[5]{b})^2$
		§ $(\sqrt[16]{\ddot{o}^4})$
	}
	\begin{vastaus}
		\alakohdatm{
			§ $z^2$
			§ $y^\frac{1}{2}$
			§ $b^\frac{2}{5}$
			§ $\ddot{o}^\frac{1}{4}$
		}
	\end{vastaus}
\end{tehtava}

Sievennä.

\begin{tehtava}
	\alakohdatm{
		§ $x^\frac{1}{5}$
		§ $x^\frac{4}{3}$
		§ $x^\frac{8}{4}$
		§ $x^\frac{25}{100}$
}
	\begin{vastaus}
		\alakohdatm{	
			§ $\sqrt[5]{x}$
			§ ${\sqrt[3]{x}}^4$
			§ $x^2$
			§ $\sqrt[4]{x}$
		} 
	\end{vastaus}
\end{tehtava}

\begin{tehtava}
	\alakohdatm{
		§ $9^\frac{1}{2}$
		§ $8^\frac{1}{3}$
		§ $4^\frac{3}{2}$
		§ $81^\frac{3}{4}$
	}
	\begin{vastaus}
		\alakohdatm{
			§ $3$
			§ $2$
			§ $8$
			§ $27$ 
		}
	\end{vastaus}
\end{tehtava}

\subsubsection*{Hallitse kokonaisuus}

%Tehtävän lisännyt Aleksi Sipola 10.11.2013
\begin{tehtava} (YO S00/1) Sievennä seuraavat lausekkeet.
\alakohdat{
§ $(x^{(n-1)})^{(n-1)}\cdot(x^n)^{2-n}$
§ $\sqrt[3]{a}(\sqrt[3]{a^2}-\sqrt[3]{a^5})$
}
	\begin{vastaus}
\alakohdat{
§ $x$
§ $a(1-a)$
}
	\end{vastaus}
\end{tehtava}

\begin{tehtava}
Sievennä
	\alakohdatm{
		§ $x^{-\frac{1}{3}}$ 
		§ $x^\frac{5}{-2}$
		§ $x^{3 \frac{1}{2}}$
		§ $x^{2 \frac{4}{7}}$.
	}
	\begin{vastaus}
		\alakohdatm{
			§ $\frac{1}{\sqrt[3]{x}}$ 
			§ $\frac{1}{\sqrt{x^5}}$ 
			§ $\sqrt{x^3}$ 
			§ $\sqrt[7]{x^8}$ 
		}
	\end{vastaus}
\end{tehtava}

\begin{tehtava}
Laske
	\alakohdatm{
		§ $4^\frac{3}{4}$
		§ $2^\frac{5}{1}$ 
		§ $16^\frac{2}{3}$
		§ $5^\frac{5}{3}$.
	}
	\begin{vastaus}
		\alakohdatm{
			§ $2\sqrt{2}$
			§ $32$ 
			§ $(\sqrt[3]{16})^2$ 
			§ $5(\sqrt[3]{5})^2$ 
		}
	\end{vastaus}
\end{tehtava}

\begin{tehtava}
Muuta murtopotenssimuotoon.
	\alakohdatm{
		§ $\sqrt{\sqrt{k}}$ 
		§ $\sqrt[3]{\sqrt[4]{k}}$
		§ $\sqrt{m\sqrt{m}}$
		§ $\sqrt[5]{m\sqrt[7]{m}}$
	}
	\begin{vastaus}
		\alakohdatm{
			§ $k^\frac{1}{4}$ 
			§ $k^\frac{1}{12}$ 
			§ $m^\frac{3}{4}$ 
			§ $m^\frac{8}{35}$
		}
	\end{vastaus}
\end{tehtava}

    \begin{tehtava}
Laske $x^{\frac{4}{3}}$, kun tiedetään, että
    \alakohdatm{
      § $x=27$
      § $x^{\frac{3}{4}}=27$
      § $x^{- \frac{3}{4}}=27$
      § $x^{- \frac{4}{3}}=27$
     }
\begin{vastaus}
\alakohdatm{
 § $81$
 § $350,48$
 § $0,037$
 § $0,0029$
}
\end{vastaus}
    \end{tehtava}
 
\begin{tehtava}
Sievennä lausekkeet.
\alakohdatm{
§ $4 \cdot \sqrt[x]{\frac{2^{x^2}}{(2^x)^2}}$
§ $\left( \frac{2^{\frac{x}{y}}}{2^{\frac{y}{x}}} \right)^{xy}$
§ $\sqrt[3]{ x^{\frac{1}{x}} \cdot x^{\frac{2}{x}} }$
}
	\begin{vastaus}
	\alakohdatm{
§ $2^x$
§ $2^{x^2 - y^2}$
§ $\sqrt[x]{x}$
}
	\end{vastaus}
\end{tehtava}

\subsubsection*{Lisää tehtäviä}

\begin{tehtava}
Sievennä.
	\alakohdatm{
		§ $\sqrt[3]{\sqrt[3]{\alpha}^2}$ 
		§ $\sqrt[5]{q^2\sqrt{q}}$ 
		§ $\left(\sqrt{P^4\sqrt{P}}\right)^3$
	}
	\begin{vastaus}
		\alakohdatm{
			§ $\alpha^\frac{2}{9}$
			§ $q^\frac{1}{2}$ 
			§ $P^\frac{27}{4}$
		}
	\end{vastaus}
\end{tehtava}

\begin{tehtava}
%Laatinut Jaakko Viertiö 2013-11-10
 Sievennä
	\alakohdatm{
		§ $\sqrt{\sqrt[3]{27}}$
		§ $\sqrt[3]{{\sqrt[16]{{\sqrt{2}}^{16}}}^2}$
		§ ${\sqrt{\sqrt[3]{{\sqrt[4]{\sqrt[5]{{\sqrt[6]{42}}^{22}}}}^2}}}^3$.
	}
	  \begin{vastaus}
		\alakohdatm{
		 § $\sqrt{3}$
		 § $\sqrt[3]{2}$
		 § ${\sqrt[60]{42}}^{11}$
		}
	\end{vastaus}
\end{tehtava}
 
 \begin{tehtava}
	\alakohdat{
	$\star$ Oletetaan, että $a\leq0$. Sievennä
		§ $\frac{\sqrt[3]{a^2}-\sqrt[3]{a}}{\sqrt[3]{a}}$
		§ $\sqrt{\frac{\sqrt{a}^2\sqrt[3]{a^6}}{4a}}$
		§ $\sqrt[7]{\frac{a-\sqrt{a}^2}{a^0}}$ 
		§ $\sqrt{\sqrt{a^3}\left(\frac{a^7}{\sqrt[5]{a^3}}\right)^0\sqrt{\frac{a^5}{a^2}}}$.
	}
	\begin{vastaus}
		\alakohdat{
			§ $\sqrt[3]{a}-1$ 
			§ $\frac{1}{2}a$ 
			§ $1$ 
			§ $a\sqrt{a}$
		}
	\end{vastaus}
\end{tehtava}
 
\begin{tehtava}
	$\star$ Tutki laskimella, miten laskutoimitukset \[ x^x, \quad x^{x^x}, \quad x^{x^{x^x}}, \quad \ldots\] käyttäytyvät, kun eksponenttien määrää kasvatetaan mielivaltaisen suureksi.
	\alakohdat{
		§ Tutki, mitä tapahtuu tapauksissa $x=1,3$, $x=1,7$ ja $x=0,05$.
		§ Millä luvun $x$ positiivisilla arvoilla potenssitornien arvot vakiintuvat tiettyyn lukuun?
	}
	\begin{vastaus}
		\alakohdat{
			§ Kun $x=1,3$, tornien arvot vakiintuvat lukuun $1,4709\ldots$;
				kun $x=1,7$, tornien arvot kasvavat mielivaltaisen suuriksi;
				kun $x=0,05$, tornien arvot vuorottelevat lukujen $0,136\ldots$
				ja $0,664\ldots$ välillä.
			§ Arvot vakiintuvat, kun $0,00659 \ldots \leq x \leq 1,444667\ldots$. Tämän voi esittää muodossa $\frac{1}{e^e} \leq x \leq \sqrt[e]{e}$, missä $e$ on irrationaalivakio nimeltään Neperin luku. Desimaalikehitelmänä $e = 2,718281828459\ldots$ (Kyseistä ratkaisua ei tämän kurssin tiedoilla voida saavuttaa.)i
		}
\end{vastaus}
\end{tehtava}

\begin{tehtava} %jonnekin yhtälölukuun
	$\star$ Potenssitornissa $x^{x^{x^{x^{\mathstrut^{.^{.^{.}}}}}}}$ on äärettömän monta päällekkäistä eksponenttia. Yhtälöllä $ x^{x^{x^{x^{\mathstrut^{.^{.^{.}}}}}}} =2$ on yksi ratkaisu. Etsi se.
\begin{vastaus}
	$x = \sqrt{2}$
\end{vastaus}
\end{tehtava}

\end{tehtavasivu}