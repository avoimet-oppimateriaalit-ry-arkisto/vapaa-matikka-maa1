\begin{tehtavasivu}

%\subsubsection*{Opi perusteet}

\begin{tehtava}
	Mitkä seuraavista luvuista ovat jaollisia luvulla $4$? Jos luku $a$ on jaollinen luvulla $4$, kerro, millä kokonaisluvulla $b$ pätee $a = 4 \cdot b$. 
	\alakohdat{
		§ $1$
		§ $12$
		§ $13$
		§ $2$
		§ $-20$
		§ $0$
	}
	\begin{vastaus}
		\alakohdat{
			§ Ei ole jaollinen luvulla $4$
			§ On jaollinen luvulla $4$, sillä $12 = 4 \cdot 3$
			§ Ei ole jaollinen luvulla $4$
			§ Ei ole jaollinen luvulla $4$
			§ On jaollinen luvulla $4$, sillä $-20 = 4 \cdot (-5)$
			§ On jaollinen luvulla $4$, sillä $0 = 4 \cdot 0$ 
		}
    \end{vastaus}
\end{tehtava}

\begin{tehtava}
Ovatko väitteet tosia vai epätosia?
\alakohdat{
§ $2|4$
§ $4|2$
§ $4\nmid 10$
§ $10 \nmid 100 $
}
	\begin{vastaus}
	\alakohdat{
	§ tosi
	§ epätosi
	§ tosi
	§ epätosi
	}
	\end{vastaus}

\end{tehtava}

\begin{tehtava}
    Jaa alkutekijöihin
	\alakohdat{
		§ $12$
		§ $15$
		§ $28$
		§ $30$
		§ $64$
		§ $90$
		§ $100$.
	}
    \begin{vastaus}
		\alakohdat{
			§ $12 = 2\cdot 2 \cdot 3$
			§ $15 = 3 \cdot 5$
			§ $28 = 2\cdot 2 \cdot 7$
			§ $30 = 2 \cdot 3 \cdot 5$
			§ $64 = 2\cdot2\cdot2\cdot2\cdot2\cdot2$
			§ $90 = 2 \cdot 3 \cdot 3 \cdot 5$
			§ $100 = 2\cdot 2 \cdot 5\cdot 5$
		}
    \end{vastaus}
\end{tehtava}

	\begin{tehtava}
Jaa luvut tekijöihin. Mitkä luvuista ovat alkulukuja?
	\alakohdat{
	§ $111$
	§ $75$
	§ $97$
	§ $360$
	 }
	 \begin{vastaus}
	  \alakohdat{
	   § Luku $111$ voidaan kirjoittaa tulona $3 \cdot 37$, joten $111$ ei ole alkuluku.
	   § Luku $74$ voidaan kirjoittaa tulona $3 \cdot 5 \cdot 5$, joten $74$ ei ole alkuluku.
	   § $97$ on alkuluku.
	   § Luku $360$ voidaan kirjoittaa tulona (esimerkiksi) $2 \cdot 5 \cdot 6 \cdot 6 $, joten $360$ ei ole alkuluku.
	  }
	 \end{vastaus}
	\end{tehtava}
	
\begin{tehtava}
Pikaruokaketju myy kananuggetteja $4$, $6$, $9$, ja $20$ nuggetin pakkauksissa. Luettele ne alle $25$ nuggettin tilaukset, joita et voi tilata yhdistelemällä edellä mainittuja annoskokoja. Entä jos et kehtaa ostaa $4$ nuggettin lasten annoksia?
	\begin{vastaus}
		Saavuttamattomissa olevat annoskoot ovat $1$, $2$, $3$, $5$, $7$ ja $11$. Jos neljän pakkaukset eivät innosta, mahdottomia annoskokoja on paljon enemmän: $1$, $2$, $3$, $4$, $5$, $7$, $8$, $10$, $11$, $13$, $14$, $16$, $17$, $19$, $22$ ja $23$.
	\end{vastaus}
\end{tehtava}

\begin{tehtava}
Onko kokonaislukujen jaollisuusrelaatio $\mid$ (esim. $8\mid 64$)
\alakohdat{
§ refleksiivinen
§ symmetrinen
§ $\star$ transitiivinen?
}
	\begin{vastaus}
	\alakohdat{
	§ kyllä 
	§ ei
	§ kyllä
	}
	\end{vastaus}
\end{tehtava}

\begin{tehtava}
Positiivisia kokonaislukuja, jotka ovat itsestään poikkeavien tekjöidensä summia, sanotaan \termi{täydellinen luku}{täydellisiksi luvuiksi}. Esimerkiksi $6$ on täydellinen luku, sillä $6=1+2+3$. Tällä hetkellä ei tiedetä, onko täydellisiä lukuja äärettömän montaa -- vuonna 2013 löydettiin tietokoneella $48.$ täydellinen luku, jossa on lähes $35$ miljoonaa numeroa. Myöskään ei tiedetä, onko parittomia täydellisiä lukuja olemassa.
%\alakohdat{
%§
Määritä seuraava kuutta suurempi täydellinen luku. (Se on alle $30$.)
%§ Perustele, voivatko negatiiviset kokonaisluvut olla täydellisiä lukuja.
%}
	\begin{vastaus}
%	\alakohdat{
%	§
$28=1+2+4+7+14$
%	§ 
%	}
	\end{vastaus}
\end{tehtava}

%musiikkitehtävä, miloin seuraavan kerran rytmissä... -> yksikkölukuun jokin bpm-tehtävä myös :)

\end{tehtavasivu}