%Lisää sievennystehtäviä, joissa käytetään potenssisääntöjä sekä rationaalilukujen laskusääntöjä

\begin{tehtavasivu}

\subsection*{Opi perusteet}

\begin{tehtava}
Pyöristä kymmenen tarkkuudella
\alakohdat{
§ $8$
§ $23$
§ $65$
§ $9\,001$
§ $1$
§ $-6$
§ $-12$
§ $99\,994$.
}
\begin{vastaus}
\alakohdat{
§ $10$
§ $20$
§ $70$
§ $9\,000$
§ $0$
§ $-10$
§ $-10$
§ $99\,990$
}
\end{vastaus}
\end{tehtava}

\begin{tehtava}
Pyöristä kahden merkitsevän numeron tarkkuudella
\alakohdat{
§ $1,0003$
§ $3,69$
§ $152,8$
§ $7\,062,4$
§ $-2,05$
§ $0,00546810$
§ $-1\,337$
§ $0,0000501$.
}
\begin{vastaus}
\alakohdat{
§ $1,0$
§ $3,7$
§ $150$
§ $7\,000$
§ $-2,0$
§ $0,0055$
§ $-1\,300$
§ $0,000050$
}
\end{vastaus}
\end{tehtava}

%ja lisää tehtäviä... :) (negatiiviset, eri pyöristysskeemat, ...)
\begin{tehtava}
Montako merkitsevää numeroa on seuraavissa luvuissa
\alakohdat{
§ $5$
§ $12,0$
§ $9\,000$
§ $666$
§ $9\,000,000$
§ $0,0$
§ $-1$
§ $-0,00024$?
}
\begin{vastaus}
\alakohdat{
§ $1$
§ $2$
§ $1$
§ $3$
§ $7$
§ $2$
§ $1$
§ $2$
}
\end{vastaus}
\end{tehtava}

\begin{tehtava}
Laske pöydän pinta-ala neliösenttimetreinä mittaustarkkuus huomioiden, kun pöytälevyn sivujen pituuksiksi on mitattu $72,5$\,cm ja $81,5$\,cm.
\begin{vastaus}
$5\,910$\,cm$^2$
\end{vastaus}
\end{tehtava}

pyöristys, kellonajat vartin tarkkuudella
\subsection*{Hallitse kokonaisuus}

\begin{tehtava}
(YO K00/3) Suorakulmaisen särmiön muotoinen hautakivi on $80$\,cm korkea, $2,10$\,m pitkä ja $32$\,cm leveä.
Voidaanko kivi nostaa nosturilla, joka pystyy nostamaan enintään kahden tonnin painoisen kuorman? Hautakiven tiheys on $2,7 \cdot 10^3$\,kg/m$^3$.
\begin{vastaus}
Kyllä voidaan: hautakiven tilavuus on noin $0,54$\,m$^3$, joten se painaa noin $1,5$ tonnia.
\end{vastaus}
\end{tehtava}

\begin{tehtava}
Astmalääke Bricanylia annetaan injektiona $30$\,kg painavalle lapselle. Bricanyl-injektionesteen vahvuus on $0,35$\,mg/ml ja annos $11$\,$\upmu$g/kg. Kuinka monta mikrolitraa injektionestettä lapsi saa? Anna vastaus 1\,$\mu$l:n tarkkuudella. %UPMU?
 \begin{vastaus}
$943$\,$\mu$l. (Lapselle annettava Bricanyl-annos on $330$\,$\mu$l.) %UPMU?
 \end{vastaus}
\end{tehtava}

\begin{tehtava}
Erään antibioottimikstuuran vahvuus on $0,23$\,mg/ml. Lapselle annostus on $1.$ päivänä $42$\,mg (aloitusannos) 
ja $2$.--$7$. päivänä $12$\,mg (ylläpitoannokset). Mikstuuran pakkauskoko on $90,0$\,ml. Kuinka monta pakkausta tarvitaan, ja montako millilitraa jää käyttämättä? Ilmoita tulos $0,1$\,ml:n tarkkuudella.
 \begin{vastaus}
Tarvitaan $6$ pakkausta. Mikstuuraa jää käyttämättä $44,3$\,ml.
 \end{vastaus}
\end{tehtava}


\subsection*{Lisää tehtäviä}




\end{tehtavasivu}
