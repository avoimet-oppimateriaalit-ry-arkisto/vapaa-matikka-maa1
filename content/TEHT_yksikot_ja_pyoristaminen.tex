%Lisää sievennystehtäviä, joissa käytetään potenssisääntöjä sekä rationaalilukujen laskusääntöjä

\begin{tehtavasivu}

\subsection*{Opi perusteet}

\begin{tehtava}
Muuta \emph{järkevimpään} yksikköön
\begin{alakohdat}
\alakohta{$0,3$\,km + $200$\,m}
\alakohta{$0,04$\,m + $10$\,mm}
\alakohta{$0,4$\,km + $7\,000$\,m + $1\,000\,000$\,mm}
\alakohta{$0,2$\,cm + $4$\,cm.}
\end{alakohdat}
\begin{vastaus}
\begin{alakohdat}
\alakohta{$500$\,m}
\alakohta{$5$\,cm}
\alakohta{$8,4$\,km}
\alakohta{$4,2$\,cm}
\end{alakohdat}
\end{vastaus}
\end{tehtava}

\begin{tehtava}
Muuta
      \begin{alakohdat}
    \alakohta{$1$\,h $36$\,min tunneiksi}
    \alakohta{$2$\,h $45$\,min minuuteiksi}
    \alakohta{$1,5$\,h sekunneiksi.}
      \end{alakohdat}
  \begin{vastaus}
    \begin{alakohdat}
    \alakohta{$1,6$\,h}
    \alakohta{$165$\,min}
    \alakohta{$5\,400$\,h}
      \end{alakohdat}
  \end{vastaus}
\end{tehtava}

\begin{tehtava}
Elektroniikkayhtiö on ilmoittanut, että laitteen täyteen ladattu akku kestää käyttöä 450 minuuttia. Laitetta on käytetty lataamisen jälkeen 3 h 30 min. Kuinka monta tuntia akun voi olettaa vielä kestävän?
\begin{vastaus}
4 tuntia
\end{vastaus}
\end{tehtava}

\begin{tehtava}
Esitä luku ilman kymmenpotenssia
\begin{alakohdat}
\alakohta{$3,2 \cdot 10^4$}
\alakohta{$-7,03 \cdot 10^{-5}$}
\alakohta{$10,005 \cdot 10^{-2}$.}
\end{alakohdat}
\begin{vastaus}
\begin{alakohdat}
\alakohta{$32000$}
\alakohta{$-0,0000703$}
\alakohta{$0,10005$}
\end{alakohdat}
\end{vastaus}
\end{tehtava}


% FIXME \todo{TEHTÄVÄ: sanallinen tehtävä, jossa pitää laskea esim. km ja m yhteen}

\begin{tehtava}
Esitä luku ilman etuliitettä tai kymmenpotenssimuotoa
\begin{alakohdat}
\alakohta{$0,5$ dl}
\alakohta{$233$ mm}
\alakohta{$33$ cm}
\alakohta{$16$ kg}
\alakohta{$2$ MJ}
\alakohta{$4$ kt}
\alakohta{$0,125$ Mt.}
\end{alakohdat}
\begin{vastaus}
\begin{alakohdat}
\alakohta{$0,05$\,l}
\alakohta{$0,233$\,m}
\alakohta{$0,33$\,m}
\alakohta{$16\,000$\,g}
\alakohta{$2\,000\,000$\,J}
\alakohta{$4096$\,tavua}
\alakohta{$131072$\,tavua}
\end{alakohdat}
\end{vastaus}
\end{tehtava}

\begin{tehtava}
Laske
\begin{alakohdat}
\alakohta{$3,0\,\textrm{ml} \cdot 25\,\textrm{mg/ml}$}
\alakohta{$150\,\textrm{m} : 3\,\textrm{m/s}$}
\alakohta{$3,0\,\textrm{g} : 2,0\,\textrm{dm}^2$.}
\end{alakohdat}
\begin{vastaus}
\begin{alakohdat}
\alakohta{75\,mg}
\alakohta{50\,s}
\alakohta{1,5\,g/dm$^2$}
\end{alakohdat}
\end{vastaus}
\end{tehtava}

\begin{tehtava}
Laske
\begin{alakohdat}
\alakohta{$3,0\,\textrm{cm} \cdot 2,5\,\textrm{\,cm} \cdot 6,0\,\textrm{cm}$}
\alakohta{$0,4\,\textrm{km} : 8\,\textrm{m/s}$}
\alakohta{$2\,\textrm{g} : 50\,\textrm{mg/ml}$.}
\end{alakohdat}
\begin{vastaus}
\begin{alakohdat}
\alakohta{45\,cm$^3$}
\alakohta{50\,s}
\alakohta{40\,ml}
\end{alakohdat}
\end{vastaus}
\end{tehtava}



\begin{tehtava}
Liuoksen pitoisuus voidaan laskea kaavalla $c=m/V$, missä $c$ on pitoisuus, $m$ on liuenneen aineen määrä ja $V$ on liuoksen tilavuus. Laske suolaliuoksen pitoisuus (yksikkönä mg/ml), kun 4,5 grammaa suolaa on liuotettu 500 millilitraan liuosta.
\begin{vastaus}
9 mg/ml
\end{vastaus}
\end{tehtava}

\begin{tehtava}
Laske oma pituutesi ja painosi tuumissa ja paunoissa.
\end{tehtava}

\begin{tehtava}
Muuta seuraavat pituudet SI-muotoon (1 tuuma = 2,54\,cm, 1 jaardi = 0,914\,m, 1 jalka = 0,305\,m, 1 maili = 1,609\,km) \\
\begin{alakohdat}
\alakohta{5 tuumaa senttimetreiksi}
\alakohta{0,3 tuumaa millimetreiksi}
\alakohta{79 jaardia metreiksi}
\alakohta{80 mailia kilometreiksi}
\alakohta{5 jalkaa ja 7 tuumaa senttimetreiksi}
\alakohta{330 jalkaa kilometreiksi.}
\end{alakohdat}
\begin{vastaus}
\begin{alakohdat}
\alakohta{12,7\,cm}
\alakohta{7,62\,mm}
\alakohta{72,206\,m}
\alakohta{128,72\,km}
\alakohta{170,28\,cm}
\alakohta{100,65\,m}
\end{alakohdat}
\end{vastaus}
\end{tehtava}

\begin{tehtava}
Jasper-Korianteri ja Kotivalo vertailivat keppejään. Jasper-Korianteri mittasi oman keppinsä 5,9 tuumaa pitkäksi ja Kotivalo omansa 14,8\,cm pitkäksi. Kummalla on pidempi keppi?
\begin{vastaus}
Jasper-Korianterilla, sillä 5,9 tuumaa = 14,986\,cm.
\end{vastaus}
\end{tehtava}

\begin{tehtava}
Pyöristä kymmenen tarkkuudella
\begin{alakohdat}
\alakohta{$8$}
\alakohta{$23$}
\alakohta{$65$}
\alakohta{$9\,001$}
\alakohta{$1$}
\alakohta{$-6$}
\alakohta{$-12$}
\alakohta{$99\,994$.}
\end{alakohdat}
\begin{vastaus}
\begin{alakohdat}
\alakohta{$10$}
\alakohta{$20$}
\alakohta{$70$}
\alakohta{$9\,000$}
\alakohta{$0$}
\alakohta{$-10$}
\alakohta{$-10$}
\alakohta{$99\,990$}
\end{alakohdat}
\end{vastaus}
\end{tehtava}

\begin{tehtava}
Pyöristä kahden merkitsevän numeron tarkkuudella
\begin{alakohdat}
\alakohta{$1,0003$}
\alakohta{$3,69$}
\alakohta{$152,8$}
\alakohta{$7062,4$}
\alakohta{$-2,05$}
\alakohta{$0,00546810$}
\alakohta{$-1337$}
\alakohta{$0,0000501$.}
\end{alakohdat}
\begin{vastaus}
\begin{alakohdat}
\alakohta{$1,0$}
\alakohta{$3,7$}
\alakohta{$150$}
\alakohta{$7\,000$}
\alakohta{$-2,0$}
\alakohta{$0,0055$}
\alakohta{$-1\,300$}
\alakohta{$0,000050$}
\end{alakohdat}
\end{vastaus}
\end{tehtava}

\begin{tehtava}
Montako merkitsevää numeroa on seuraavissa luvuissa
\begin{alakohdat}
\alakohta{$5$}
\alakohta{$12,0$}
\alakohta{$9\,000$}
\alakohta{$666$}
\alakohta{$9\,000,000$}
\alakohta{$0,0$}
\alakohta{$-1$}
\alakohta{$-0,00024$?}
\end{alakohdat}
\begin{vastaus}
\begin{alakohdat}
\alakohta{$1$}
\alakohta{$2$}
\alakohta{$1$}
\alakohta{$3$}
\alakohta{$7$}
\alakohta{$2$}
\alakohta{$1$}
\alakohta{$2$}
\end{alakohdat}
\end{vastaus}
\end{tehtava}


\begin{tehtava}
Laske pöydän pinta-ala neliösenttimetreinä mittaustarkkuus huomioiden, kun pöytälevyn sivujen pituuksiksi on mitattu 72,5\,cm ja 81,5\,cm.
\begin{vastaus}
5\,910\,cm$^2$
\end{vastaus}
\end{tehtava}

\begin{tehtava}
Suorakulmaisen särmiön muotoinen hautakivi on 80\,cm korkea, 2,10\,m pitkä ja 32\,cm leveä.
Voidaanko kivi nostaa nosturilla, joka pystyy nostamaan enintään kahden tonnin painoisen kuorman? Hautakiven tiheys on $2,7 \cdot 10^3$\,kg/m$^3$. (YO k2000/3)
\begin{vastaus}
Kyllä voidaan: hautakiven tilavuus on noin 0,54\,m$^3$, joten se painaa n. 1,5 tonnia.
\end{vastaus}
\end{tehtava}

\begin{tehtava}
Astmalääke Bricanylia annetaan injektiona 30\,kg painavalle lapselle. Bricanyl-injektionesteen vahvuus on 0,35\,mg/ml ja annos 11\,$\upmu$g/kg. Kuinka monta mikrolitraa injektionestettä lapsi saa? Anna vastaus 1\,$\mu$l:n tarkkuudella.
 \begin{vastaus}
943\,$\mu$l. (Lapselle annettava Bricanyl-annos on 330\,$\mu$l.)
 \end{vastaus}
\end{tehtava}

\begin{tehtava}
Erään antibioottimikstuuran vahvuus on 0,23\,mg/ml. Lapselle annostus on 1. päivänä 42\,mg (aloitusannos) 
ja 2.--7. päivänä 12\,mg (ylläpitoannokset). Mikstuuran pakkauskoko on 90,0\,ml. Kuinka monta pakkausta tarvitaan, ja montako millilitraa jää käyttämättä? Ilmoita tulos 0,1\,ml:n tarkkuudella.
 \begin{vastaus}
Tarvitaan 6 pakkausta. Mikstuuraa jää käyttämättä 44,3 ml.
 \end{vastaus}
\end{tehtava}

\subsection*{Hallitse kokonaisuus}

\begin{tehtava}
Kuinka monta litraa maitoa mahtuu metri kertaa metri kertaa metri -laatikkoon? (Jos kuution särmän pituus on $a$, on kuution tilavuus $a^3$.)
	\begin{vastaus}
	1\,000\l
	\end{vastaus}
\end{tehtava}

\subsection*{Lisää tehtäviä}


\begin{tehtava}
Muuta minuuteiksi
\begin{alakohdat}
\alakohta{$1$\,h $17$\,min}
\alakohta{$2$\,h $45$\,min}
\alakohta{$1,5$\,h}
\alakohta{$1,75$\,h}
\end{alakohdat}
\begin{vastaus}
\begin{alakohdat}
\alakohta{$77$\,min}
\alakohta{$165$\,min}
\alakohta{$90$\,min}
\alakohta{$105$\,min}
\end{alakohdat}
\end{vastaus}
\end{tehtava}
\begin{tehtava}
Muuta sekunneiksi
\begin{alakohdat}
\alakohta{$1$ h $42$ min}
\alakohta{$3$ h $32$ min}
\alakohta{$1,25$ h}
\alakohta{$4,5$ h.}
\end{alakohdat}
\begin{vastaus}
\begin{alakohdat}
\alakohta{$6120$ s}
\alakohta{$12720$ s}
\alakohta{$4500$ s}
\alakohta{$16200$ s}
\end{alakohdat}
\end{vastaus}
\end{tehtava}

\begin{tehtava}
Muuta tunneiksi ja minuuteiksi
\begin{alakohdat}
\alakohta{$125$ min}
\alakohta{$667$ min}
\alakohta{$120$ min}
\alakohta{$194$ min.}
\end{alakohdat}
\begin{vastaus}
\begin{alakohdat}
\alakohta{$2$ h $5$ min}
\alakohta{$11$ h $7$ min}
\alakohta{$2$ h}
\alakohta{$3$ h $14$ min}
\end{alakohdat}
\end{vastaus}
\end{tehtava}

\begin{tehtava}
Liisa ohjelmoi tietokoneensa sammumaan 14\,400 sekunnin kuluttua. Kuinka monen tunnin kuluttua Liisan tietokone sammuu?
\begin{vastaus}
4 tunnin kuluttua
\end{vastaus}
\end{tehtava}

\begin{tehtava}
(Muunnelma fyysikko Enrico Fermin esittämästä ongelmasta.) Chicagossa asuu osapuilleen
$5\,000\,000$ asukasta. Kussakin kotitaloudessa asuu keskimäärin kaksi asukasta.
Karkeasti joka kahdennessakymmenennessä kotitaloudessa on säännöllisesti viritettävä piano.
Säännöllisesti viritettävät pianot viritetään keskimäärin kerran vuodessa.
Pianonvirittäjällä kestää noin kaksi tuntia virittää piano, kun matkustusajat huomioidaan.
Kukin pianonvirittäjä työskentelee kahdeksan tuntia päivässä, viisi päivää viikossa
ja $50$ viikkoa vuodessa. Mikä arvio Chicagon pianonvirittäjien lukumäärälle saadaan näillä
arvioilla?
\begin{vastaus}
125 pianonvirittäjää
\end{vastaus}
\end{tehtava}

\end{tehtavasivu}
