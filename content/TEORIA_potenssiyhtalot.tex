Eräs tärkeä yhtälöiden tyyppi on \termi{potenssiyhtälö}{potenssiyhtälöt}.
\laatikko{Potenssiyhtälö on muotoa $x^n=a$ oleva yhtälö, jossa $n$ on jokin rationaaliluku. ($n$ voi ilman ongelmia olla mikä tahansa reaalilukukin, mutta rationaalitarkastelu riittää tälle kurssille.)}

Huomaa miinus- ja murtopotenssiyhtälöissä tarkistaa, millä kantaluvuilla $x$ potenssit on määritelty. Murtopotenssia määriteltäessä vaadittiin, että kantaluku $x\geq0$ ja negatiivisen potenssin tapauksessa on huomioitava, että kantaluku $x\neq0$, sillä $x^{-a}=\frac{1}{x^a}$, eikä nimittäjässä sallita olevan lukua $0$.

Eksponentin $n$ ollessa kokonaisluku sen arvoa kutsutaan potenssiyhtälön \termi{aste (potenssiyhtälö)}{asteeksi}. Esimerkiksi potenssiyhtälön $x^2=0$ aste on $2$. Potenssiyhtälöitä tarvitaan esimerkiksi tilanteissa, joissa lasketaan korolle korkoa. Myös pinta-ala- ja tilavuuslaskuissa esiintyy potenssiyhtälöitä.

\luettelolaatikko{Potenssiyhtälön ratkaiseminen}{
	§ Jos potenssiyhtälön aste $n$ on parillinen ja $a \ge 0$, yhtälöllä on kaksi ratkaisua, $$ x = \pm \sqrt[n]{a} \textrm{.} $$ ($\pm$ tarkoittaa, että sekä positiivinen että negatiivinen arvo käyvät: $\pm 5$ tarkoittaa sekä lukuja $5$ että $-5$.)
	§ Jos aste on parillinen, sillä on yksi ratkaisu vain kun $a = 0$, sillä $ x = \pm \sqrt[n]{0} = \pm 0 = 0$
	§ Jos aste on parillinen ja $a < 0$, potenssiyhtälöllä ei ole yhtään ratkaisua.
	§ Siis parillisen asteen potenssifunktiolla voi olla yksi, kaksi tai ei yhtään ratkaisua.
	§ Jos aste on pariton, ja $a \neq 0$  yhtälöllä on aina täsmälleen yksi ratkaisu, $x = \sqrt[n]{a}.$
	§ Jos aste on pariton, ja $a = 0$, yhtälöllä ei ole yhtään ratkaisua.
}

\begin{esimerkki}
Ratkaise yhtälöt a) $x^2 = 0$, b) $x^2 - 9 = 0$ ja c) $x^2 + 9 = 0$

a)	\begin{align*}
	x^2 &= 0 \\
	x &= 0
	\end{align*}
	Yhtälöllä on yksi ratkaisu $x = 0$.

b)	\begin{align*}
	x^2 - 9 &= 0 \\
	x^2 &= 9 \\
	x &= \pm 3
	\end{align*}
	Yhtälöllä on kaksi ratkaisua $x = 3$ ja $x = -3$, sillä $3^2 = 9$ ja $(-3)^2 = 9$.

c)	\begin{align*}
	x^2 + 9 &= 0 \\
	x^2 &= -9 \\
	x &= \sqrt{-9}
	\end{align*}
	Yhtälöllä ei ole reaalista ratkaisua.

\end{esimerkki}

\laatikko{Mikäli potenssiyhtälön asteluku $n$ on pariton, yhtälöllä on aina tasan yksi reaalilukuratkaisu (kun $a \neq 0$).}

\begin{esimerkki}
Ratkaise yhtälö $2x^3 + 16 = 0$

	\begin{align*}
	2x^3 + 16 &= 0 \\
	2x^3 &= -16 \\
	x^3 &= -8  \\
	x = \sqrt[3]{-8} &= -2
	\end{align*}
\end{esimerkki}

\begin{esimerkki}
	\alakohdat{
		§ Yhtälö $27x^3=7$ on potenssiyhtälö, sillä jakamalla se puolittain luvulla $27$ saadaan $x^3 = \frac{7}{27}$. Tämän potenssiyhtälön aste on 3.
		§ Yhtälö $2x^{4}-7=3$ on potenssiyhtälö, sillä se voidaan muokata muotoon $x^n = a$, ja sen aste on 4. \\
				\[2x^{4} -7 = 3\]
				\[2x^{4} = 3+7\]
				\[x^{4} = \frac{10}{2}\]
				\[x^{4} = 5.\]
	}
\end{esimerkki}

\begin{luoKuva}{potenssiyhtaloratk}
	kuvaaja.pohja(-4, 6, -50, 110, leveys = 9, korkeus = 6, nimiX = "$x$", nimiY = "$y$")
	with vari("red"): kuvaaja.piirra(lambda x: x**3, kohta = -3, suunta = -45, nimi = r"$y = x^3$")
	with vari("green!70!black"): kuvaaja.piirra(lambda x: x**4, kohta = -2.86, suunta = -180, nimi = r"$y = x^4$")
	with vari("blue"): kuvaaja.piirra(lambda x: x**6, kohta = -2, suunta = 45, nimi = r"$y = x^6$")
	A1 = geom.piste(-50**0.25, 50, suunta = -135, nimi = "\\tiny$(-\sqrt[4]{50}, 50)$")
	A2 = geom.piste(50**0.25, 50, suunta = 45, nimi = "\\tiny$(\sqrt[4]{50}, 50)$")
	B = geom.piste(100**(1./3), 100, suunta = -20, nimi = "\\tiny$(\sqrt[3]{100}, 100)$")
	
	def xproj(p): return (p[0], 0)
	def yproj(p): return (0, p[1])
	def kulma(p):
		geom.jana(xproj(p), p)
		geom.jana(yproj(p), p)
	
	kulma(A1)
	kulma(A2)
	kulma(B)
\end{luoKuva}
\begin{esimerkki}
	\alakohdat{
		§ Potenssiyhtälön $x^3 = 100$ ratkaisu on $x=\sqrt[3]{100}$.
		§ Potenssiyhtälöllä $x^6 = -1$ ei ole ratkaisua, sillä $x^6 = (x^3)^2 \ge 0$ kaikilla $x$.
		§ Potenssiyhtälöllä $x^4=50$ on kaksi ratkaisua $x=\sqrt[4]{50}=2,6591...$ ja $x=-\sqrt[4]{50}=-2,6591...$.
	}
	\begin{center}
		\naytaKuva{potenssiyhtaloratk}
	\end{center}
\end{esimerkki}

\begin{esimerkki}
Etsitään luku $x$, joka toteuttaa yhtälön $\frac{x^3}{3}=\frac{x^2}{2}$. Tehdään tämä kahdella tapaa vaiheittain

		\begin{align*}
			\frac{x^3}{3}&=\frac{x^2}{2} && \text{| Kerrotaan molemmat puolet kolmella.} \\
			x^3 &=\frac{3x^2}{2}   && \text{| Kerrotaan molemmat puolet kahdella.} \\
			2x^3 &=3x^2 && \text{| Vähennetään puolittain oikean puolen termillä.} \\
			2x^3 -3x^2&=0 && \text{| Jaetaan $x^2$. Huomataan ja merkitään että $x\neq0$.} \\
			2x -3&=0 && \text{| Lisätään molemmille puolille $3$ ja jaetaan kahdella.} \\ 
			x&=\frac{3}{2} && \\
		\end{align*}
Tutkitaan vielä erikseen tilanne $x=0$, joka ei ole määritelty yllä olevassa osamäärässä. Sijoitetaan $x=0$ yhtälöön $\frac{x^3}{3}=\frac{x^2}{2}$ ja saadaan $\frac{0^3}{3}=\frac{0^2}{2}$, joka pätee sillä $0=0$. Näin ollen siis myös $x=0$ on ratkaisu. 

Toinen tapa
\begin{align*}
\frac{x^3}{3}&=\frac{x^2}{2} && \text{| Jaetaan molemmat puolet oikean puolen termillä. } \\
\frac{x^3}{3}:\frac{x^2}{2}&=1 && \text{| Kerrotaan jakajan käänteisluvulla.} \\
\frac{x^3\cdot2}{3\cdot x^2}&=1 && \text{| Sievennetään. Huomataan ja merkitään, että nyt $x\neq0$.} \\
\frac{x\cdot2}{3}&=1 && \text{| Jaetaan puolet kahdella ja kerrotaan kolmella.} \\
x&=\frac{3}{2} && \\
\end{align*}

Tutkitaan vielä erikseen tilanne $x=0$, joka ei ole määritelty yllä olevassa osamäärässä. Sijoitetaan $x=0$ yhtälöön $\frac{x^3}{3}=\frac{x^2}{2}$ ja saadaan $\frac{0^3}{3}=\frac{0^2}{2}$, joka pätee, sillä $0=0$. Näin ollen siis myös $x=0$ on ratkaisu.

Eli molemmin tavoin saatiin sama tulos, vaikka yhtälöä muokattiin eri keinoin.

\end{esimerkki}

\begin{esimerkki}
Taulutelevision kooksi (lävistäjäksi) on ilmoitettu mainoksessa $46,0$ tuumaa ($116,8$ cm) ja kuvasuhteeksi 16:9. Kuinka leveä televisio on (senttimetreinä)?

%FIXME: KIMMON KUVITUSKUVA!

{\bf Ratkaisu.}

Taulutelevision halkaisija, alareuna ja toinen sivu muodostavat suorakulmaisen kolmion. Kolmion hypotenuusa ($c$) on television halkaisija ja kateetit ($a$ ja $b$) alareuna ja toinen sivu.

Kuvasuhteen perusteella kateettien pituuksia voidaan merkitä $16x$ ja $9x$. Pythagoraan lauseesta ($c^2 = a^2 + b^2$) saadaan
\[
116,8^2 = (16x)^2 + (9x)^2
\]
\[
13\,642,24 = (256+81)x^2.
\]
\[
x^2 = \frac{13\,642,24}{337}
\]
\[
x= \sqrt{\frac{13\,642,24}{337}} \approx 6,36.
\]
Television leveys on noin $16x = 16\cdot 6,36\approx 102$\,cm.

{\bf Vastaus.} Noin $102$\,cm.
\end{esimerkki}


\begin{esimerkki}
Suursijoittaja Nalle Mursulla on $5\,000$ euroa ylimääräistä rahaa, jonka hän aikoo sijoittaa $30$ vuodeksi.  Nalle Mursu haluaa sijoittamansa pääoman kasvavan $100\,000$ euroksi $30$ vuodessa.  Kuinka suuren vuotuisen korkokannan Nalle Mursu tarvitsee sijoitukselleen? 
	\begin{esimratk}
		Olkoon vuotuinen korkokanta $r$. Korkoa korolle -periaatteen nojalla $5\,000$ euron sijoitus kasvaa $30$ vuodessa summaksi $5\,000\cdot(1+r)^{30}$. Merkitsemällä $x=1+r$ saamme yhtälön $5\,000\cdot x^{30} = 100\,000$. Jakamalla yhtälö puolittain luvulla $5\,000$ päädymme potenssiyhtälöön
		\[ x^{30} = 20, \] 
		jonka ratkaisuksi saadaan $x=20^{\frac{1}{30}} \approx 1,105$. Näin ollen suursijoittaja Nalle Mursun vaatima korkokanta sijoitukselleen on noin $r=1-x=1-1,105=0,105=10,5\,\%$.
	\end{esimratk}
\end{esimerkki}
