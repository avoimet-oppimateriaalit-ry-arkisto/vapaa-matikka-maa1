\begin{tehtavasivu}

\subsubsection*{Opi perusteet}

\begin{tehtava} Supista niin pitkälle kuin mahdollista.
	\alakohdat{
		§ $\frac{15}{20}$
		§ $\frac{27}{30}$
		§ $\frac{2}{32}$
		§ $\frac{65}{105}$
		§ $\frac{144}{243}$
	}
\begin{vastaus}
	\alakohdat{
		§ $\frac{3}{4}$
		§ $\frac{9}{10}$
		§ $\frac{1}{16}$
		§ $\frac{13}{21}$
		§ $\frac{16}{27}$
	}
\end{vastaus}
\end{tehtava}

%\begin{tehtava}
%Supista. \quad
%a) $\frac{3\cdot 5 \cdot6}{4\cdot5\cdot 6}$ \qquad b) $\frac{3a+ab}{a}$ \qquad c) $\frac{3a\cdot a\cdot(4b)\cdot b}{(4a)b}$
%\begin{vastaus}
%a) $\frac{3}{4}$ \qquad b) $3+b$\qquad c) $3ab$
%\end{vastaus}
%\end{tehtava}

\begin{tehtava}
Lavenna samannimisiksi
\alakohdat{
§ $\frac{2}{3}$ ja $\frac{4}{5}$ 
§ $\frac{5a}{6}$ ja $\frac{7b}{9}$
§ $\frac{2}{3a}$ ja $\frac{5b}{2}$.
}
\begin{vastaus}
\alakohdat{
§ $\frac{10}{15}$ ja $\frac{12}{15}$
§ $\frac{15a}{18}$ ja $\frac{14b}{18}$
§ $\frac{4}{6a}$ ja $\frac{15ab}{6a}$
}
\end{vastaus}
\end{tehtava}

\begin{tehtava}
Laske
	\alakohdat{
		§ $\frac{3}{11}+\frac{5}{11}$
		§ $\frac{4}{5}-\frac{1}{5}$
		§ $\frac{2}{3}+\frac{1}{6}+\frac{2}{6}$
		§ $ \frac{11}{12}-\frac{5}{6}+\frac{3}{2}$.
	}
	\begin{vastaus}
		\alakohdat{
			§ $\frac{8}{11}$
			§ $\frac{3}{5}$
			§ $\frac{7}{6}$
			§ $\frac{19}{12}$
		}
	\end{vastaus}
\end{tehtava}

\begin{tehtava}
Esitä sekamurtolukuna
\alakohdat{
§ $\frac{15}{2}$ 
§ $\frac{177}{16}$
§ $\frac{2\,013}{15}$.
}
\begin{vastaus}
\alakohdat{
§ $7\frac{1}{2}$
§ $11\frac{1}{16}$
§ $134\frac{3}{15}$
}
\end{vastaus}
\end{tehtava}

\begin{tehtava}
Muunna murtoluvuksi
\alakohdat{
§ $3\frac{2}{5}$
§ $4\frac{1}{34}$
§ $2\frac{6}{3\,145}$.
}
\begin{vastaus}
\alakohdat{
§ $\frac{17}{5}$
§ $\frac{137}{34}$
§ $\frac{6\,296}{3\,145}$
}
\end{vastaus}
\end{tehtava}

\begin{tehtava}
Laske
	\alakohdat{
		§ $1\frac{2}{9}+\frac{5}{9}$
		§ $\frac{1}{3}+2\frac{1}{3}$
		§ $2+\frac{5}{4}+3\frac{3}{4}$
		§ $\frac{3}{2}+1-\frac{5}{6}$.
	}
	\begin{vastaus}
		\alakohdat{
			§ $\frac{16}{9}$ tai $1\frac{7}{9}$  %kaikki vastaukset sekä murtona että sekana!
			§ $\frac{8}{3}$
			§ $\frac{7}{1}$
			§ $1\frac{2}{3}$
		}
	\end{vastaus}
\end{tehtava}

\begin{tehtava}
Laske
	\alakohdat{
		§ $\frac{2}{3}\cdot \frac{4}{5}$
		§ $\frac{3}{5} \cdot \frac{5}{4}$
		§ $2\cdot \frac{2}{7}\cdot\frac{3}{4}$
		§ $\frac{3}{2}\cdot\frac{4}{5}\cdot 10$.
	}
	\begin{vastaus}
		\alakohdat{
			§ $\frac{8}{15}$
			§ $\frac{15}{20}=\frac{3}{4}$
			§ $\frac{3}{7}$
			§ $12$
		}
	\end{vastaus}
\end{tehtava}


\begin{tehtava} Laske
	\alakohdat{
		§ $\frac{2}{3} : \frac{7}{11}$
		§ $\frac{4}{3}:\left(\frac{-13}{4}\right)$
		§ $\frac{7}{8}:4$
		§ $\frac{1}{\frac{3}{7}-\frac{9}{21}}$.
	}
\begin{vastaus}
	\alakohdat{
		§ $1\frac{1}{21}$
		§ $-\frac{16}{39}$
		§ $\frac{7}{32}$
		§ ei määritelty (nollalla jakaminen)
	}
\end{vastaus}
\end{tehtava}

\begin{tehtava}
Aseta luvut suuruusjärjestykseen
\alakohdat{
§ $\frac{-15}{11}, \frac{-19}{9}, \frac{11}{12}, \frac{3}{4}, -7$
§ $\frac{15}{21}, \frac{7}{-3}, \frac{-3}{-4}, \frac{8}{9}, \frac{5}{-2}$.
}
\begin{vastaus}
\alakohdat{
§ $-7, \frac{-19}{9}, \frac{-15}{11}, \frac{3}{4}, \frac{11}{12}.$
§ $\frac{5}{-2}, \frac{7}{-3}, \frac{15}{21}, \frac{-3}{-4}, \frac{8}{9}.$
}
\end{vastaus}
\end{tehtava}


\begin{tehtava}
Laske luvun $\frac{3}{4}$
	\alakohdat{
		§ vastaluku
		§ käänteisluku
		§ vastaluvun vastaluku
		§ vastaluvun käänteisluku
		§ käänteisluvun vastaluku
		§ käänteisluvun käänteisluku.
		§ Mitkä tehtävän luvuista ovat samoja kuin alkuperäinen luku? Miksi?
		§ Mitkä tehtävän luvuista ovat samoja keskenään? Miksi?
	}
	\begin{vastaus}
		\alakohdat{
			§ $\frac{-3}{4}$
			§ $\frac{4}{3}$
			§ $\frac{3}{4}$
			§ $-\frac{4}{3}$
			§ $-\frac{4}{3}$
			§ $\frac{3}{4}$
			§ kohdat c ja f; Luvun vastaluvun vastaluku on sama kuin luku itse -- samoin käänteisluvun käänteisluku.
			§ kohdat d ja e: käänteisluvun vastaluku on sama kuin vastaluvun käänteisluku. 
		}
	\end{vastaus}
\end{tehtava}


\begin{tehtava}
Laske lukujen $\frac{5}{4}$ ja $\frac{1}{8}$ 
	\alakohdat{
		§ vastalukujen summa
		§ käänteislukujen summa
		§ summan vastaluku
		§ summan käänteisluku
	}
Huomioi, mitkä luvuista ovat samoja.
	\begin{vastaus}
		\alakohdat{
			§ $\frac{-11}{8}$
			§ $8\frac{5}{4}=9\frac{1}{4}$
			§ $-\frac{11}{8}$
			§ $-\frac{8}{11}$
		}
		
	\end{vastaus}
\end{tehtava}

\subsubsection*{Hallitse kokonaisuus}

\begin{tehtava}
Laske murtolukujen $\frac{5}{6}$ ja $-\frac{2}{15}$\\ a) summa \qquad b) erotus \qquad c) tulo \qquad d) osamäärä.
\begin{vastaus}
a) $\frac{7}{10}$ \qquad b) $\frac{29}{30}$ \qquad c) $-\frac{1}{9}$ \qquad d) $-\frac{25}{4}$ eli $-6\frac{1}{4}$
\end{vastaus}
\end{tehtava}

\begin{tehtava} Laske
a) $\frac{5}{8}\cdot(\frac{3}{5}+\frac{2}{5})$ \qquad b) $\frac{1}{3}+\frac{1}{4}\cdot\frac{6}{5}$.
\begin{vastaus}
a) $\frac{5}{8}$ \qquad b) $\frac{19}{30}$
\end{vastaus}
\end{tehtava}

%lisää alakohtia
\begin{tehtava}
Supista \quad
a) $\frac{12a+2ab}{4a}$ \qquad b) $\frac{a\cdot a\cdot (3b)}{3a}$.
\begin{vastaus}
a) $\frac{6+b}{2}$\qquad b) $ab$
\end{vastaus}
\end{tehtava}

\begin{tehtava}Laske
a) $\frac{\frac{1}{2}:\frac{3}{2}}{\frac{3}{2}+\frac{1}{3}}$ \qquad b) $\frac{\frac{2}{3}+\frac{3}{4}}{\frac{5}{6}-\frac{7}{12}}$.
\begin{vastaus}
a) $\frac{2}{11}$ \qquad b) $5\frac{2}{3}$
\end{vastaus}
\end{tehtava}

\begin{tehtava}
Millä luvun $x$ arvolla lukujen 
\alakohdat{
§ $x$ ja $\frac{3}{8}$ tulo on $0$?
§ $x$ ja $\frac{3}{8}$ tulo on $1$?
§ $x$, $\frac{7}{91}$ ja $\frac{23}{2}$ tulo on $1$?
}
\begin{vastaus}
\alakohdat{
§ Kun $x=0$ (ja vain silloin; tulosta tulee nolla vain nollalla)
§ $\frac{8}{3}$ (käänteisluvun määritelmä)
§ $\frac{26}{23}$
}
\end{vastaus}
\end{tehtava}

\begin{tehtava} 
        Laatikossa on palloja, joista kolmasosa on mustia, neljäsosa valkoisia ja viidesosa harmaita. Loput palloista ovat punaisia. Kuinka suuri osuus palloista on punaisia?
        
        \begin{vastaus}
            $1-(\frac{1}{3}+\frac{1}{4}+\frac{1}{5})
            = \frac{60}{60}-\frac{20}{60}-\frac{15}{60}-\frac{12}{60}
            = \frac{60}{60}-\frac{47}{60}
            = \frac{13}{60}$
        \end{vastaus}
    \end{tehtava}
    
\begin{tehtava} 
Eräässä pitkän matematiikan ensimmäisen kurssin ryhmässä on $16$ opiskelijaa. Heistä $8$ on tyttöjä ja tytöistä neljänneksellä on siniset silmät. Kuinka suuri osa luokan oppilaista on sinisilmäisiä tyttöjä?
        \begin{vastaus}
			Yksi kahdeksasosa
        \end{vastaus}
\end{tehtava}
%Laatinut Henri Ruoho 9.11.2013

%ryhmätehtävä, jossa ... poikia, tyttöjä ja muita :)

\begin{tehtava}
Laske lausekkeen $\frac{x}{2-3x}$ arvo, kun $x$ on
\alakohdat{
§ $4$
§ $-\frac{1}{2}$
§ $\frac{7}{10}$.
}
\begin{vastaus}
\alakohdat{
§ $-\frac{2}{5}$
§ $-\frac{1}{7}$
§ $-7$
}
\end{vastaus}
\end{tehtava}

\begin{tehtava}
Selvitä, millä nollasta poikkeavilla luvuilla on se ominaisuus, että luvun vastaluvun käänteisluku on yhtä suuri kuin luvun käänteisluvun vastaluku.
\begin{vastaus}
Kaikilla, jos $a \neq 0$, $-\frac{1}{a} = \frac{1}{-a}$
\end{vastaus}
\end{tehtava}

\subsubsection*{Lisää tehtäviä}

\begin{tehtava}
Laske lausekkeen $\frac{x+y}{2x-y}$ arvo, kun
\alakohdat{
§ $x=\frac{1}{2}$ ja $y= \frac{1}{4}$
§ $x=\frac{1}{4}$ ja $y= -\frac{3}{8}$.
}
\begin{vastaus}
\alakohdat{
§ $1$
§ $-\frac{1}{7}$
}
\end{vastaus}
\end{tehtava}
 
        \begin{tehtava} Laske %ja esitä vastaus murtolukuna.(?)
            \alakohdat{
        	§ $\frac{3}{5} + \frac{1}{5}$
        	§ $\frac{5}{7} + \frac{4}{7}$
        	§ $2 + \frac{2}{3}$
        	§ $3 + \frac{3}{5} + \frac{2}{5}$.
            }
            \begin{vastaus}
        		\alakohdat{
        			§ $\frac{4}{5}$
        			§ $\frac{9}{7} = 1 \frac{2}{7}$
        			§ $2 \frac{2}{3} = \frac{8}{3}$
        			§ $4$
        		}
            \end{vastaus}
        \end{tehtava}
        
        \begin{tehtava}
        \alakohdat{
        	§ $\frac{6}{2} + \frac{3}{5}$
        	§ $\frac{7}{8} - \frac{1}{4}$
        	§ $2 \frac{1}{3} + \frac{4}{6}$
        	§ $4 \frac{7}{2} - 6 \frac{5}{4}$
        }
            \begin{vastaus}		
        		\alakohdat{
        			§ $\frac{18}{5}$
        			§ $\frac{5}{8}$
        			§ $3$
        			§ $\frac{1}{4}$
        		}
            \end{vastaus}
        \end{tehtava}
        
        \begin{tehtava}
        \alakohdat{
        	§ $2 \cdot \frac{2}{5}$
        	§ $2 \cdot \frac{2}{3}$
        	§ $\frac{5}{4} \cdot 2 \cdot 3$
        	§ $\frac{\frac{3}{7}}{4}$ 
        }
            \begin{vastaus}
        		\alakohdat{
        			§ $\frac{4}{5}$
        			§ $\frac{4}{3} = 1 \frac{1}{3}$
        			§ $\frac{15}{2} = 7 \frac{1}{2}$
        			§ $\frac{3}{28}$
        		}
            \end{vastaus}
        \end{tehtava}
        
        \begin{tehtava}
        \alakohdat{
        	§ $\frac{1}{3} \cdot \frac{6}{5}$
        	§ $\frac{5}{4} \cdot (-\frac{2}{3})$ 
        	§ $\frac{2}{5} (2 - \frac{3}{4})$
        	§ $(\frac{5}{6} - \frac{1}{3})(\frac{7}{4} - \frac{3}{2})$
        }
            \begin{vastaus}		
        		\alakohdat{
        			§ $\frac{2}{5}$
        			§ $-\frac{5}{6}$
        			§ $\frac{1}{2}$
        			§ $\frac{1}{8}$ 
        		}
            \end{vastaus}
        \end{tehtava}
        
        \begin{tehtava} 
        \alakohdat{
        	§ $\displaystyle \frac{\frac{3}{7} + \frac{5}{4}}{3}$
        	§ $\displaystyle \frac{\frac{10}{8}}{\frac{5}{2}}$
        	§ $\displaystyle \frac{\frac{1}{3} - \frac{5}{10}}{\frac{3}{4} + \frac{1}{2}}$
        	§ $\displaystyle 3\frac{\frac{4}{2} + \frac{10}{4}}{\frac{3}{2} - \frac{2}{3}}$
        }
            \begin{vastaus}		
        		\alakohdat{
        			§ $\frac{47}{84}$
        			§ $\frac{1}{2}$
        			§ $-\frac{2}{15}$
        			§ $\frac{27}{5}$
        		}
            \end{vastaus}
        \end{tehtava}
        
        \begin{tehtava}
	\alakohdat{
		§ $\frac{4}{9} : \frac{1}{5}$
		§ $\frac{2}{7} : \frac{5}{9}$
		§ $\frac{2}{3}:\frac{4}{3}$
	}
	\begin{vastaus}
		\alakohdat{
			§ $\frac{20}{9}$
			§ $\frac{18}{35}$
			§ $\frac{1}{2}$
		}
	\end{vastaus}
\end{tehtava}

\begin{tehtava}
    Laske 
    $\frac{10}{9}\cdot \frac{9}{8}\cdot \frac{8}{7}\cdot \frac{7}{6}\cdot \frac{6}{5}
    \cdot \frac{5}{4}\cdot \frac{4}{3}\cdot \frac{3}{2}$.
    \begin{vastaus}
		$5$
    \end{vastaus}        
\end{tehtava}
    
    \begin{tehtava}
        Mira, Pontus, Jarkko-Kaaleppi ja Milla leipoivat lanttuvompattipiirakkaa. Pontus kuitenkin söi piirakasta kolmanneksen ennen muita, ja loput piirakasta jaettiin muiden kanssa tasan. Kuinka suuren osan muut saivat?
        
        \begin{vastaus}
            Muut saivat piirakasta kuudesosan.
        \end{vastaus}
    \end{tehtava}
    
\begin{tehtava}
    Huvipuiston sisäänpääsylippu maksaa $20$ euroa, ja lapset pääsevät sisään puoleen hintaan.
	\alakohdat{
		§ Kuinka paljon kolmen lapsen yksinhuoltajaperheelle maksaa päästä sisään?
		§ Kuinka paljon sisäänpääsy maksaa perheelle avajaispäivänä, kun silloin sisään pääsee neljänneksen ($25$\,\%) halvemmalla?
    }
    \begin{vastaus}
		\alakohdat{
			§ $50$ euroa 
			§ $50$ euroa 
			§ $37,50$ euroa
		} 
    \end{vastaus}
\end{tehtava}  

\begin{tehtava}
	Eräässä kaupassa on käynnissä loppuunmyynti, ja kaikki tuotteet myydään puoleen hintaan. Lisäksi kanta-asiakkaat saavat aina viidenneksen alennusta ostoksistaan. Paljonko kanta-asiakas maksaa nyt tuotteesta, joka normaalisti maksaisi $40$ euroa?
    \begin{vastaus}
		$40\cdot \frac{1}{2} \cdot \frac{4}{5}=40\cdot \frac{4}{10}= 16$. 
	\end{vastaus}
\end{tehtava}
    
\begin{tehtava}
	Kokonaisesta kakusta syödään maanantaina iltapäivällä puolet, ja jäljelle jääneestä palasta syödään tiistaina iltapäivällä taas puolet. Jos kakun jakamista ja syömistä jatketaan samalla tavalla koko viikko, kuinka suuri osa alkuperäisestä kakusta on jäljellä seuraavana maanantaiaamuna?
	\begin{vastaus}
		Toisena päivänä aamulla kakkua on jäljellä puolet, kolmantena päivänä aamulla
		$1-\left(\frac{1}{2} + \frac{1}{4}\right) = \frac{1}{4}$, 
		neljäntenä päivänä
		$1-\left(\frac{1}{2} + \frac{1}{4} + \frac{1}{8}\right)
		= \frac{1}{8}$, jne.
		Siis seitsemän päivän jälkeen kakkua on jäljellä
		$1-\left(\frac{1}{2} + \frac{1}{4} + \frac{1}{8} +
		\frac{1}{16} + \frac{1}{32} + \frac{1}{64} + \frac{1}{128}\right)
		= \frac{1}{128}$.  
	\end{vastaus}
\end{tehtava}

\begin{tehtava}
	Laske lausekkeen $\frac{1}{n}-\frac{1}{m}$ arvo, kun tiedetään, että $n = \frac{1}{9}$ ja $m=n+1$.
	\begin{vastaus}
		$\frac{81}{10}$
	\end{vastaus}
\end{tehtava}

\begin{tehtava}
 Tarkastellaan lukuja $\frac{3+a}{a}$ ja $\frac{\frac{5}{3}+a}{a}$. Kumpi luvuista on suurempi, kun $a$ on kokonaisluku ja suurempi tai yhtäsuuri kuin $1$?
 \begin{vastaus}
  Luku $\frac{3+a}{a}$, sillä $3>\frac{5}{3}$. Vakio a on molemmissa sama luku, joten se vaikuttaa samalla tavalla molempien lukujen suuruuteen.
 \end{vastaus}
\end{tehtava}

\begin{tehtava}
	Laske lausekkeen $\frac{1}{n}-\frac{1}{2n}+\frac{1}{3n}$ arvo, kun tiedetään, että $n = 10$.
	\begin{vastaus}
		$\frac{1}{12}$
	\end{vastaus}
\end{tehtava}
\begin{tehtava}
	Määritä nollasta poikkeavien rationaalilukujen \(r\) ja \(s\) käänteislukujen summan käänteisluku. Minkä arvon saat, jos \(r=\frac{2}{3}\) ja \(s=3\)?
	\begin{vastaus}
		Summa on $\frac{rs}{r+s}$. Jos \(r=\frac{2}{3}\) ja \(s=3\), niin lausekkeen arvo on \(\frac{6}{11}\).
	\end{vastaus}
\end{tehtava}
%Laatinut Henri Ruoho 9.11.2013

\begin{tehtava} $\star$ \alakohdat{
		§ Jos $n$ on positiivinen kokonaisluku, laske lukujen $n$ ja $(n+1)$ käänteislukujen erotus.
		§ Laske summa $\frac{1}{1\cdot 2}+\frac{1}{2 \cdot 3}+ \ldots + \frac{1}{(n-1)n} $.
	}
	\begin{vastaus}
		\alakohdat{
			§ $\frac{1}{n(n+1)}$
			§ $1-\frac{1}{n}$
		}
	\end{vastaus}
\end{tehtava}

\begin{tehtava} $\star$
%Laatinut Jaakko Viertiö 2013-11-9
	Määritä ne positiiviset reaaliluvut $x$, jotka ovat käänteislukuaan $\frac{1}{x}$ suurempia.
	\begin{vastaus}
	 Kun $x>1$.
	\end{vastaus}
\end{tehtava}

\begin{tehtava} $\star $
%Laatinut Jaakko Viertiö 2013-11-9
Sievennä
 \alakohdat{
  § $a+{b}\cdot{\frac{a}{b}}\cdot{\frac{\frac{b}{c}}{\frac{ab}{c}}}$
  § $\frac{208ab+52b+26b({\frac{a}{26}}-{\frac{6a}{3a}})-ab}{-42ab+52b+42ba}$.
 }
 \begin{vastaus}
	\alakohdat{
		§ $a+1$
		§ $4a$
	}
 \end{vastaus}
\end{tehtava}

\begin{tehtava}
	$\star$ Vanhalla matemaatikolla on kolme lasta. Eräänä päivänä hän antaa lapsilleen laatikollisen vuosien varrella ongelmanratkaisukilpailuista voitettuja palkintoja. Hän kertoo antavansa vanhimmalle lapselleen puolet saamistaan arvoesineistä, keskimmäiselle neljäsosan ja nuorimmalle kuudesosan. Laatikossa on kuitenkin vain $11$ palkintoa. Miten  palkinnot jaetaan ja kuinka monta arvoesinettä matemaatikko pitää itsellään?
	\begin{vastaus}
		Vanhin sai $6$ esinettä, keskimmäinen $3$ esinettä ja nuorin $2$ esinettä. Vanha
		matemaatikko pitää yhden palkinnon itsellään, sillä $\frac{1}{2} + \frac{1}{4}
		+ \frac{1}{6} = \frac{11}{12}$. (Tämä vastaus on oikein. Jos ihmetyttää, kannattaa lukea tarkkaan, mitä tehtävässä oikein sanotaan.)
	\end{vastaus}
\end{tehtava}

\begin{tehtava}
	$\star$ Fibonaccin luvut $0$, $1$, $1$, $2$, $3$, $5$, $8$, $13$, $21$, $\ldots$ määritellään seuraavasti: Kaksi ensimmäistä Fibonaccin lukua ovat $0$ ja $1$, ja siitä seuraavat saadaan kahden edellisen summana: $0+1=1$, $1+1=2$, $1+2 = 3$, $2+3=5 $ ja niin edelleen. Tutki, miten Fibonaccin luvut liittyvät lukuihin
	$ \frac{1}{1+1}$, $\frac{1}{1+\frac{1}{1+1}}$, 
	$\frac{1}{1+\frac{1}{1+\frac{1}{1+1}}}$, 
	$\frac{1}{1+\frac{1}{1+\frac{1}{1+\frac{1}{1+1}}}}$, $\ldots\ $
	\begin{vastaus}
		Luvut ovat sievennettynä peräkkäisten Fibonaccin lukujen osamääriä:
		\[\frac{1}{2}, \ \frac{2}{3}, \ \frac{3}{5}, \frac{5}{8} \ldots  \]
	\end{vastaus}
\end{tehtava}

\begin{tehtava}
% Ryhmittely
Sievennä ryhmittelemällä.
	\alakohdat{
	§ $2x^2+3x+5x^2$
	§ $x^2+3x^3+x^2+x^3+2x^2$
	§ $ax^2+bx+cx$
	§ $ax^3+bx+cy^3+dx+ey^3+fx^3$
	}

\begin{vastaus}
	\alakohdat{
	§ $7x^2+3x$
	§ $4(x^2+x^3)$ tai $4x^2+4x^3$
	§ $ax^2+(b+c)x$ tai $ax^2+bx+cx$
	§ $(a+f)x^3+(b+d)x+(c+e)y^3$
	}
\end{vastaus}
\end{tehtava}

\begin{tehtava}
Sievennä.
	\alakohdat{
	§ $ \frac{1-x}{3} + \frac{x-2}{6}$
	§ $ \frac{5x-1}{3} - \frac{2x+5}{2}$
	§ $\frac{x}{6y} \cdot \frac{3y}{2}$
	}

\begin{vastaus}
	\alakohdat{
	§ $ -\frac{x}{6}$
	§ $ \frac{2}{3} x - \frac{17}{6}$
	§ $\frac{x}{4}$
	}
\end{vastaus}
\end{tehtava}

\begin{tehtava} $\star$
Sievennä $(2ab+4b^2-8b^3c):(a+2b-4b^2c)$.
	\begin{vastaus}
	$2b$
	\end{vastaus}
\end{tehtava}

\end{tehtavasivu}