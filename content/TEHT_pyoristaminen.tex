\begin{tehtavasivu}
\subsubsection*{Opi perusteet}

\begin{tehtava}
Pyöristä kymmenen tarkkuudella
\alakohdat{
§ $8$
§ $23$
§ $65$
§ $9\,001$
§ $1$
§ $-6$
§ $-12$
§ $99\,994$.
}
\begin{vastaus}
\alakohdat{
§ $10$
§ $20$
§ $70$
§ $9\,000$
§ $0$
§ $-10$
§ $-10$
§ $99\,990$
}
\end{vastaus}
\end{tehtava}

%ja lisää tehtäviä... :) (negatiiviset, eri pyöristysskeemat, ...)
\begin{tehtava}
Montako merkitsevää numeroa on seuraavissa luvuissa
\alakohdat{
§ $5$
§ $12,0$
§ $9\,000$
§ $666$
§ $9\,000,000$
§ $0,0$
§ $-1$
§ $-0,00024$?
}
	\begin{vastaus}
\alakohdat{
§ $1$
§ $2$
§ $1$
§ $3$
§ $7$
§ $2$
§ $1$
§ $2$
}
	\end{vastaus}
\end{tehtava}

\begin{tehtava}
Pyöristä kahden merkitsevän numeron tarkkuudella
\alakohdat{
§ $1,0003$
§ $3,69$
§ $152,8$
§ $7\,062,4$
§ $-2,05$
§ $0,00546810$
§ $-1\,337$
§ $0,0000501$.
}
	\begin{vastaus}
\alakohdat{
§ $1,0$
§ $3,7$
§ $150$
§ $7\,000$
§ $-2,0$
§ $0,0055$
§ $-1\,300$
§ $0,000050$
}
	\end{vastaus}
\end{tehtava}

%pyöristys, kellonajat vartin tarkkuudella
%\subsubsection*{Hallitse kokonaisuus}



\subsection*{Lisää tehtäviä}

\begin{tehtava}
$\star$ Kirjoita haluamallasi ohjelmointikielellä koodi, joka määrittää annetusta luvusta, kuinka monta merkitsevää numeroa siinä on.
\end{tehtava}

\end{tehtavasivu}