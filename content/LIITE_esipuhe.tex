Matematiikka on kaikkien luonnontieteiden perusta. Se tarjoaa työkaluja asioiden täsmälliseen jäsentämiseen, päättelyyn ja mallintamiseen, joita ilman yhteiskunnan teknologinen ja tieteellinen kehittyminen ei olisi mahdollista. Alasta riippuen käsittelemme matematiikassa erilaisia \textbf{objekteja}: Geometriassa tarkastelemme kaksiulotteisia \textbf{tasokuvioita} ja kolmiulotteisia \textbf{avaruuskappaleita}. Algebra tutkii \textbf{lukualueita} ja niissä määriteltäviä \textbf{laskutoimituksia}. Todennäköisyyslaskenta tarkastelee satunnaisten \textbf{tapahtumien} esiintymistä. Matemaattinen analyysi tutkii \textbf{funktioita} ja niiden ominaisuuksia, esimerkiksi \textbf{jatkuvuutta}, \textbf{derivoituvuutta} ja \textbf{integroituvuutta}. Matemaattista analyysiä käsittelevät pitkässä matematiikassa kurssit 7, 8, 10 ja 13 sekä osin kurssit 9 ja 12. Voidaankin sanoa, että analyysi on keskeisin aihealue lukion pitkässä matematiikassa.\footnote[1]{Tämä Suomessa käytetty lähestymistapa on käytössä monissa länsimaissa. Sen sijaan esimerkiksi Balkanilla geometrialle ja matemaattiselle todistamiselle annetaan edelleen paljon suurempi painoarvo.} Matematiikka opettaa loogista päättelytaitoa ja luovaa ongelmanratkaisukykyä, mistä on hyötyä niin opinnoissa kuin elämässä yleensäkin. Sitä hyödynnetään jollakin tavalla jokaisella tieteenalalla ja myös taiteessa. 

Jokaiseen tarkastelukohteeseen liitetään myös niille ominaisia \textbf{operaatioita}. Pitkän matematiikan ensimmäinen kurssi MAA1 Funktiot ja yhtälöt käsittelee lähinnä lukuja ja niiden operaatioita eli laskutoimituksia. Kirjassa esittelemme luvun käsitteen ja yleisimmin käytetyt lukualueet laskutoimituksineen, ja jatkamme niistä \textbf{yhtälöihin} ja funktioihin. Kurssilla luodaan tietopohja nimensä mukaisiin aiheisiin ja tutustutetaan opiskelija lukion matematiikkaan.

\newpage
\section*{Matematikan opetussuunnitelma}

Lukion pitkän matematiikan kurssien tavoitteena on, että opiskelija ymmärtää matemaattisen ajattelun perusteet ja oppii ilmaisemaan itseään matematiikan kielellä sekä mallintamaan itse käytännön asioita matemaattisesti. Matematiikan pitkä oppimäärä antaa hyvät valmiudet luonnontieteiden opiskeluun.

Uusimpien (vuoden 2003) lukion opetussuunnitelman perusteiden mukaan pitkän matematiikan ensimmäisen kurssin tavoitteena on, että opiskelija
\luettelo{
§ vahvistaa yhtälön ratkaisemisen ja prosenttilaskennan taitojaan
§ syventää verrannollisuuden, neliöjuuren ja potenssin käsitteiden ymmärtämistään
§ tottuu käyttämään neliöjuuren ja potenssin laskusääntöjä
§ syventää funktiokäsitteen ymmärtämistään tutkimalla potenssi- ja eksponenttifunktioita
§ oppii ratkaisemaan potenssiyhtälöitä.
}

Opetussuunnitelman perusteet määrittelevät kurssin keskeisiksi sisällöiksi
\luettelo{
§ potenssifunktion
§ potenssiyhtälön ratkaisemisen
§ juuret ja murtopotenssin
§ eksponenttifunktion.
}

Avoimet oppimateriaalit ry tuottaa ja julkaisee oppimateriaaleja ja kirjoja, jotka ovat kaikille vapaita käyttää. Vapaa matikka -sarja on suunnattu lukion pitkän matematiikan kursseille ja täyttää valtakunnallisen opetussuunnitelman vaatimukset.

\newpage

\section*{Kirjan käyttämisestä ja matematiikan opiskelusta}

Kirjan kirjoituksessa on nähty erityistä vaivaa siinä, että \textit{mitää ei oleteta osattavan etukäteen}. Lähtötaso on lähes nolla. Jos kirja ei anna tarpeeksi esimerkkejä jostain ''pikkujutusta'', ota yhteyttä, ja asia korjataan välittömästi. Syy ei ole sinun, vaan oppimateriaalien. :)

Kirjan tehtävät on suurimmassa osaa luvuista jaettu kolmeen kategoriaan.

Tähdellä $\star$ merkityt tehtävät ovat selvästi opetussuunnitelman ulkopuolisia tehtäviä, erittäin soveltavia, tekijöiden mielestä erityisen kiinnostavia, oppiainerajat ylittäviä tehtäviä, pitkiä tai koettu suhteellisen haastaviksi. Näiden tehtävien osaaminen ei missään nimessä ole vaatimus kympin kurssiarvosanaan.

Alkusanoihin: kirjassa on kahdenlaisia välivaiheita: niitä, joita vain käytetään havainnollistamaan, miten edetään, esim. x=1 kertaa x, niitä jotka yleensä katsotaan välttämättömiksi pisteytystä varten \ldots.
\newpage