\begin{tehtavasivu}

\subsubsection*{Opi perusteet}

\begin{tehtava}
Mitä yhtälölle $ax+b = 0$ tapahtuu, jos kerroin $a$ saa arvon nolla? Onko yhtälöllä ratkaisuja?
	\begin{vastaus}
Jos $b = 0$, yhtälö toteutuu kaikilla $x$:n arvoilla. Jos $b \neq 0$, yhtälö ei toteudu millään $x$:n arvolla. Periaatteessa kyseessä ei kuitenkaan enää ole ensimmäisen asteen yhtälö, mikäli $a = 0$.
	\end{vastaus}
\end{tehtava}

\begin{tehtava}
Ratkaise yhtälöistä tuntematon $x$.

\alakohdatm{
§ $x + 4 = 5$
§ $1 - x = -3$
§ $7x = 35$
§ $-2x = 4$
§ $10 - 2x = x$
§ $9x + 4 = 6 - x$
§ $\frac{2x}{5} = 4$
§ $\frac{x}{3} + 1 = \frac{5}{6} - x$
}
	\begin{vastaus}
\alakohdatm{
§ $x=1$
§ $x=4$
§ $x=35/7$
§ $x=-2$
§ $x=10/3$
§ $x=1/5$
§ $x=10$
§ $x=-1/8$
}
	\end{vastaus}
\end{tehtava}

\begin{tehtava}
\alakohdatm{
§ $4x + 3 = -3x + 4$
§ $100x = 101x - 2$
§ $\frac{5}{6} x + \frac{4}{5} = x - 1$
§ $2\cdot(x+4)-x = x+4$
§ $5\cdot(x-8) + \frac{3}{2}\cdot(x-7) = 3x$
}
\begin{vastaus}
\alakohdatm{
§ $x = \frac{1}{7}$
§ $x = 2$
§ $x = \frac{54}{5}$
§ Ei ratkaisua
§ $x = \frac{101}{7}$
}
\end{vastaus}
\end{tehtava}

\begin{tehtava}
Ratkaise yhtälöt ja tarkista sijoittamalla.
\alakohdatm{
    § $-2x+5=2(5+x)$
    § $4(x-1) - 3x = 15-4x$
    § $6 - (2x+3) = 3-2x$
    § $x - 100\,000 = -0,25x$
    § $10^6 x - 3 \cdot 10^9 = 0$
    § $5(1-x) = -5x+3$
}
\begin{vastaus}
\alakohdatm{
    § $x=-\frac 5 4$
    § $x=\frac{19}{5}$
    § Yhtälö toteutuu kaikilla reaaliluvuilla.
    § $x=80\,000$
    § $x=3\,000$
    § Yhtälö ei toteudu millään reaaliluvulla.
}
\end{vastaus}
\end{tehtava}

\begin{tehtava}
Huvipuistossa yksittäinen laitelippu maksaa $7$ euroa, ja sillä pääsee yhteen laitteeseen. Rannekkeella pääsee käymään päivän aikana niin monessa laitteessa kuin haluaa, ja se maksaa $37$ euroa. Kuinka monessa laitteessa on käytävä, jotta rannekkeen ostaminen kannattaa?
	\begin{vastaus}
$6$ laitteessa
	\end{vastaus}
\end{tehtava}

\subsubsection*{Hallitse kokonaisuus}

\begin{tehtava}
Ratkaise yhtälöt ja tarkista sijoittamalla.
\alakohdatm{
    § $\frac{8x}{3} = 12$
    § $3 - \frac{x}{4} = x$
    § $-\frac{7x}{8} = x+2$
    § $x+7 = 1 - \frac{x-1}{2}$
    § $\frac{2x-1}{3} - 1 = \frac{4x}{9}$
    § $\frac{x+5}{3} = x+3-\frac{2x+1}{4}$
}
	\begin{vastaus}
\alakohdatm{
    § $x=4\frac 1 2$
    § $x=2\frac{2}{5}$
    § $x=-1\frac{1}{15}$
    § $x = -3\frac 2 3$
    § $x=6$
    § $x=-6\frac{1}{2}$
}
	\end{vastaus}
\end{tehtava}

\begin{tehtava}
Kaisa, Jonas, Aada ja Tomi menevät elokuviin. Elokuvalippu 3D-näytökseen maksaa $12$\,euroa. Viiden lipun sarjalippu maksaa $38$\,euroa.
\alakohdat{
    § Kannattaako heidän ostaa sarjalippu ja maksaa sillä elokuva vai ostaa liput suoraan tiskiltä?
    § Jos sarjalipun hinta jaetaan tasan kaikille niin, että Kaisa pitää ylijäävän lipun, paljonko Kaisa joutuu maksamaan sarjalipusta?
}
	\begin{vastaus}
\alakohdat{
    § Sarjalippu on kannattavampi. (Liput maksaisivat erikseen ostettuina $48$\,euroa.)
    § $7,60$\,euroa
}
	\end{vastaus}
\end{tehtava}

\begin{tehtava}
Ratkaise
\alakohdat{
    § $\frac{3x+6}{9-4x} = -5$
    § $\frac{9x^2-6}{16-3x} = -3x$.
}
	\begin{vastaus}
\alakohdat{
     § $x=3$
     § $x=\frac{1}{8}$
}
	\end{vastaus}
\end{tehtava}

\begin{tehtava}
Kännykkäliittymän kuukausittainen perusmaksu on $2,90$ euroa. Lisäksi jokainen puheminuutti ja tekstiviesti maksaa $6,9$ senttiä. Liaran kännykkälasku kuukauden ajalta oli $27,05$ euroa.

\alakohdat{
	§ Kuinka monta puheminuuttia tai tekstiviestiä Pekka käytti kuukauden aikana yhteensä?
	§ Liara lähetti kaksi tekstiviestiä jokaista viittä puheminuuttia kohden. Kuinka monta tekstiviestiä Pekka lähetti?
}
	\begin{vastaus}
		\alakohdat{
			§ $350$ puheminuuttia tai tekstiviestiä
			§ $100$ tekstiviestiä
		}
	\end{vastaus}
\end{tehtava}

\begin{tehtava}
Määritä luvulle $a$ sellainen arvo, että
  \alakohdat{
     § yhtälön $5x-8-3ax=4-x$ ratkaisu on $-1$
     § yhtälön $5x+a = \frac{6x}{3} + \frac{7}{2}$ ratkaisu on $2$.
  }
\begin{vastaus}
   \alakohdat{
        § $a=6$
        § $a=-\frac{5}{2}$
      }
\end{vastaus}
\end{tehtava}

\begin{tehtava} %VASTAAVA ESIMERKKI!
Kylpyhuoneessa on kolme hanaa. Hana A täyttää kylpyammeen $60$ minuutissa, hana B $30$ minuutissa ja hana C $15$ minuutissa. Kuinka kauan kylpyammeen täyttymisessä kestää, jos kaikki hanat ovat yhtäaikaa auki?
	\begin{vastaus}
$8$\,min $34$\,s
	\end{vastaus}
\end{tehtava}

\subsubsection*{Lisää tehtäviä}

\begin{tehtava}
%Laatinut Jaakko Viertiö 2013-11-10
Määritä ne neljä peräkkäistä paritonta kokonaislukua, joiden summa on $72$.
	\begin{vastaus}
	 $15, 17, 19, 21$
	\end{vastaus}
\end{tehtava}

\begin{tehtava}
Fysiikassa ja geometriassa kaavoissa esiintyy muitakin muuttujia kuin $x$. Esimerkiksi kuljettu matka on nopeus kertaa aika eli $s=vt$. ($s= \text{spatium}$ latinaksi ja englanniksi $v= \text{velocity}$ ja $t= \text{time}$) Ratkaise kysytty tuntematon yhtälöstä.

\alakohdat{
§ $F=ma$, $m=?$ (Voima on massa kerrottuna kiihtyvyydellä.)
§ $p=\frac{F}{A}$, $F=?$ (Paine on voima jaettuna alalla.)
§ $A=\pi r^2$, $r=?$ (Ympyrän pinta-ala on pii kerrottuna säteen neliöllä.)
§ $V=\frac{1}{3} \pi r^2 h$, $h=?$ (Kartion tilavuus on piin kolmasosa
kerrottuna säteen neliöllä ja kartion korkeudella.)
}
\begin{vastaus}
\alakohdat{
§ $m=\frac{F}{a}$
§ $F=p A$
§ $r=\sqrt{\frac{A}{\pi}}$
§ $h=\frac{3V}{ \pi r^2 }$
}
\end{vastaus}
\end{tehtava}

\begin{tehtava}
 Taksimatkan perusmaksu on arkisin $5,90$ euroa. Taksimatkan hinta oli $28,00$ euroa. Mikä oli taksimatkan pituus, kun
\alakohdat{
     § Petteri matkusti yksin, jolloin taksa oli $1,52$ euroa kilometriltä?
     § Arttu otti taksin kuuden ystävänsä kanssa, jolloin taksa oli $2,13$ euroa kilometriltä?
}
	\begin{vastaus}
		\alakohdat{
			§ $14,5$ kilometriä
			§ $10,4$ kilometriä
		}
	\end{vastaus}
\end{tehtava}

\begin{tehtava}
Sadevesikeräin näyttää vesipatsaan korkeuden millimetreinä. Eräänä aamuna keräimessä oli $5$\,mm vettä. Seuraavana aamuna samaan aikaan keräimessä oli $23$\,mm vettä. Muodosta yhtälö ja selvitä, kuinka paljon vettä oli keskimäärin satanut kuluneen vuorokauden aikana tunnissa.
	\begin{vastaus}
	$0,75$\,mm/h
	\end{vastaus}
\end{tehtava}

\begin{tehtava}
London and Manchester are $200$ miles apart. Jem travels from London to Manchester at an average speed of $70$\,mph. Robbie travels in the opposite direction, from Manchester to London, at an average speed of $30$\,mph. Knowing that they leave at the same time, in how much time will their paths cross each other?
	\begin{vastaus}
		$2$ hours
	\end{vastaus}
\end{tehtava}

%\begin{tehtava}
%Kirjoita tilanne yhtälöparina. (Ei tarvitse ratkaista -- tämä on mallinnusharjoitus)
%  \alakohdat{
%§ Lukujen $k$ ja $l$ summa on $10$ ja tulo $20$.
%§ Suorakulmion muotoisen aitauksen, jonka leveys on $x$ ja pituus $y$, pinta-ala $30$ neliömetriä. Aitaa on käytettävissä yhteensä $150$ metriä.
%§ Nuorisojärjestö teki ryhmätilauksen demopartyille: $35$ euron sisäänpääsylippuja ja $70$ euron konepaikkalippuja (sis. sisäänpääsyn) ostettiin yhteensä $15$ kappaletta, ja koko tilaus maksoi yhteensä $875$ euroa.
%  }
%  \begin{vastaus}
%    \alakohdat{
%    § $$\left\{    
%    \begin{array}{rcl}
%        k+l&=&10 \\
%        kl&=&20 \\
%    \end{array}
%    \right. $$
%    § $$\left\{    
%    \begin{array}{rcl}
%        xy&=&30\,\text{m}^2 \\
%        2x+2y&=&150 \\
%    \end{array}
%    \right. $$
%   § $$\left\{    
%    \begin{array}{rcl}
%		x+y&=&15 \\
%		35x+70y&=&875 \\
%    \end{array}
%    \right. ,$$ missä $x$ on sisäänpääsylipullisten määrä ja $y$ konepaikallisten määrä
%    }
%  \end{vastaus}
%\end{tehtava}
%
%\begin{tehtava}
%Pekka on melomassa joella. Hän kulkee ensiksi $3,7$\,km matkan vastavirtaan, ja palaa sitten samaa reittiä takaisin lähtöpisteeseen. Pekalla kuluu menomatkaan aikaa $42$ minuuttia, ja paluumatkaan $27$ minuuttia. Pekka meloo tasaisella nopeudella koko matkan. Laske joen virtausnopeus ja Pekan melomisnopeus. Ilmoita nopeudet yksikössä km/h kahden desimaalin tarkkuudella.
%		\begin{vastaus}
%			Joen virtausnopeus on noin $1,47$\,km/h ja Pekan melomisnopeus noin $6,75\,\frac{km}{h}$.
%		\end{vastaus}
%\end{tehtava}

%\begin{tehtava}
%Millä vakion $a$ arvoilla yhtälöparilla 
%$$\left\{    
%    \begin{array}{rcl}
%        ax+1&=&y \\
%        x&=&-y \\
%    \end{array}
%    \right. $$
%    ei ole ratkaisua?
%	\begin{vastaus}
%Ratkaisua ei ole, kun $a=-1$. Tämä havaitaan, kun ratkaistaan yhtälöparista joko $x$ tai $y$ -- tapaus $a=-1$ johtaisi nollalla jakamiseen.
%	\end{vastaus}
%\end{tehtava}

\end{tehtavasivu}