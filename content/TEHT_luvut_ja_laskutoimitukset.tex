\begin{tehtavasivu}

\subsubsection*{Opi perusteet}

\begin{tehtava}
   	Kirjoita laskut näkyviin: Lukusuoralla on saavuttu pisteeseen $-2$. Mihin pisteeseen päädytään, kun liikutaan
    \begin{alakohdat}
        \alakohta{3 askelta negatiiviseen suuntaan?}
        \alakohta{6 askelta positiiviseen ja sitten 2 negatiiviisen suuntaan?}
        \alakohta{2 askelta positiiviseen ja sitten 6 negatiiviseen suuntaan?}
    \end{alakohdat}

    \begin{vastaus}
    	Pisteeseen 
        \begin{alakohdat}
            \alakohta{$-2+(-3)=-5$}
            \alakohta{$-2+6+(-2)=2$}
            \alakohta{$-2+2+(-6)=-6$}
        \end{alakohdat}
    \end{vastaus}
\end{tehtava}


\begin{tehtava}
    Kirjoita laskutoimitukseksi (laskuun ei tarvitse merkitä yksikköjä eli celsiusasteita %tai euroja.)
    )

    \begin{alakohdat}
        \alakohta{Pakkasta on aluksi $-10~^{\circ}$C, ja sitten se lisääntyy kahdella pakkasasteella.}
        \alakohta{Pakkasta on aluksi $-20~^{\circ}$C, ja sitten se hellittää (vähentyy) kolme (pakkas)astetta.}
        \alakohta{Lämpötila on aluksi $17~^{\circ}$C, ja sitten se vähentyy viisi astetta.}
        \alakohta{Lämpötila on aluksi $5~^{\circ}$C, ja sitten se kasvaa kuusi astetta.}
%        \alakohta{Mies on mafialle $30~000$ euroa velkaa ja menehtyy. Hänen kolme 
%            poikaansa jakavat velan tasan keskenään. Kuinka paljon kukin on
%            velkaa mafialle? Merkitse velkaa negatiivisella luvulla.}
    \end{alakohdat}

    \begin{vastaus}
        \begin{alakohdat}
            \alakohta{$-10+(-2)=-12$}
            \alakohta{$-20-(-3)=-17$}
            \alakohta{$17-5=12$}
            \alakohta{$5+6=11$}
           % \alakohta{$\dfrac{-30~000}{3}=10~000$}
        \end{alakohdat}
    \end{vastaus}
\end{tehtava}

\begin{tehtava}
Kirjoita erotukset summana
	\begin{alakohdat}
		\alakohta{$130-(-5)$}
		\alakohta{$-4-7$}
		\alakohta{$2a-b$, kun $a$ ja $b$ ovat kokonaislukuja.}
	\end{alakohdat}
\begin{vastaus}
	\begin{alakohdat}
		\alakohta{$130+5$}
		\alakohta{$-4+(-7)$}
		\alakohta{$2a+(-b)$}
	\end{alakohdat}
\end{vastaus}
\end{tehtava}

\begin{tehtava}
Kirjoita summat erotuksena
	\begin{alakohdat}
		\alakohta{$31+4$}
		\alakohta{$-15+(-92)$}
		\alakohta{$-a+2b$, kun $a$ ja $b$ ovat kokonaislukuja.}
	\end{alakohdat}
\begin{vastaus}
	\begin{alakohdat}
		\alakohta{$31-(-4)$}
		\alakohta{$-15-92$}
		\alakohta{$-a-(-2b)$}
	\end{alakohdat}
\end{vastaus}
\end{tehtava}


\begin{tehtava}
Laske
    \begin{alakohdat}
        \alakohta{$3+(-8)$}
        \alakohta{$5+(+7)$}
        \alakohta{$-8-(-5)$}
        \alakohta{$+(+8)-(+5)$}
        \alakohta{$-(-8)-(+8)$.}
    \end{alakohdat}
\begin{vastaus}
    \begin{alakohdat}
        \alakohta{$-5$}
        \alakohta{$12$}
        \alakohta{$-3$}
        \alakohta{$3$}
        \alakohta{$0$}
    \end{alakohdat}
\end{vastaus}
\end{tehtava}

\begin{tehtava}
Laske
    \begin{alakohdat}
        \alakohta{$3\cdot (-6)$}
        \alakohta{$-7\cdot (+7)$}
        \alakohta{$-8\cdot (-5)$}
        \alakohta{$-(-8)\cdot (-8)$}
        \alakohta{$-(-8)\cdot (-(+5))$}
        \alakohta{$-(-8)\cdot (-5)\cdot (-1)\cdot (-2)$.}
    \end{alakohdat}
\begin{vastaus}
    \begin{alakohdat}
        \alakohta{$-36$}
        \alakohta{$-49$}
        \alakohta{$40$}
        \alakohta{$-64$}
        \alakohta{$-40$}
        \alakohta{$-80$}
    \end{alakohdat}
\end{vastaus}
\end{tehtava}

\begin{tehtava}
% Laatinut Sampo Tiensuu 2013-12-14 %liian isoja hyppyjä, eli lisää tehtäviä väliin! T: JoonasD6
Laske laskujärjestyssääntöjä noudattaen
	\begin{alakohdat}
	    \alakohta{$2+3\cdot(-1)$}
	    \alakohta{$(5-(2-3))\cdot 2$}
	    \alakohta{$1+9:3:3-(3-5):2$.}
	\end{alakohdat}
\begin{vastaus}
	\begin{alakohdat}
	    \alakohta{$-1$}
	    \alakohta{$12$}
	    \alakohta{$3$}
	\end{alakohdat}
\end{vastaus}
\end{tehtava}


\begin{tehtava}
	Mitkä seuraavista luvuista ovat jaollisia luvulla $4$?
	Jos luku $a$ on jaollinen luvulla $4$, kerro, millä kokonaisluvulla $b$ pätee $a = 4 \cdot b$. \\
	\begin{alakohdatrivi}
		\alakohta{$1$}
		\alakohta{$12$}
		\alakohta{$13$}
		\alakohta{$2$}
		\alakohta{$-20$}
		\alakohta{$0$}
	\end{alakohdatrivi}
	\begin{vastaus}
		\begin{alakohdat}
			\alakohta{Ei ole jaollinen luvulla $4$}
			\alakohta{On jaollinen luvulla $4$, $12 = 4 \cdot 3$}
			\alakohta{Ei ole jaollinen luvulla $4$}
			\alakohta{Ei ole jaollinen luvulla $4$}
			\alakohta{On jaollinen luvulla $4$, $-20 = 4 \cdot (-5)$}
			\alakohta{On jaollinen luvulla $4$, $0 = 4 \cdot 0$} 
		\end{alakohdat}
    \end{vastaus}
\end{tehtava}

\begin{tehtava}
    Jaa alkutekijöihin \\
	\begin{alakohdatrivi}
		\alakohta{$12$}
		\alakohta{$15$}
		\alakohta{$28$}
		\alakohta{$30$}
		\alakohta{$64$}
		\alakohta{$90$}
		\alakohta{$100$.}
	\end{alakohdatrivi}
    \begin{vastaus}
		\begin{alakohdat}
			\alakohta{$12 = 2^2 \cdot 3$}
			\alakohta{$15 = 3 \cdot 5$}
			\alakohta{$28 = 2^2 \cdot 7$}
			\alakohta{$30 = 2 \cdot 3 \cdot 5$}
			\alakohta{$64 = 2^6$}
			\alakohta{$90 = 2 \cdot 3^2 \cdot 5$}
			\alakohta{$100 = 2^2 \cdot 5^2$}
		\end{alakohdat}
    \end{vastaus}
\end{tehtava}

\begin{tehtava}
% Laatinut Sampo Tiensuu 2013-12-14
$\star$ Laske laskujärjestyssääntöjä noudattaen
	\begin{alakohdat}
	    \alakohta{$\left(\dfrac{(5+3\cdot 2)\cdot 6+2}{-(2-12:2)}+(-(-2)-1)\right):2$}
	    \alakohta{$-2+5\cdot\left[3+(2-3)\cdot 2-\dfrac{3-102}{121:11}\right]$}
	    \alakohta{$\{3\cdot 88:4:2+13:3\cdot \left[21:(14:2)\right]\}:9+1$.}
	\end{alakohdat}
\begin{vastaus}
	\begin{alakohdat}
	    \alakohta{$9$}
	    \alakohta{$48$}
	    \alakohta{$6$}
	\end{alakohdat}
\end{vastaus}
\end{tehtava}

\end{tehtavasivu}
