Rationaaliluvun esitystä kokonaislukujen osamääränä
$\frac{a}{b}$ kutsutaan \termi{murtoluku}{murtoluvuksi}. Luku $a$ on murtoluvun
\termi{osoittaja}{osoittaja} ja luku $b$ on
\termi{nimittäjä}{nimittäjä}. Nimittäjä on aina erisuuri kuin nolla ($b\neq0 $). Määritelmän mukaan kaikki rationaaliluvut
voidaan esittää murtolukuina. Usein rationaaliluvut voidaan esittää myös desimaaliesityksen avulla.


Rationaalilukuja esitetään toisinaan myös \termi{sekamurtoluku}{sekamurtolukuina} eli lyhyemmin
sekalukuina. Sekalukuesityksessä luku esitetään summana kokonaisosasta ja murto-osasta (yleensä luku nollan
ja yhden väliltä), mutta yhteenlaskumerkki jätetään merkitsemättä. Esimerkiksi sekaluku $3\frac{1}{4}$
tarkoittaa lukua $3 + \frac{1}{4}$, jonka voi esittää murtolukuna $\frac{13}{4}$. Jos luku on negatiivinen,
miinusmerkki merkitään koko sekaluvun eteen, siis $-6\frac{4}{5}$ tarkoittaa lukua $-(6 + \frac{4}{5})$.
Merkintä saattaa aiheuttaa sekaannusta, sillä $3\frac{5}{6}$ saattaisi myös viitata tuloon
$3\cdot \frac{5}{6}$. Sekalukumerkintää käytetään vain tunnetuilla lukuarvoilla, siis $a\frac{b}{c}$ tarkoittaa aina
tuloa $a\cdot \frac{b}{c}$. 

\begin{esimerkki}
        Rationaaliluvulla $\frac{5}{4}$ on kolme erilaista esitystapaa: murtolukuesitys, sekalukuesitys ja desimaaliesitys.
        \[\frac{5}{4} = 1\frac{1}{4}=1,25 \]
    \end{esimerkki}

Murtolukujen laskutoimituksia varten sekaluvut kannattaa usein muuttaa murtolukuesitykseksi. 
\newpage

\laatikko[Laventaminen ja supistaminen]{
	Murtoluvun arvo pysyy samana, kun sekä osoittajassa että nimittäjässä oleva luku kerrotaan (\termi{laventaminen}{lavennetaan})
	samalla luvulla. Vastaavasti, osoittaja ja nimittäjä voidaan jakaa (\termi{supistaminen}{supistaa}) samalla luvulla, 
	jos jako menee tasan.
	\[ \text{laventaminen}\]
	\[ \longrightarrow\]
% \FIXME Nuoliviritys toimisi, mutta rivivälit kaipaavat muokkausta (laventaminen ja nuoli lähemmäs toisiaan). 
	 \[
	 \frac{a}{b} = \frac{ca}{cb}
	 \]
	 
 	\[ \longleftarrow\]
 	\[ \text{supistaminen}\]
 	
	missä $b \neq 0$ ja $c \neq 0$.
}

Laventamisen mahdollistaa se, että jokainen luku $c(\neq0)$ voidaan esittää muodossa $\frac{c}{c}=1$. Kun mielivaltaista lukua kerrotaan luvulla $1$, luvun arvo ei muutu. Huomaa, että murtoluvun laventaminen ei tarkoita samaa kuin murtoluvun kertominen.

Koska murtolukuesitystä voidaan laventaa millä tahansa (nollasta eriävällä) kokonaisluvulla, jokaisella murtoluvulla on monta esitystapaa; esimerkiksi $\frac{1}{2}=\frac{3}{6}$. Konkreettisemmin voidaan ajatella, että puolikas pizza ei vähene vaikka molemmat puolikkaat jaetaan edelleen kolmeen osaan.

\subsection*{Murtolukujen yhteen- ja vähennyslasku}

\laatikko{
    Jos murtolukujen nimittäjät ovat samat, 
    voidaan murtoluvut laskea yhteen laskemalla osoittajat yhteen.
    \[
    \frac{a}{c} + \frac{b}{c} = \frac{a+b}{c}
    \]
    missä $c \neq 0$
}

Murtolukuja, joiden nimittäjät ovat samat, sanotaan \termi{samanniminen}{samannimisiksi}. Jos yhteenlaskettavilla murtoluvuilla on eri nimittäjät, murtoluvut lavennetaan ensin samannimisiksi ja sitten osoittajat lasketaan yhteen. Jos siis $\frac{a}{b}$ ja $\frac{c}{d}$ ovat murtolukuja (missä $b \neq 0$ ja $d \neq 0$), lasketaan

\laatikko{
    \[
    \frac{a}{b} + \frac{c}{d} = \frac{ad}{bd} + \frac{bc}{bd} = \frac{ad+bc}{bd}
    \]
    Tässä $\frac{a}{b}$ lavennetaan luvulla $d$ ja $\frac{c}{d}$ lavennetaan
    luvulla $b$. Nyt saadaan kaksi samannimistä murtolukua, joiden kummankin
    nimittäjäksi tulee yhteenlaskettavien nimittäjien tulo $bd$.
 }    
 
Yllä esitetty menettely, missä murtoluvut on lavennettu toistensa nimittäjillä. toimii aina, mutta toisinaan pelkästään toisen murtoluvun laventaminen riittää, kuten seuraavassa esimerkissä. 

\begin{esimerkki}
        Laske
        \[
        \frac{1}{2} + \frac{1}{6} + \frac{2}{6}.
        \]
        
        \textbf{Ratkaisu.}
        Lavennetaan nimittäjät samannimisiksi ja lasketaan osoittajat yhteen.
        %lisätäänkö lavennusmerkki? teknisesti hankala? käytetäänkö maailmalla? opiskelijoille kuitenkin tuttu
        %tai sitten voisi sanallisesti kertoa, millä lavennetaan.
        \begin{align*}
            \frac{1}{2} + \frac{1}{6} + \frac{2}{6} &=\frac{3\cdot 1}{3\cdot 2} + \frac{1}{6} + \frac{2}{6}\\
            										&=\frac{3}{6} + \frac{1}{6} + \frac{2}{6}\\
           											&= \frac{3+1+2}{6}\\
           											&= \frac{6}{6} = 1.
        \end{align*}
    \end{esimerkki}
    
Murtolukujen vähennyslasku toimii periaatteessa samalla tavalla kuin yhteenlaskukin (vähennyslasku voidaan ajatella vastaluvun lisäämisenä). Ensin lavennetaan samannimisiksi, sitten suoritetaan vähennyslasku osoittajassa.

\begin{esimerkki}
        Laske
        \[
        2\frac{2}{3} - \frac{3}{7}.
        \]
        
        \textbf{Ratkaisu.}
        Aloitetaan muuttamalla sekalukuesitys murtoluvuksi. Nimittäjillä ei ole yhteisiä tekijöitä, joten lavennetaan ne toistensa nimittäjillä. %selitä, miksi tuo toimii: koska kertolaskun vaihdannaisuus
        %lisätäänkö lavennusmerkki? teknisesti hankala? käytetäänkö maailmalla? opiskelijoille kuitenkin tuttu
        %tai sitten voisi sanallisesti kertoa, millä lavennetaan.
        \begin{align*}
            2\frac{2}{3} - \frac{3}{7} = \frac{2\cdot3+2}{3} - \frac{3}{7} \\
            			&=\frac{8}{3} - \frac{3}{7}\\
            			&=\frac{7\cdot8}{7\cdot3} - \frac{3\cdot3}{3\cdot7}\\
            			&=\frac{56}{21} - \frac{9}{21}\\
            			&=\frac{56-9}{21}\\
            			&=\frac{45}{21}.\\
        \end{align*}
        Erotuksena syntyvän murtoluvun osoittaja ja nimittäjä ovat molemmat jaollisia kolmella, eli murtolukua voidaan supistaa. 
        Esitetään vastaus yksinkertaisimmassa (sievimmässä) muodossa.
        \begin{align*}
            \frac{45}{21} = \frac{45:3}{21:3}\\
		  &=\frac{15}{7}\\
		  &=2\frac{1}{7}.\\
        \end{align*}
    \end{esimerkki}

Mikä tahansa luku jaettuna itsellään on yksi.
    
\subsection*{Murtolukujen kertolasku}
    
\begin{esimerkki}
	Laske
	\[
        \frac{3}{4}\cdot \frac{6}{5}.
        \]
	
        \textbf{Ratkaisu.}
        Murtolukujen kertolaskussa tekijöitä ei tarvitse laventaa samannimisiksi. 
        Osoittajat kerrotaan keskenään ja nimittäjät kerrotaan keskenään. 
      \[
        \frac{3}{4}\cdot \frac{6}{5}= \frac{3\cdot 6}{4\cdot 5}= \frac{18}{20}=\frac{9}{10}
        \]
    \end{esimerkki}
    
\laatikko{
    Murtolukujen $\frac{a}{b}$ ja $\frac{c}{d}$ ($b \neq 0$ ja $d \neq 0$) tulo lasketaan kertomalla 
    lukujen osoittajat ja nimittäjät keskenään:
    % Onko tämä toistoa? sama sanottiin jo yllä olevassa esimerkissä.
    \[
    \frac{a}{b}\cdot \frac{c}{d} = \frac{a\cdot c}{b\cdot d} = \frac{ac}{bd}
    \]
}

%\missingfigure{tähän Sampon paperille suunnittelema havainnollistus kertolaskusäännöstä}
\includegraphics[scale=0.4]{pictures/Kuva3-1-1.pdf}
\includegraphics[scale=0.4]{pictures/Kuva3-1-2.pdf}
\includegraphics[scale=0.4]{pictures/Kuva3-1-3.pdf}
\includegraphics[scale=0.4]{pictures/Kuva3-1-4.pdf}
\includegraphics[scale=0.4]{pictures/Kuva3-1-5.pdf}

\subsection*{Murtolukujen jakolasku}

\laatikko{
    Luvun $a$ \termi{käänteisluku}{käänteisluku} on sellainen luku $b$, joka kerrottuna luvulla a on yksi, $ab=1$.
    
    Rationaaliluvun $a$ käänteisluku on $\frac{1}{a}$ ($a\neq 0$), sillä
    \[
    a\cdot \frac{1}{a} = 1.
    \]
    Vastaavasti rationaaliluvun $\frac{a}{b}$ käänteisluku on $\frac{b}{a}$ ($a\neq 0$ ja $b\neq 0$), sillä
    \[
    \frac{a}{b}\cdot \frac{b}{a} = 1.
    \]    

  %  Murtolukujen $p=\frac{a}{b}$ ja $q=\frac{c}{d}\neq 0$ \termi{osamäärä}{osamäärä} $p : q$ saadaan, kun kerrotaan luku $p$ luvun $q$ käänteisluvulla,
 %   \[
 %\frac{p}{q} = p\cdot q^{-1} = \frac{a}{b}\cdot\Big(\frac{c}{d}\Big)^{-1} = \frac{a}{b}\cdot \frac{d}{c}
 %   = \frac{ad}{bc}.
 %  \]
 }

 \begin{esimerkki}
	Luvun $5$ käänteisluku on $\frac{1}{5}$, koska
	\[
	 5\cdot \frac{1}{5}=1.
	\]
	Vastaavasti luvun $-\frac{2}{3}$ käänteisluku on $-\frac{3}{2}$, koska
	\[
	 -\frac{2}{3}\cdot (-\frac{3}{2})=1.
	\]
  \end{esimerkki}
  
Käänteislukua tarvitaan muun muassa murtolukujen jakolaskuissa.
 
\begin{esimerkki}
Murtolukujen jakolasku. Laske $\frac 3 5 : \frac 2 7$.

\textbf{Ratkaisu.}
Jakolaskun määritelmän mukaan osamäärän tulisi olla sellainen luku, joka kerrottuna jakajalla antaa tulokseksi jaettavan. 
Jos merkitään laskun $\frac 3 5 : \frac 2 7$ vastausta kirjaimella $x$, pitää siis olla $x \cdot \frac 2 7 = \frac 3 5$.  
Tätä yhtälöä kutsutaan jakolaskun $\frac 3 5 : \frac 2 7$ \termi{jakoyhtälö}{jakoyhtälöksi.}
% Jakoyhtälö-termin käyttö on tässä ongelmallista, koska yhtälöt käsitellään kirjassa vasta myöhemmin.

Kerrotaan jakoyhtälön molemmat puolet luvun $\frac 2 7$ käänteisluvulla $\frac 7 2$. Koska käänteislukujen tulo on $1$, saadaan
\[
	\text{vasen puoli} = x \cdot \underbrace{\frac 2 7 \cdot \frac 7 2}_{= 1} = x \quad \text{ja} \quad \text{oikea puoli} = \frac 3 5 \cdot \frac 7 2.
\]

% Voisi olla selkeyden vuoksi hyvä, jos käytettäisiin tässä esim. värikoodia konkretisoimaan sitä, mistä tulee x (vasen puoli) ja mistä tulee murtolukujen tulo (oikea puoli).
Kun yhtälön molemmat puolet kerrotaan samalla luvulla, ovat myös näin saadut luvut yhtä suuria. Siis on saatu
\[
	x = \frac 3 5 \cdot \frac 7 2.
\]
Koska $x$:llä merkittiin alkuperäistä jakolaskua, on nyt onnistuttu muuttamaan jakolasku kertolaskuksi:
\[
	\frac 3 5 : \frac 2 7 = \frac 3 5 \cdot \frac 7 2 = \frac{3 \cdot 7}{5 \cdot 2} = \frac{21}{10} = 2 \frac{1}{10}.
\]
 \end{esimerkki}
 
\laatikko{
Olkoon $b \neq 0$, $c \neq 0$ ja $d \neq 0$. Murtolukujen osamäärä $\frac a b : \frac c d$ lasketaan kertomalla jaettava jakajan käänteisluvulla:
\[
	\frac a b : \frac c d = \frac a b \cdot \frac d c = \frac{ad}{bc}.
\]
}

Samannimisten murtolukujen vertailu on helppoa: jos $a>0$, niin $\frac{b}{a} < \frac{c}{a}$ täsmälleen silloin kun $b < c$. Murtolukujen vertailua varten luvut siis lavennetaan samannimisiksi.

\laatikko{
Kahta murtolukua voidaan vertailla laventamalla ne samannimisiksi siten, että yhteinen nimittäjä on positiivinen, ja sitten vertaamalla osoittajia.
}
    
% Pitäisikö antaa esimerkki myös tilanteesta, jossa toinen nimittäjistä on negatiivinen? Ilman sitä positiivisuuden korostaminen tuntuu 
% hieman turhalta (kun on kuitenkin sanottu että vertailtavat murtoluvut ovat samannimisiä.
    
    \begin{esimerkki}
        Mozzarellapizza jaetaan kuuteen ja salamipizza neljään yhtä suureen
        siivuun. Vesa saa kaksi siivua mozzarellapizzaa ja yhden siivun salamipizzaa.
        Minttu saa kaksi siivua salamipizzaa. Kumpi saa enemmän pizzaa, jos
        molemmat pizzat ovat saman kokoisia?
        
        \begin{center}        
          \includegraphics[scale=1.0]{pictures/Kuva3-1-6-pizzat.pdf}
        \end{center}

        \textbf{Ratkaisu.}
        
        Pizzan kokonaismäärän vertailua varten luvut on lavennettava samannimisiksi.
        Yhteiseksi nimittäjäksi tarvitaan luku, joka on jaollinen sekä kuudella että neljällä.
        Huomataan, että $12 = 3\cdot 4 = 2\cdot 6$. Murtolukujen nimittäjään tarvitaan siis luku $12$.
        
        Vesan saama määrä pizzaa on
        \begin{align*}
           \frac{2}{6} + \frac{1}{4} &= \frac{2\cdot 2}{2\cdot 6} + \frac{3\cdot 1}{3\cdot 4} \\ 
	       							 &= \frac{4}{12}+\frac{3}{12} \\ 
	       							 &= \frac{7}{12}.
        \end{align*}
        
        Mintun saama määrä pizzaa on
        \[
            \frac{2}{4} =
            \frac{3\cdot 2}{3\cdot 4} =
            \frac{6}{12}.
        \]
        Koska $6/12 < 7/12$, Vesa saa enemmän.
    \end{esimerkki}
    
    Kaikki rationaaliluvut voidaan esittää murtolukumuodossa, mutta myös
    kokonaisluvut voidaan esittää murtolukuina asettamalla murtoluvun
    nimittäjäksi yksi. Tätä voidaan käyttää, kun lasketaan yhteen
    kokonaislukuja ja murtolukuja.
    
    \begin{esimerkki}
        Laske
        \[
            2 + \frac{1}{3}.
        \]
        
        \textbf{Ratkaisu.}
        
%        Kirjoitetaan aluksi
%        \[
%            2=\frac{2}{1}.
%        \]
		Kirjoitetaan lausekkeen kokonaisluku $2$ murtolukuna, minkä
		jälkeen murtoluvut voidaan laventaa samannimisiksi
		ja laskea yhteen:
        \begin{align*}
           2 + \frac{1}{3} &= \frac{2}{1} + \frac{1}{3}  \\ 
	       				   &= \frac{3 \cdot 2}{3 \cdot 1} + \frac{1}{3} \\ 
	       				   &= \frac{6+1}{3} \\ 
	       				   &= \frac{7}{3}.
        \end{align*}
    \end{esimerkki}

    \subsection*{Murtolausekkeiden sieventäminen}

\laatikko{
Jos murtoluvun osoittajassa tai nimittäjässä on summa, jonka osilla on yhteinen tekijä, sen voi ottaa \emph{yhteiseksi tekijäksi} sulkujen eteen. Jos osoittajassa ja nimittäjässä on sen jälkeen sama kerroin, sen voi jakaa pois molemmista eli \emph{supistaa} pois.
\begin{equation}
\frac{ac+bc}{c} = \frac{ \cancel{c} (a+b)}{\cancel{c}} = a+b
\end{equation}


Joskus murtolauseke sieventyy, jos sen esittääkin kahden murtoluvun summana.
\begin{equation}
\frac{ca+b}{c} = \frac{ca}{c} + \frac{b}{c} = a + \frac{b}{c}
\end{equation}
}

Kun jakaa kolme erikokoista nallekarkkipussia ($a$, $b$ ja $c$) tasan kolmen ihmisen kesken, on sama, laittaako kaikki ensin samaan kulhoon ja jakaa ne sitten ($\frac{a+b+c}{3}$) vai jakaako jokaisen pussin erikseen ($ \frac{a}{3} + \frac{b}{3} + \frac{c}{3}$).

Jos taas samat kolme henkilöä jakavat keskenään pussin tikkareita ($6$ kpl) ja yhden pussin nallekarkkeja ($n$ kpl), niin saadaan seuraavanlainen lasku: $ \frac{6\text{ tikkaria}+n\text{ nallekarkkia}}{3} = \frac{6\text{ tikkaria}}{3} + \frac{n\text{ nallekarkkia}}{3} = \frac{\cancel{3} \cdot 2\text{ tikkaria}}{\cancel{3}} + \frac{n\text{ nallekarkkia}}{3} = 2\text{ tikkaria} + \frac{n\text{ nallekarkkia}}{3}$. Toisin sanoen, kukin saa kaksi tikkaria ja kuinka paljon ikinä onkaan kolmasosa kaikista nallekarkeista.

\laatikko{
Samantyyppiset asiat voidaan laskea yhteen tai \emph{ryhmitellä}.
\begin{equation}
ax^2 + bx + cx^2 + dy + ex = (a+c)x^2 + (b+e)x + dy
\end{equation}
}

%Tämä alla oleva esimerkki rationaalukujen laskutoimitusosioon kirjan alkupäähän ja niin, että se RIVITETÄÄN ja kerrotaan VAIHEITTAIN, mitä tehtiin! :) T: Joonas

\begin{esimerkki}

$ \frac{1}{6} + \frac{3}{2} = \frac{1}{2\cdot 3} + \frac{3}{2} = \frac{1}{2 \cdot 3} + \frac{3 \cdot 3}{2 \cdot 3} = \frac{1}{6} + \frac{9}{6} = \frac{10}{6} = \frac{\cancel{2} \cdot 5}{\cancel{2} \cdot 3} = \frac{5}{3}$

\end{esimerkki}
