%Tässä uusi, lisättä, EI POISTETA


\section{Esipuhe}

Matematiikka on kaikkien luonnontieteiden perusta. Se tarjoaa työkaluja asioiden täsmälliseen jäsentämiseen, päättelyyn ja mallintamiseen. Alasta riippuen käsittelemme matematiikassa erilaisia \textbf{objekteja}: Geometriassa tarkastelemme kaksiulotteisia \textbf{tasokuvioita} ja kolmiulotteisia \textbf{avaruuskappaleita}. Algebra tutkii \textbf{lukualueita} ja niissä määriteltäviä \textbf{laskutoimituksia}. Todennäköisyyslaskenta tarkastelee satunnaisten \textbf{tapahtumien} esiintymistä. Matemaattinen analyysi tutkii \textbf{funktioita} ja niiden ominaisuuksia, esimerkiksi \textbf{jatkuvuutta}, \textbf{derivoituvuutta} ja \textbf{integroituvuutta}. Matemaattista analyysiä käsittelevät pitkässä matematiikassa kurssit 7, 8, 10 ja 13 sekä osin kurssit 9 ja 12. Voidaankin sanoa, että analyysi on keskeisin aihealue lukion pitkässä matematiikassa.\footnote[1]{Tämä Suomessa käytetty lähestymistapa on käytössä monissa länsimaissa. Sen sijaan esimerkiksi Balkanilla geometrialle ja matemaattiselle todistamiselle annetaan edelleen paljon suurempi painoarvo.} Matematiikka opettaa loogista päättelytaitoa ja luovaa ongelmanratkaisukykyä, mistä on hyötyä niin opinnoissa kuin elämässä yleensäkin. Sitä hyödynnetään jollakin tavalla jokaisella tieteenalalla ja myös taiteessa. 

Jokaiseen tarkastelukohteeseen liitetään myös niille ominaisia \textbf{operaatioita}. Pitkän matematiikan ensimmäinen kurssi MAA1 Funktiot ja yhtälöt käsittelee lähinnä lukuja ja niiden operaatioita eli laskutoimituksia. Kirjassa esittelemme luvun käsitteen ja yleisimmin käytetyt lukualueet laskutoimituksineen, ja jatkamme niistä \textbf{yhtälöihin} ja funktioihin. Kurssilla luodaan tietopohja nimensä mukaisiin aiheisiin ja tutustutetaan opiskelija lukion matematiikkaan. Oppikirja on rakennettu siten, että aiheet esitellään lukujen alussa ja havainnollistetaan esimerkein. Tehtäviä on runsaasti ja niiden tarkoituksena on saada opiskelija sisäistämään opiskellut asiat ja siirtämään ne käytäntöön.

Lukion pitkän matematiikan kurssien tavoitteena on, että opiskelija ymmärtää matemaattisen ajattelun perusteet ja oppii ilmaisemaan itseään matematiikan kielellä sekä mallintamaan itse käytännön asioita matemaattisesti. Matematiikan pitkä oppimäärä antaa hyvät valmiudet luonnontieteiden opiskeluun.

Uusimpien (vuoden 2003) lukion opetussuunnitelman perusteiden mukaan pitkän matematiikan ensimmäisen kurssin tavoitteena on, että opiskelija
\begin{itemize}
\item vahvistaa yhtälön ratkaisemisen ja prosenttilaskennan taitojaan
\item syventää verrannollisuuden, neliöjuuren ja potenssin käsitteiden ymmärtämistään
\item tottuu käyttämään neliöjuuren ja potenssin laskusääntöjä
\item syventää funktiokäsitteen ymmärtämistään tutkimalla potenssi- ja eksponenttifunktioita
\item oppii ratkaisemaan potenssiyhtälöitä.
\end{itemize}

Opetussuunnitelman perusteet määrittelevät kurssin keskeisiksi sisällöiksi
\begin{itemize}
\item potenssifunktion
\item potenssiyhtälön ratkaisemisen
\item juuret ja murtopotenssin
\item eksponenttifunktion.
\end{itemize}

Avoimet oppimateriaalit ry tuottaa ja julkaisee oppimateriaaleja ja kirjoja, jotka ovat kaikille vapaita käyttää. Vapaa matikka -sarja on suunnattu lukion pitkän matematiikan kursseille ja täyttää valtakunnallisen opetussuunnitelman vaatimukset.
