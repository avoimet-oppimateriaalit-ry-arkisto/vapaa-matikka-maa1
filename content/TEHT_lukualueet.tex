\begin{tehtavasivu}

\subsubsection*{Opi perusteet}

\begin{tehtava}
   	Mitkä seuraavista väitteistä pitävät paikkansa?
    \alakohdat{
        § $-1 \notin \zz$
        § $-1 \in \nn$
        § $\frac{1}{4} \notin \zz$
        § $-\frac{2}{4} \in \qq$
        § $-\frac{412}{97} \in \rr$
%        § $7,333\ldots \in \qq$
    }
    \begin{vastaus}
    	\alakohdat{
        § epätosi
        § epätosi
        § tosi
        § tosi
        § tosi
%        § epätosi
        }
    \end{vastaus}
\end{tehtava}

%tehtävä jonkin relaation transitiivisuudesta jne.

\begin{tehtava}
   	Kirjoita laskut näkyviin: Lukusuoralla on saavuttu pisteeseen $-2$. Mihin pisteeseen päädytään, kun liikutaan
    \alakohdat{
        § 3 askelta negatiiviseen suuntaan?
        § 6 askelta positiiviseen ja sitten 2 negatiiviisen suuntaan?
        § 2 askelta positiiviseen ja sitten 6 negatiiviseen suuntaan?
    }

    \begin{vastaus}
    	Pisteeseen 
        \alakohdat{
            § $-2+(-3)=-5$
            § $-2+6+(-2)=2$
            § $-2+2+(-6)=-6$
        }
    \end{vastaus}
\end{tehtava}


\begin{tehtava}
    Kirjoita laskutoimitukseksi (laskuun ei tarvitse merkitä yksikköjä eli celsiusasteita %tai euroja.)
    )

    \alakohdat{
           § Lämpötila on aluksi $17\,\textdegree$C, ja sitten se vähentyy viisi astetta.
        § Lämpötila on aluksi $5\,\textdegree$C, ja sitten se kasvaa kuusi astetta.
        § Pakkasta on aluksi $-10\,\textdegree$C, ja sitten se lisääntyy kahdella pakkasasteella.
        § Pakkasta on aluksi $-20\,\textdegree$C, ja sitten se hellittää (vähentyy) kolme (pakkas)astetta.
 
%        § Mies on mafialle $30\,000$ euroa velkaa ja menehtyy. Hänen kolme 
%            poikaansa jakavat velan tasan keskenään. Kuinka paljon kukin on
%            velkaa mafialle? Merkitse velkaa negatiivisella luvulla.
    }

    \begin{vastaus}
        \alakohdat{
                    § $17-5=12$
            § $5+6=11$
            § $-10+(-2)=-12$
            § $-20-(-3)=-17$
           % § $\frac{-30\,000}{3}=10\,000$
        }
    \end{vastaus}
\end{tehtava}

\begin{tehtava}
Laske
    \alakohdat{
        § $3+(-8)$
        § $5+(+7)$
        § $-8-(-5)$
        § $+(+8)-(+5)$
        § $-(-8)-(+8)$.
    }
\begin{vastaus}
    \alakohdat{
        § $-5$
        § $12$
        § $-3$
        § $3$
        § $0$
    }
\end{vastaus}
\end{tehtava}

\begin{tehtava}
Laske
    \alakohdat{
        § $3\cdot (-6)$
        § $-7\cdot (+7)$
        § $-8\cdot (-5)$
        § $-(-8)\cdot (-8)$
        § $-(-8)\cdot (-(+5))$
        § $-(-8)\cdot (-5)\cdot (-1)\cdot (-2)$.
    }
\begin{vastaus}
    \alakohdat{
        § $-36$
        § $-49$
        § $40$
        § $-64$
        § $-40$
        § $-80$
    }
\end{vastaus}
\end{tehtava}

\begin{tehtava}
	Luet kirjaa sivulta $27$ sivulle $86$. Montako sivua luet?
	\begin{vastaus}
		$60$
	\end{vastaus} 
\end{tehtava}

\subsubsection*{Hallitse kokonaisuus}

\begin{tehtava}
Kirjoita ohjeen mukainen lauseke ja laske sen arvo, jos mahdollista
	\alakohdat{
		§ lukujen $5$ ja $-15$ osamäärä
		§ lukujen $2$ ja $8$ tulon vastaluku
		§ luvun $-2$ vastaluvun ja lukujen $x$ ja $y$ erotuksen osamäärä.
	}
\begin{vastaus}
	\alakohdat{
		§ $5:(-15)=-\frac{1}{3}$
		§ $-(2\cdot8)=-16$
		§ $\frac{-(-2)}{x-y}=\frac{2}{x-y}$
	}
\end{vastaus}
\end{tehtava}

\begin{tehtava}
Kirjoita erotukset summana
	\alakohdat{
		§ $130-(-5)$
		§ $-4-7$
		§ $2a-b$, kun $a$ ja $b$ ovat kokonaislukuja.
	}
\begin{vastaus}
	\alakohdat{
		§ $130+5$
		§ $-4+(-7)$
		§ $2a+(-b)$
	}
\end{vastaus}
\end{tehtava}

\begin{tehtava}
Kirjoita summat erotuksena
	\alakohdat{
		§ $31+4$
		§ $-15+(-92)$
		§ $-a+2b$, kun $a$ ja $b$ ovat kokonaislukuja.
	}
\begin{vastaus}
	\alakohdat{
		§ $31-(-4)$
		§ $-15-92$
		§ $-a-(-2b)$
	}
\end{vastaus}
\end{tehtava}

\begin{tehtava}
% Laatinut Sampo Tiensuu 2013-12-14 %liian isoja hyppyjä, eli lisää tehtäviä väliin! T: JoonasD6
Laske laskujärjestyssääntöjä noudattaen
	\alakohdat{
	    § $2+3\cdot(-1)$
	    § $(5-(2-3))\cdot 2$
	    § $1+9:3:3-(3-5):2$.
	}
\begin{vastaus}
	\alakohdat{
	    § $-1$
	    § $12$
	    § $3$
	}
\end{vastaus}
\end{tehtava}

\begin{tehtava}
% Laatinut Sampo Tiensuu 2013-12-14
$\star$ Laske.
	\alakohdat{
	    § $\left(\frac{(5+3\cdot 2)\cdot 6+2}{-(2-12:2)}+(-(-2)-1)\right):2$
	    § $-2+5\cdot\left[3+(2-3)\cdot 2-\frac{3-102}{121:11}\right]$
	    § $\{3\cdot 88:4:2+13:3\cdot \left[21:(14:2)\right]\}:9+1$
	}
\begin{vastaus}
	\alakohdat{
	    § $9$
	    § $48$
	    § $6$
	}
\end{vastaus}
\end{tehtava}

\begin{tehtava}
%Laatinut Paula Thitz 2014-02-09
$\star$ Lisää sulkuja siten, että yhtäsuuruusmerkin molemmilla puolilla olevilla lausekkeilla on sama arvo. \\
	\alakohdat{
		§ $5-1\cdot3+2=1+4\cdot4-2-4-8$
		§ $2\cdot-1:8+3\cdot4=2+1\cdot8-1$.
		§ $7-9+25\cdot5-1:61-6-5=3+4\cdot7\cdot1-5+14\cdot6+8$
	}
    \begin{vastaus}
		\alakohdat{
			§ $(5-1)\cdot3+2=(1+4)\cdot(4-2)-(4-8)$
			§ $2\cdot(-1:8+3)\cdot4=(2+1)\cdot8-1$
			§ $7-9+25\cdot(5-1):(61-6-5)=(3+4)\cdot7\cdot(1-5)+14\cdot(6+8)$	
		}
    \end{vastaus}
\end{tehtava}


\end{tehtavasivu}