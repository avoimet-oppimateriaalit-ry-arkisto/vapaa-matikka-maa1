\begin{tehtavasivu}

\begin{tehtava}
   	Mitkä seuraavista väitteistä pitävät paikkansa?
    \alakohdat{
        § $-1 \notin \zz$
        § $-1 \in \nn$
        § $\frac{1}{4} \notin \zz$
        § $-\frac{2}{4} \in \qq$
        § $-\frac{412}{97} \in \rr$
    }
    \begin{vastaus}
    	\alakohdat{
        § epätosi
        § epätosi
        § tosi
        § tosi
        § tosi
        }
    \end{vastaus}
\end{tehtava}

\begin{tehtava}
   	Kirjoita laskut näkyviin: Lukusuoralla on saavuttu pisteeseen $-2$. Mihin pisteeseen päädytään, kun liikutaan
    \alakohdat{
        § $3$ askelta negatiiviseen suuntaan?
        § $6$ askelta positiiviseen ja sitten $2$ negatiiviisen suuntaan?
        § $2$ askelta positiiviseen ja sitten $6$ negatiiviseen suuntaan?
    }

    \begin{vastaus}
    	Pisteeseen 
        \alakohdat{
            § $-2+(-3)=-5$
            § $-2+6+(-2)=2$
            § $-2+2+(-6)=-6$
        }
    \end{vastaus}
\end{tehtava}

\begin{tehtava}
    Kirjoita laskutoimitukseksi (laskuun ei tarvitse merkitä yksikköjä eli celsiusasteita).
    \alakohdat{
        § Lämpötila on aluksi $17\,\textdegree$C, ja sitten se vähentyy viisi astetta.
        § Lämpötila on aluksi $5\,\textdegree$C, ja sitten se kasvaa kuusi astetta.
        § Pakkasta on aluksi $-10\,\textdegree$C, ja sitten se lisääntyy kahdella pakkasasteella.
        § Pakkasta on aluksi $-20\,\textdegree$C, ja sitten se hellittää kolme astetta.
    }
    \begin{vastaus}
        \alakohdat{
            § $17-5=12$
            § $5+6=11$
            § $-10+(-2)=-12$
            § $-20-(-3)=-17$
        }
    \end{vastaus}
\end{tehtava}

\begin{tehtava}
Laske
    \alakohdat{
        § $3+(-8)$
        § $5+(+7)$
        § $-8-(-5)$
        § $+(+8)-(+5)$
        § $-(-8)-(+8)$.
    }
	\begin{vastaus}
    \alakohdat{
        § $-5$
        § $12$
        § $-3$
        § $3$
        § $0$
    }
	\end{vastaus}
\end{tehtava}

\begin{tehtava}
Laske
    \alakohdat{
        § $3\cdot (-6)$
        § $-7\cdot (+7)$
        § $-8\cdot (-5)$
        § $-(-8)\cdot (-8)$
        § $-(-8)\cdot (-(+5))$
        § $-(-8)\cdot (-5)\cdot (-1)\cdot (-2)$.
    }
\begin{vastaus}
    \alakohdat{
        § $-36$
        § $-49$
        § $40$
        § $-64$
        § $-40$
        § $-80$
    }
\end{vastaus}
\end{tehtava}

\begin{tehtava}
	\alakohdat{
	§ Kaverisi pänttää historiankokeeseen ja suunnittelee opiskelevansa $15$ sivua päivässä. Hän on nyt sivun $47$ alussa. Monennelle sivulle hän pääsee kolmessa päivässä?
	§ Päädyt töihin tietokantaohjelmoijaksi, ja työnantajasi antaa sinulle nimellisen tavoitteen kirjoittaa $2\,450$ riviä ohjelmakoodia viikossa (tässä tapauksessa viidessä arkipäivässä). Jos etenisit tasaiseen tahtiin, kuinka monta riviä koodia kirjoittaisit päivässä?
	§ Luet kirjaa sivun $100$ alusta sivun $120$ loppuun. Montako sivua luet?
	§ Luet toista kirjaa sivun $27$ alusta sivun $86$ loppuun. Montako sivua luet?
	}
	\begin{vastaus}
		\alakohdat{
		§ sivulle $92$
		§ $490$ riviä (koodirivien laskeminen on erittäin huono tapa seurata työn tehokkuutta)
		§ $121$ sivua
		§ $60$ sivua
		}
	\end{vastaus} 
\end{tehtava}

\begin{tehtava}
\alakohdat{
§ Olet Ruotsin-laivalla Turusta Tallinnaan, on jo yömyöhä, ja sinua väsyttää. Aamulla pitäisi nousta maihin (paikallista aikaa) kello $10$, jotta ehdit ajoissa Tukholmassa sijaitsevaan scifikirjakauppaan ennen muita päivän toimia. Tarvitset puoli tuntia aamutoimiin ennen maihinnousua. Käytät puhelintasi herätyskellona, mutta et ole heikon verkkosignaalin vuoksi varma, näyttääkö laite Suomen vai Ruotsin aika. Et jaksa mennä tarkistamaan oikeaa aikaa muualta. Ruotsi sijaitsee aikavyöhykkeellä, jonka aika on tunnin jäljessä Suomen aikaa. Haluat tietenkin maksimoida nukkumisaikasi, joten moneksi asetat herätyksen, jotta ehdit silti varmasti hoitaa aamutoimet? Ei haittaa, vaikka olisit ajoissa valmis poistumaan laivasta, mutta myöhästyminen olisi tuhoisaa. 
§ Sama tilanne, mutta yöjuna Helsingistä Moskovaan, missä aika on kaksi tuntia Suomea edellä.
}
	\begin{vastaus}
	\alakohdat{
	§ Aseta herätysajaksi $9.30$. Jos puhelimesi näyttää Ruotsin aikaa, ongelmaa ei ole. Jos puhelimesi on Suomen ajassa, puhelimesi herättää sinut tuntia aiemmin, mutta tällöin et ainakaan myöhästy. Sitä aikaisemmaksi on turha herätystä asettaa -- se vie vain unelta tunteja.
	§ Herätysajaksi kannattaa asettaa $7.30$. Jos puhelimesi näyttää Suomen aikaa, Moskovassa on kello tällöin juuri sopivasti $9.30$. Jos puhelimesi näyttää Moskovan aikaa (tai välissä olevaa tunnin Suomesta edellä olevaa aikaa), herätys on aikaisintaan kello $7.30$, jolloin olet kuitenkin ajoissa.
	}
	\end{vastaus}
\end{tehtava} %FIXME \todo teoriaosuuteen esimerkkejä aikavyöhykelaskuista

\begin{tehtava}
Kirjoita ohjeen mukainen lauseke ja laske sen arvo, jos mahdollista
	\alakohdat{
		§ lukujen $5$ ja $-15$ osamäärä
		§ lukujen $2$ ja $8$ tulon vastaluku
		§ luvun $-2$ vastaluvun ja lukujen $x$ ja $y$ erotuksen osamäärä.
	}
	\begin{vastaus}
	\alakohdat{
		§ $5:(-15)=-\frac{1}{3}$
		§ $-(2\cdot8)=-16$
		§ $\frac{-(-2)}{x-y}=\frac{2}{x-y}$
	}
	\end{vastaus}
\end{tehtava}

\begin{tehtava}
Kirjoita erotukset summana
	\alakohdat{
		§ $130-(-5)$
		§ $-4-7$
		§ $2a-b$, kun $a$ ja $b$ ovat kokonaislukuja.
	}
	\begin{vastaus}
	\alakohdat{
		§ $130+5$
		§ $-4+(-7)$
		§ $2a+(-b)$
	}
	\end{vastaus}
\end{tehtava}

\begin{tehtava}
Kirjoita summat erotuksena
	\alakohdat{
		§ $31+4$
		§ $-15+(-92)$
		§ $-a+2b$, kun $a$ ja $b$ ovat kokonaislukuja.
	}
	\begin{vastaus}
	\alakohdat{
		§ $31-(-4)$
		§ $-15-92$
		§ $-a-(-2b)$
	}
	\end{vastaus}
\end{tehtava}

\subsubsection*{Lisää tehtäviä}

\begin{tehtava}
% Laatinut Sampo Tiensuu 2013-12-1
Laske.
	\alakohdat{
	    § $2+3\cdot(-1)$
	    § $(5-(2-3))\cdot 2$
	    § $1+9:3:3-(3-5):2$.
	}
	\begin{vastaus}
	\alakohdat{
	    § $-1$
	    § $12$
	    § $3$
	}
	\end{vastaus}
\end{tehtava}

\begin{tehtava}
Tarkista laskemalla, onko lausekepareilla sama lukuarvo.
	\alakohdat{
	§ $(2\cdot 11)\cdot 3$ ja $2\cdot (11\cdot 3)$
	§ $(24:12):2$ ja $24:(12:2)$ 
	§ $(37-48)-11$ ja $37-(48-11)$
	}
	\begin{vastaus}
	\alakohdat{
	§ Molempien lausekkeiden lukuarvo on $66$.
	§ Ensimmäisestä tulee $1$ ja jälkimmäisestä $4$.
	§ Ensimmäisestä tulee $-22$ ja jälkimmäisestä $0$.
	}
	\end{vastaus}
\end{tehtava}

\begin{tehtava}
% Laatinut Sampo Tiensuu 2013-12-14
$\star$ Laske.
	\alakohdat{
	    § $\left(\frac{(5+3\cdot 2)\cdot 6+2}{-(2-12:2)}+(-(-2)-1)\right):2$
	    § $-2+5\cdot\left[3+(2-3)\cdot 2-\frac{3-102}{121:11}\right]$
	    § $\{3\cdot 88:4:2+13:3\cdot \left[21:(14:2)\right]\}:9+1$
	}
\begin{vastaus}
	\alakohdat{
	    § $9$
	    § $48$
	    § $6$
	}
\end{vastaus}
\end{tehtava}

\begin{tehtava}
%Laatinut Paula Thitz 2014-02-09
$\star$ Lisää sulkuja siten, että yhtäsuuruusmerkin molemmilla puolilla olevilla lausekkeilla on sama arvo.
	\alakohdat{
		§ $5-1\cdot3+2=1+4\cdot4-2-4-8$
		§ $2\cdot-1:8+3\cdot4=2+1\cdot8-1$.
		§ $7-9+25\cdot5-1:61-6-5=3+4\cdot7\cdot1-5+14\cdot6+8$
	}
    \begin{vastaus}
		\alakohdat{
			§ $(5-1)\cdot3+2=(1+4)\cdot(4-2)-(4-8)$
			§ $2\cdot(-1:8+3)\cdot4=(2+1)\cdot8-1$
			§ $7-9+25\cdot(5-1):(61-6-5)=(3+4)\cdot7\cdot(1-5)+14\cdot(6+8)$	
		}
    \end{vastaus}
\end{tehtava}

\end{tehtavasivu}