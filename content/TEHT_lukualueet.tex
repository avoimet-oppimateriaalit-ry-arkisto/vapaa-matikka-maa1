\begin{tehtavasivu}

\begin{tehtava}
   	Mitkä seuraavista väitteistä pitävät paikkansa?
    \alakohdat{
        § $-1 \notin \zz$
        § $-1 \in \nn$
        § $\frac{1}{4} \notin \zz$
        § $-\frac{2}{4} \in \qq$
        § $-\frac{412}{97} \in \rr$
    }
    \begin{vastaus}
    	\alakohdat{
        § epätosi
        § epätosi
        § tosi
        § tosi
        § tosi
        }
    \end{vastaus}
\end{tehtava}

\begin{tehtava}
   	Kirjoita laskut näkyviin: Lukusuoralla on saavuttu pisteeseen $-2$. Mihin pisteeseen päädytään, kun liikutaan
    \alakohdat{
        § $3$ askelta negatiiviseen suuntaan?
        § $6$ askelta positiiviseen ja sitten $2$ negatiiviisen suuntaan?
        § $2$ askelta positiiviseen ja sitten $6$ negatiiviseen suuntaan?
    }

    \begin{vastaus}
    	Pisteeseen 
        \alakohdat{
            § $-2+(-3)=-5$
            § $-2+6+(-2)=2$
            § $-2+2+(-6)=-6$
        }
    \end{vastaus}
\end{tehtava}

\begin{tehtava}
    Kirjoita laskutoimitukseksi (laskuun ei tarvitse merkitä yksikköjä eli celsiusasteita).
    \alakohdat{
        § Lämpötila on aluksi $17\,\textdegree$C, ja sitten se vähentyy viisi astetta.
        § Lämpötila on aluksi $5\,\textdegree$C, ja sitten se kasvaa kuusi astetta.
        § Pakkasta on aluksi $-10\,\textdegree$C, ja sitten se lisääntyy kahdella pakkasasteella.
        § Pakkasta on aluksi $-20\,\textdegree$C, ja sitten se hellittää kolme astetta.
    }
    \begin{vastaus}
        \alakohdat{
            § $17-5=12$
            § $5+6=11$
            § $-10+(-2)=-12$
            § $-20-(-3)=-17$
        }
    \end{vastaus}
\end{tehtava}

\begin{tehtava}
Laske
    \alakohdat{
        § $-(-h)$
        § $-(-(-h))$
        § $-(-h)-(-h)$
    }
	\begin{vastaus}
    \alakohdat{
        § $h$
        § $-h$
        § $0$
    }
	\end{vastaus}
\end{tehtava}

\begin{tehtava}
Laske
    \alakohdat{
        § $3+(-8)$
        § $5+(+7)$
        § $-8-(-5)$
        § $+(+8)-(+5)$
        § $-(-8)-(+8)$.
    }
	\begin{vastaus}
    \alakohdat{
        § $-5$
        § $12$
        § $-3$
        § $3$
        § $0$
    }
	\end{vastaus}
\end{tehtava}

\begin{tehtava}
Laske
    \alakohdat{
        § $3\cdot (-6)$
        § $-7\cdot (+7)$
        § $-8\cdot (-5)$
        § $-(-8)\cdot (-8)$
        § $-(-8)\cdot (-(+5))$
        § $-(-8)\cdot (-5)\cdot (-1)\cdot (-2)$.
    }
	\begin{vastaus}
    \alakohdat{
        § $-36$
        § $-49$
        § $40$
        § $-64$
        § $-40$
        § $-80$
    }
	\end{vastaus}
\end{tehtava}

\begin{tehtava}
Kirjoita ohjeen mukainen lauseke ja laske sen arvo, jos mahdollista
	\alakohdat{
		§ lukujen $5$ ja $-15$ osamäärä
		§ lukujen $2$ ja $8$ tulon vastaluku
		§ luvun $-2$ vastaluvun ja lukujen $x$ ja $y$ erotuksen osamäärä.
	}
	\begin{vastaus}
	\alakohdat{
		§ $5:(-15)=-\frac{1}{3}$
		§ $-(2\cdot8)=-16$
		§ $\frac{-(-2)}{x-y}=\frac{2}{x-y}$
	}
	\end{vastaus}
\end{tehtava}

\begin{tehtava} Koita oheisia pieniä laskuongelmia ratkaistaessa kiinnittämään huomiota omiin ajatuksiisi -- miten päädyit tulokseen? Kirjoita ratkaisumalliasi vastaava laskutoimitus ylös. %teoriaosuuteen formalismista, eli miten päässään kirjotietana laksutoimituksiksi
	\alakohdat{
	§ Kaverisi pänttää historiankokeeseen ja suunnittelee opiskelevansa $15$ sivua päivässä. Hän on nyt sivun $47$ alussa. Monennelle sivulle hän pääsee kolmessa päivässä?
	§ Päädyt töihin tietokantaohjelmoijaksi, ja työnantajasi antaa sinulle nimellisen tavoitteen kirjoittaa $2\,450$ riviä ohjelmakoodia viikossa (tässä tapauksessa viidessä arkipäivässä). Jos etenisit tasaiseen tahtiin, kuinka monta riviä koodia kirjoittaisit päivässä?
	§ Luet kirjaa sivun $100$ alusta sivun $120$ loppuun. Montako sivua luet?
	§ Äidinkielen kurssille luetaan Mika Waltarin \textit{Sinuhe Egyptiläinen}. Lukemassasi painoksessa on $738$ sivua. Olet juuri lukenut sivun $680$. Kuinka monta sivua sinulla on vielä lukematta?
	§ Luet toista kirjaa sivun $27$ alusta sivun $86$ loppuun. Montako sivua luet?
	§ Nettisivulla pyörii tietovisa maailman valtioista. Kysymyksiä on yhteensä kaksikymmentä, ja jokaisesta oikeasta vastauksesta saa pisteen. Yhdennentoista kysymyksen kohdalla toteat pistemääräsi olevan kahdeksan. Mikä on suurin kokonaispistemäärä, jonka pystyt vielä saavuttamaan?
	§ Lasagnea pidetään uunissa erään ohjeen mukaan $25$ minuuttia, ja lasagne pistettiin uuniin kello $19.15$. Toteat kellon myöhemmin olevan $19.32$. Kauanko lasagnen tulee olla vielä uunissa?
	}
	\begin{vastaus}
		\alakohdat{
		§ Sivulle $92$. Luku saadaan laskutoimituksesta $47+3\cdot 15$.
		§ $490$ riviä. Luku saadaan laskutoimituksesta $2\,450:5$. (Kirjoitettujen rivien laskeminen on muuten oikeasti \textit{erittäin} huono tapa seurata työn tehokkuutta.)
		§ $121$ sivua
		§ $58$ sivua
		§ $60$ sivua
		§ $17$ pistettä. Luku saadaan laskutoimituksesta $8+(20-11)$. (Sulkeet selkeyden vuoksi.)
		§ $8$ minuuttia. Vastaukseen päädytään laskemalla ensin kulunut aika minuuteissa $32-15$ ja vähentämällä tämä $25$ minuutista: $25-(32-15)$. (Tai $25-7$, kun $32-15$ on jo laskettu.)
		}
	\end{vastaus} 
\end{tehtava}

\begin{tehtava}
\alakohdat{
§ Olet Ruotsin-laivalla Turusta Tallinnaan, on jo yömyöhä, ja sinua väsyttää. Aamulla pitäisi nousta maihin (paikallista aikaa) kello $10$, jotta ehdit ajoissa Tukholmassa sijaitsevaan scifikirjakauppaan ennen muita päivän toimia. Tarvitset puoli tuntia aamutoimiin ennen maihinnousua. Käytät puhelintasi herätyskellona, mutta et ole heikon verkkosignaalin vuoksi varma, näyttääkö laite Suomen vai Ruotsin aikaa. Et jaksa mennä tarkistamaan oikeaa aikaa muualta. Ruotsi sijaitsee aikavyöhykkeellä, jonka aika on tunnin jäljessä Suomen aikaa. Haluat tietenkin toissijaisesti maksimoida nukkumisaikasi, joten moneksi asetat herätyksen, jotta ehdit silti varmasti hoitaa aamutoimet? Ei haittaa, vaikka olisit ajoissa valmis poistumaan laivasta, mutta myöhästyminen olisi tuhoisaa. 
§ Sama tilanne, mutta yöjuna Helsingistä Moskovaan, missä aika on kaksi tuntia Suomea edellä.
}
	\begin{vastaus}
	\alakohdat{
	§ Aseta herätysajaksi $9.30$. Jos puhelimesi näyttää Ruotsin aikaa, ongelmaa ei ole. Jos puhelimesi on Suomen ajassa, puhelimesi herättää sinut tuntia aiemmin, mutta tällöin et ainakaan myöhästy. Sitä aikaisemmaksi on turha herätystä asettaa -- se vie vain unelta tunteja.
	§ Herätysajaksi kannattaa asettaa $7.30$. Jos puhelimesi näyttää Suomen aikaa, Moskovassa on kello tällöin juuri sopivasti $9.30$. Jos puhelimesi näyttää Moskovan aikaa (tai välissä olevaa tunnin Suomesta edellä olevaa aikaa), herätys on aikaisintaan kello $7.30$, jolloin olet kuitenkin ajoissa.
	}
	\end{vastaus}
\end{tehtava} %FIXME \todo teoriaosuuteen esimerkkejä aikavyöhykelaskuista ja kellonajoista

\begin{tehtava}
Näköä harjoittavassa nopeuspelissä saa pisteitä, kun napauttaa näytölle ilmestyvää kuviota mahdollisimman nopeasti. Mitä kauemmin kuvion löytämisessä kestää, sitä vähemmän pisteitä saa; kaikkein nopeiten kohdetta napauttamalla saa nelinkertaisen pistemäärän. Jos pelaajan pistemäärä on ensin $33$ ja mahdollisimman nopean napsautuksen jälkeen $49$, kuinka monta pistettä kohteiden napautuksesta saa ilman nopeuskertoimia? 
	\begin{vastaus}
$(49-33):4=16:4=4$ eli $4$ pistettä
	\end{vastaus}
\end{tehtava}

\begin{tehtava}
Matematiikan ylioppilaskokeessa vastataan korkeintaan $10$ tehtävään. Kuinka monta minuuttia aikaa kokeessa on käytettävissä tehtävää kohden, jos koeaikaa on
\alakohdat{
§ kuusi tuntia
§ pidennetty kahdeksan tuntia (esimerkiksi lukihäiriön vuoksi)?
}
	\begin{vastaus}
	\alakohdat{
	§ $36$ minuuttia
	§ $48$ minuuttia
	}
	\end{vastaus}
\end{tehtava}

\begin{tehtava}
Kirjoita erotukset summana
	\alakohdat{
		§ $130-(-5)$
		§ $-4-7$
		§ $2a-b$, kun $a$ ja $b$ ovat kokonaislukuja.
	}
	\begin{vastaus}
	\alakohdat{
		§ $130+5$
		§ $-4+(-7)$
		§ $2a+(-b)$ ($a$:n ja $b$:n tarkempi määrittely teki laskutoimituksesta ylipäätään hyvin määritellyn, mutta tehtävän suorituksen kannalta ei ole olennaista, minkälaisia lukuja $a$ ja $b$ ovat.)
	}
	\end{vastaus}
\end{tehtava}

\begin{tehtava}
Kirjoita summat erotuksena
	\alakohdat{
		§ $31+4$
		§ $-15+(-92)$
		§ $-a+2b$, kun $a$ ja $b$ ovat rationaalilukuja.
	}
	\begin{vastaus}
	\alakohdat{
		§ $31-(-4)$
		§ $-15-92$
		§ $-a-(-2b)$ ($a$:n ja $b$:n tarkempi määrittely teki laskutoimituksesta ylipäätään hyvin määritellyn, mutta tehtävän suorituksen kannalta ei ole olennaista, minkälaisia lukuja $a$ ja $b$ ovat.)
	}
	\end{vastaus}
\end{tehtava}

%hallitse kokonaisuus, tehtäviä relaatioista ja jostain...

\subsubsection*{Lisää tehtäviä}

\begin{tehtava}
Koulussa on $240$ oppilasta, ja opettajia on sen verran, että luokkaryhmiä voidaan muodostaa korkeintaan neljätoista. Ryhmistä halutaan mahdollisimman samankokoisia. Kuinka suuria ryhmistä tulee, ja kuinka monta oppilasta ''jää yli''? ''Ylijäämäoppilaat'' sijoitetaan kaikki johonkin valmiiseen ryhmään. Kuinka suuri ryhmästä tällöin tulee?
	\begin{vastaus}
$240:14$ on hiukan yli seitsemäntoista, eli jokaiseen ryhmään tulee (alustavasti) seitsemäntoista oppilasta. $14\cdot 17=238$, eli $240-238=2$ oppilasta ''jää yli''. Kun nämä kaksi sijoitetaan johonkin olemassaolevista ryhmistä, ryhmään tulee $19$ oppilasta.
	\end{vastaus}
\end{tehtava}

%tahtinumeroiden laskeminen (kuva tahtinumeroista), toisella aukikirjoitettu ja toisella alku plus kertausosio

\begin{tehtava}
Tarkista laskemalla, onko laskutoimituksilla sama lukuarvo.
	\alakohdat{
	§ $(2\cdot 11)\cdot 3$ ja $2\cdot (11\cdot 3)$
	§ $(24:12):2$ ja $24:(12:2)$ 
	§ $(37-48)-11$ ja $37-(48-11)$
	}
	\begin{vastaus}
	\alakohdat{
	§ Molempien lausekkeiden lukuarvo on $66$.
	§ Ei -- ensimmäisestä tulee $1$ ja jälkimmäisestä $4$.
	§ Ei -- ensimmäisestä tulee $-22$ ja jälkimmäisestä $0$.
	}
	\end{vastaus}
\end{tehtava}

\begin{tehtava}
% Laatinut Sampo Tiensuu 2013-12-14
$\star$ Laske.
	\alakohdat{
	    § $\left(\frac{(5+3\cdot 2)\cdot 6+2}{-(2-12:2)}+(-(-2)-1)\right):2$
	    § $-2+5\cdot\left[3+(2-3)\cdot 2-\frac{3-102}{121:11}\right]$
	    § $\{3\cdot 88:4:2+13:3\cdot \left[21:(14:2)\right]\}:9+1$
	}
\begin{vastaus}
	\alakohdat{
	    § $9$
	    § $48$
	    § $6$
	}
\end{vastaus}
\end{tehtava}

\begin{tehtava}
%Laatinut Paula Thitz 2014-02-09
$\star$ Lisää sulkuja siten, että yhtäsuuruusmerkin molemmilla puolilla olevilla lausekkeilla on sama arvo.
	\alakohdat{
		§ $5-1\cdot3+2=1+4\cdot4-2-4-8$
		§ $2\cdot-1:8+3\cdot4=2+1\cdot8-1$.
		§ $7-9+25\cdot5-1:61-6-5=3+4\cdot7\cdot1-5+14\cdot6+8$
	}
    \begin{vastaus}
		\alakohdat{
			§ $(5-1)\cdot3+2=(1+4)\cdot(4-2)-(4-8)$
			§ $2\cdot(-1:8+3)\cdot4=(2+1)\cdot8-1$
			§ $7-9+25\cdot(5-1):(61-6-5)=(3+4)\cdot7\cdot(1-5)+14\cdot(6+8)$	
		}
    \end{vastaus}
\end{tehtava}

\begin{tehtava}
$\star$ Käytössäsi on neljä paria sukkia, kahdeksan paitaa ja kolmet housut. Kuinka monta erilaista asukokonaisuutta näistä voidaan muodostaa? (Ohje: Keksi nopeampi tapa yhdistelmien laskemiseen kuin kaikkien yhdistelmien kirjoittaminen ja lasku ''yksi, kaksi, kolme, ...''. Kaavion piirtäminen saattaa auttaa.)
	\begin{vastaus}
	$96$ erilaista
	\end{vastaus}
\end{tehtava}
%+Friedmann numbers?
\end{tehtavasivu}