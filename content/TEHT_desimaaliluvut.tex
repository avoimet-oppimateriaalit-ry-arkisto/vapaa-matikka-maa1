\begin{tehtavasivu}

\begin{tehtava}
Laske ja totea murtolukujen 
 $ \frac{1}{10} = 0,1$ , 
$ \frac{1}{100} = 0,01$ , 
 $ \frac{1}{2} = 0,5$ , 
$ \frac{1}{4} = 0,25$ , 
$ \frac{3}{4} = 0,75$
desimaaliesitysten paikkansapitävyys.
%Oli toivottu tehtäviä jakokulmasta, joten voisiko tämän tehtävän tehtävänantoon lisätä 
%``desimaaliesitysten paikkansapitävyys laskemalla jakokulmassa.'' (P Thitz 2014-02-08.)
\end{tehtava}

\begin{tehtava}
Esitä murtolukuna

a) $0,5$ \qquad b) $0,333...$ \qquad c) $1,234$ \qquad d) $-0,2$.
\begin{vastaus}
a) $\frac{1}{2}$ \qquad b) $\frac{1}{3}$\qquad c) $\frac{1234}{1000}=\frac{617}{500}$ d) $-\frac{1}{5}$
\end{vastaus}
\end{tehtava}

\begin{tehtava}
Esitä desimaalilukuna

a) $-\frac{3}{2}$ \qquad b) $\frac{3}{4}$\qquad c) $\frac{12}{20}$ d) $-1\frac{1}{6}$.
\begin{vastaus}
a) $-1.5$ \qquad b) $0,75$ \qquad c) $0,6$ \qquad d) $-1.1666...$
\end{vastaus}
\end{tehtava}

\begin{tehtava}
Muuta murtoluvuksi
%selkeys voitti täsmällisyyden ei siis murtolukumuotoon
	\begin{alakohdat}
		\alakohta{$43{,}532$}
		\alakohta{$5{,}031$}
		\alakohta{$0{,}23$}
		\alakohta{$0{,}3002$}
		\alakohta{$0{,}101$.}
	\end{alakohdat}
\begin{vastaus}
	\begin{alakohdat}
		\alakohta{$ \frac{43532}{1000}$}
		\alakohta{$ \frac{5031}{1000}$}
		\alakohta{$ \frac{23}{100}$}
		\alakohta{$ \frac{3002}{1000}$}
		\alakohta{$ \frac{101}{1000}$}
	\end{alakohdat}
\end{vastaus}
\end{tehtava}

\begin{tehtava}%sovteht tai vaikea tehtävä? sivevennyksen takia
Muuta murtoluvuksi ja sievennä
%selkeys voitti täsmällisyyden ei siis murtolukumuotoon
	\begin{alakohdat}
		\alakohta{$0{,}01$}
		\alakohta{$0{,}0245$}
		\alakohta{$0{,}004$}
		\alakohta{$0{,}001004$.}
	\end{alakohdat}
\begin{vastaus}
	\begin{alakohdat}
		\alakohta{$ \frac{1}{100}$}
		\alakohta{$ \frac{49}{200}$}
		\alakohta{$ \frac{1}{250}$}
		\alakohta{$ \frac{251}{250~000}$}
	\end{alakohdat}
\end{vastaus}
\end{tehtava}

\begin{tehtava}
Muuta murtoluvuksi
	\begin{alakohdat}
		\alakohta{$0,77777\ldots$}
		\alakohta{$0,151515 \ldots$}
		\alakohta{$2,05\overline{631}$}
		\alakohta{$0,99999\ldots$.}
	\end{alakohdat}
\begin{vastaus}
	\begin{alakohdat}
		\alakohta{$\frac{7}{9}$ }
		\alakohta{$\frac{15}{99}=\frac{5}{33}$}
		\alakohta{$\frac{205\ 426}{99\ 900} = \frac{102\ 713}{49\ 950}$}
		\alakohta{$\frac{9}{9} = 1$}
	\end{alakohdat}
\end{vastaus}
\end{tehtava}

\begin{tehtava}
Muuta desimaaliluvuksi
	\begin{alakohdatrivi}
		\alakohta{$\frac{151}{250}$}
		\alakohta{$\frac{251}{625}$}
		\alakohta{$\frac{386}{1\ 250}$}
		\alakohta{$\frac{493}{500}$.}
	\end{alakohdatrivi}
\begin{vastaus}
	\begin{alakohdat}
		\alakohta{$0,604$}
		\alakohta{$0,4016$}
		\alakohta{$0,3088$}
		\alakohta{$0,986$}
	\end{alakohdat}
\end{vastaus}
\end{tehtava}

\begin{tehtava}
Muuta murtoluvuksi
	\begin{alakohdatrivi}
		\alakohta{$0,\overline{649}$}
		\alakohta{$0,\overline{2154}$.}
	\end{alakohdatrivi}
\begin{vastaus}
	\begin{alakohdat}
		\alakohta{$\frac{649}{999}$}
		\alakohta{$\frac{718}{3333}$.}
	\end{alakohdat}
\end{vastaus}
\end{tehtava}

\begin{tehtava}
Muuta desimaaliluvuksi
	\begin{alakohdatrivi}
		\alakohta{$\frac{42}{11}$}
		\alakohta{$\frac{37}{13}$}
		\alakohta{$\frac{38}{99}$}
		\alakohta{$\frac{14}{15}$.}
	\end{alakohdatrivi}
\begin{vastaus}
	\begin{alakohdat}
		\alakohta{$3,\overline{81}$}
		\alakohta{$2,\overline{846153}$}
		\alakohta{$0,\overline{38}$}
		\alakohta{$0,9\overline{3}$}
	\end{alakohdat}
\end{vastaus}
\end{tehtava}



\begin{tehtava}
	Tunnissa on 60 minuuttia ja minuutissa on 60 sekuntia. Muuta seuraavat ajat tunneiksi
	\begin{alakohdat}
		\alakohta{73 minuuttia}
		\alakohta{649 sekuntia}
		\alakohta{15 minuuttia ja 50 sekuntia}
		\alakohta{42 minuuttia ja 54 sekuntia}
	\end{alakohdat}
	\begin{vastaus}
		\begin{alakohdat}
			\alakohta{1,21666... tuntia}
			\alakohta{0,1802777... tuntia}
			\alakohta{0,263888... tuntia}
			\alakohta{0,715 tuntia}
		\end{alakohdat}
	\end{vastaus}
\end{tehtava}

%\begin{tehtava}
%	Tarkastellaan jaksollista desimaalilukua \(a=0,1212\ldots\) Jakson pituus on 2 ja \(100a=12,1212\ldots\) Nyt \(100a-a=12\), joten \[a=\frac{12}{99}.\] Vastaavasti jos \(b=0,314314\ldots\), niin jakson pituus on 3 ja \(1000b=314,314314\ldots\) Siis \(999b=314\) ja niinpä \[b=\frac{314}{999}.\] Määritä samalla tekniikalla jaksollisten desimaalilukujen
%	\begin{alakohdat}
%		\alakohta{$0,2020\ldots$}
%		\alakohta{$0,118118\ldots$}
%		\alakohta{$0,333\ldots$}
%		\alakohta{$3,1414\ldots$}
%	\end{alakohdat}
%murtolukuesitykset.	
%	\begin{vastaus}
%		\begin{alakohdat}
%			\alakohta{$20/99$}
%			\alakohta{$118/999$}
%			\alakohta{$1/3$}
%			\alakohta{$3+14/99$}
%		\end{alakohdat}
%	\end{vastaus}
%\end{tehtava}

\end{tehtavasivu}
