Ylioppilaskokeissa on aina tehtäviä tai tehtävien alakohtia, joiden ratkaiseminen on mahdollista ensimmäisen kurssin tiedoin.

%FIXME: osa näistä kaipaa vastauksia...

\subsubsection*{Lyhyen oppimäärän tehtäviä}

\begin{tehtava} (K2014/1b,c) 
	\alakohdat{
		§ Laske lausekkeen {\Large$\frac{a^2-b^2}{a-b}$} arvo, kun $a=1$ ja $b=\frac{1}{2}$.
		§ Ratkaise yhtälö {\Large$\frac{x}{3}=\frac{x-1}{4}$}.
	}
    \begin{vastaus}
	\alakohdat{
		 § $\frac{3}{2}$
		 § $x=-3$
	}
    \end{vastaus}	
\end{tehtava}
%lisännyt jaakko viertiö 22.3.2014

\begin{tehtava}(K2014/2b)
	 \alakohdat{
%		 § a. Missä pisteessä suora $x-5y=4$ leikkaa $y$-akselin? %tällaisesta ei ole esimerkkiä kirjassa :(
		 § Ratkaise yhtälö $4x^3=48$. Anna tarkka arvo ja kolmidesimaalinen likiarvo.
	 }
    \begin{vastaus}
	 \alakohdat{
%		  § pisteessä $(0,-\frac{4}{5})$
		  § $x=\sqrt[3]{12}\approx2,289$
	 }
    \end{vastaus}	
\end{tehtava}
%lisännyt jaakko viertiö 22.3.2014

\begin{tehtava} (K2014/5) Boolimaljassa on $4,0$ litraa sekoitusta, jonka tilavuudesta $70\,\%$ on kuohuviiniä ja $30\,\%$ mansikkamehua. Kuinka paljon siihen täytyy lisätä kuohuviiniä, jotta mehun osuus on $20\,\%$?
    \begin{vastaus}
	    Kuohuviiniä on lisättävä $2,0$ litraa.
    \end{vastaus}
\end{tehtava}
%lisännyt jaakko viertiö 22.3.2014

\begin{tehtava} (K2014/6) Kiinalainen arvoitus $5\,000$ vuoden takaa: Häkissä on fasaaneja ja kaniineja. Oletetaan, että yhdellä fasaanilla on yksi pää ja kaksi jalkaa, ja yhdellä kaniinilla yksi pää ja neljä jalkaa. Niillä on yhteensä $35$ päätä ja $94$ jalkaa. Kuinka monta fasaania ja kuinka monta kaniinia häkissä on?
	\begin{vastaus}
		Kaniineja on $12$ ja fasaaneja $23$.
	\end{vastaus}
\end{tehtava}
%lisännyt jaakko viertiö 22.3.2014

\begin{tehtava} (K2014/14) Helsingin kaupunki teetti ennusteen kaupungin väestönkasvusta vuodesta $2012$ 
alkaen. Ennusteen mukaan asukasluku kasvaa lineaarisesti aikavälillä $2012--2030$ niin, että kaupungissa 
on $607\,417$ asukasta vuoden $2014$ alussa ja $629\,894$ asukasta vuoden $2018$ alussa. Ennusteessa 
ei otettu huomioon mahdollisia kuntaliitoksia.
	\alakohdat{
		§ Ennusteen mukaan asukasluku $y$ toteuttaa yhtälön \[y=a(x-2014)+6,\] kun $x$
		on vuosiluku. Määritä vakioiden $a$ ja $b$ tarkat arvot käyttämällä yllä mainittuja tietoja.
		§ Kuinka paljon asukasluku kasvaa ennusteen mukaan aikavälillä $2014--2030$? Anna vastaus 
		$1\,000$ asukkaan tarkkuudella.
		%§ Piirrä asukasluvun $y$ kuvaaja välillä $2014 \leq x \leq 2030$.
	}
    \begin{vastaus}
	\alakohdat{
		§ $a=5\,619,25$ ja $b=607\,417$
		§ $90\,000$
	%	§ Kuvaaja?
	}
    \end{vastaus}
\end{tehtava}
%lisännyt jaakko viertiö 22.3.2014


\subsubsection*{Pitkän oppimäärän tehtäviä}

\begin{tehtava} (K2013/1a) Ratkaise yhtälö $(x-4)^2=(x-4)(x+4)$.
	\begin{vastaus}
	$x=4$
	\end{vastaus}
\end{tehtava}

\begin{tehtava}
	(K2013/3b) Sievennä lauseke $(x^\frac{1}{3} + y^\frac{1}{3})(x^\frac{2}{3} - x^\frac{1}{3}y^\frac{1}{3} + y^\frac{2}{3})$.
		\begin{vastaus}
		$x+y$
		\end{vastaus}
\end{tehtava}
						
\begin{tehtava} (S2012/1a) Ratkaise yhtälö $2(1-3x+3x^2)=3(1+2x+2x^2)$.
	\begin{vastaus}
		$-\frac{1}{12}$
		\end{vastaus}
\end{tehtava}

\begin{tehtava}(S2012/1c) Ratkaise yhtälö $1-x=\frac{1}{1-x}$.
			\begin{vastaus}
				$x=0$ tai $x=2$
				\end{vastaus}
\end{tehtava}

\begin{tehtava}(K2012/1b) Ratkaise yhtälö $\frac{x}{6} - \frac{x-3}{2} - \frac{7}{9} = 0$. 
                        \begin{vastaus}
				$\frac{13}{6}$
			\end{vastaus}
\end{tehtava}

\begin{tehtava}(K2012/2a) Laske lausekkeen $ \frac{15}{4} - \left( \frac{6}{3} \right)^2 $ arvo.
\begin{vastaus}
				$-\frac{1}{4}$
				\end{vastaus}
\end{tehtava}

\begin{tehtava}(S2011/1b) Ratkaise yhtälö $\frac{4x - 1}{5} = \frac{x + 1}{2} + \frac{3 - x}{4}$.
                        \begin{vastaus}
				$x=\frac{29}{11}$
				\end{vastaus}
\end{tehtava}

\begin{tehtava}(S2011/2a) Sievennä välivaiheet esittäen lauseke $\frac{1}{\sqrt{2} + \frac{1}{2 + \sqrt{2}}}$.
                     \begin{vastaus}
				$2-\sqrt{2}$
				\end{vastaus}
\end{tehtava}


\begin{tehtava}(K2011/1a) Ratkaise yhtälö $\frac{2}{x} = \frac{3}{x - 2}$.
\begin{vastaus}
				$x=-4$
				\end{vastaus}
\end{tehtava}

\begin{tehtava}(K2011/2a) Osakkeen arvo oli $35,50$ euroa. Se nousi ensin $12\,\%$, mutta laski seuraavana päivänä $10\,\%$. Kuinka monta prosenttia arvo nousi yhteensä näiden muutosten jälkeen?
  \begin{vastaus} $0,8\,\%$
  \end{vastaus}
\end{tehtava}


\begin{tehtava}(S2010/1a) Sievennä lauseke $(a + b)^2 - (a - b)^2$.
\begin{vastaus}
				$4ab$
				\end{vastaus}
\end{tehtava}

\begin{tehtava}(K2010/1b) Sievennä lauseke $ (\sqrt{a} + 1)^2 - a - 1 $.
\begin{vastaus}
				$2\sqrt{2}$
				\end{vastaus}
\end{tehtava}

\begin{tehtava}(S2009/1c) Osoita, että $\sqrt{27 - 10 \sqrt{2}} = 5 - \sqrt{2} $.
\begin{vastaus}
				$(5-\sqrt{2})^2=27-10\sqrt{2}$ %välivaiheita
				\end{vastaus}
\end{tehtava}

\begin{tehtava}(S2009/2b) Ratkaise yhtälö $ \sqrt{x + 2 } = 3$.
\begin{vastaus}
				$x=7$
				\end{vastaus}
\end{tehtava}

\begin{tehtava}(K2009/1a) Sievennä $ \frac{a^2}{3} - \left( \frac{-a}{3} \right)^2 $.
\begin{vastaus}
				$\frac{2a^2}{9}$
				\end{vastaus}
\end{tehtava}

\begin{tehtava}(S2008/1b) Sievennä lauseke $\frac{1}{x} - \frac{1}{x^2} + \frac{1 + x}{x^2}$.
                        \begin{vastaus}
				$\frac{2}{x}$
				\end{vastaus}
\end{tehtava}

\begin{tehtava}(S2009/2b) Ratkaise yhtälö $\frac{x}{6} - \frac{x - 2}{3} = \frac{5}{12}$.
                        \begin{vastaus}
				$x=\frac{3}{2}$
				\end{vastaus}
\end{tehtava}

\begin{tehtava}(K2008/4) Vuonna 2007 alennettiin parturimaksujen arvonlisäveroa $22$ prosentista $8$ prosenttiin. Jos alennus olisi siirtynyt täysimääräisenä parturimaksuihin, kuinka monta prosenttia ne olisivat alentuneet? Arvonlisävero ilmoitetaan prosentteina verottomasta hinnasta ja se on osa tuotteen tai palvelun hintaa.
\end{tehtava}

\begin{tehtava}(S2007/1c) Ratkaise $L$ yhtälöstä $t=\frac{1}{2\pi\sqrt{LC}}$.
\end{tehtava}

\begin{tehtava}(S2007/4) Tuotteen hintaa korotettiin $p$ prosenttia, jolloin menekki väheni. Tämän johdosta hinta päätettiin alentaa takaisin alkuperäiseksi. Kuinka monta prosenttia korotetusta hinnasta alennus oli?
\end{tehtava}

\begin{tehtava}(K2007/1c) Sievennä lauseke $ \sqrt[3]{a \sqrt{a}} \quad (a > 0) $.
\end{tehtava}

\begin{tehtava}(K2007/3a) Merivettä, jossa on $4,0$ painoprosenttia suolaa, haihdutetaan altaassa, kunnes sen massa on vähentynyt $28$\,\%. Mikä on suolapitoisuus haihduttamisen jälkeen? Anna vastaus prosentin kymmenesosan tarkkuudella. 
\end{tehtava}

\begin{tehtava}(K2007/3b) Mikä on vuotuinen korkoprosentti, jos tilille talletettu rahamäärä kasvaa korkoa korolle $1,5$-kertaiseksi $10$ vuodessa? Lähdeveroa ei oteta huomioon. Anna vastaus prosentin sadasosan tarkkuudella.
\end{tehtava}

\begin{tehtava}(S2006/5) Hopean ja kuparin seoksesta tehty esine painaa $150$\,g, ja sen tiheys on $10,1$\,kg/dm$^3$. Kuinka monta painoprosenttia esineessä on hopeaa ja kuinka monta kuparia, kun hopean tiheys on $10,5$\,kg/dm\(^3\) ja kuparin $9,0$\,kg/dm$^3$?
\end{tehtava}

\begin{tehtava}(K2006/1a) Ratkaise $x$ yhtälöstä $4x + 2 =  3 - 2(x + 4)$.
\end{tehtava}

\begin{tehtava}(K2006/1c) Sievennä lauseke 
                        $ \frac{1}{a - 1} \left( a - \frac{1}{a} \right) $
\end{tehtava}

\begin{tehtava}(K2006/4) Kesämökin rakentaminen tuli $25\,\%$ arvioitua kalliimmaksi. Rakennustarvikkeet olivat $19\,\%$ ja muut kustannukset $28\,\% $ arvioitua kalliimpia. Mikä oli rakennustarvikkeiden arvioitu osuus ja mikä lopullinen osuus kokonaiskustannuksista?
\end{tehtava}

\begin{tehtava}(S2005/1a) Ratkaise reaalilukualueella yhtälö 
                       $ 2(x - 1) + 3(x + 1 ) = -x $
\end{tehtava}

\begin{tehtava}(S2005/1c) Ratkaise reaalilukualueella yhtälö $ x^{16} = 256 $.
\end{tehtava}

\begin{tehtava}(K2005/1a) Sievennä lauseke
                        $ \frac{x}{1 - x} + \frac{x}{1 + x} $
\end{tehtava}

\begin{tehtava}(K2005/2a) Ratkaise yhtälöryhmä
                      $
                        \left\{
                        \begin{aligned}
                             x + y &= a \\
                             x - y &= 2a
                        \end{aligned}
                        \right.
                    $
\end{tehtava}

\begin{tehtava}(K2005/3) Asuinrakennuksesta saadut vuokrat ovat $12\,\%$ pienemmät kuin ylläpitokustannukset. Kuinka monta prosenttia vuokria oli korotettava, jotta ne tulisivat $10\,\%$ suuremmiksi kuin ylläpitokustannukset, jotka samanaikaisesti kohoavat $4\,\%$?
\end{tehtava}

\begin{tehtava}(K2004/3) Perheen vuokramenot olivat $25\,\%$ tuloista. Vuokramenot nousivat $15\,\%$. Montako prosenttia vähemmän rahaa riitti muuhun käyttöön korotuksen jälkeen?
\end{tehtava}

\begin{tehtava}(S2003/5) Päärynämehusta ja omenamehusta tehdyn sekamehun sokeripitoisuus on $11\,\%$. Määritä mehujen sekoitussuhde, kun päärynämehun sokeripitoisuus on $14\,\%$ ja omenamehun $7\,\%$.
\end{tehtava}


\begin{tehtava}(K2003/1) Sievennä lausekkeet
        \alakohdat{
            § $ \sqrt{3\frac{3}{4}} \big/ \sqrt{1\frac{2}{3}} $
            § $ \left( \frac{x}{y} + \frac{y}{x} - 2 \right) \big/ \left( \frac{x}{y} - \frac{y}{x} \right) $.
        }
\end{tehtava}

\begin{tehtava}(S2002/2) Vuoden 1960 jälkeen on nopeimman junayhteyden matka-aika Helsingin ja Lappeenrannan välillä lyhentynyt $37$ prosenttia. Laske, kuinka monta prosenttia keskinopeus on tällöin noussut. Oletetaan, että radan pituus ei ole muuttunut.
\end{tehtava}

\begin{tehtava}(S2002/4a) Olkoon $ a \neq 0$ ja $b \neq 0 $. Sievennä lauseke
                        $
                            \frac{a + \frac{b^2}{a} } {b + \frac{a^2}{b} }
                        $
\end{tehtava}

\begin{tehtava}(K2002/3) Vuonna 2001 erään liikeyrityksen ulkomaille suuntautuvan myynnin arvo kasvoi $10$\,\% vuoteen 2000 verrattuna. Samaan aikaan myynnin arvo kotimaassa väheni $5$\,\%. Tällöin koko myynnin arvo kasvoi $6$\,\%. Laske, kuinka monta prosenttia myynnistä meni vuonna 2000 ulkomaille.
\end{tehtava}

\begin{tehtava}(S2001/1) Ratkaise lineaarinen yhtälöryhmä
                       $
                         \left\{
                          \begin{aligned}
                             3x - 2y &= 1 \\
                             4x + 5y &= 2                      
                         \end{aligned}
                         \right.
                       $
\end{tehtava}

\begin{tehtava}(S2001/3) Juna lähtee Tampereelta klo 8.06 ja saapuu Helsinkiin klo 9.58. Vastakkaiseen suuntaan kulkeva juna lähtee Helsingistä klo 8.58 ja saapuu Tampereelle klo 11.02. Matkan pituus on $187$ kilometriä. Oletetaan, että junat kulkevat tasaisella nopeudella, eikä pysähdyksiin kuluvia aikoja oteta huomioon. Laske kummankin junan keskinopeus. Millä etäisyydellä Helsingistä junat kohtaavat, ja paljonko kello tällöin on? 
\end{tehtava}

\begin{tehtava}(K2001/4) Säiliö sisältää $2,3$\,kg ilmaa, ja pumppu poistaa jokaisella vedolla $5$\,\% säiliössä olevasta ilmasta. Kuinka monen vedon jälkeen säiliössä on vähemmän kuin $0,2$\,kg ilmaa?
\end{tehtava}

\begin{tehtava}(S2000/1) Sievennä seuraavat lausekkeet:
                        \alakohdat{
                            § $ \left( x^{n - 1} \right)^{n - 1} \cdot \left( x^{n} \right)^{2 - n} $
                            § $ \sqrt[3]{a} \; ( \sqrt[3]{a^2} - \sqrt[3]{a^5}) $
                        }
\end{tehtava}

\begin{tehtava}(S2000/3) Matkaa kuljetaan tasaisella nopeudella. Kun matkasta on jäljellä $40$\,\%, nopeutta lisätään $20$\,\%. Kuinka monta prosenttia koko matkaan kuluva aika tällöin lyhenee?
\end{tehtava}


