Sana \termi{prosentti}{prosentti} tulee latinan kielen ilmaisusta \termi{pro centum}{pro centum},
joka tarkoittaa kirjaimellisesti sataa kohden. 
Prosentit ovat siis sadasosia.
Prosentteja käytetään ilmaisemaan suhteellista osuutta.
Prosentin merkki on \%.

\laatikko{$1\,\textnormal{prosentti} \; = 1\,\% = \frac{1}{100} = 0,01$}

\begin{esimerkki}
Tarmo leikkaa pizzan neljään osaan ja syö näistä kolme.

§ Kuinka monta prosenttia pizzasta Tarmo syö? \\
b) Kuinka monta prosenttia pizzasta jää jäljelle?

\textbf{Ratkaisu.}

§ Tarmo syö 3 palaa neljästä eli $\frac{3}{4} = 0,75$. Tämä voidaan muuttaa prosenttiluvuksi ilmaisemalla luku sadasosina. Tämä saadaan kertomalla luku sadalla: $0,75 \cdot 100 = 75\,\%$. Siis tarmo söi $75\,\%$ pizzasta
\newline
§ Koko pizzasta otetaan pois Tarmon syömä osa jolloin jäljelle jää $100\,\% - 75\,\% = 25\,\%$.
\end{esimerkki}


\begin{esimerkki}
Eräs kauppa myy tuotteensa $5$ prosentin alennuksella. Tämä tarkoittaa, että jokaisesta hinnasta vähennetään $\frac{5}{100}$ kertaa tuotteen hinta. Jos tuotteen alkuperäinen hinta on esimerkiksi $100$\,euroa, saadaan $5\,\%$ alennus laskettua $100$\,eurosta kertomalla $100$\,euroa $5\,\%$:lla eli viidellä sadasosalla 
\[
\frac{5}{100} \cdot 100\,\text{euroa} = 5\,\text{euroa}
\]
Alennettu hinta saadaan vähentämällä alkuperäisestä hinnasta alennus eli $100\,\text{euroa} - 5\,\text{euroa} = 95\,\text{euroa}$.

Jos tuotteen alkuperäinen hinta on $14,50$\,euroa, on alennuksen määrä
\[
	\frac{5}{100} \cdot 14,50\,\text{euroa} = 0,725\,\text{euroa}.
\]
Alennettu hinta on tällöin $(14,50 - 0,725)\,\text{euroa} \approx 13,78\,\text{euroa}$.

Samaan tulokseen pääsee vielä kätevämmin, kun ajattelee alennuksen määrän sijaan sitä, kuinka suuri osa hinnasta jää jäljelle. Jos alennus on $5\,\%$, jää hinnasta jäljelle $100\,\% - 5\,\% = 95\,\%$. Jos alkuperäinen hinta on $14,50$\,euroa, on alennettu hinta 
\[
	\frac{95}{100} \cdot 14,50\,\text{euroa} \approx 13,78\,\text{euroa}.
\]
\end{esimerkki}

\begin{esimerkki}
    Prosenttiluvut voidaan esittää myös muilla tavoin.
    \alakohdat{
        § $6\,\% = \frac{6}{100} = 0,06$
        § $48,2\,\% = \frac{48,2}{100} = 0,482$
        § $140\,\% = \frac{140}{100} = 1,40$ 
    }
\end{esimerkki}

% PERUSARVO
\laatikko{Lukua, josta suhde lasketaan, kutsutaan \termi{perusarvo}{perusarvoksi}.}

\begin{esimerkki}
    Jos sadan euron hintaisen tuotteen hintaa on alennettu $25$\,prosenttia,
    niin alennettu hinta on $75$\,euroa. Jos sen sijaan alkuperäinen
    hinta nousee $15$\,prosenttia, niin tuotteen uusi hinta on $115$\,euroa.
    Perusarvo on molemmissa tapauksissa $100$\,euroa.
    
    \begin{center}
        \includegraphics[scale=.25]{pictures/Kuva13-1-100.pdf}
        \includegraphics[scale=.25]{pictures/Kuva13-2-75.pdf}
        \includegraphics[scale=.25]{pictures/Kuva13-3-115.pdf}
    \end{center}
\end{esimerkki}

% MUUTOSPROSENTTI
\laatikko{
    Prosentteja käytetään usein ilmaisemaan suureiden muutoksia, esimerkiksi luku $a$ kasvaa luvuksi $b$.
    \termi{muutosprosentti}{Muutosprosenttia} laskettaessa muutoksen suuruutta verrataan alkuperäiseen lukuun.
    Perusarvona on siis alkuperäinen arvo, johon nähden muutos on tapahtunut. Muutosta merkitään yleensä symbolilla
    $\Delta$.
    
    \termi{absoluuttinen muutos}{Absoluuttinen muutos} luvusta $a$ lukuun $b$ on $b-a$.
    \termi{suhteellinen muutos}{Suhteellinen muutos} saadaan suhteuttamalla absoluuttinen muutos alkuperäiseen lukuun $a$ eli laskemalla
    
    \[ \Delta_{\text{suhteellinen}} = \frac{\Delta_{\text{absoluuttinen}}}{a} = \frac{b-a}{a} \]
    
    Muutosprosentti saadaan suhteellisesta muutoksesta muuttamalla se prosenttiluvuksi
    
    \[ \Delta_{\text{prosentti}} = \Delta_{\text{suhteellinen}} = \frac{b-a}{a} \] tulkittuna prosenteissa.
    
    
}

\begin{esimerkki}
    Vesan paino on tammikuussa $68$\,kg ja kesäkuussa $64$\,kg.
    \newline
§ Mikä on Vesan painon \textit{absoluuttinen muutos}?
    \newline
§ Mikä on Vesan painon \textit{muutosprosentti}?
    
    \textbf{Ratkaisu.}
§ Halutaan tietää Vesan painon \textit{absoluuttinen muutos} eli muutos kiloina tammikuusta kesäkuuhun.
    \[
       \Delta_{\text{absoluuttinen}} = b - a = 64\,\text{kg} - 68\,\text{kg} = -4\,\text{kg}
    \]

    Vesan paino on muuttunut $-4$ kiloa, eli Vesa on laihtunut 4 kiloa.

b) Halutaan tietää Vesan painon muutos \textit{prosentteina} tammikuusta kesäkuuhun.
    
    \[
        \Delta_{\text{prosentti}}
        = \frac{\Delta_{\text{absoluuttinen}}}{a}
        = \frac{-4}{68}
        \approx -0,06
        = -6\,\% 
    \]
    
    Vesan paino on muuttunut kuudella prosentilla negatiiviseen suuntaan,
    eli Vesa on laihtunut kuusi prosenttia.
    
    \textbf{Vastaus.}
    
§ Vesa on laihtunut 4 kiloa.
b) Vesa on laihtunut $6\,\%$.
\end{esimerkki}

% EROTUSPROSENTTI
\laatikko{
    Muutosprosentille läheinen käsite on \termi{erotusprosentti}{erotusprosentti}.
    Erotusprosentti ilmaisee kuinka monta prosenttia jokin on suurempi tai pienempi kuin joku toinen.
    
    Suhteellinen erotus saadaan laskemalla lukujen absoluuttinen erotus $|b-a|$ ja vertaamalla sitä perusarvoon, joka on aina ``kuin''-sanan jälkeinen arvo.
    
    Jos luku $a$ on $p\,\%$ pienempi tai suurempi kuin luku $b$, pätee
    \[ p = \frac{|b-a|}{b}. \]
}
    
\begin{esimerkki}
    Miniluumutomaatit maksavat normaalisti 4,80\,euroa kilolta. Nyt ne ovat kuitenkin alennuksessa ja maksavat 2,50\,euroa kilolta.
     \newline
§ Kuinka monta prosenttia enemmän miniluumutomaatit maksavat normaalisti kuin alennuksessa?
     \newline
§ Kuinka monta prosenttia vähemmän miniluumutomaatit maksavat alennuksessa kuin normaalisti?
     
     \textbf{Ratkaisu.}
     
§ Verrataan absoluuttista erotusta alennettuihin miniluumutomaatteihin:
\begin{align*}
     &\frac{|b-a|}{b}  = \frac{|2,50-4,80|}{2,50} = \frac{|-2,30|}{2,50} \\
     = &\frac{2,30}{2,50}  = 0,92 = 92,0\,\%.
\end{align*}
    
    
b) Verrataan absoluuttista erotusta normaalihintaisiin miniluumutomaatteihin:
\begin{align*}
     &\frac{|b-a|}{a} = \frac{|2,50-4,80|}{4,80} = \frac{|-2,30|}{4,80} \\
     = &\frac{2,30}{4,80}  \approx 0,479  = 47,9\,\%.
\end{align*}

     \textbf{Vastaus.}
     
    
§ Miniluumutomaatit maksavat normaalisti 92,0\,\% enemmän kuin alennuksessa.
     \newline
§ Miniluumutomaatit maksavat alennuksessa 47,9\,\% vähemmän kuin normaalisti.
     
     Huomataan, että se mihin verrataan on merkittävää.
     \end{esimerkki}
    
% VERTAILUPROSENTTI
\laatikko{
    \termi{vertailuprosentti}{Vertailuprosentilla} tarkoitetaan sitä, kuinka paljon jokin on jostakin.
    
    Vertailuprosentilla vastataan siis kysymykseen ''kuinka monta prosenttia luku $a$ on luvusta $b$?''
    Vertailuprosentti on tässä tapauksessa $\frac{a}{b} $.
}

\begin{esimerkki}
    Vesa ansaitsee kuukaudessa ${3\,200}$ euroa ja Antero ${2\,300}$ euroa.
    Kuinka monta prosenttia Anteron tulot ovat Vesan tuloista? 
    
    \textbf{Ratkaisu.}
    
    Lasketaan vertailuprosentti. Perusarvo on tehtävänannon mukaisesti
    Vesan palkka eli ${3\,200}$ euroa.
    
    \[
        \frac{2\,300}{3\,200} 
        \approx 0,72
        = 72\,\%.
    \]
    
    \textbf{Vastaus.}
    
    Anteron tulot ovat $72\,\%$ Vesan tuloista.
\end{esimerkki}

Joissain tilanteissa perusarvo on tuntematon. Se ratkeaa usein kätevästi yhtälöllä.
% MUUTOSKERTOIMESTA PIENI TEORIAPLÄJÄYS?
\begin{esimerkki}
Tuotteen hintaa korotettiin $15$\,\%, jolloin hinnaksi muodostui $175$\,euroa. Kuinka suuri oli alkuperäinen korottamaton hinta?

\textbf{Ratkaisu.} 
Olkoon alkuperäinen hinta $x$\,euroa. Koska hintaa on korotettu $15$\,\%, on uusi hinta $115$\,\% alkuperäisestä eli alkuperäinen hinta $x$ on kerrottu $1,15$:llä:
\begin{align*}
	1,15x	&= 175	&	&|\, \text{Jaetaan} 1,15\text{:llä}.\\
	x	&= \frac{175}{1,15} \approx 152,17
\end{align*}
    \textbf{Vastaus.}
    Tuotteen alkuperäinen hinta oli $152,17$\,euroa.
\end{esimerkki}

Huomaa, että edellisessä esimerkissä saatuun kertoimeen $1,15 = 115\,\%$ oltaisiin päädytty myös laskemalla $x + 0,15x = (1 + 0,15)x = 1,15x$.

Lauseella ''kaksi kertaa enemmän'' tarkoitetaan usein yleiskielessä kaksinkertaista.
Matemaattisesti nämä ovat kuitenkin eri asioita.
Sekaannuksen vuoksi on syytä olla käyttämättä tällaista ilmaisua.

% PROSENTTIYKSIKKÖ
\laatikko{\, \,
    \termi{prosenttiyksikkö}{Prosenttiyksikköä} käytetään mittaamaan prosenttilukujen absoluuttista muutosta.
    Esimerkiksi $3\,\%$ on yhden prosenttiyksikön suurempi kuin $2\,\%$, mutta $50\,\%$
    suurempi kuin $2\,\%$. Jos prosenttiluku muuttuu, muutos voidaan ilmaista joko
    prosentteina (suhteellinen muutos) %aikaisemmin: ``prosenttimuutos''
    tai prosenttiyksikköinä (absoluuttinen muutos).
    
    Prosentin ja prosenttiyksikön merkitysero on keskeinen esimerkiksi
    talousuutisten tulkinnassa.
}

\begin{esimerkki}
    Tuotteen markkinaosuus on vuoden tammikuussa $10$\,\% ja kesäkuussa $15$\,\%.
    \newline
§ Kuinka monta prosenttia tuotteen markkinaosuus on noussut tammikuusta kesäkuuhun?
    \newline
§ Kuinka monta prosenttiyksikköä tuotteen markkinaosuus on noussut tammikuusta kesäkuuhun?

    \textbf{Ratkaisu.}
    
§ Tuotteen markkinaosuuden muutos prosentteina:
          \[
                \frac{15-10}{10} = \frac{5}{10} = 0,5 = 50\,\%.
          \]
  
b) Tuotteen markkinaosuuden muutos prosenttiyksiköinä $15\,\%-10\,\%=5\,\%$

\textbf{Vastaus.}

§ $50$ prosenttia
b) $5$ prosenttiyksikköä
\end{esimerkki}
