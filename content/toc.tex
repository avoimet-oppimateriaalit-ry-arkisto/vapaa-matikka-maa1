\newcommand{\osa}[1]{\chapter{#1}} % osa
\newcommand{\nosa}[1]{\chapter*{#1} \addcontentsline{toc}{chapter}{#1}} % numeroimaton osa
\newcommand{\luku}[2]{\section{#2} \input{chapters/TEORIA_#1} \input{chapters/TEHT_#1}} % luku
\newcommand{\nluku}[2]{\section*{#2} \addcontentsline{toc}{section}{#2} \input{chapters/#1}} % numeroimaton luku

\Opensolutionfile{ans}[chapters/LIITE_vastaukset]

\osa{Luvut ja laskutoimitukset}
    \luku{rationaaliluvut}{Kokonais- ja rationaaliluvut}
    \luku{potenssi}{Potenssi}
    \luku{desimaaliluvut}{Luvun desimaaliesitys}
    \luku{yksikot_ja_pyoristaminen}{Yksiköt ja vastaustarkkuus}
    \luku{reaaliluvut}{Reaaliluvut}
    \luku{juuret}{Juuret}
    \luku{murtopotenssi}{Murtopotenssi}

\osa{Yhtälöt}
    \luku{yhtalo}{Yhtälö}
    \luku{ensimmainen}{Ensimmäisen asteen yhtälö}
    \luku{verrannollisuus}{Suoraan ja kääntäen verrannollisuus}
    \luku{prosenttilaskenta}{Prosenttilaskenta}
    \luku{potenssiyhtalot}{Potenssiyhtälöt}

\osa{Funktiot}
    \luku{funktio}{Funktio ja sen esitystavat} % ok?
	\luku{potenssifunktio}{Potenssifunktio}
	\luku{eksponenttifunktio}{Eksponenttifunktio}
	
%\nosa{Tiivistelmä}

\nosa{Kertaustehtäviä}
    \nluku{TEHT_harjoituskokeet}{Harjoituskokeita}
    \nluku{TEHT_ylioppilaskokeet}{Ylioppilaskoetehtäviä}
	\nluku{TEHT_paasykokeet}{Pääsykoetehtäviä}

\Closesolutionfile{ans}

\nosa{Lisämateriaalia}
	\nluku{LIITE_lahtotasotesti}{Lähtötasotesti} % ok?
	\nluku{LIITE_saannot}{Laskusääntöjen todistuksia}
	\nluku{LIITE_lukujarjestelmat}{Lukujärjestelmät} % ok?
	\nluku{LIITE_lukuteoria}{Lukuteorian alkeet} % ok?
	%\nluku{LIITE_joukot}{Joukko-opin alkeet}
	\nluku{LIITE_aksioomat}{Reaalilukujen aksioomat} % ok?
    %\nluku{LIITE_sanasto}{Sanasto}
	%\nluku{LIITE_symbolit}{Symbolit}
	\nluku{LIITE_vastaukset}{Vastaukset}
