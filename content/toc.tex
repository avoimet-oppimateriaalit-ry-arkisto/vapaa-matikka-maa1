\newcommand{\osa}[1]{\chapter{#1}} % osa
\newcommand{\nosa}[1]{\chapter*{#1} \addcontentsline{toc}{chapter}{#1}} % numeroimaton osa
\newcommand{\luku}[2]{\section{#2} \input{chapters/#1}} % luku
\newcommand{\nluku}[2]{\section*{#2} \addcontentsline{toc}{section}{#2} \input{chapters/#1}} % numeroimaton luku

\Opensolutionfile{ans}[chapters/LIITE_vastaukset]

\osa{Luvut ja laskutoimitukset}
    \luku{rationaaliluvut}{Kokonais- ja rationaaliluvut}
    \luku{potenssi}{Potenssi}
    \luku{desimaaliluvut}{Luvun desimaaliesitys}
    \luku{reaaliluvut}{Reaaliluvut}
    \luku{juuret}{Juuret}
    \luku{murtopotenssi}{Murtopotenssi}

\osa{Yhtälöt}
    \luku{yhtalo}{Yhtälö}
    \luku{ensimmaisen-asteen-yhtalo}{Ensimmäisen asteen yhtälö}
    \luku{verrannollisuus}{Suoraan ja kääntäen verrannollisuus}
    \luku{prosenttilaskenta}{Prosenttilaskenta}
    \luku{potenssiyhtalot}{Potenssiyhtälöt}

\osa{Funktiot}
    \luku{funktio}{Funktio ja sen esitystavat}
	\luku{potenssifunktio}{Potenssifunktio}
	\luku{eksponenttifunktio}{Eksponenttifunktio}

\nosa{Kertaustehtäviä}
    \nluku{harjoituskokeet}{Harjoituskokeita}
    \nluku{ylioppilaskokeet}{Ylioppilaskoetehtäviä}
	\nluku{paasykokeet}{Pääsykoetehtäviä}

\Closesolutionfile{ans}

\nosa{Lisämateriaalia}
    \nluku{LIITE_lahtotasotesti}{Lähtötasotesti}
	\nluku{LIITE_kaytanto}{Yksiköt ja vastaustarkkuus}
	\nluku{LIITE_saannot}{Laskusääntöjen todistuksia}
	\nluku{LIITE_lukuteoria}{Lukuteorian alkeet}
	\nluku{LIITE_joukot}{Logiikan ja joukko-opin alkeet}
	\nluku{LIITE_lukujarjestelmat}{Lukujärjestelmistä}
	\nluku{LIITE_aksioomat}{Reaalilukujen aksioomat}
	\nluku{LIITE_vastaukset}{Vastaukset}
    \nluku{LIITE_sanasto}{Sanasto}

% Kertaustiivistelmä
% makeindex
% Symbolitaulukko
