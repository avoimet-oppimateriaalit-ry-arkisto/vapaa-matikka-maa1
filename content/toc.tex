\newcommand{\osa}[1]{\chapter{#1}} % osa
\newcommand{\nosa}[1]{\chapter*{#1} \addcontentsline{toc}{chapter}{#1}} % numeroimaton osa
\newcommand{\luku}[2]{\section{#2} \input{chapters/TEORIA_#1} \input{chapters/TEHT_#1}} % luku
\newcommand{\nluku}[2]{\section*{#2} \addcontentsline{toc}{section}{#2} \input{chapters/#1}} % numeroimaton luku
\newcommand{\vast}{\section*{Vastaukset} \addcontentsline{toc}{section}{Vastaukset}


\begin{vastaussivu} \input{chapters/LIITE_vastaukset} \end{vastaussivu}}

\Opensolutionfile{ans}[chapters/LIITE_vastaukset]


\osa{Luvut ja laskutoimitukset}
    \luku{luvut_ja_laskutoimitukset}{Lukualueet}
    \luku{rationaaliluvut}{Murtoluvuilla laskeminen}
    \luku{sieventaminen}{Lausekkeiden sieventäminen}
    \luku{potenssi}{Potenssi}
    \luku{desimaaliluvut}{Luvun desimaaliesitys}
    \luku{yksikot_ja_pyoristaminen}{Yksiköt ja vastaustarkkuus}
    \luku{juuret}{Juuret}
    \luku{murtopotenssi}{Murtopotenssi}
    \luku{reaaliluvut}{Reaaliluvut lukusuoralla}

%\nosa{Tiivistelmä}    
    
\osa{Yhtälöt}
    \luku{yhtalo}{Yhtälö}
    \luku{ensimmainen}{Ensimmäisen asteen yhtälö}
    \luku{verrannollisuus}{Suoraan ja kääntäen verrannollisuus}
    \luku{prosenttilaskenta}{Prosenttilaskenta}
    \luku{potenssiyhtalot}{Potenssiyhtälöt}

\osa{Funktiot}
    \luku{funktio}{Funktio ja sen esitystavat}
	\luku{potenssifunktio}{Potenssifunktio}
	\luku{eksponenttifunktio}{Eksponenttifunktio}
	
%\nosa{Tiivistelmä}

\nosa{Kertaustehtäviä}
    \nluku{TEHT_harjoituskokeet}{Harjoituskokeita}
    \nluku{TEHT_ylioppilaskokeet}{Ylioppilaskoetehtäviä}
	\nluku{TEHT_paasykokeet}{Pääsykoetehtäviä}

\nosa{Lisämateriaalia}
	\nluku{LIITE_lahtotasotesti}{Lähtötasotesti}
	\nluku{LIITE_saannot}{Laskusääntöjen todistuksia}
	\nluku{LIITE_lukujarjestelmat}{Lukujärjestelmät}
	%\nluku{LIITE_lukuteoria}{Lukuteorian alkeet}
	%\nluku{LIITE_joukot}{Joukko-opin alkeet}
	%\nluku{LIITE_aksioomat}{Reaalilukujen aksioomat}
    \nluku{LIITE_sanasto}{Hakemisto ja suomi–ruotsi–englanti-sanasto}
	%\nluku{LIITE_symbolit}{Symbolit}

\Closesolutionfile{ans}
%	\vast
