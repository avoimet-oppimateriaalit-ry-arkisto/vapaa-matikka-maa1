Harjoituskokeissa on kussakin kymmenen tehtävää, jotka on suunniteltu niin, että jokaisesta (alakohtineen) saa korkeintaan kuusi pistettä. %Ensimmäisessä, toisessa ja kolmannessa harjoituskokeessa laskimen käyttö on suotavaa -- neljäs koe on suunniteltu tehtäväksi ilman laskinta.

\subsection*{Harjoituskoe 1}

\begin{tehtava}
 \alakohdat{
	§ Perustele, onko $341$ alkuluku. ($2$\,p.)
	§ Mainitse jokin aina epätosi yhtälö. ($1$\,p.)
	§ Mainitse jokin reaaliluku, joka ei ole rationaaliluku. ($1$\,p.)
	§ Esitä jokin laskutoimitus, jota ei ole määritelty reaaliluvuilla, ja perustele, miksi on niin. ($2$\,p.)
	 }
	 \begin{vastaus}
\alakohdat{
 § Luvun $341$ pystyy kirjoittamaan kahden pienemmän luonnollisen luvun tulona, $341=11\cdot31$, joten kyseessä ei ole alkuluku. (Kyseinen alkulukukehitelmä selviää kokeilemalla rohkeasti jakaa $341$ eri luvuilla.)
 § Esim. $1=2$ tai $x=x+1$
 § Vastaukseksi käy mikä tahansa irrationaaliluku käy vastaukseksi, esim. $\pi$ tai $\sqrt{2}$ .
 § Vastaukseksi käy jokin seuraavista:
 §§ Nollalla jakaminen (esim. $3/0$ tai $0/0$) -- nollalla ei ole olemassa käänteislukua. Käänteisluvun määritelmän perusteella olisi $0\cdot0^{-1}=1$, mutta tämä on ristiriidassa sen kanssa, että mikä tahansa luku kerrottuna nollalla pitäisi olla nolla.
§§ Nollan korottaminen nollanteen potenssiin (esim. $0^0$) -- palautuu nollalla jakamisen ongelmaan, koska potensissääntöjen mukaan pitäisi olla $0^0=0^{1-1}=\frac{0^1}{0^1}=\frac{0}{0}$.
 §§ Parillisen juuren ottaminen negatiivisesta luvusta -- ei ole olemassa reaalilukua, joka kerrottuna itsellään parillista määrää kertoja (ts. korotettuna parilliseen potenssiin) tuottaisi tulona negatiivisen luvun. Nollan kokonaislukupotenssit ovat nollia, ja sekä negatiivinen että positiivinen luku kerrottuna itsellään tuottavat positiivisen luvun, ei milloinkaan negatiivista.
 §§ Nollan korottaminen negatiiviseen murtopotenssiin -- aiheuttaa ongelmia, koska negatiivisesta eksponentista seuraa nollan potenssin siirtyminen nimittäjään, joka taas on ongelmallista nollalla jakamisen vuoksi.
}
    \end{vastaus}
	\end{tehtava}
	
	\begin{tehtava}
Sievennä lausekkeet.
	\alakohdat{
	§ $ \sqrt[3]{3^4}\cdot \sqrt[4]{3^3} $
	§ $\frac{2x-4}{2-x}$
	§ $\frac{5,7\cdot 10^{1\,002} + 3\cdot 10^{1\,001}}{60}$
	 }
	 \begin{vastaus}
\alakohdat{
 § $9\sqrt[12]{3}$
 § $-2$
 § $10^{1\,001}$
}
\end{vastaus}
	\end{tehtava}
	
	\begin{tehtava}
	\alakohdat{
	§ Ratkaise yhtälö $x^3 - \frac{14}{27} = \frac{20}{6}+\frac{14}{18}$. ($3$\,p.)
	§ Esitä luku $0,3888\ldots$ murtolukuna. ($3$\,p.)
	 }
	 \begin{vastaus}
	  \alakohdat{
	   § $x = \frac{5}{3}$
	   § $\frac{7}{18}$
	  }
	 \end{vastaus}
	\end{tehtava}
	
	\begin{tehtava}
	\alakohdat{
	§ Eräs bakteerikanta kasvaa päivässä $75$ prosentilla. Montako prosenttia bakteerikannan alkuperäinen koko on bakteerikannan määrästä kolmen päivän kuluttua? ($3$\,p.)
	§ Mursupuvun hinnasta puolet tulee valmistuskuluista. Valmistuskuluista $75\,$\% on materiaalikuluja. Montako prosenttia materiaalikulujen pitää laskea, jotta mursupuvun hinta tippuu $10$\,\%? ($3$\,p.)
	}
	\begin{vastaus}
	  \alakohdat{
	  § $19\,\%$
	  § $20\,\%$
	  	 }
	\end{vastaus}
	\end{tehtava}
	
	\begin{tehtava}
	\alakohdat{
§ Lounaspalvelu myy lounasta hintaan $8,0$ euroa. Asiakas voi myös ostaa kerralla tarjouspaketin hintaan $170$ euroa, johon sisältyy lounas kuukaudeksi ($30$ päivää). Kuinka monena päivänä kuukaudessa vähintään on käytävä lounaalla, jotta tarjouspaketti tulisi halvemmaksi?
§ Kaupassa myydään kuuden kolajuomatölkin (à $33$\,cl) pakkausta litrahinnalla $2,96$\,€/l. Kanta-asiakkaat voivat ostaa pakkauksen hintaan $4,90$\,€. Kuinka monen prosentin alennuksen kanta-asiakkaat saavat? Anna vastaus prosentin tarkkuudella.
}
	 \begin{vastaus}
	 \alakohdat{
	§ Tarjouspaketti on kannattavampi kuin yksittäisten lounaiden osto, kun lounaita on kuukaudessa vähintään $22$.
	§ $16$\,\%
	}
	 \end{vastaus}
	\end{tehtava}
	
%leipomisverrannollisuus	
	
	\begin{tehtava}
	\alakohdat{
	§ Laske $f(2)$, kun $f(x)=5x^3-12x^{-2}-2$.
	§ Tutki kuvaajasta, mitkä ovat funktion $f(x)=2x^2-3x-2$ nollakohdat ja lisäksi arvo, kun $x=1$. Millä muuttujan arvoilla $f$ saa arvon $3$? Laske $f(7)$. %is dum :(
	 }
	\begin{vastaus}
	 \alakohdat{
	  § $x = 35$
	  § Nollakohdat ovat $x=-0,5$ ja $x = 2$. $f(1) = -3$. Funktio saa arvon $3$, kun $x = -1$ ja $x = 2,5$. $f(7) = 75$
	 }
	\end{vastaus}
	\end{tehtava}
	\begin{center}
		\begin{kuvaajapohja}{1.0}{-3}{5}{-4}{4}
			\kuvaaja{2*x**2-3*x-2}{$f(x)=2x^2-3x-2$}{black}
		\end{kuvaajapohja}
	\end{center}
	
	\begin{tehtava}
Jaana-Petterillä on suorakulmainen pala kartonkia, jonka pinta-ala on $36$\,cm$^2$, lyhyen sivu pituus on $4$\,cm ja pitemmän sivun pituus on $x$\,cm. Hän haluaa taivuttaa kartongista putken niin, että lyhyet sivut ovat yhdessä. Laske putken tilavuus, kun putkelle soveltuva tilavuuden kaava on $V=\pi r^2 h$, jossa $V$ on tilavuus, $h$ on kartonkipalan lyhyen sivun pituus, ja $x=2\pi r$.    
\begin{vastaus}
 $81\pi$
\end{vastaus}
\end{tehtava}

\begin{tehtava}
Kaikki funktiot ovat relaatioita, mutta kaikki relaatiot eivät ole funktioita.
\alakohdat{
§ Anna esimerkki relaatiosta, joka ei ole funktio, ja selitä, miksei se ole funktio. (Muista, että relaation ei tarvitse käsitellä lukuarvoja.)
§ Valitse funktiolle $f,f(x)=|x-1|$ sopiva määrittelyjoukko niin, että $f$ on relaationa symmetrinen.
}

\begin{vastaus}
\alakohdat{
§ Määritellään relaatio esimerkiksi ''$x$ on $y$:n äiti'' siten, että $x$ ja $y$ ovat mielivaltaisia ihmisiä.Kyseessä ei ole tällöin funktio monestakin syystä (yksi riittää): 1) muuttuja $x$ voidaan yhdistää useaan arvoon (eli yhdellä äidillä voi olla monta lasta) -- funktion arvojen pitäisi olla yksikäsitteisiä 2) jokainen funktion määrittelyjoukon alkio pitää voida liittää funktion arvoon -- sen sijaan kaikki ihmiset eivät ole äitejä.
§ Määrittelyjoukoksi sopii ${0,1}$. Tällöin $f(0)=1$ ja toisaalta $f(1)=0$, eli nolla ja yksi kuvautuvat toisikseen.
}	
	\end{vastaus}
\end{tehtava}

\newpage

\subsection*{Harjoituskoe 2}

	\begin{tehtava}
	\alakohdat{
	§ Mainitse jokin ehdollisesti tosi yhtälö ja milloin se on tosi. ($1$\,p.)
	§ Mainitse jokin kokonaisluku, joka ei ole luonnollinen luku. ($1$\,p.)
	§ Perustele, onko $61$ alkuluku. ($2$\,p.)
	§ Näytä esimerkillä, että vähennyslasku ei ole vaihdannainen. ($2$\,p.)
	}
	\begin{vastaus}
	 \alakohdat{
	 § Esim. $x=1$, joka on tosi vain, kun $x=1$
	  § Vastaukseksi käy mikä tahansa negatiivinen kokonaisluku, esimerkiksi $-42$, ja nolla. (Opettaja kertokoon, käyttääkö hän nollaa luonnollisena lukuna vai ei.)
	  § $61$ on alkuluku, sillä se ei ole jaollinen millään sitä pienemmällä luonnollisella luvulla. (Tai: sitä ei voi kirjoittaa kahden luonnollisen luvun tulona, joista kumpikaan ei ole $1$ tai $61$.)
	  § Koska (esimerkiksi) $2-3=-1\neq 1=3-2$, vähennyslasku ei ole vaihdannainen.
	  }
	\end{vastaus}
	\end{tehtava}
	
	\begin{tehtava}
Ratkaise
	\alakohdat{

	§ $\frac{5x}{4}-1=\frac{4}{5}x$
	§ $\sqrt{2}x+\sqrt{2}=2x$.
	}
	\begin{vastaus}
	 \alakohdat{

	  § $\frac{20}{9}$
	  § $-1-\sqrt{2}$
	 }
	\end{vastaus}
	\end{tehtava}
	
	\begin{tehtava}
yhdistä funktion tyyppi kuvaajaan	
\end{tehtava}
	
	\begin{tehtava}
Muuta desimaaliluvuksi
	\alakohdat{
	§ $\frac{10}{3}$
	§ $\frac{1}{4}-\frac{1}{16}$
	§ $\frac{0}{17}$
	§ $\frac{1}{2+\frac{1}{3}}$.
	}
	\begin{vastaus}
	 \alakohdat{
	  § $3,\overline{3}$
	  § $0,1875$
	  § $0$
	  § $0,\overline{428571}$
	 }
	\end{vastaus}
	\end{tehtava}
	
	\begin{tehtava}
Tuoreessa ananaksessa veden osuus on $80$\,\% ananaksen massasta ja A-, B- ja C-vitamiinien yhteenlaskettu osuus $0,05$\,\% massasta. Ananas kuivatetaan niin, että veden osuus laskee $8$ prosenttiin ananaksen massasta. Kuinka suuri on A-, B- ja C-vitamiinien osuus kuivatun ananaksen massasta? (Luvut eivät ole faktuaalisia.)
	\begin{vastaus}
	 Noin $18$\,\%.
	\end{vastaus}
	\end{tehtava}
	
	\begin{tehtava}
Pertsa ajaa kotoansa mummolaan tunnissa, jos ajovauhti on lupsakka $60$\,km/h. Nyt Pertsalla on kuitenkin kiire ja hän yrittää keretä mummolaan kahdessa kolmasosatunnissa. Kuinka nopeasti Pertsan pitää ajaa?
	\begin{vastaus}
	 $90$\,km/h
	\end{vastaus}
	\end{tehtava}
	
	\begin{tehtava} 
Sievennä
	\alakohdat{
%	§  $2 \sqrt{5}-6\sqrt{2}-\sqrt{5}+5\sqrt{2}$
	§ $\frac{a^2 b^2}{a}$, $a \neq 0$
	§ $3(a^2+1)-2(a^2-1)$
	§ $ab(a+2a)$
	§ $(a^3 b^2 c)^2$.
		§ $\frac{\sqrt{a}}{\sqrt{2}}(2a)^{\frac{-5}{2}}$  
	}
		\begin{vastaus}
	 \alakohdat{
%	 §
	  § $a b^2$
	  § $a^2 + 5$
	  § $3a^2 b$
	  § $a^5 + b^4 + c^2$
	  § $\frac{1}{8a^2}$
	 }
	\end{vastaus}
	\end{tehtava}
	
	\begin{tehtava}
	\alakohdat{
§ Olkoon $f(t) = 35 \cdot 2^t$ bakteerien lukumäärä soluviljelmässä ajanhetkellä $t$ (sekuntia). Monenko kokonaisen sekunnin kuluttua bakteereita on yli $1\,000$?
§ yksikkötehtävä
}
	\begin{vastaus}
	\alakohdat{
	 § $6$ sekuntia
	 § yksikkötehtävä
	 }
	\end{vastaus}
		\end{tehtava}
		
	\begin{tehtava}
Määritellään funktio $f$ lausekkeella $f(t) = 2t-3$. Ratkaise
\alakohdat{
§ $f(t-1)+f(t)+f(t+1) = 3$
§ $f(f(t^2))=0$.
}
\begin{vastaus}
\alakohdat{
§ $t = 2$
§ $t=\pm \frac{3}{2}$
}
\end{vastaus}
\end{tehtava}

\newpage

\subsection*{Harjoituskoe 3}

\begin{tehtava}
\alakohdat{
§ Esitä yhdistetyn luvun $124$ alkulukukehitelmä. ($1$\,p.)
§ Osoita esimerkillä, että jakolasku ei ole liitännäinen. ($2$\,p.)
§ Olkoon ihmisten joukossa relaatio ''$a$ on $b$:n sisar'', jota merkitään $a \bowtie b$. Perustele, onko relaatio $\bowtie$
	§§ refleksiivinen
	§§ symmetrinen
	§§ transitiivinen. ($3$\,p.)
}

	\begin{vastaus}
	\alakohdat{
	§ $124=2^2\cdot 31$
	§ Esimerkiksi $(8/4)/2=2/2=1$, mutta toisaalta $8/(4/2)=8/2=4$, joten jakolasku ei ole liitännäinen. (Vastauksesta täytyy selvästi käydä ilmi, että opiskelija ymmärtää, mitä laskutoimituksen liitännäisyys tarkoittaa.)
	§ Yksi piste per alakohta:
		§§ Kukaan ei ole itsensä sisar, joten relaatio ei ole refleksiivinen.
		§§ Sisaruus on aina molemminpuolista (jos $a$ on $b$:n sisar, niin myös $b$ on $a$:n sisar), joten relaatio on symmetrinen.
		§§ Jokainen mielivaltaisen henkilön sisar on myös muiden kyseisen henkilön sisarten sisar (sisaruus ''välittyy eteenpäin''), joten relaatio on transitiivinen.
	}
	\end{vastaus}
\end{tehtava}
%kuvaajatehtäviä!
%taulukoiden analyysiä!

	\begin{tehtava}
Sievennä
	\alakohdat{

	 	§ $-(-x+y)-(-y)$
	 			§ $(x+1)^2$
	 	§ $a-(ab)^2+a^2 b^2$
	 	§ $\frac{ab+ca}{a} \cdot (b-c)$.
	}
	\begin{vastaus}
	 \alakohdat{
	  	§ $x$
	  	§ $x^2+2x+1$
	  	§ $a$
	  	§ $b^2-c^2$
	 }
	\end{vastaus}
	\end{tehtava}
	
%	\begin{tehtava}
%Muuta sekaluvut murtoluvuiksi.
%	 \alakohdat{
%	  § $2\frac{3}{4}$
%	  § $\frac{4}{13}$
%	  § $-7\frac{2}{11}$
%	  § $4$
%	 }
%	 \begin{vastaus}
%	 \alakohdat{
%	  § $\frac{11}{4}$
%	  § $\frac{4}{13}$
%	  § $-\frac{79}{11}$
%	  § $-\frac{4}{1}$
%	 }
%	 \end{vastaus}
%	\end{tehtava}
	
	\begin{tehtava}
Sievennä
	 \alakohdat{
	 § $a^{\frac{1}{3}} \cdot a^{\frac{1}{4}} \cdot a^{\frac{1}{5}} \cdot a^{\frac{13}{60}}$
	 § $(\sqrt[3]{a})^5 \cdot a^\frac{8}{6}$.
	 }
	 \begin{vastaus}
	  \alakohdat{
	   § $a$
	   § $a^3$
	  }
	 \end{vastaus}
	 \end{tehtava}
	
	\begin{tehtava}
Ostat vuoden voimassa olevan parturikortin, jolla voi käydä parturissa niin usein kuin haluaa. Kortti maksaa $230$\,€ ja kertakäynti $25$\,€. Kuinka monta kertaa sinun pitäisi käydä parturissa vuoden aikana, jotta ostos olisi kannattava?
	\begin{vastaus}
	 $10$ kertaa
	\end{vastaus}
	\end{tehtava}
	
	\begin{tehtava}
	\alakohdat{
§ Mansikasta noin $99\,$\% on vettä. Ostat $3,00$ kiloa mansikoita ja laitat ne kuivumaan, kunnes kaksi kolmasosaa mansikoiden vedestä on haihtunut. Kuinka monta kilogrammaa mansikoita sinulla on kuivatuksen jälkeen?
§ Laitat suuren lottovoiton tilille, jolla talletuksen arvo kasvaa $1\,$\% vuodessa. Nostat rahat $20$ vuoden päästä. Kuinka monta prosenttia enemmän rahaa tilillä on silloin?
}
	\begin{vastaus}
		\alakohdat{
	§ $1,02$\,kg
	§ $22$\,\%
	}
	\end{vastaus}
	\end{tehtava}
	
	\begin{tehtava}
Funktion $T$ arvot määritellään yhtälöllä $T(x)=\frac{3}{4x^2-2}$. Määritä funktiolle mahdollisimman laaja reaalinen määrittelyjoukko ja sitä vastaava arvojoukko.
	\begin{vastaus}

Funktio on määritelty, kun lausekkeen nimittäjä on erisuuri kuin nolla. Tutkitaan, millä muuttujan arvoilla nimittäjään tulee nolla: $4x^2-2=0 \leftrightarrow x=\pm \frac{1}{\sqrt{2}}$. Funktion määrittelyjoukko on siis $\mathbb{R}\setminus{\frac{1}{\sqrt{2}}, -\frac{1}{\sqrt{2}}}$ eli reaalilukujen joukko, josta on poistettu $\pm \frac{1}{\sqrt{2}}$.

Arvojoukon selvittäminen aloitetaan ratkaisemalla $x$ yhtälöstä $\frac{3}{4x^2-2}=a$, missä $a$ esittää mahdollisia arvoja, joita funktio voi saada: $\frac{3}{4x^2-2}=a \leftrightarrow x=\pm \sqrt{\frac{\frac{3}{a}+2}{4}}$. Lauseke on hyvin määritelty, kun juurrettavna on epänegatiivinen luku. Lausekkeen $\frac{\frac{3}{a}+2}{4}$ pitää siis olla nolla tai positiivinen. Nolla saadaan yhtälön $\frac{\frac{3}{a}+2}{4}=0$ ratkaisuna: $\frac{3}{a}+2=0 \leftrightarrow a=-\frac{3}{2}$. Tätä suuremmat $a$:n arvot tekevät juurrettavasta positiivisen (jolloin $x$ on hyvin määritelty kyseinen funktion arvo $a$ on oikeasti saavutettavissa). Funktion arvojoukko koostuu siis reaaliluvuista tasan $-\frac{3}{2}$ tai sitä suuremmista luvuista. (MAA2-kurssilla opitaan kirjoittamaan tämä lyhyemmin: $[-\frac{3}{2},\inf[$.)
	%FIXME:kurssiviite
	\end{vastaus}
	\end{tehtava}

\newpage	
	
\subsection*{Harjoituskoe 4}

%esitetään yhtälönratkaisu: selosta, mitä käytettiin eri vaiheissa (osittelulaki, vaihdannaisuus, ...)
\begin{tehtava}
\alakohdat{
	§ Mainitse jokin aina tosi yhtälö. ($1$\,p.)
	§ Miksei reaalilukuja käytettäessä voida ottaa parillista juurta negatiivisesta luvusta? ($2$\,p.)
	§ Järjestä joukot $\mathbb{R}, \mathbb{N}, \mathbb{Q}$ ja $\mathbb{Z}$ järjestykseen pienimmästä suurimpaan ts. siten, että edellinen sisältyy seuraavaan. ($1$\,p.)
	§ Esitä luku $1705,02$ kymmenen potenssien summana. Kirjoita kaikki kertoimet ja eksponentit selvästi näkyviin. ($1$\,p.)
	§ Esitä luvun $81$ alkutekijäkehitelmä. ($1$\,p.)
	
}
	\begin{vastaus}
	\alakohdat{
	§ Esim. $x=x$, $1=1$, $\sqrt{16}=4$, \ldots
	§ Kaikki reaaliluvut ovat joko negatiivisia, positiivisia tai nolla. Nollan potenssit ovat nollia, ja mikä tahansa positiivinen tai negatiivinen luku korotettuna parilliseen potenssiin tuottaa positiivisen luvun. Koska  mistään reaaliluvusta ei saada parilliseen potenssiin korottomalla negatiivista, ei myöskään parillinen juuri ole määritelty negatiivisille luvuille.
	§ $\mathbb{N}, \mathbb{Z}, \mathbb{Q}, \mathbb{R}$
	§ $1\cdot 10^3+7\cdot10^2+0\cdot10^1+5\cdot10^0+0\cdot10^{-1}+2\cdot10^{-2}$
	§ $81=3^4$ (tai $3\cdot3\cdot3\cdot3$)
	}
	\end{vastaus}
\end{tehtava}

\begin{tehtava}
\alakohdat{
§ Esitä $0,\overline{285714}$ murtolukuna.
§ Mikä on funktion $f$ määrittelyjoukko, kun funktion arvot määritellään kaavalla $f(x)=\frac{1}{\sqrt{x}}$?
§ Määritä yhtälön $x^42=42$ kaikki reaaliset ratkaisut. Anna vastaus sadasosien tarkkuudella.
}
\begin{vastaus}
\alakohdat{
	§ $\frac{2}{7}$
	§ $\mathbb{R}_-$ eli negatiivisten reaalilukujen joukko. (Nolla ei kuulu kyseiseen määrittelyjoukkoon.)
	}
	\end{vastaus}
	\end{tehtava}

\begin{tehtava}
Itämeren keskisyvyys on $55$ metriä, pinta-ala $415\,000$\,km$^2$ ja veden keskisuolaisuus alle $1$\,\% (painoprosentteina). Jos keskimääräinen suolapitoisuus olisi $1$\,\%, kuinka monta tonnia suolaa itämeressä olisi yhteensä? Veden tiheydeksi oletetaan $1\,$kg\,dm$^{-3}$.
	\begin{vastaus}
	$228$ miljardia tonnia ($2,28\cdot10^{14}$)
	\end{vastaus}
\end{tehtava}	
	
%	\begin{tehtava}
%Jarskilla on kaksi hattua, kolme paitaa ja kahdet housut. Kuinka monella tavalla Jarski voi pukeutua?
%	 \begin{vastaus}
%	  $2 \cdot 3 \cdot 2 = 12$.
%	 \end{vastaus}
%	 \end{tehtava}
	 
	\begin{tehtava}
Mirjami heittää pallon $10$ metrin korkeuteen. Pallo pomppaa aina $80$\,\% edellisen pompun korkeudesta. Monennenko pompun jälkeen pallo ei enää nouse yli $5$ metrin?
	\begin{vastaus}
	 $4$. pompun jälkeen
	\end{vastaus}
	\end{tehtava}
	
	\begin{tehtava}
Alennusmyynneissä $50$\,\% alennetut sukat maksavat $5$ euroa. Kuinka monta prosenttia alennettua hintaa täytyy korottaa, jotta päästäisiin takaisin alkuperäiseen hintaan?
	\begin{vastaus}
	 $100$\,\%.
	\end{vastaus}
	\end{tehtava}
	
	\begin{tehtava}
	
	Käyt ravintolassa kahden kaverisi kanssa.	
	\end{tehtava}	
	
%	\begin{tehtava}
%Millä $a$:n arvoilla
%	\alakohdat{
%	 § $\sqrt{a+1} = 3$?
%	 § $(a+1)^2 = 1$?
%	 § $\sqrt{a^2} = -a$?
%	 § $\sqrt{a} = -1$?
%	}
%	\begin{vastaus}
%	 \alakohdat{
%	  § $a = 8$
%	  § $a = 0$ ja $a = -2$
%	  § $a \le 0$
%	  § Ei millään $a$:n arvolla
%	 }
%	\end{vastaus}
%	\end{tehtava}
	
	\begin{tehtava}
Laske käyttämällä potenssien laskusääntöjä
	 \alakohdat{
	  § $\frac{7^{10\,002}}{7^{10\,000}}$
	  § $x^2 \cdot 4^{28} \cdot x^{-1} \cdot 0,25^{27}$
	  § $8^{7a + \frac{a}{3}}-2^{21a + a}$.
	 }
	 \begin{vastaus}
	  \alakohdat{
	   § $49$
	   § $4x$
	   § $0$
	  }
	 \end{vastaus}
	\end{tehtava}
	
% tämä, mutta kysytään hintaa: Lounaspalvelu myy etillisten lounaiden lisäksi a tarjouspaketin hintaan $170$ euroa, johon sisältyy lounas kuukaudeksi ($30$ päivää). Miten pitää hinnoitella, jotta asiakas saa taloudellista hyötyöä yksittäisiin lounaisiin verrattuna käydessään 20 päivänä kuussa lounaalla?
	
	\begin{tehtava}
Timanttien arvo on karkeasti suoraan verrannollinen niiden massan neliöön. Kuuluisa timantti \textit{The Great Mogul} painaa noin $800$ karaattia. \textit{The Sancy} -timantti painaa noin $50$ karaattia. Kuinka paljon
	\alakohdat{
	 § suurempi The Great Mogul on verrattuna The Sancyyn?
	 § arvokkaampi The Great Mogul on verrattuna The Sancyyn?
	}
	\begin{vastaus}
	
	\alakohdat{
	 § noin $16$ kertaa suurempi
	 § noin $256$ kertaa arvokkaampi
	}
	\end{vastaus}
	\end{tehtava}
	
	\begin{tehtava}
	 Laboratoriossa viljeltävän bakteeriviljelmän massa kolminkertaistuu päivässä. Kun bakteereja on kasvatettu $5$ päivän ajan, niitä on $243$ grammaa. Kuinka paljon bakteereja
	 \alakohdat{
	  § on $4$ päivän päästä?
	  § oli kasvatuksen alussa?
	 }
	 \begin{vastaus}
	  \alakohdat{
	   § $29,7$\,kg
	   § $1,00$\,g
	  }
	 \end{vastaus}
	\end{tehtava}

\begin{tehtava}
\alakohdat{	
§ Voiko funktio olla relaationa sekä transitiivinen että refleksiivinen? Jos voi, anna esimerkki. Jos ei, perustele.
§ Anna esimerkki reaalifunktion $f$ lausekkeesta, jolle pätee $f(f(x))=x$ kaikilla $x \in \mathbb{R}$.
}
\begin{vastaus}
\alakohdat{
§ Esimerkkifunktio (identiteettifunktio) $f, f(x)=x$ toteuttaa molemmat, koska jokainen luku on yhtä suuri itsensä kanssa.
§ Sopiva funktion lauseke (muuttajana $x$) on esimerkiksi käänteisluoperaatio $\frac{1}{x}$ tai $1-x$.
}
\end{vastaus}
\end{tehtava}
