Seuraavissa harjoituskokeissa on kussakin kymmenen tehtävää, jotka on suunniteltu niin, että jokaisesta (alakohtineen) saa korkeintaan kuusi pistettä. %tulevaisuudessa täydet ratkaisut pisteytysohjeineen
%yhtälöpari+pythagoras?

\subsection*{Harjoituskoe 1}

\begin{tehtava}
 \alakohdat{
	§ Perustele, onko $341$ alkuluku. ($2$\,p.)
	§ Mainitse jokin aina epätosi yhtälö. ($0,5$\,p.)
	§ Mainitse jokin reaaliluku, joka ei ole rationaaliluku. ($0,5$\,p.)
	§ Esitä jokin laskutoimitus, jota ei ole määritelty reaaliluvuilla, ja perustele, miksi on niin. ($2$\,p.)
	§ Kuinka monta eri reaalijuurta on yhtälöllä $x^4=9$? ($1$\,p.)
	 }
	 \begin{vastaus}
\alakohdat{
 § Luvun $341$ pystyy kirjoittamaan kahden pienemmän luonnollisen luvun tulona, $341=11\cdot31$, joten kyseessä ei ole alkuluku. (Kyseinen alkulukukehitelmä selviää kokeilemalla rohkeasti jakaa $341$ eri luvuilla.)
 § Esim. $1=2$ tai $x=x+1$
 § Vastaukseksi käy mikä tahansa irrationaaliluku käy vastaukseksi, esim. $\pi$ tai $\sqrt{2}$ .
 § Vastaukseksi käy jokin seuraavista:
 §§ Nollalla jakaminen (esim. $3/0$ tai $0/0$) -- nollalla ei ole olemassa käänteislukua. Käänteisluvun määritelmän perusteella olisi $0\cdot0^{-1}=1$, mutta tämä on ristiriidassa sen kanssa, että mikä tahansa luku kerrottuna nollalla pitäisi olla nolla.
§§ Nollan korottaminen nollanteen potenssiin (esim. $0^0$) -- palautuu nollalla jakamisen ongelmaan, koska potensissääntöjen mukaan pitäisi olla $0^0=0^{1-1}=\frac{0^1}{0^1}=\frac{0}{0}$.
 §§ Parillisen juuren ottaminen negatiivisesta luvusta -- ei ole olemassa reaalilukua, joka kerrottuna itsellään parillista määrää kertoja (ts. korotettuna parilliseen potenssiin) tuottaisi tulona negatiivisen luvun. Nollan kokonaislukupotenssit ovat nollia, ja sekä negatiivinen että positiivinen luku kerrottuna itsellään tuottavat positiivisen luvun, ei milloinkaan negatiivista.
 §§ Nollan korottaminen negatiiviseen murtopotenssiin -- aiheuttaa ongelmia, koska negatiivisesta eksponentista seuraa nollan potenssin siirtyminen nimittäjään, joka taas on ongelmallista nollalla jakamisen vuoksi.
 § Kaksi ratkaisua
}
    \end{vastaus}
	\end{tehtava}
	
	\begin{tehtava}
Sievennä lausekkeet.
	\alakohdat{
	§ $ \sqrt[3]{3^4}\cdot \sqrt[4]{3^3} $
	§ $\frac{2x-4}{2-x}$
	  § $8^{7a + \frac{a}{3}}-2^{21a + a}$	
	 }
	 \begin{vastaus}
\alakohdat{
 § $9\sqrt[12]{3}$
 § $-2$
 § $0$
}
\end{vastaus}
	\end{tehtava}
	
	\begin{tehtava}
	\alakohdat{
	§ Ratkaise yhtälö $x^3 - \frac{14}{27} = \frac{20}{6}+\frac{14}{18}$. ($3$\,p.)
	§ Esitä luku $0,3888\ldots$ murtolukuna. ($3$\,p.)
	 }
	 \begin{vastaus}
	  \alakohdat{
	   § $x = \frac{5}{3}$
	   § $\frac{7}{18}$
	  }
	 \end{vastaus}
	\end{tehtava}
	
	\begin{tehtava}
	\alakohdat{
	§ Eräs bakteerikanta kasvaa päivässä $75$ prosentilla. Montako prosenttia bakteerikannan alkuperäinen koko on bakteerikannan määrästä kolmen päivän kuluttua? ($3$\,p.)
	§ Mursupuvun hinnasta puolet tulee valmistuskuluista. Valmistuskuluista $75\,$\% on materiaalikuluja. Montako prosenttia materiaalikulujen pitää laskea, jotta mursupuvun hinta tippuu $10$\,\%? ($3$\,p.)
	}
	\begin{vastaus}
	  \alakohdat{
	  § $19\,\%$
	  § $20\,\%$
	  	 }
	\end{vastaus}
	\end{tehtava}
	
	\begin{tehtava}
	\alakohdat{
§ Lounaspalvelu myy lounasta hintaan $8,00$ euroa. Asiakas voi myös ostaa kerralla tarjouspaketin hintaan $170$ euroa, johon sisältyy lounas kuukaudeksi ($30$ päivää). Kuinka monena päivänä kuukaudessa vähintään on käytävä lounaalla, jotta tarjouspaketti tulisi halvemmaksi?
§ Kaupassa myydään kuuden kolajuomatölkin (à $33$\,cl) pakkausta litrahinnalla $2,96$\,€/l. Kanta-asiakkaat voivat ostaa pakkauksen hintaan $4,90$\,€. Kuinka monen prosentin alennuksen kanta-asiakkaat saavat? Anna vastaus prosentin tarkkuudella.
}
	 \begin{vastaus}
	 \alakohdat{
	§ Tarjouspaketti on kannattavampi kuin yksittäisten lounaiden osto, kun lounaita on kuukaudessa vähintään $22$.
	§ $16$\,\%
	}
	 \end{vastaus}
	\end{tehtava}
	
\begin{tehtava}
\alakohdat{
§ Pullaohjeen mukaan $4$\,dl maitoa ja $2$\,dl sokeria riittää $20$ pullaan. Jon leipoo niin monta pullaa kuin litralla maitoa on mahdollista. Riittääkö leipomiseen tällöin täysi kilogramman paketti sokeria? Jos riittää, niin kuinka monta grammaa sokeripakkauksesta jää tällöin jäljelle? Tavallisen pöytäsokerin eli sakkaroosin tiheys on $1,59$\,g\,cm$^{-3}$.
§ Bordellin kymmenen miesprostituoitua hoitaa keskimäärin kaksikymmentä asiakasta kolmessa tunnissa. Kuinka monta asiakasta kolmekymmentä prostituoitua ehtii hoitaamaan tunnissa?
}

	\begin{vastaus}
	\alakohdat{
§ Maidosta riittää $2,5$-kertaiseen annokseen ($50$ pullaa), eli myös sokeria tarvitaan $2,5$-kertainen määrä: $2\cdot2,5$\,dl$=5$\,dl. Tiheys $1,59$\,g\,cm$^{-3}$ ilmaistuna desilitrojen avulla on $159$\,g/dl. Tarvittava sokerin määrä on siis $5$\,dl$\cdot159$\,g/dl$=795$\,g. Kilogramman pakkauksesta jää siis jäljelle $1000$\,g$-795$\,g$=205$\,g.
§ Kaksikymmentä asiakasta
}
	\end{vastaus}

\end{tehtava}
	
	\begin{tehtava}
	\alakohdat{
	§ Laske $f(2)$, kun $f(x)=5x^3-12x^{-2}-2$.
	§ Tutki kuvaajasta, mitkä ovat funktion $f$ nollakohdat. Millä muuttujan arvoilla funktio saa arvon $3$? Arvioi funktion pienin arvo.
	 }
	\begin{vastaus}
	 \alakohdat{
	  § $x = 35$
	  § Nollakohdat ovat $x=-0,5$ ja $x = 2$. Funktio saa arvon $3$, kun $x = -1$ ja $x = 2,5$. Funktion pienin arvo on kuvaajan perusteella noin $-3$.
	 }
	\end{vastaus}
	\end{tehtava}
	\begin{center}
		\begin{kuvaajapohja}{1.0}{-3}{5}{-4}{4}
			\kuvaaja{2*x**2-3*x-2}{$f(x)=2x^2-3x-2$}{black}
		\end{kuvaajapohja}
	\end{center}
	
	\begin{tehtava}
Jaana-Petterillä on suorakulmainen pala kartonkia, jonka pinta-ala on $36$\,cm$^2$, lyhyen sivu pituus on $4$\,cm ja pitemmän sivun pituus on $x$\,cm. Hän haluaa taivuttaa kartongista putken niin, että lyhyet sivut ovat yhdessä. Laske putken tilavuus, kun putkelle soveltuva tilavuuden kaava on $V=\pi r^2 h$, jossa $V$ on tilavuus, $h$ on kartonkipalan lyhyen sivun pituus, ja $x=2\pi r$.    
\begin{vastaus}
 $81\pi$
\end{vastaus}
\end{tehtava}

\begin{tehtava}
Kaikki funktiot ovat relaatioita, mutta kaikki relaatiot eivät ole funktioita.
\alakohdat{
§ Anna esimerkki relaatiosta, joka ei ole funktio, ja selitä, miksei se ole funktio. (Muista, että relaation ei tarvitse käsitellä lukuarvoja.)
§ Valitse funktiolle $f,f(x)=|x-1|$ sopiva määrittelyjoukko niin, että $f$ on relaationa symmetrinen.
}

\begin{vastaus}
\alakohdat{
§ Määritellään relaatio esimerkiksi ''$x$ on $y$:n äiti'' siten, että $x$ ja $y$ ovat mielivaltaisia ihmisiä.Kyseessä ei ole tällöin funktio monestakin syystä (yksi riittää): 1) muuttuja $x$ voidaan yhdistää useaan arvoon (eli yhdellä äidillä voi olla monta lasta) -- funktion arvojen sen sijaan pitäisi olla yksikäsitteisiä 2) jokainen funktion määrittelyjoukon alkio pitää voida liittää funktion arvoon -- sen sijaan kaikki ihmiset eivät ole äitejä.
§ Määrittelyjoukoksi sopii ${0,1}$. Tällöin $f(0)=1$ ja toisaalta $f(1)=0$, eli nolla ja yksi kuvautuvat toisikseen.
}	
	\end{vastaus}
\end{tehtava}

\newpage

\subsection*{Harjoituskoe 2}

	\begin{tehtava}
	\alakohdat{
	§ Mainitse jokin ehdollisesti tosi yhtälö ja milloin se on tosi. ($1$\,p.)
	§ Mainitse jokin kokonaisluku, joka ei ole luonnollinen luku. ($0,5$\,p.)
	§ Mainitse jokin reaaliluku, joka ei ole irrationaaliluku. ($0,5$\,p.)
	§ Perustele, onko $61$ alkuluku. ($2$\,p.)
	§ Näytä esimerkillä, että vähennyslasku ei ole vaihdannainen. ($2$\,p.)
	}
	\begin{vastaus}
	 \alakohdat{
	 § Esim. $x=1$, joka on tosi vain, kun $x=1$
	  § Vastaukseksi käy mikä tahansa negatiivinen kokonaisluku, esimerkiksi $-42$, ja nolla. (Opettaja kertokoon, käyttääkö hän nollaa luonnollisena lukuna vai ei.)
	  § Vastaukseksi käyvät kaikki reaaliset ei-irrationaaliluvut, esimerkiksi $-3$, $0$ tai $\frac{99}{98}.$ (Ei siis $\pi$, $\sqrt{2}$ jne.)
	  § $61$ on alkuluku, sillä se ei ole jaollinen millään sitä pienemmällä luonnollisella luvulla. (Tai: sitä ei voi kirjoittaa kahden luonnollisen luvun tulona, joista kumpikaan ei ole $1$ tai $61$.)
	  § Koska (esimerkiksi) $2-3=-1\neq 1=3-2$, vähennyslasku ei ole vaihdannainen.
	  }
	\end{vastaus}
	\end{tehtava}
	
	\begin{tehtava}
Ellulla meni filosofian toinen kurssi ensimmäistä huonommin, ja nyt filosofian keskiarvo (ei-pyöristettynä) kahden suoritetun kurssin jälkeen on $7,5$.
	\alakohdat{
	§ Minkä päättöarvosanan Ellu saa filosofiasta (keskiarvo pyöristettynä), kun hän saa viimeisestä eli kolmannesta kurssista ansaitsemansa kympin? (Oletetaan, ettei Ellu korota päättöarvosanaa mahdollisessa erilliskokeessa.)
	§ Taulukoi kaikki mahdolliset kurssiarvosanat, jotka Ellu on voinut saada kahdesta ensimmäisestä kurssista.
	§ Ellu opiskelee myös kolme kurssia terveystietoa. Hän on on saanut ensimmäisestä kurssista arvosanaksi $7$ ja toisesta $8$, ja hänen pyöristetty kurssikeskiarvonsa on $8$. Minkä arvosanan Ellu sai kolmannesta kurssista?
	}
	\begin{vastaus}
	 \alakohdat{
	 § $\frac{2\cdot 7,5+10}{3}\approx8$ ($\frac{7,5+10}{2}$ on väärin!) 
	§
	\begin{tabular}{|c|c|}
	\hline 
	FI$1$ & FI$2$ \\ 
	\hline 
	$8$ & $7$ \\ 
	\hline 
	$9$ & $6$ \\ 
	\hline 
	$10$ & $5$ \\ 
	\hline 
	\end{tabular} 
	§ $8$, $9$ tai $10$
	 }
	\end{vastaus}
	\end{tehtava}
	
%	\begin{tehtava}
%yhdistä funktion tyyppi kuvaajaan	
%\end{tehtava}
	
	\begin{tehtava}
	\alakohdat{
	§ Muuta desimaaliluku $-5,\overline{20}$ sekamurtoluvuksi.
	§ Ratkaise $z$ yhtälöstä $-\sqrt{a}z-\sqrt{a}=-az$.
	}
	\begin{vastaus}
	 \alakohdat{
	  § $-5\,\frac{2}{9}$
	  § $z=1+\sqrt{a}$
	 }
	\end{vastaus}
	\end{tehtava}
	
	\begin{tehtava}
Tuoreessa ananaksessa veden osuus on $80$\,\% ananaksen massasta ja A-, B- ja C-vitamiinien yhteenlaskettu osuus $0,05$\,\% massasta. Ananas kuivatetaan niin, että veden osuus laskee $8$ prosenttiin ananaksen massasta. Kuinka suuri on A-, B- ja C-vitamiinien osuus kuivatun ananaksen massasta? (Luvut eivät ole faktuaalisia.)
	\begin{vastaus}
	 Noin $18$\,\%.
	\end{vastaus}
	\end{tehtava}
	
	\begin{tehtava}
	\alakohdat{
	§ Pertsa ajaa kotoansa mummolaan tunnissa, jos ajovauhti on lupsakka $60$\,km/h. Nyt Pertsalla on kuitenkin kiire, ja hän yrittää keretä mummolaan kahdessa kolmasosatunnissa. Kuinka nopeasti Pertsan pitää ajaa?
	§ Pertsan ajaessa nopeudella $60$\,km/h juuri ennen jarrutusta, jarrutusmatka oli $100$ metriä. Kuinka monta metriä pitkä pysähtymismatka on samoissa ajo-olosuhteissa nopeudella $100$\,km/h, kun otetaan lisäksi huomioon Pertsan yhden sekunnin reaktioaika? Auton jarrutusmatka on suoraan verrannollinen nopeuden neliöön, ja pysähtymisaika on reaktioajan ja jarrutusajan summa.
	}
	\begin{vastaus}
	\alakohdat{
	 § $90$\,km/h
	 § $306$ metriä
	 }
	\end{vastaus}
	\end{tehtava}
	
	\begin{tehtava} 
Sievennä.
	\alakohdat{ %murtolukuja ja potensseja
		§ $3(a^2+1)-2(a^2-1)$
			§ $\frac{x^2+1}{x}-x$
	§ $\frac{1}{a}:(\frac{1}{a}-1)$
§ $\frac{\sqrt{a}}{\sqrt{2}}(2a)^{\frac{-5}{2}}$  
	}
		\begin{vastaus}
	 \alakohdat{
	 	  § $a^2 + 5$
	  § $x$
	  § $\frac{1}{1-a}$
	  § $\frac{1}{8a^2}$
	 }
	\end{vastaus}
	\end{tehtava}

	\begin{tehtava}
	\alakohdat{
§ Olkoon $f(t) = 35 \cdot 2^t$ bakteerien lukumäärä soluviljelmässä ajanhetkellä $t$ (sekuntia). Monenko kokonaisen sekunnin kuluttua bakteereita on yli $1\,000$?
§ Aape Ronkalla on kuution muotoisia kananmunia, joiden tiheys on $4,0$\,g\,cm$^{-3}$. Jos $20$ munan yhteismassa on yksi kilogramma, kuinka monta senttimetriä pitkä on kunkin munan särmä? Kaikki munat ovat saman kokoisia. Kuution tilavuus lasketaan kaavalla $V=a^3$, missä $a$ on kuution särmän pituus.
}
	\begin{vastaus}
	\alakohdat{
	 § $6$ sekuntia
	 § $2,3$ senttimetriä
	 }
	\end{vastaus}
		\end{tehtava}
		
	\begin{tehtava}
Määritellään funktio $f$ lausekkeella $f(t) = 2t-3$. Ratkaise
\alakohdat{
§ $f(t-1)+f(t)+f(t+1) = 3$
§ $f(f(t^2))=0$.
}
\begin{vastaus}
\alakohdat{
§ $t = 2$
§ $t=\pm \frac{3}{2}$
}
\end{vastaus}
\end{tehtava}

\newpage

\subsection*{Harjoituskoe 3}

\begin{tehtava}
\alakohdat{
§ Esitä yhdistetyn luvun $124$ alkulukukehitelmä. ($1$\,p.)
§ Osoita esimerkillä, että jakolasku ei ole liitännäinen. ($2$\,p.)
§ Olkoon ihmisten joukossa relaatio ''$a$ on $b$:n sisar'', jota merkitään $a \bowtie b$. Perustele, onko relaatio $\bowtie$
	§§ refleksiivinen
	§§ symmetrinen
	§§ transitiivinen. ($3$\,p.)
}

	\begin{vastaus}
	\alakohdat{
	§ $124=2^2\cdot 31$
	§ Esimerkiksi $(8/4)/2=2/2=1$, mutta toisaalta $8/(4/2)=8/2=4$, joten jakolasku ei ole liitännäinen. (Vastauksesta täytyy selvästi käydä ilmi, että opiskelija ymmärtää, mitä laskutoimituksen liitännäisyys tarkoittaa.)
	§ Yksi piste per alakohta:
		§§ Kukaan ei ole itsensä sisar, joten relaatio ei ole refleksiivinen.
		§§ Sisaruus on aina molemminpuolista (jos $a$ on $b$:n sisar, niin myös $b$ on $a$:n sisar), joten relaatio on symmetrinen.
		§§ Jokainen mielivaltaisen henkilön sisar on myös muiden kyseisen henkilön sisarten sisar (sisaruus ''välittyy eteenpäin''), joten relaatio on transitiivinen.
	}
	\end{vastaus}
\end{tehtava}
%kuvaajatehtäviä!
%taulukoiden analyysiä!

	\begin{tehtava}
Avaa mahdolliset sulut ja sievennä.
	\alakohdat{
	 	§ $-(-x+y)-(-y)$
	 	§ $(x+1)^2$
	 		 § $a^{\frac{1}{3}} \cdot a^{\frac{1}{4}} \cdot a^{\frac{1}{5}} \cdot a^{\frac{13}{60}}$
	}
	\begin{vastaus}
	 \alakohdat{
	  	§ $x$
	  	§ $x^2+2x+1$
	  		   § $a$
	 }
	\end{vastaus}
	\end{tehtava}
		

%yksikkötehtävä
%verrrannollisuustehtävä	
	
	\begin{tehtava}
Ostat vuoden voimassa olevan parturikortin, jolla voi käydä parturissa niin usein kuin haluaa. Kortti maksaa $230$\,€ ja kertakäynti $25$\,€. Kuinka monta kertaa sinun pitäisi käydä parturissa vuoden aikana, jotta ostos olisi kannattava?
	\begin{vastaus}
	 $10$ kertaa
	\end{vastaus}
	\end{tehtava}
	
		\begin{tehtava}
Mirjami heittää pallon $10$ metrin korkeuteen. Pallo pomppaa aina $80$\,\% edellisen pompun korkeudesta. Kuinka korkealle pallo nousee toisen pompuun jälkeen? Monennenko pompun jälkeen pallo ei enää nouse yli $5$ metrin?
	\begin{vastaus}
	 $6,4$ metriä; $4$. pompun jälkeen
	\end{vastaus}
	\end{tehtava}
	
	
	\begin{tehtava}
	\alakohdat{
§ Mansikasta noin $99\,$\% on vettä. Ostat $3,00$ kiloa mansikoita ja laitat ne kuivumaan, kunnes kaksi kolmasosaa mansikoiden vedestä on haihtunut. Kuinka monta kilogrammaa mansikoita sinulla on kuivatuksen jälkeen?
§ Laitat suuren lottovoiton tilille, jolla talletus kasvaa $1,5\,$\% vuodessa. Nostat rahat $20$ vuoden päästä. Kuinka monta prosenttia arvokkaampi talletus tällöin on, kun inflaatio on pienentänyt joka vuosi rahan arvoa yhdellä prosentilla?
}
	\begin{vastaus}
		\alakohdat{
	§ $1,02$\,kg
	§ $10,2$\,\%
	}
	\end{vastaus}
	\end{tehtava}
	
\begin{tehtava}
Kuutiosenttimetrissä ilmaa on keskimäärin $100$ bakteeria. Bakteerisolujen oletetaan olevan pallon muotoisia (säde yksi mikrometri) ja vastaavan tiheydeltään vettä. Puhtaan ilman tiheys (ilman bakteereita) on $1,275$\,kg\,m$^{-3}$. Kuinka monta miljardisosaa (yksikössä ppb, parts per billion) ilman massasta on bakteereja? Pallon tilavuus lasketaan kaavalla $V=\frac{4}{3}\pi r^2$, missä $r$ on pallon säde eli puolet halkaisijasta/läpimitasta.
	\begin{vastaus}
	$330$\,ppb %FIXME: tarkista vastaustarkkuus
	\end{vastaus}
\end{tehtava}	
	
	\begin{tehtava}
Funktion $T$ arvot määritellään yhtälöllä $T(x)=\frac{3}{4x^3-2}$. Määritä funktiolle mahdollisimman laaja reaalinen määrittelyjoukko ja sitä vastaava arvojoukko.
	\begin{vastaus}
Funktio on määritelty, kun lausekkeen nimittäjä on erisuuri kuin nolla. Tutkitaan, millä muuttujan arvoilla nimittäjään tulee nolla: $4x^3-2=0 \leftrightarrow x=\frac{1}{\sqrt[3]{2}}$. Funktion määrittelyjoukko on siis $\mathbb{R}\setminus\lbrace\frac{1}{\sqrt{3}} \rbrace$ eli reaalilukujen joukko, josta on poistettu $\frac{1}{\sqrt[3]{2}}$.

Arvojoukon selvittäminen aloitetaan ratkaisemalla $x$ yhtälöstä $\frac{3}{4x^3-2}=a$, missä $a$ esittää mahdollisia arvoja, joita funktio voi saada: $\frac{3}{4x^3-2}=a \leftrightarrow x= \sqrt[3]{\frac{\frac{3}{a}+2}{4}}$. Lauseke on hyvin määritelty, kun nollalla ei jaeta -- $a$ ei siis voi olla nolla. (Kuutiojuuren sisällä voi olla mikä tahansa reaaliluku, joten se ei rajoita mitään.) Funktion arvojoukko koostuu siis kaikista nollasta poikkeavista reaaliluvuista: $\mathbb{R}\setminus \lbrace 0 \rbrace$.
	\end{vastaus}
	\end{tehtava}

\newpage	
	
\subsection*{Harjoituskoe 4}

%esitetään yhtälönratkaisu: selosta, mitä käytettiin eri vaiheissa (osittelulaki, vaihdannaisuus, ...)
\begin{tehtava}
\alakohdat{
	§ Mainitse jokin aina tosi yhtälö. ($0,5$\,p.)
	§ Sievennä $x+2x$. ($0,5$\,p.)
	§ Miksei reaalilukuja käytettäessä voida ottaa parillista juurta negatiivisesta luvusta? ($2$\,p.)
	§ Järjestä joukot $\mathbb{R}, \mathbb{N}, \mathbb{Q}$ ja $\mathbb{Z}$ järjestykseen pienimmästä suurimpaan ts. siten, että edellinen sisältyy seuraavaan. ($1$\,p.)
	§ Esitä luku $1\,705,02$ kymmenen potenssien summana. Kirjoita kaikki kertoimet ja eksponentit selvästi näkyviin. ($1$\,p.)
	§ Esitä luvun $81$ alkutekijäkehitelmä. ($1$\,p.)
	
}
	\begin{vastaus}
	\alakohdat{
	§ Esim. $x=x$, $1=1$, $\sqrt{16}=4$, \ldots
	§ $3x$
	§ Kaikki reaaliluvut ovat joko negatiivisia, positiivisia tai nolla. Nollan potenssit ovat nollia, ja mikä tahansa positiivinen tai negatiivinen luku korotettuna parilliseen potenssiin tuottaa positiivisen luvun. Koska  mistään reaaliluvusta ei saada parilliseen potenssiin korottomalla negatiivista, ei myöskään parillinen juuri ole määritelty negatiivisille luvuille.
	§ $\mathbb{N}, \mathbb{Z}, \mathbb{Q}, \mathbb{R}$
	§ $1\cdot 10^3+7\cdot10^2+0\cdot10^1+5\cdot10^0+0\cdot10^{-1}+2\cdot10^{-2}$
	§ $81=3^4$ (tai $3\cdot3\cdot3\cdot3$)
	}
	\end{vastaus}
\end{tehtava}

\begin{tehtava}
\alakohdat{
§ Esitä $0,\overline{285714}$ murtolukuna.
§ Mikä on funktion $f$ määrittelyjoukko, kun funktion arvot määritellään kaavalla $f(x)=\frac{1}{\sqrt{x}}$?
§ Määritä yhtälön $x^{42}=42$ kaikki reaaliset ratkaisut. Anna vastaus sadasosien tarkkuudella.
}
\begin{vastaus}
\alakohdat{
	§ $\frac{2}{7}$
	§ $\mathbb{R}_+$ eli positiivisten reaalilukujen joukko. (Nolla ei kuulu kyseiseen määrittelyjoukkoon.)
	§ $1,09$ ja $-1,09$
	}
	\end{vastaus}
	\end{tehtava}

	\begin{tehtava}
Sievennä.
	 \alakohdat{
	 § $\frac{x-y}{y-x}$
	§ $\frac{5,7\cdot 10^{1\,002} + 3\cdot 10^{1\,001}}{60}$
	  § $x^2 \cdot 4^{28} \cdot x^{-1} \cdot 0,25^{27}$

	 }
	 \begin{vastaus}
	  \alakohdat{
	  § $-1$
 § $10^{1\,001}$
	   § $4x$
	  }
	 \end{vastaus}
	\end{tehtava}

\begin{tehtava}
Itämeren keskisyvyys on $55$ metriä ja pinta-ala $415\,000$\,km$^2$. Jos veden keskisuolaisuus olisi $1$\,\% (painoprosentteina eli massaa suolaa per massa suolavettä), kuinka monta tonnia suolaa itämeressä olisi yhteensä? Veden tiheydeksi oletetaan $1\,$kg\,dm$^{-3}$.
	\begin{vastaus}
	$228$ miljardia tonnia ($2,28\cdot10^{14}$)
	\end{vastaus}
\end{tehtava}	

	\begin{tehtava}
	\alakohdat{
§ Alennusmyynneissä $50$\,\% alennetut sukat maksavat $5$ euroa. Kuinka monta prosenttia alennettua hintaa täytyy korottaa, jotta päästäisiin takaisin alkuperäiseen hintaan?
%§ Käyt kahden ystäväsi kanssa ravintolassa.<>
}
	\begin{vastaus}
	\alakohdat{
	§ $100$\,\%.
%	§
	}
	\end{vastaus}
	\end{tehtava}
	
\begin{tehtava}
Kuntosaliyritys myy palveluitaan kahdella maksusuunnitelmalla: asiakas ostaa joko yksittäisiä käyntejä tai kuukausikortin. Yksittäinen käynti maksaa $7$ euroa. Kuinka paljon kuukausikortin hinnaksi tulee, kun yritys suunnittelee sen olevan $15$ päivänä kuukaudessa ($30$ päivää) kuntoilevalle asiakkaalle $20$ prosenttia halvempi kuin vastaavien käyntien ostaminen erikseen? Kuinka monta kertaa kuukaudessa salilla tulee tällöin käydä, jotta kuukausikortti olisi edullisempaa kuin yksittäiset maksut? (Oletetaan asiakkaan käyvän kuntosalilla korkeintaan yhden kerran vuorokaudessa.)
	\begin{vastaus}
Kuukausikortin hinnaksi tulee $84,00$ euroa. Kuukausikortti on edullisempi, jos käy vähintään $13$ kertaa viikossa ($12$ käynnillä hinta on sama.)
	\end{vastaus}	
\end{tehtava}	
	
	\begin{tehtava}
Timanttien arvo on karkeasti suoraan verrannollinen niiden massan neliöön. Kuuluisa timantti \textit{The Great Mogul} painaa noin $800$ karaattia. \textit{The Sancy} -timantti painaa noin $50$ karaattia. Kuinka paljon
	\alakohdat{
	 § suurempi The Great Mogul on verrattuna The Sancyyn?
	 § arvokkaampi The Great Mogul on verrattuna The Sancyyn?
	}
	\begin{vastaus}
	\alakohdat{
	 § noin $16$ kertaa suurempi
	 § noin $256$ kertaa arvokkaampi
	}
	\end{vastaus}
	\end{tehtava}
	
	\begin{tehtava}
	 Laboratoriossa viljeltävän bakteeriviljelmän massa kolminkertaistuu päivässä. Kun bakteereja on kasvatettu $5$ päivän ajan, niitä on $243$ grammaa. Kuinka paljon bakteereja
	 \alakohdat{
	  § on $4$ päivän päästä?
	  § oli kasvatuksen alussa?
	 }
	 \begin{vastaus}
	  \alakohdat{
	   § $29,7$\,kg
	   § $1,00$\,g
	  }
	 \end{vastaus}
	\end{tehtava}

\begin{tehtava}
Tuotteen hintaa laskettiin ensin $p$\,\%, minkä jälkeen tuotteen hintaa korotettiin samat $p$\,\%. Kokonaisuudessaan tuotteen hinta laski $15$\,\%. Määritä $p$:n lukuarvo kahden desimaalin tarkkuudella.
	\begin{vastaus}
	Merkitään tuotteen alkuperäistä hintaa (esim.) $a$:lla. (Hinta on lisäksi erisuuri kuin nolla.) Ratkaistava yhtälö on prosentteina $(1-\frac{p}{100})(1+\frac{p}{100})a=(1-\frac{15}{100})a$ tai joidenkin mielestä helpommassa desimaalimuodossa (sijoituksella $p/100=x$) $(1-x)(1+x)a=(1-\frac{15}{100})a$. Tästä $a$ katoaa jakolaskulla, ja oikean puolen vähennyslasku voidaan sieventää: $(1-x)(1+x)=\frac{85}{100}$. Yhtälön vasemman puolen tulon voi kirjoittaa uudelleen avaamalla sulut osittelulain avulla, jolloin päädytään potenssiyhtälöön $1-x^2=\frac{85}{100}$. Tästä edelleen $x^2=\frac{15}{100}$, josta $x=\pm \sqrt{\frac{15}{100}}=\pm \frac{\sqrt{15}}{10}$. Aiemmin tehdyn sijoituksen mukaan $\frac{p}{100}=x$, joten $p$ voidaan ratkaista yhtälöstä $\frac{p}{100}=\pm \frac{\sqrt{15}}{10}$ kertomalla se puolittain nimittäjällä $100$. Saadun yhtälön $p=\pm 10\sqrt{15}$ kahdesta ratkaisusta hyväksytään vain positiivinen vastaus, koska prosentuaalisen muutoksen ''suunta'' on jo otettu huomioon alkuperäisessä yhtälössä yhteen- ja vähennyslaskujen avulla. Lopullinen vastaus on siis $p=10\sqrt{15}\approx 38,73$.
	\end{vastaus}
\end{tehtava}

\begin{tehtava}
\alakohdat{	
§ Voiko funktio olla relaationa sekä transitiivinen että refleksiivinen? Jos voi, anna esimerkki. Jos ei, perustele.
§ Anna esimerkki reaalifunktion $f$ lausekkeesta, jolle pätee $f(f(x))=x$ kaikilla $x \in \mathbb{R}$.
}
\begin{vastaus}
\alakohdat{
§ Esimerkkifunktio (identiteettifunktio) $f, f(x)=x$ toteuttaa molemmat, koska jokainen luku on yhtä suuri itsensä kanssa.
§ Sopivia funktion lausekkeita (muuttajana $x$) ovat esimerkiksi $\frac{1}{x}$ tai $1-x$.
}
\end{vastaus}
\end{tehtava}
