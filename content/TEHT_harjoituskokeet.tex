\subsection*{Harjoituskoe 1}

\begin{tehtava}
Mainitse jokin
 \begin{alakohdat}
	\alakohta{aina tosi yhtälö.}
	\alakohta{joskus tosi yhtälö (ja milloin se on tosi).}
	\alakohta{aina epätosi yhtälö.}
	 \end{alakohdat}
	 \begin{vastaus}
\begin{alakohdat}
 \alakohta{esim. $x=x$}
 \alakohta{esim. $x=1$, tosi vain kun $x=1$}
 \alakohta{esim. $1=2$}
\end{alakohdat}
    \end{vastaus}
	\end{tehtava}
	
	\begin{tehtava}
Sievennä
	\begin{alakohdat}
	\alakohta{$\sqrt{144}$}
	\alakohta{$64^\frac{3}{2}$}
	\alakohta{$ \sqrt[3]{3^4}\cdot \sqrt[4]{3^3} $}
	\alakohta{$\frac{\sqrt{a}}{\sqrt{2}}(2a)^{\frac{-5}{2}}$.}
	 \end{alakohdat}
	 \begin{vastaus}
\begin{alakohdat}
 \alakohta{$12$}
 \alakohta{$512$}
 \alakohta{$9\sqrt[12]{3}$}
 \alakohta{$\frac{1}{8a^2}$}
\end{alakohdat}
\end{vastaus}
	\end{tehtava}
	\begin{tehtava}
Muuta murtoluvuksi
	\begin{alakohdat}
	\alakohta{$0,45$}
	\alakohta{$0,333\ldots$}
	\alakohta{$0,\overline{285714}$}
	\alakohta{$0,3\overline{8}$.}
	 \end{alakohdat}
	 \begin{vastaus}
	  \begin{alakohdat}
	   \alakohta{$\frac{9}{20}$}
	   \alakohta{$\frac{1}{3}$}
	   \alakohta{$\frac{2}{7}$}
	   \alakohta{$\frac{7}{18}$}
	  \end{alakohdat}
	 \end{vastaus}
	\end{tehtava}
	\begin{tehtava}
	Eräs bakteerikanta kasvaa päivässä $75$ prosentilla. Montako prosenttia bakteerikannan alkuarvo $1$ g on bakteerikannan määrästä kolmen päivän kuluttua?
	\begin{vastaus}
	  $536 \% $
	\end{vastaus}
	\end{tehtava}
		\begin{tehtava}
	Mursupuvun hinnasta puolet tulee valmistuskuluista. Valmistuskuluista $75 \%$ on materiaalikuluja. Montako prosenttia materaalikulujen pitää laskea, jotta mursupuvun hinta tippuu $10$ \%?
	\begin{vastaus}
	  $20 \%$
	\end{vastaus}
	\end{tehtava}
	\begin{tehtava} 
Ratkaise
	\begin{alakohdat}
	\alakohta{$x^3 - \frac{14}{27} = \frac{20}{6}+\frac{14}{18}$}
	\alakohta{$2(x-1)=\frac{3x}{4}$.}
	 \end{alakohdat}
	 \begin{vastaus}
	  \begin{alakohdat}
	   \alakohta{$x = \frac{5}{3}$}
	   \alakohta{$x = \frac{8}{5}$}
	  \end{alakohdat}
	 \end{vastaus}
	\end{tehtava}
	\begin{tehtava}
	\begin{alakohdat}
	\alakohta{Laske $f(2)$, kun $f(x)=5x^{3}-12x^{-2}-2$}
	\alakohta{Tutki kuvaajasta, mitkä ovat funktion $f(x)=2x^2-3x-2$ nollakohdat ja lisäksi arvo, kun $x=1$. Millä muuttujan arvoilla $f$ saa arvon $3$? Laske $f(7)$.}
	 \end{alakohdat}
	\begin{vastaus}
	 \begin{alakohdat}
	  \alakohta{$x = 35$}
	  \alakohta{Nollakohdat ovat $x = -0,5$ ja $x = 2. f(1) = -3$. Funktio saa arvon 3 kun $x = -1$ ja $x = 2,5$. $f(7) = 75$}
	 \end{alakohdat}
	\end{vastaus}
	\end{tehtava}
	\begin{center}
		\begin{kuvaajapohja}{1.0}{-3}{5}{-4}{4}
			\kuvaaja{2*x**2-3*x-2}{$f(x)=2x^2-3x-2$}{black}
		\end{kuvaajapohja}
	\end{center}
	\begin{tehtava}
Kalle-Petterillä on suorakulmainen pala kartonkia, jonka pinta-ala on $36$ cm$^2$, lyhyen sivu pituus on $4$ cm ja pitemmän sivun pituus on $x$ cm. Hän haluaa taivuttaa kartongista putken niin että lyhyet sivut ovat yhdessä. Laske putken tilavuus, kun tilavuuden kaava on $V=\pi r^2 h$, jossa $V$ on tilavuus, $h$ on kartonkipalan lyhyen sivun pituus, ja $x=2\pi r$.    
\begin{vastaus}
 $81 \pi$
\end{vastaus}
\end{tehtava}

\subsection*{Harjoituskoe 2}

	\begin{tehtava}
Mainitse jokin
	\begin{alakohdat}
	\alakohta{kokonaisluku, joka ei ole luonnollinen luku}
	\alakohta{rationaaliluku, joka ei ole kokonaisluku}
	\alakohta{reaaliluku, joka ei ole rationaaliluku.}
	\end{alakohdat}
	\begin{vastaus}
	 \begin{alakohdat}
	  \alakohta{$-1$}
	  \alakohta{$\frac{1}{2}$}
	  \alakohta{$\pi$}
	  \end{alakohdat}
	\end{vastaus}
	\end{tehtava}
	\begin{tehtava}
Ratkaise
	\begin{alakohdat}
	\alakohta{$11x=77$}
	\alakohta{$8x+174=50x$}
	\alakohta{$\frac{5}{4}x-1=\frac{4}{5}x$}
	\alakohta{$\sqrt{2}x+\sqrt{2}=2x$.}
	\end{alakohdat}
	\begin{vastaus}
	 \begin{alakohdat}
	  \alakohta{$7$}
	  \alakohta{$\frac{29}{7}$}
	  \alakohta{$\frac{20}{9}$}
	  \alakohta{$-1-\sqrt{2}$}
	 \end{alakohdat}
	\end{vastaus}
	\end{tehtava}
	\begin{tehtava}
Muuta desimaaliluvuksi
	\begin{alakohdat}
	\alakohta{$\frac{10}{3}$}
	\alakohta{$\frac{1}{4}-\frac{1}{16}$}
	\alakohta{$\frac{0}{17}$}
	\alakohta{$\frac{1}{2+\frac{1}{3}}$}
	\end{alakohdat}
	\begin{vastaus}
	 \begin{alakohdat}
	  \alakohta{$3,\overline{3}$}
	  \alakohta{$0,1875$}
	  \alakohta{$0$}
	  \alakohta{$0,\overline{428571}$}
	 \end{alakohdat}
	\end{vastaus}
	\end{tehtava}
	\begin{tehtava}
Tuoreessa ananaksessa veden osuus on 80\% ananaksen massasta ja A-, B- ja C-vitamiinien yhteenlaskettu osuus $0,05$ \% massasta. Ananas kuivatetaan niin, että veden osuus laskee 8 prosenttiin ananaksen massasta. Kuinka suuri on A-, B- ja C-vitamiinien osuus kuivatun ananaksen massasta? (Luvut eivät ole faktuaalisia.)
	\begin{vastaus}
	 Noin 18 \%.
	\end{vastaus}
	\end{tehtava}
	\begin{tehtava}
Pertsa ajaa kotoansa mummolaan tunnissa, jos ajovauhti on lupsakka $60$ km/h. Nyt Pertsalla on kuitenkin kiire ja hän yrittää keretä mummolaan kahdessa kolmasosa tunnissa. Kuinka nopeasti Pertsan pitää ajaa?
	\begin{vastaus}
	 90 km/h.
	\end{vastaus}
	\end{tehtava}
	\begin{tehtava} 
Sievennä
	\begin{alakohdat}
	\alakohta{$\frac{a^2 b^2}{a}$, $a \neq 0$}
	\alakohta{$3(a^2+1)-2(a^2-1)$}
	\alakohta{$ab(a+2a)$}
	\alakohta{$(a^3 b^2 c)^2$.}
	\end{alakohdat}
		\begin{vastaus}
	 \begin{alakohdat}
	  \alakohta{$a b^2$}
	  \alakohta{$a^2 + 5$}
	  \alakohta{$3a^2 b$}
	  \alakohta{$a^5 + b^4 + c^2$}
	 \end{alakohdat}
	\end{vastaus}
	\end{tehtava}
	\begin{tehtava}
Olkoon $f(t) = 35 \cdot 2^t$ bakteerien lukumäärä soluviljelmässä ajanhetkellä $t$ (sekuntia). Monenko sekunnin kuluttua bakteereita on yli 1000? Yhden sekunnin tarkkuus ylöspäin pyöristettynä riittää.
	\begin{vastaus}
	 6 sekuntia.
	\end{vastaus}
	\end{tehtava}
	\begin{tehtava}
Määritellään funktio $f$ lausekkeella $f(x) = 2x-3$. Ratkaise yhtälö
\[
f(x-1)+f(x)+f(x+1) = 3
\] 
\begin{vastaus}
$x = 2$
\end{vastaus}
\end{tehtava}

\subsection*{Harjoituskoe 3}

	\begin{tehtava}
Sievennä
	\begin{alakohdat}
		 \alakohta{$-x \cdot (-x)$}
	 	\alakohta{$-(-x+y)-(-y)$}
	 	\alakohta{$a-(ab)^2+a^2 b^2$}
	 	\alakohta{$\frac{ab+ca}{a} \cdot (b-c)$.}
	\end{alakohdat}
	\begin{vastaus}
	 \begin{alakohdat}
	  	\alakohta{$x^2$}
	  	\alakohta{$x$}
	  	\alakohta{$a$}
	  	\alakohta{$b^2-c^2$}
	 \end{alakohdat}
	\end{vastaus}
	\end{tehtava}
	
	\begin{tehtava}
Muuta sekaluvut murtoluvuiksi
	 \begin{alakohdat}
	  \alakohta{$2\frac{3}{4}$}
	  \alakohta{$\frac{4}{13}$}
	  \alakohta{$-7\frac{2}{11}$}
	  \alakohta{$4$}
	 \end{alakohdat}
	 \begin{vastaus}
	 \begin{alakohdat}
	  \alakohta{$\frac{11}{4}$}
	  \alakohta{$\frac{4}{13}$}
	  \alakohta{$-\frac{79}{11}$}
	  \alakohta{$4$}
	 \end{alakohdat}
	 \end{vastaus}
	\end{tehtava}
	
	\begin{tehtava}
Sievennä
	 \begin{alakohdat}
	 \alakohta{$a^{\frac{1}{3}} \cdot a^{\frac{1}{4}} \cdot a^{\frac{1}{5}} \cdot a^{\frac{13}{60}}$}
	 \alakohta{$(\sqrt[3]{a})^5 \cdot a^\frac{8}{6}$.}
	 \end{alakohdat}
	 \begin{vastaus}
	  \begin{alakohdat}
	   \alakohta{$a$}
	   \alakohta{$a^3$}
	  \end{alakohdat}
	 \end{vastaus}
	 \end{tehtava}
	 
	\begin{tehtava}
Kuinka monta prosenttia
	\begin{alakohdat}
	\alakohta{suurempi $30$ euroa on verrattuna $10$ euroon?}
	\alakohta{lyhyempi $6$ metriä on verrattuna $30$ metriin?}
	\alakohta{$1800c$ on $2400c$:stä?}
	\end{alakohdat}
	\begin{vastaus}
	 \begin{alakohdat}
	  \alakohta{200\,\% suurempi.}
	  \alakohta{80\,\% pienempi.}
	  \alakohta{75\,\%.}
	  \end{alakohdat}
	\end{vastaus}
	\end{tehtava}
	
	\begin{tehtava}
Ostat vuoden voimassa olevan parturikortin, jolla voi käydä parturissa niin usein kuin haluaa. Kortti maksaa 230\,€ ja kertakäynti 25\,€. Kuinka monta kertaa sinun pitäisi käydä parturissa vuoden aikana, jotta ostos olisi kannattava?
	\begin{vastaus}
	 10 kertaa.
	\end{vastaus}
	\end{tehtava}
	
	\begin{tehtava}
Isä avasi kukkaronsa nyörejä ja ostaa paukautti suuren taulutelevision, jonka mitat ovat 300\,cm x 210\,cm. Mahtuuko televisio edes sisään ulko-ovesta, jonka mitat on 200\,cm x 80\,cm?
	\begin{vastaus}
	 Kyllä mahtuu. (Oven lävistäjä on suurempi kuin telkkarin korkeus.)
	\end{vastaus}
	\end{tehtava}
	
	\begin{tehtava}
Mansikasta noin $99 \%$ on vettä. Ostat 3,00 kiloa mansikoita ja laitat ne kuivumaan, kunnes kaksi kolmasosaa mansikoiden vedestä on haihtunut. Kuinka monta kilogrammaa mansikoita sinulla on kuivatuksen jälkeen?
	\begin{vastaus}
	 $1,02$ kg.
	\end{vastaus}
	\end{tehtava}
	
	\begin{tehtava}
Laitat suuren lottovoiton tilille, jolla talletuksen arvo kasvaa $1 \%$ vuodessa. Nostat rahat 70 vuoden päästä. Kuinka monta prosenttia enemmän rahaa tilillä on silloin?
	\begin{vastaus}
	 10\, \%.
	\end{vastaus}
	\end{tehtava}
	
\subsection*{Harjoituskoe 4}
	
	\begin{tehtava}
Jarskilla on kaksi hattua, kolme paitaa ja kahdet housut. Kuinka monella tavalla Jarski voi pukeutua?
	 \begin{vastaus}
	  $2 \cdot 3 \cdot 2 = 12$.
	 \end{vastaus}
	 \end{tehtava}
	 
	\begin{tehtava}
Mirjami heittää pallon 10 metrin korkeuteen. Pallo pomppaa aina 80\,\% edellisen pompun korkeudesta. Monennenko pompun jälkeen pallo ei enää nouse yli 5 metrin?
	\begin{vastaus}
	 4. pompun jälkeen.
	\end{vastaus}
	\end{tehtava}
	
	\begin{tehtava}
Alennusmyynneissä 50\,\% alennetut sukat maksavat 5 euroa. Kuinka monta prosenttia alennettua hintaa täytyy korottaa, jotta päästäisiin takaisin alkuperäiseen hintaan?
	\begin{vastaus}
	 100\,\%.
	\end{vastaus}
	\end{tehtava}
	
	\begin{tehtava}
Laske.
	\begin{alakohdat}
	 \alakohta{$\sqrt{9+16}$}
	 \alakohta{$\sqrt{9} + \sqrt{16}$}
	 \alakohta{$\sqrt[3]{\sqrt{(9 + 16)^3}}$}
	 \alakohta{$\sqrt{\frac{(a+b)^2 + (a-b)^2}{a^2 + b^2}}$.}
	\end{alakohdat}
	\begin{vastaus}
	 \begin{alakohdat}
	  \alakohta{$5$}
	  \alakohta{$7$}
	  \alakohta{$5$}
	  \alakohta{$\sqrt{2}$}
	 \end{alakohdat}
	\end{vastaus}
	\end{tehtava}
	\begin{tehtava}
Millä a:n arvoilla
	\begin{alakohdat}
	 \alakohta{$\sqrt{a+1} = 3$?}
	 \alakohta{$(a+1)^2 = 1$?}
	 \alakohta{$\sqrt{a^2} = -a$?}
	 \alakohta{$\sqrt{a} = -1$?}
	\end{alakohdat}
	\begin{vastaus}
	 \begin{alakohdat}
	  \alakohta{$a = 8$}
	  \alakohta{$a = 0$ ja $a = -2$}
	  \alakohta{$a \le 0$}
	  \alakohta{Ei millään $a$:n arvolla.}
	 \end{alakohdat}
	\end{vastaus}
	\end{tehtava}
	
	\begin{tehtava}
Laske käyttämällä potenssien laskusääntöjä
	 \begin{alakohdat}
	  \alakohta{$\frac{7^{10002}}{7^{10000}}$}
	  \alakohta{$x^2 \cdot 4^{28} \cdot x^{-1} \cdot 0,25^{27}$}
	  \alakohta{$8^{7a + \frac{a}{3}}-2^{21a + a}$.}
	 \end{alakohdat}
	 \begin{vastaus}
	  \begin{alakohdat}
	   \alakohta{$49$}
	   \alakohta{$4x$}
	   \alakohta{$0$}
	  \end{alakohdat}
	 \end{vastaus}
	\end{tehtava}
	
	\begin{tehtava}
Timanttien arvo on karkeasti suoraan verrannollinen niiden massan neliöön. Kuuluisa timantti The Great Mogul painaa noin 800 karaattia. The Sancy -timantti painaa noin 50 karaattia. Kuinka paljon
	\begin{alakohdat}
	 \alakohta{suurempi The Great Mogul on verrattuna The Sancyyn?}
	 \alakohta{arvokkaampi The Great Mogul on verrattuna The Sancyyn?}
	\end{alakohdat}
	\begin{vastaus}
	
	\begin{alakohdat}
	 \alakohta{noin 16 kertaa suurempi.}
	 \alakohta{noin 256 kertaa arvokkaampi.}
	\end{alakohdat}
	\end{vastaus}
	\end{tehtava}
	
	\begin{tehtava}
	 Laboratoriossa viljeltävän bakteeriviljelmän massa kolminkertaistuu päivässä. Kun bakteereja on kasvatettu 5 päivän ajan, niitä on 243 grammaa. Kuinka paljon bakteereja
	 \begin{alakohdat}
	  \alakohta{on 4 päivän päästä?}
	  \alakohta{oli kasvatuksen alussa?}
	 \end{alakohdat}
	 \begin{vastaus}
	  \begin{alakohdat}
	   \alakohta{$29,7$\,kg.}
	   \alakohta{$1,00$\,g.}
	  \end{alakohdat}
	 \end{vastaus}
	\end{tehtava}
