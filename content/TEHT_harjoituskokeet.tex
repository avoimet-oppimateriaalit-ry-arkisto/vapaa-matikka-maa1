\subsection*{Harjoituskoe 1}

\begin{tehtava}
Mainitse jokin
 \alakohdat{
	§ aina tosi yhtälö
	§ joskus tosi yhtälö (ja milloin se on tosi)
	§ aina epätosi yhtälö.
	 }
	 \begin{vastaus}
\alakohdat{
 § esim. $x=x$
 § esim. $x=1$, tosi vain kun $x=1$
 § esim. $1=2$
}
    \end{vastaus}
	\end{tehtava}
	
	\begin{tehtava}
Sievennä %yhdistä potensseja!
	\alakohdat{
	§ $\sqrt{144}$
	§ $64^\frac{3}{2}$
	§ $ \sqrt[3]{3^4}\cdot \sqrt[4]{3^3} $
	§ $\frac{\sqrt{a}}{\sqrt{2}}(2a)^{\frac{-5}{2}}$.
	 }
	 \begin{vastaus}
\alakohdat{
 § $12$
 § $512$
 § $9\sqrt[12]{3}$
 § $\frac{1}{8a^2}$
}
\end{vastaus}
	\end{tehtava}
	\begin{tehtava}
Muuta murtoluvuksi
	\alakohdat{
	§ $0,45$
	§ $0,333\ldots$
	§ $0,\overline{285714}$
	§ $0,3\overline{8}$.
	 }
	 \begin{vastaus}
	  \alakohdat{
	   § $\frac{9}{20}$
	   § $\frac{1}{3}$
	   § $\frac{2}{7}$
	   § $\frac{7}{18}$
	  }
	 \end{vastaus}
	\end{tehtava}
	\begin{tehtava}
	Eräs bakteerikanta kasvaa päivässä $75$ prosentilla. Montako prosenttia bakteerikannan alkuarvo $1$\,g on bakteerikannan määrästä kolmen päivän kuluttua?
	\begin{vastaus}
	  $536\,\% $
	\end{vastaus}
	\end{tehtava}
		\begin{tehtava}
	Mursupuvun hinnasta puolet tulee valmistuskuluista. Valmistuskuluista $75\,\%$ on materiaalikuluja. Montako prosenttia materiaalikulujen pitää laskea, jotta mursupuvun hinta tippuu $10$\,\%?
	\begin{vastaus}
	  $20\,\%$
	\end{vastaus}
	\end{tehtava}
	\begin{tehtava} 
Ratkaise
	\alakohdat{
	§ $x^3 - \frac{14}{27} = \frac{20}{6}+\frac{14}{18}$
	§ $2(x-1)=\frac{3x}{4}$.
	 }
	 \begin{vastaus}
	  \alakohdat{
	   § $x = \frac{5}{3}$
	   § $x = \frac{8}{5}$
	  }
	 \end{vastaus}
	\end{tehtava}
	\begin{tehtava}
	\alakohdat{
	§ Laske $f(2)$, kun $f(x)=5x^3-12x^{-2}-2$.
	§ Tutki kuvaajasta, mitkä ovat funktion $f(x)=2x^2-3x-2$ nollakohdat ja lisäksi arvo, kun $x=1$. Millä muuttujan arvoilla $f$ saa arvon $3$? Laske $f(7)$. %is dum :(
	 }
	\begin{vastaus}
	 \alakohdat{
	  § $x = 35$
	  § Nollakohdat ovat $x=-0,5$ ja $x = 2$. $f(1) = -3$. Funktio saa arvon 3 kun $x = -1$ ja $x = 2,5$. $f(7) = 75$
	 }
	\end{vastaus}
	\end{tehtava}
	\begin{center}
		\begin{kuvaajapohja}{1.0}{-3}{5}{-4}{4}
			\kuvaaja{2*x**2-3*x-2}{$f(x)=2x^2-3x-2$}{black}
		\end{kuvaajapohja}
	\end{center}
	\begin{tehtava}
Jaana-Petterillä on suorakulmainen pala kartonkia, jonka pinta-ala on $36$\,cm$^2$, lyhyen sivu pituus on $4$ cm ja pitemmän sivun pituus on $x$\,cm. Hän haluaa taivuttaa kartongista putken niin että lyhyet sivut ovat yhdessä. Laske putken tilavuus, kun tilavuuden kaava on $V=\pi r^2 h$, jossa $V$ on tilavuus, $h$ on kartonkipalan lyhyen sivun pituus, ja $x=2\pi r$.    
\begin{vastaus}
 $81\pi$
\end{vastaus}
\end{tehtava}

\subsection*{Harjoituskoe 2}

	\begin{tehtava}
Mainitse jokin
	\alakohdat{
	§ kokonaisluku, joka ei ole luonnollinen luku
	§ rationaaliluku, joka ei ole kokonaisluku
	§ reaaliluku, joka ei ole rationaaliluku.
	}
	\begin{vastaus}
	 \alakohdat{
	  § $-1$
	  § $\frac{1}{2}$
	  § $\pi$
	  }
	\end{vastaus}
	\end{tehtava}
	\begin{tehtava}
Ratkaise
	\alakohdat{
	§ $11x=77$
	§ $8x+174=50x$
	§ $\frac{5}{4}x-1=\frac{4}{5}x$
	§ $\sqrt{2}x+\sqrt{2}=2x$.
	}
	\begin{vastaus}
	 \alakohdat{
	  § $7$
	  § $\frac{29}{7}$
	  § $\frac{20}{9}$
	  § $-1-\sqrt{2}$
	 }
	\end{vastaus}
	\end{tehtava}
	\begin{tehtava}
Muuta desimaaliluvuksi
	\alakohdat{
	§ $\frac{10}{3}$
	§ $\frac{1}{4}-\frac{1}{16}$
	§ $\frac{0}{17}$
	§ $\frac{1}{2+\frac{1}{3}}$.
	}
	\begin{vastaus}
	 \alakohdat{
	  § $3,\overline{3}$
	  § $0,1875$
	  § $0$
	  § $0,\overline{428571}$
	 }
	\end{vastaus}
	\end{tehtava}
	\begin{tehtava}
Tuoreessa ananaksessa veden osuus on 80\,\% ananaksen massasta ja A-, B- ja C-vitamiinien yhteenlaskettu osuus $0,05$\,\% massasta. Ananas kuivatetaan niin, että veden osuus laskee 8 prosenttiin ananaksen massasta. Kuinka suuri on A-, B- ja C-vitamiinien osuus kuivatun ananaksen massasta? (Luvut eivät ole faktuaalisia.)
	\begin{vastaus}
	 Noin 18 \%.
	\end{vastaus}
	\end{tehtava}
	\begin{tehtava}
Pertsa ajaa kotoansa mummolaan tunnissa, jos ajovauhti on lupsakka $60$\,km/h. Nyt Pertsalla on kuitenkin kiire ja hän yrittää keretä mummolaan kahdessa kolmasosa tunnissa. Kuinka nopeasti Pertsan pitää ajaa?
	\begin{vastaus}
	 90 km/h.
	\end{vastaus}
	\end{tehtava}
	\begin{tehtava} 
Sievennä
	\alakohdat{
	§ $\frac{a^2 b^2}{a}$, $a \neq 0$
	§ $3(a^2+1)-2(a^2-1)$
	§ $ab(a+2a)$
	§ $(a^3 b^2 c)^2$.
	}
		\begin{vastaus}
	 \alakohdat{
	  § $a b^2$
	  § $a^2 + 5$
	  § $3a^2 b$
	  § $a^5 + b^4 + c^2$
	 }
	\end{vastaus}
	\end{tehtava}
	\begin{tehtava}
Olkoon $f(t) = 35 \cdot 2^t$ bakteerien lukumäärä soluviljelmässä ajanhetkellä $t$ (sekuntia). Monenko sekunnin kuluttua bakteereita on yli 1\,000? Yhden sekunnin tarkkuus ylöspäin pyöristettynä riittää.
	\begin{vastaus}
	 6 sekuntia
	\end{vastaus}
	\end{tehtava}
	\begin{tehtava}
Määritellään funktio $f$ lausekkeella $f(x) = 2x-3$. Ratkaise yhtälö
\[
f(x-1)+f(x)+f(x+1) = 3
\] 
\begin{vastaus}
$x = 2$
\end{vastaus}
\end{tehtava}

\subsection*{Harjoituskoe 3}

	\begin{tehtava}
Sievennä
	\alakohdat{
		§ $-x \cdot (-x)$
	 	§ $-(-x+y)-(-y)$
	 	§ $a-(ab)^2+a^2 b^2$
	 	§ $\frac{ab+ca}{a} \cdot (b-c)$.
	}
	\begin{vastaus}
	 \alakohdat{
	  	§ $x^2$
	  	§ $x$
	  	§ $a$
	  	§ $b^2-c^2$
	 }
	\end{vastaus}
	\end{tehtava}
	
	\begin{tehtava}
Muuta sekaluvut murtoluvuiksi.
	 \alakohdat{
	  § $2\frac{3}{4}$
	  § $\frac{4}{13}$
	  § $-7\frac{2}{11}$
	  § $4$
	 }
	 \begin{vastaus}
	 \alakohdat{
	  § $\frac{11}{4}$
	  § $\frac{4}{13}$
	  § $-\frac{79}{11}$
	  § $-\frac{4}{1}$
	 }
	 \end{vastaus}
	\end{tehtava}
	
	\begin{tehtava}
Sievennä
	 \alakohdat{
	 § $a^{\frac{1}{3}} \cdot a^{\frac{1}{4}} \cdot a^{\frac{1}{5}} \cdot a^{\frac{13}{60}}$
	 § $(\sqrt[3]{a})^5 \cdot a^\frac{8}{6}$.
	 }
	 \begin{vastaus}
	  \alakohdat{
	   § $a$
	   § $a^3$
	  }
	 \end{vastaus}
	 \end{tehtava}
	 
	\begin{tehtava}
Kuinka monta prosenttia
	\alakohdat{
	§ suurempi $30$ euroa on verrattuna $10$ euroon?
	§ lyhyempi $6$ metriä on verrattuna $30$ metriin?
	§ $1\,800c$ on $2\,400c$:stä?
	}
	\begin{vastaus}
	 \alakohdat{
	  § $200$\,\% suurempi
	  § $80$\,\% pienempi
	  § $75$\,\%.
	  }
	\end{vastaus}
	\end{tehtava}
	
	\begin{tehtava}
Ostat vuoden voimassa olevan parturikortin, jolla voi käydä parturissa niin usein kuin haluaa. Kortti maksaa $230$\,€ ja kertakäynti $25$\,€. Kuinka monta kertaa sinun pitäisi käydä parturissa vuoden aikana, jotta ostos olisi kannattava?
	\begin{vastaus}
	 10 kertaa
	\end{vastaus}
	\end{tehtava}
	
%puuttunee tehtävä

	\begin{tehtava}
Mansikasta noin $99\,\%$ on vettä. Ostat $3,00$ kiloa mansikoita ja laitat ne kuivumaan, kunnes kaksi kolmasosaa mansikoiden vedestä on haihtunut. Kuinka monta kilogrammaa mansikoita sinulla on kuivatuksen jälkeen?
	\begin{vastaus}
	 $1,02$\,kg.
	\end{vastaus}
	\end{tehtava}
	
	\begin{tehtava}
Laitat suuren lottovoiton tilille, jolla talletuksen arvo kasvaa $1\,\%$ vuodessa. Nostat rahat $70$ vuoden päästä. Kuinka monta prosenttia enemmän rahaa tilillä on silloin?
	\begin{vastaus}
	 $10$\,\%
	\end{vastaus}
	\end{tehtava}
	
\subsection*{Harjoituskoe 4}
	
	\begin{tehtava}
Jarskilla on kaksi hattua, kolme paitaa ja kahdet housut. Kuinka monella tavalla Jarski voi pukeutua?
	 \begin{vastaus}
	  $2 \cdot 3 \cdot 2 = 12$.
	 \end{vastaus}
	 \end{tehtava}
	 
	\begin{tehtava}
Mirjami heittää pallon $10$ metrin korkeuteen. Pallo pomppaa aina $80$\,\% edellisen pompun korkeudesta. Monennenko pompun jälkeen pallo ei enää nouse yli $5$ metrin?
	\begin{vastaus}
	 4. pompun jälkeen
	\end{vastaus}
	\end{tehtava}
	
	\begin{tehtava}
Alennusmyynneissä $50$\,\% alennetut sukat maksavat 5 euroa. Kuinka monta prosenttia alennettua hintaa täytyy korottaa, jotta päästäisiin takaisin alkuperäiseen hintaan?
	\begin{vastaus}
	 $100$\,\%.
	\end{vastaus}
	\end{tehtava}
	
	\begin{tehtava}
Laske.
	\alakohdat{
	 § $\sqrt{9+16}$
	 § $\sqrt{9} + \sqrt{16}$
	 § $\sqrt[3]{\sqrt{(9 + 16)^3}}$
	}
	\begin{vastaus}
	 \alakohdat{
	  § $5$
	  § $7$
	  § $5$
	 }
	\end{vastaus}
	\end{tehtava}
	\begin{tehtava}
Millä $a$:n arvoilla
	\alakohdat{
	 § $\sqrt{a+1} = 3$?
	 § $(a+1)^2 = 1$?
	 § $\sqrt{a^2} = -a$?
	 § $\sqrt{a} = -1$?
	}
	\begin{vastaus}
	 \alakohdat{
	  § $a = 8$
	  § $a = 0$ ja $a = -2$
	  § $a \le 0$
	  § Ei millään $a$:n arvolla
	 }
	\end{vastaus}
	\end{tehtava}
	
	\begin{tehtava}
Laske käyttämällä potenssien laskusääntöjä
	 \alakohdat{
	  § $\frac{7^{10\,002}}{7^{10\,000}}$
	  § $x^2 \cdot 4^{28} \cdot x^{-1} \cdot 0,25^{27}$
	  § $8^{7a + \frac{a}{3}}-2^{21a + a}$.
	 }
	 \begin{vastaus}
	  \alakohdat{
	   § $49$
	   § $4x$
	   § $0$
	  }
	 \end{vastaus}
	\end{tehtava}
	
	\begin{tehtava}
Timanttien arvo on karkeasti suoraan verrannollinen niiden massan neliöön. Kuuluisa timantti \textit{The Great Mogul} painaa noin $800$ karaattia. \textit{The Sancy} -timantti painaa noin $50$ karaattia. Kuinka paljon
	\alakohdat{
	 § suurempi The Great Mogul on verrattuna The Sancyyn?
	 § arvokkaampi The Great Mogul on verrattuna The Sancyyn?
	}
	\begin{vastaus}
	
	\alakohdat{
	 § noin $16$ kertaa suurempi
	 § noin $256$ kertaa arvokkaampi
	}
	\end{vastaus}
	\end{tehtava}
	
	\begin{tehtava}
	 Laboratoriossa viljeltävän bakteeriviljelmän massa kolminkertaistuu päivässä. Kun bakteereja on kasvatettu $5$ päivän ajan, niitä on $243$ grammaa. Kuinka paljon bakteereja
	 \alakohdat{
	  § on $4$ päivän päästä?
	  § oli kasvatuksen alussa?
	 }
	 \begin{vastaus}
	  \alakohdat{
	   § $29,7$\,kg
	   § $1,00$\,g
	  }
	 \end{vastaus}
	\end{tehtava}
