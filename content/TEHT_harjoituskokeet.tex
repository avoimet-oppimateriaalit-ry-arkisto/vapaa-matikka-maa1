\subsection*{Harjoituskoe 1}

\begin{tehtava}
Mainitse jokin
 \begin{alakohdat}
	\alakohta{aina tosi yhtälö}
	\alakohta{joskus tosi yhtälö (ja milloin se on tosi)}
	\alakohta{aina epätosi yhtälö.}
	 \end{alakohdat}
	 \begin{vastaus}
\begin{alakohdat}
 \alakohta{$x=x$}
 \alakohta{$x=1$, tosi vain kun $x=1$}
 \alakohta{$1=2$}
\end{alakohdat}
    \end{vastaus}
	\end{tehtava}
	\begin{tehtava}
Sievennä
	\begin{alakohdat}
	\alakohta{$\sqrt{144}$}
	\alakohta{$64^\frac{3}{2}$}
	\alakohta{$ 6^\frac{3}{2}\cdot 8^\frac{2}{3} $}
	\alakohta{$\frac{\sqrt{a}}{\sqrt{2}}(2a)^{\frac{-5}{2}}$.}
	 \end{alakohdat}
	 \begin{vastaus}
\begin{alakohdat}
 \alakohta{$12$}
 \alakohta{$512$}
 \alakohta{$\frac{1}{8a^2}$}
\end{alakohdat}
\end{vastaus}
	\end{tehtava}
	\begin{tehtava}
Muuta murtoluvuksi
	\begin{alakohdat}
	\alakohta{$0,45$}
	\alakohta{$0,33\ldots$}
	\alakohta{$0,2857142\ldots$}
	\alakohta{$0,388\ldots$.}
	 \end{alakohdat}
	 \begin{vastaus}
	  \begin{alakohdat}
	   \alakohta{$\frac{9}{20}$}
	   \alakohta{$\frac{1}{3}$}
	   \alakohta{$\frac{2}{7}$}
	   \alakohta{$\frac{7}{18}$}
	  \end{alakohdat}
	 \end{vastaus}
	\end{tehtava}
	\begin{tehtava}
Sievennä
	\begin{alakohdat}
	\alakohta{Eräs bakteerikanta kasvaa päivässä $75$:llä prosentilla. Montako prosenttia bakteerikannan alkuarvo $1$ g on bakteerikannan määrästä kolmen päivän kuluttua?}
	\alakohta{Mursupuvun hinnasta puolet tulee valmistuskuluista. Valmistuskuluista $75$ \% on materiaalikuluja. Montako prosenttia materaalikulujen pitää laskea, että mursupuvun hinta tippuu $10$ \%?}
	\end{alakohdat}
	\begin{vastaus}
	 \begin{alakohdat}
	  \alakohta{536 \%}
	  \alakohta{13,3 \%}
	 \end{alakohdat}
	\end{vastaus}

	\end{tehtava}
	\begin{tehtava}
Jaa luvut tekijöihin. Mitkä luvuista ovat alkulukuja?
	\begin{alakohdat}
	\alakohta{$111$}
	\alakohta{$75$}
	\alakohta{$97$}
	\alakohta{$360$.}
	 \end{alakohdat}
	 \begin{vastaus}
	  \begin{alakohdat}
	   \alakohta{$3 \cdot 37$}
	   \alakohta{$5 \cdot 5 \cdot 3$}
	   \alakohta{97 on alkuluku}
	   \alakohta{$6 \cdot 6 \cdot 5 \cdot 2$}
	  \end{alakohdat}
	 \end{vastaus}
	\end{tehtava}
	\begin{tehtava} 
Ratkaise
	\begin{alakohdat}
	\alakohta{$x^3 - \frac{14}{27} = \frac{20}{6}+\frac{14}{18}$}
	\alakohta{$2(x-1)=\frac{3x}{4}$}
	 \end{alakohdat}
	 \begin{vastaus}
	  \begin{alakohdat}
	   \alakohta{$x = \frac{5}{3}$}
	   \alakohta{$x = \frac{8}{5}$}
	  \end{alakohdat}
	 \end{vastaus}
	\end{tehtava}
	\begin{tehtava}
Tee näitä.
	\begin{alakohdat}
	\alakohta{Laske $f(2)$, kun $f(x)=5x^{3}-12x^{-2}-2$}
	\alakohta{Tutki kuvaajasta, mitkä ovat funktion $f(x)=2x^2-3x-2$ nollakohdat ja lisäksi arvo, kun $x=1$. Millä muuttujan arvoilla $f$ saa arvon $3$? Laske $f(7)$.}
	 \end{alakohdat}
	\begin{vastaus}
	 \begin{alakohdat}
	  \alakohta{$x = 35$}
	  \alakohta{Nollakohdat ovat $x = -0,5$ ja $x = 2. f(1) = -3$. Funktio saa arvon 3 kun $x = -1$ ja $x = 2,5$. $f(7) = 75$}
	 \end{alakohdat}

	\end{vastaus}

	\end{tehtava}
	\begin{center}
		\begin{kuvaajapohja}{1.0}{-3}{5}{-4}{4}
			\kuvaaja{2*x**2-3*x-2}{$f(x)=2x^2-3x-2$}{black}
		\end{kuvaajapohja}
	\end{center}

	\begin{tehtava}
Kalle-Petterillä on suorakulmainen pala kartonkia, jonka pinta-ala on $36$ cm$^2$, lyhyen sivu pituus on $4$ cm ja pitemmän sivun pituus on $x$ cm. Hän haluaa taivuttaa kartongista putken niin että lyhyet sivut ovat yhdessä. Laske putken tilavuus, kun tilavuuden kaava on $V=\pi\cdot r^2\cdot h$, jossa $V$ on tilavuus, $h$ on kartonkipalan lyhyen sivun pituus, ja $x=2\pi\cdot r$.    
\begin{vastaus}
 $81 \pi$
\end{vastaus}

\end{tehtava}

\subsection*{Harjoituskoe 2}

	\begin{tehtava}
Mainitse jokin
	\begin{alakohdat}
	\alakohta{kokonaisluku, joka ei ole luonnollinen luku}
	\alakohta{rationaaliluku, joka ei ole kokonaisluku}
	\alakohta{reaaliluku, joka ei ole rationaaliluku.}
	\end{alakohdat}
	\end{tehtava}
	\begin{tehtava}
Ratkaise
	\begin{alakohdat}
	\alakohta{$11x=77$}
	\alakohta{$8x+174=50x$}
	\alakohta{$\frac{5}{4}x-1=\frac{4}{5}x$}
	\alakohta{$\sqrt{2}x+\sqrt{2}=2x$.}
	\end{alakohdat}
	\end{tehtava}
	\begin{tehtava}
Muuta a ja b desimaaliluvuksi sekä c ja d murtopotensseiksi sieventäen.
	\begin{alakohdat}
	\alakohta{$\frac{3}{10}$}
	\alakohta{$\frac{3}{16}$}
	\alakohta{$a\cdot \sqrt[3]{a^4}$}
	\alakohta{$\frac{\sqrt[3]{a}}{\sqrt[3]{a}}-\frac{1}{\sqrt[3]{3a}}$ .}
	\end{alakohdat}
	\end{tehtava}
	\begin{tehtava}
Tuoreessa ananaksessa veden osuus on 80\% ananaksen massasta ja A-, B- ja C-vitamiinien yhteenlaskettu osuus $0,05$ \% massasta. Ananas kuivatetaan niin, että veden osuus laskee 8 prosenttiin ananaksen massasta. Kuinka suuri on A-, B- ja C-vitamiinien osuus kuivatun ananaksen massasta? (Luvut eivät ole faktuaalisia.)
	\end{tehtava}
	\begin{tehtava}
Pertsa ajaa kotoansa mummolaan tunnissa, jos ajovauhti on lupsakka $60$ km/h.
	\begin{alakohdat}
	\alakohta{Nyt Pertsalla on kuitenkin kiire ja hän yrittää keretä mummolaan kahdessa kolmasosa tunnissa. Kuinka nopeasti Pertsan pitää ajaa?}
	\alakohta{Pertsan ajaessa tätä vauhtia, tielle ryntää yhtäkkiä kirahvi. Jos jarrutusmatka on suoraan verranollinen ajonopeuden neliöön, kuinka kaukana kirahvista täytyy Pertsan ruveta jarruttamaan, että ei käy huonosti?}
	\end{alakohdat}
	\end{tehtava}
	\begin{tehtava} 
Sievennä
	\begin{alakohdat}
	\alakohta{$\frac{a^2 b^2}{a}$, $a \neq 0$}
	\alakohta{$3(a^2+1)-2(a^2-1)$}
	\alakohta{$ab(a+2a)$}
	\alakohta{$(a^3 b^2 c)^2$.}
	\end{alakohdat}
	\end{tehtava}
	\begin{tehtava}
Olkoon $f(t) = 35 \cdot 2^t$ bakteerien lukumäärä soluviljelmässä ajanhetkellä $t$ (sekuntia). Monenko sekunnin kuluttua bakteereita on yli 1000? Yhden sekunnin tarkkuus ylöspäin pyöristettynä riittää.
	\end{tehtava}
	\begin{tehtava}
Määritellään kahdelle järjestetylle lukunelikolle $(a, b, c, d)$ laskutoimitus $\odot$ seuraavasti: $(a_1, b_1, c_1, d_1) \odot (a_2, b_2, c_2, d_2) = (a_1 a_2 + b_1 c_2, a_1 b_2 + b_1 d_2, c_1 a_2 + d_1 c_2, c_1 b_2 + d_1 d_2)$. Laske $(1, 1, 1, 0) \odot (1, 1, 1, 0)$.
\end{tehtava}

\subsection*{Harjoituskoe 3}

	\begin{tehtava}
Sievennä
	\begin{alakohdat}
	 \alakohta{$-x \cdot (-x)$}
	 \alakohta{$-(-x+y)-(-y)$}
	 \alakohta{$a-(ab)^2+a^2 b^2$}
	 \alakohta{$\frac{ab+a}{a} \cdot (b-1)$}
	\end{alakohdat}
	\end{tehtava}
	\begin{tehtava}
Muuta samaan yksikköön
	 \begin{alakohdat}
	  \alakohta{$3$ km + $2000$ m}
	  \alakohta{$2$ dl + $70$ ml}
	  \alakohta{$4$ h + $30$ min + $1800$ s}
	 \end{alakohdat}
	\end{tehtava}
	\begin{tehtava}
Sievennä
	 \begin{alakohdat}
	 \alakohta{$a^{\frac{1}{3}} + a^{\frac{1}{4}} + a^{\frac{1}{5}}$}
	 \alakohta{$(\sqrt[3]{a})^5 \cdot a^\frac{8}{6}$}
	 \end{alakohdat}
	 \end{tehtava}
	\begin{tehtava}
Kuinka monta prosenttia
	\begin{alakohdat}
	\alakohta{suurempi $30$ euroa on verrattuna $10$ euroon}
	\alakohta{lyhyempi $6$ metriä on $30$ metriä}
	\alakohta{$1800$ c on $2400$ c:stä}
	\end{alakohdat} 
	\end{tehtava}
	\begin{tehtava}
Isä avasi kukkaron nyörejään ja osti suuren taulutelevision, jonka mitat on 300 cm x 210 cm. Mahtuuko televisio ulko-ovesta, jonka mitat on 200 cm x 80 cm?
	\end{tehtava}
	\begin{tehtava}
Mansikasta noin 99 \% on vettä. Ostat 3 kiloa mansikoita ja laitat ne kuivumaan, kunnes kaksi kolmas osaa mansikoiden vedestä on haihtunut. Kuinka paljon mansikoita sinulla on kuivatuksen jälkeen?
	\end{tehtava}
	\begin{tehtava}
Ostat vuoden voimassa olevan parturikortin, jolla voit käydä parturissa niin usein kuin haluat. Kortti maksaa 230 e ja kerta käynti 25 e. Kuinka monta kertaa sinun tulee käydä parturissa vuoden aikana, jotta ostos oli kannattava?
	\end{tehtava}
	\begin{tehtava}
Laitat suuren lotto voiton tilille, jossa se kasvaa 1 \% vuodessa. Nostat rahat 70 vuoden päästä. Kuinka monta prosentia enemmän rahaa sinulla on silloin?
	\end{tehtava}
	
\subsection*{Harjoituskoe 4}
	
	\begin{tehtava}
Jarskilla on kaksi hattua, kolme paitaa ja kahdet housut. Kuinka monella tavalla Jarski voi pukeutua?
	 \end{tehtava}
	\begin{tehtava}
Mirjami heittää pallon 10 metrin korkeuteen. Pallo pomppaa aina 80 \% edellisestä korkeudesta. Monennenko pompun jälkeen pallo ei enää nouse yli 5 metrin?
	\end{tehtava}
	\begin{tehtava}
50 \% alennusmyynnissä sukat maksavat 5 euroa. Kuinka monta prosenttia hintaa täytyy nostaa, jotta päästäisiin takaisin alkuperäiseen hintaan?
	\end{tehtava}
	\begin{tehtava}
Laske
	\begin{alakohdat}
	 \alakohta{$\sqrt{9+16}$}
	 \alakohta{$\sqrt{9} + \sqrt{16}$}
	 \alakohta{$\sqrt[3]{9 + 16}$}
	 \alakohta{$\sqrt{\frac{(a+b)^2 + (a-b)^2}{a^2 + b^2}} \cdot \sqrt{2}$}
	\end{alakohdat}
	\end{tehtava}
	\begin{tehtava}
Millä a:n arvoilla
	\begin{alakohdat}
	 \alakohta{$\sqrt{a} = a$}
	 \alakohta{$a + a^2 = 0$}
	 \alakohta{$\sqrt{a^2} = -a$}
	 \alakohta{$a^a = 0$}
	\end{alakohdat}
	\end{tehtava}
	\begin{tehtava}
Laske käyttämällä potenssien laskusääntöjä
	 \begin{alakohdat}
	  \alakohta{$\frac{7^{78}}{7^{76}}$}
	  \alakohta{$4^{28} \cdot 0,25^{27}$}
	  \alakohta{$8^{17}-2^{19}$}
	 \end{alakohdat}
	\end{tehtava}
	\begin{tehtava}
Timanttien arvo on karkeasti suoraan verrannollinen niiden massan neliöön. Kuulu timantti The Great Mogul painaa 800 karaattia. The Sancy --timantti painaa noin 50 karaattia. Kuinka paljon arvokkaampi The Great Mogul on verrattuna The Sancyyn?
	\end{tehtava}
	\begin{tehtava}
	 Laboratoriossa viljellään erästä bakteeria, joka kolminkertaistuu tunnissa. 10 tunnin kasvatuksen jälkeen bakteereja on 30 grammaa. Kuinka paljon bakteereja on
	 \begin{alakohdat}
	  \alakohta{5 tunnin päästä}
	  \alakohta{10 tuntia sitten}
	 \end{alakohdat}
	\end{tehtava}
