%  %Tähän tulee potenssien (ja ehkä muidenkin?) laskusääntöjen todistuksia, kunhan joku laatii ne.
% 
% \subsection*{Murtolausekkeiden sieventäminen}
% 
% \laatikko{
% Jos murtoluvun osoittajassa tai nimittäjässä on summa, jonka osilla on yhteinen tekijä, sen voi ottaa \emph{yhteiseksi tekijäksi} sulkujen eteen. Jos osoittajassa ja nimittäjässä on sen jälkeen sama kerroin, sen voi jakaa pois molemmista eli \emph{supistaa} pois.
% \begin{equation}
% \frac{ac+bc}{c} = \frac{ \cancel{c} (a+b)}{\cancel{c}} = a+b
% \end{equation}
% 
% 
% Joskus murtolauseke sieventyy, jos sen esittääkin kahden murtoluvun summana.
% \begin{equation}
% \frac{ca+b}{c} = \frac{ca}{c} + \frac{b}{c} = a + \frac{b}{c}
% \end{equation}
% }
% 
% Kun jakaa kolme erikokoista nallekarkkipussia ($a$, $b$ ja $c$) tasan kolmen ihmisen kesken, on sama, laittaako kaikki ensin samaan kulhoon ja jakaa ne sitten ($\frac{a+b+c}{3}$) vai jakaako jokaisen pussin erikseen ($ \frac{a}{3} + \frac{b}{3} + \frac{c}{3}$).
% 
% Jos taas samat kolme henkilöä jakavat keskenään pussin tikkareita ($6$ kpl) ja yhden pussin nallekarkkeja ($n$ kpl), niin saadaan seuraavanlainen lasku: $ \frac{6\text{ tikkaria}+n\text{ nallekarkkia}}{3} = \frac{6\text{ tikkaria}}{3} + \frac{n\text{ nallekarkkia}}{3} = \frac{\cancel{3} \cdot 2\text{ tikkaria}}{\cancel{3}} + \frac{n\text{ nallekarkkia}}{3} = 2\text{ tikkaria} + \frac{n\text{ nallekarkkia}}{3}$. Toisin sanoen, kukin saa kaksi tikkaria ja kuinka paljon ikinä onkaan kolmasosa kaikista nallekarkeista.
% 
% \laatikko{
% Samantyyppiset asiat voidaan laskea yhteen tai \emph{ryhmitellä}.
% \begin{equation}
% ax^2 + bx + cx^2 + dy + ex = (a+c)x^2 + (b+e)x + dy
% \end{equation}
% }
% 
% %Tämä alla oleva esimerkki rationaalukujen laskutoimitusosioon kirjan alkupäähän ja niin, että se RIVITETÄÄN ja kerrotaan VAIHEITTAIN, mitä tehtiin! :) T: Joonas
% 
% \begin{esimerkki}
% 
% $ \frac{1}{6} + \frac{3}{2} = \frac{1}{2\cdot 3} + \frac{3}{2} = \frac{1}{2 \cdot 3} + \frac{3 \cdot 3}{2 \cdot 3} = \frac{1}{6} + \frac{9}{6} = \frac{10}{6} = \frac{\cancel{2} \cdot 5}{\cancel{2} \cdot 3} = \frac{5}{3}$
% 
% \end{esimerkki}
% 
% \begin{tehtavasivu}
% 
% \begin{tehtava}
% % Ryhmittely
% Sievennä
% 	\alakohdat{
% 	§ $2x^2+3x+5x^2$
% 	§ $x^2+3x^3+x^2+x^3+2x^2$
% 	§ $ax^2+bx+cx$
% 	§ $ax^3+bx+cy^3+dx+ey^3+fx^3$
% 	}
% 
% \begin{vastaus}
% 	\alakohdat{
% 	§ $7x^2+3x$
% 	§ $4(x^2+x^3)$ tai $4x^2+4x^3$
% 	§ $ax^2+(b+c)x$ tai $ax^2+bx+cx$
% 	§ $(a+f)x^3+(b+d)x+(c+e)y^3$
% 	}
% \end{vastaus}
% \end{tehtava}
% 
% \begin{tehtava}
% % Yksi termi osoittajassa
% Sievennä
% 	\alakohdat{
% 	§ $\frac{2x^3}{x}$
% 	§ $\frac{3x^3y^2}{xy}$
% 	§ $\frac{x^2yz}{xy^2}$
% 	§ $\frac{6xy^3z^2}{2xz}$
% 	}
% 
% \begin{vastaus}
% 	\alakohdat{
% 	§ $2x^2$
% 	§ $\frac{x}{y}$
% 	§ $\frac{xz}{y}$
% 	§ $3y^3z$
% 	}
% \end{vastaus}
% \end{tehtava}
% 
% \begin{tehtava}
% % Useampia termejä osoittajassa
% Sievennä
% 	\alakohdat{
% 	§ $\frac{2x^5+3x^3}{x^2}$
% 	§ $\frac{6x^2+8y}{2x^2}$
% 	§ $\frac{3x-2x^2y^3}{xy}$
% 	§ $\frac{2x^2+3xy^2z-4xz}{2xy^2z}$
% 	}
% 
% \begin{vastaus}
% 	\alakohdat{
% 	§ $2x^3+3x$
% 	§ $3+4 \frac{y}{x^2}$
% 	§ $\frac{3}{y} - 2xy^2$
% 	§ $\frac{x}{y^2z} + \frac{3}{2} + \frac{2}{y^2}$
% 	}
% \end{vastaus}
% \end{tehtava}
% 
% \begin{tehtava}
% Sievennä.
% 	\alakohdat{
% 	§ $ \frac{1-x}{3} + \frac{x-2}{6}$
% 	§ $ \frac{5x-1}{3} - \frac{2x+5}{2}$
% 	§ $\frac{4x^2+3x}{x} + \frac{5x^3y-2x^2y}{x^2y}$
% 	§ $\frac{7x+5y}{y} - \frac{3x-2y}{x}$
% 	}
% 
% \begin{vastaus}
% 	\alakohdat{
% 	§ $ -\frac{x}{6}$
% 	§ $ \frac{2}{3} x - \frac{17}{6}$
% 	§ $9x+1$
% 	§ $\frac{7x}{y} + \frac{2y}{x} +2$
% 	}
% \end{vastaus}
% \end{tehtava}
% 
% \begin{tehtava}
% % Tuloja
% Sievennä.
% 	\alakohdat{
% 	§ $\frac{x}{6y} \cdot \frac{3y}{2}$
% 	§ $x \cdot \frac{x+y}{xy}$
% 	}
% 
% \begin{vastaus}
% 	\alakohdat{
% 	§ $\frac{x}{4}$
% 	§ $\frac{x}{y} + 1$
% 	}
% \end{vastaus}
% \end{tehtava}
% 
% \begin{tehtava}
% Sievennä: \\
% $(2ab+4b^2-8b^3c):(a+2b-4b^2c)$
% 	\begin{vastaus}
% 	$2b$
% 	\end{vastaus}
% \end{tehtava}
% 
% \end{tehtavasivu}
