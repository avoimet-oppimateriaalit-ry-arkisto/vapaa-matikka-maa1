\begin{tehtavasivu}

\subsubsection*{Opi perusteet}

%Tarkistanut V-P Kilpi 2013-11-09
\begin{tehtava}
Ratkaise
\alakohdatm{
§ $\frac{x}{3} = 1$
§ $\frac{8}{y} = 2$
§ $\frac{7}{z} = \frac{16}{8}$
§ $\frac{t}{3} = \frac{1}{7}$.
}
	\begin{vastaus}
\alakohdatm{
§ $x= 3$
§ $y= 4$ ($y\neq 0$)
§ $z= \frac{7}{2}$ ($z \neq 0$)
§ $t= \frac{3}{7}$
}
	\end{vastaus}
\end{tehtava}

\begin{tehtava}
	Kun $x = 5$, $y = 8$. Mikä on $y$:n arvo silloin kun $x = 4$, ja muuttujat $x$ ja $y$ ovat
	\alakohdatm{
		§ suoraan verrannollisia?
		§ kääntäen verrannollisia?
	}
	\begin{vastaus}
		\alakohdatm{
			§ $\frac{32}{5}$
			§ $10$
		}
	\end{vastaus}
\end{tehtava}

%Laatinut V-P Kilpi 2013-11-09
\begin{tehtava}
Ovatko muuttujat $x$ ja $y$ suoraan tai kääntäen verrannolliset vai eivät kumpaakaan, jos niiden
\alakohdatm{
§ osamäärä on aina $2$
§ summa on aina $5$
§ tulo on aina $4$
§ osamäärä on aina $1$?
}
	\begin{vastaus}
\alakohdatm{
§ Suoraan verrannollisia
§ Eivät ole kumpaakaan, sillä niiden tulo tai osamäärä ei ole vakio. 
§ Kääntäen verrannollisia
§ Kääntäen verrannollisia
}
	\end{vastaus}
\end{tehtava}

%Tarkistanut V-P Kilpi 2013-11-09
\begin{tehtava}
    Tarkastele seuraavia muuttujapareja:
    \alakohdat{
        § kuljettu matka ja kulunut aika, kun nopeus on vakio
        § sivujen pituudet suorakulmiossa, jonka pinta-ala on lukittu $5$ yksikköön
        § dieselin hinta ja $50$ eurolla saatavan dieselin määrä
        § irtomyynnistä ostetun suolakurkkuerän paino ja hinta
        § neliön sivun pituus ja neliön pinta-ala        
    }
    Tutki, miten toinen muuttuja muuttuu ensimmäisen kaksinkertaistuessa tai puolittuessa. Ovatko muuttujat suoraan tai kääntäen verrannolliset vai eivät kumpaakaan?
    \begin{vastaus}
        \alakohdatm{
            § Suoraan verrannolliset
            § Kääntäen verrannolliset
            § Kääntäen verrannolliset
            § Suoraan verrannolliset
            § Eivät ole kumpaakaan.
		
        }
    \end{vastaus}
\end{tehtava}

%Tarkistanut V-P Kilpi 2013-11-09
\begin{tehtava}
Muodosta seuraavia tilanteita kuvaavat yhtälöt (jossakin muodossa). Voit käyttää vakion merkkinä esimerkiksi kirjainta $k$.
\alakohdat{
§ Kultakimpaleen arvo $x$ on suoraan verrannollinen sen massaan $m$, eli mitä painavampi kimpale, sitä arvokkaampi se on.
§ Aidan maalaamiseen osallistuvien ihmisten määrä $x$ on kääntäen verrannollinen maalaamiseen kuluvaan aikaan $t$. Toisin sanoen: mitä
enemmän maalaajia, sitä nopeammin työ on valmis.
§ Planeettojen toisiinsa aiheuttama vetovoima $F$ on suoraan verrannollinen planeettojen massojen tuloon $m_1m_2$ ja kääntäen verrannollinen niiden välisen etäisyyden $r$ neliöön.
}
	\begin{vastaus}
\alakohdatm{
§ $ \frac{x}{m}=c$ tai $x=km$
§ $ xt=k $ tai $ x=\frac{k}{t}$
§ $ \frac{Fr^2}{m_1m_2}=k$ tai $ F=k\frac{m_1m_2}{r^2}$
}
	\end{vastaus}
\end{tehtava}

%Laatinut V-P Kilpi 2013-11-09
\begin{tehtava}
Hiusten pituuskasvu on suoraan verrannollinen kuluneeseen aikaan. Iinan hiukset kasvavat noin $1,5$ senttimetriä kuukaudessa. Iina päättää kasvattaa hiuksia vuoden ajan. Kuinka paljon Iinan hiukset tänä aikana kasvavat?
	\begin{vastaus}
Hiukset kasvavat vuodessa noin $18$ senttimetriä.
	\end{vastaus}
\end{tehtava}

%Tarkistanut V-P Kilpi 2013-11-09
\begin{tehtava}
Rento pyöräilyvauhti kaupunkiolosuhteissa on noin $20$\,km/h. Lukiolta urheiluhallille on matkaa $7$\,km. Kuinka monta minuuttia kestää arviolta pyöräillä lukiolta urheiluhallille?
	\begin{vastaus}
Viiden minuutin tarkkuudella $20$ minuuttia
	\end{vastaus}
\end{tehtava}

%Tarkistanut V-P Kilpi 2013-11-09
\begin{tehtava}
Isä ja lapset ovat ajamassa mökille Sotkamoon. On ajettu jo neljä viidesosaa matkasta, ja aikaa on kulunut kaksi tuntia. ''Joko ollaan perillä?'' lapset kysyvät takapenkiltä. Kuinka pitkään vielä arviolta kuluu, ennen kuin ollaan mökillä?
    \begin{vastaus}
        $30$\,min
    \end{vastaus}
\end{tehtava}

%Laatinut V-P Kilpi 2013-11-09
\begin{tehtava}
Yhteen sokerikakkuun tarvitaan neljä kananmunaa, $1,5$ desilitraa sokeria ja kaksi desilitraa vehnäjauhoja. Timolla on keittiössään $20$ kananmunaa, litra sokeria ja $1,5$ litraa jauhoja. Kuinka monta sokerikakkua Timo voi leipoa?  
	\begin{vastaus}
Munia riittäisi viiteen, sokeria kuuteen ja jauhoja seitsemään kokonaiseen kakkuun. Kaikki aineksia tarvitaan, joten Timo voi leipoa vain viisi kakkua.
	\end{vastaus}
\end{tehtava}

\begin{tehtava}
Pullareseptin\footnote{http://kotiliesi.fi/ruoka/reseptit/pullataikina} mukaan taikinaan tarvitaan $5$\,dl maitoa, $2$\,dl sokeria ja $1,5$\,l vehnäjauhoja. Tällöin saadaan $40$ pikkupullaa.
	\alakohdat{
		§ Taija haluaa leipoa $16$\,pullaa. Kuinka paljon hän tarvitsee maitoa, sokeria ja vehnäjauhoja?
		§ Jos Taijalla on käytössä vain $3$\,dl maitoa, kuinka monta pikkupullaa hän voi enintään leipoa?
	} 
	\begin{vastaus}
		\alakohdat{
			§ $2$\,dl maitoa, $0,8$\,dl sokeria, $6$\,dl vehnäjauhoja
			§ $24$\,pullaa
		}
	\end{vastaus}
\end{tehtava}

%Tarkistanut V-P Kilpi 2013-11-09
\begin{tehtava}
	Kymmenen miestä kaivaa viisitoista kuoppaa kymmenessä tunnissa. Kuinka kauan viidellä miehellä kestää kaivaa $30$ kuoppaa?	
	\begin{vastaus}
		$40$ tuntia
	\end{vastaus}
\end{tehtava}

\subsubsection*{Hallitse kokonaisuus}

%Laatinut V-P Kilpi 2013-11-09
\begin{tehtava}
Sotkamossa isä lämmittää savusaunaa, jonka käyttöaika on suoraan verrannollinen lämmitysaikaan. Jos saunaa lämmittää kolme tuntia, pysyy se kuumana kaksi tuntia. Perhe haluaa saunoa kolme tuntia kello kahdeksasta eteenpäin. Milloin saunaa pitää alkaa lämmittää?
	\begin{vastaus}
Puoli neljältä
	\end{vastaus}
\end{tehtava}

%Tarkistanut V-P Kilpi
\begin{tehtava}
Äidinkielen kurssilla annettiin tehtäväksi lukea $300$-sivuinen romaani. Eräs opiskelija otti aikaa ja selvitti lukevansa vartissa seitsemän sivua. Kuinka monta tuntia häneltä kuluu koko romaanin lukemiseen, jos taukoja ei lasketa?
    \begin{vastaus}
        Noin $643$ minuuttia eli $10$\,h $43$\,min
    \end{vastaus}
\end{tehtava}

%Tarkistanut V-P Kilpi
\begin{tehtava}
Alkuperäiskansojen perinteitä tutkiva etnologi kerää saamelaisten kansantarinoita. Kierrettyään viikkojen ajan saamelaisperheiden vieraana palaa etnologi yliopistolle mukanaan äänitiedostoiksi tallennettuja haastatteluja. Jotta etnologi voi ryhtyä kirjoittamaan tutkimusraporttia, pitää haastattelutiedostot ensin purkaa tekstiksi eli \emph{litteroida}. Tähän etnologi pyytää apua laitoksen opiskelijoilta. Neljä opiskelijaa litteroi tiedostot $25$\,tunnissa. Kuinka kauan aikaa kuluu, jos työhön saadaan vielä kolme opiskelijaa lisää? Kuinka monta opiskelijaa tarvitaan, jotta työ saataisiin tehdyksi $9$\,tunnissa? Oletetaan, että kaikkien opiskelijoiden työteho on sama.
	\begin{vastaus}
Seitsemällä opiskelijalla työ valmistuu noin $14$\,tunnissa. Opiskelijoita tarvitaan $12$, jolloin työ valmistuu $8$\,tunnissa ja $20$ minuutissa. 
	\end{vastaus}
\end{tehtava}

\begin{tehtava}
	Kappaleen liike-energia on suoraan verrannollinen nopeuden neliöön. Kun kappale liikkui nopeudella $9,0$\,m/s, sillä oli $150$ joulea (J) liike-energiaa.
	\alakohdatm{
	§ Laske kappaleen liike-energia, kun sen nopeus on $5,0$\,m/s.
	§ Mikä saadaan verrannollisuuskertoimen yksiköksi, kun $\mathrm{J}=\mathrm{kg\,m}^2\,\mathrm{s}^{-2}$?
	}
	\begin{vastaus}
	\alakohdatm{
	§ Noin $46$ joulea
	§ kilogramma
	}
	\end{vastaus}
\end{tehtava}

\begin{tehtava}
	Venematka Nauvoon kestää tyynellä säällä $75$ minuuttia. Kuinka monta prosenttia aika lyhenee, jos ostetaan uusi vene, jonka nopeus on
	\alakohdat{
		§ $30\,\%$ suurempi kuin vanhan veneen nopeus $35$\,km/h?
		§ $45\,\%$ suurempi kuin vanhan veneen nopeus $v$?
	}
	\begin{vastaus}
	\alakohdat{
		§ vähenee $23\%$ (nopeammalla veneellä aika on noin $58$\,min)
		§ vähenee $31\%$ (aikaa kuluu noin $52$\,min)
	}
	\end{vastaus}
\end{tehtava}

%Laatinut P Thitz 2014-02-08
\begin{tehtava}
	Puun rungon tilavuutta voidaan arvioida suoran ympyräkartion tilavuuden kaavalla, $V = 1/3\cdot\pi \cdot r^2\cdot h$, missä $h$ kuvaa puun korkeutta ja $r$ puun sädettä.
	\alakohdat{
	§ Vuosien 2012--2013 aikana puun korkeus kasvoi $20\,\%$ ja säde $10\,\%$. Kuinka monta prosenttia puun tilavuus kasvoi?
	§ Samalla aikavälillä puun kuutiohinta (€/m$^3$) laski $35\,\%$. Kuinka paljon puun arvo muuttui (prosentteina)?
	}
	\begin{vastaus}
	\alakohdatm{
	§ Kasvoi $45\,\%$.
	§ Arvo laski noin $6\,\%$.
	}
	\end{vastaus}
\end{tehtava}

%Laatinut V-P Kilpi 2013-11-09
\begin{tehtava}
Pallon muotoisessa lankakerässä langan pituus on verrannollinen pallon halkaisijan kuutioon. Kun lankakerässä on $500$ metriä hyvin ohutta lankaa, on sen halkaisija $4$\,cm. Jättimäisen lankakerän halkaisijan pituus on yksi metri. Kuinka monta kertaa siinä oleva lanka riittäisi levitettäväksi maapallon ympäri? Maapallon ympärysmitta on noin $40\,000$\,km.
	\begin{vastaus}
Lankakerä sisältää $125\,000$\,km lankaa, joten kokonaisia kierroksia maapallon ympäri (isoympyrää pitkin) saadaan kolme.
	\end{vastaus}
\end{tehtava}

\subsubsection*{Lisää tehtäviä}

%Laatinut V-P Kilpi 2013-11-09
\begin{tehtava}
Kun muuttujan $x$ arvo on $6$, muuttuja $y$ saa arvon $12$. Minkä arvon muuttuja $y$ saa silloin, kun $x=18$ ja
	\alakohdat{
§ muuttujat $x$ ja $y$ ovat suoraan verrannollisia?
§ muuttujat $x$ ja $y$ ovat kääntäen verrannollisia?
} 
	\begin{vastaus}
		\alakohdatm{
§ $36$
§ $4$
} 
	\end{vastaus}
\end{tehtava}

%Laatinut V-P Kilpi 2013-11-09
\begin{tehtava}
	Metsäpalon pinta-ala on suoraan verrannollinen palamisajan neliöön. Kahdessa päivässä metsää on palanut neliökilometrin verran. 
	\alakohdat{
§ Kuinka monta neliökilometriä metsää on palanut, kun metsäpalo on kestänyt kuusi päivää?
§ Kuinka monen päivän päästä metsää on palanut yli $30$ neliökilometriä?
} 
	\begin{vastaus}
		\alakohdatm{
§ Yhdeksän neliökilometriä
§ Viiden päivän päästä
} 
	\end{vastaus}
\end{tehtava}

%Laatinut V-P Kilpi 2013-11-09
\begin{tehtava}
	Karkkikulhon tyhjentymisaika on kääntäen verrannollinen syöjien määrään. Kun syöjiä on $20$, kulho syödään tyhjäksi kahdessa tunnissa. Juhliin kutsuttiin sata ihmistä, mutta kulhon tyhjäksi syömiseen kului puoli tuntia. Kuinka moni vieraista ei syönyt karkkeja?
	\begin{vastaus}
		Sadasta vieraasta $20$ ei syönyt karkkeja.
	\end{vastaus}
\end{tehtava}

%Laatinut V-P Kilpi 2013-11-09
\begin{tehtava}
Kenkätehtaalla sata työntekijää valmistaa yhdeksässä tunnissa $500$ paria kenkiä. Tehtaan omistaja päättää siirtää tuotannon halvemman työvoiman maahan, jossa hänellä on varaa palkata $300$ työntekijää. 

\alakohdat{
§ Jos uudet työntekijät ovat yhtä tehokkaita kuin vanhat, kuinka kauan heillä kestää tehdä $500$ paria kenkiä?
§ Uudet työntekijät suostuvat tekemään $15$ tunnin työpäivää. Kuinka monta paria kenkiä yhden työpäivän aikana valmistuu olettaen, että työteho ei hiivu?
} 
	\begin{vastaus}
\alakohdatm{
§ Kolme tuntia
§ $2\,500$ paria kenkiä
} 
	\end{vastaus}
\end{tehtava}

%Laatinut V-P Kilpi 2013-11-09
\begin{tehtava}
	Kuution tilavuus on suoraan verrannollinen sen pinta-alan neliöjuuren kolmanteen potenssiin. Kun kuution tilavuus on $64$\,m$^3$, ja sen pinta-ala on $96$\,m$^2$. Jos kuution pinta-ala on $100$\,m$^2$, mikä on sen tilavuus?
	\begin{vastaus}
		${\sqrt{\frac{100}{6}}}^{3} \approx 68$
	\end{vastaus}
\end{tehtava}

\begin{tehtava} $\star$ Kappaleen putoamisen kesto maahan korkeudelta $x$ on kääntäen verrannollinen putoamiskiihtyvyyden $g$ neliöjuureen. Vakio $g$ on kullekin taivaankappaleelle ominainen ja eri puolilla taivaankappaletta likimain sama. Empire State Buildingin katolta (korkeus $381$ metriä) pudotetulla kuulalla kestää noin $6,2$ sekuntia osua maahan. Marsin putoamiskiihtyvyys on $37,6\,\%$ Maan putoamiskiihtyvyydestä. Jos Empire State Building sijaitsisi Marsissa, kuinka kauan kuluisi kuulan maahan osumiseen?
    \begin{vastaus}
        Noin $10$ sekuntia
    \end{vastaus}
\end{tehtava}
%se lankkutehtävä!
\end{tehtavasivu}