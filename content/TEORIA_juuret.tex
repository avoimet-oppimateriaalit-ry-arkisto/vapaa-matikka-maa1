\subsection*{Neliöjuuri}
\label{neliojuuri}

Ajatellaan, että neliön pinta-ala on $a$. Halutaan tietää, mikä on kyseisen neliön sivun pituus. Vastausta tähän kysymykseen kutsutaan luvun $a\ge 0$ \termi{neliöjuuri}{neliöjuureksi} ja merkitään $\sqrt{a}$. Luvun $a$ neliöjuuri on myös yhtälön $x^2 = a$ ratkaisu. Tällöin täytyy kuitenkin huomata, että myös luku $x=-\sqrt{a}$ toteuttaa kyseisen yhtälön. Neliöjuurella tarkoitetaan kyseisen yhtälön epänegatiivista ratkaisua. Tämä on luonnollista, koska neliön sivun pituus ei voi olla negatiivinen luku.

\laatikko[Neliöjuuri]{Luvun $a$ neliöjuuri on epänegatiivinen luku, jonka neliö on $a$. Tämä voidaan ilmaista myös $(\sqrt{a})^2=a$.}

Rajoitumme lukiokursseilla käsittelemään reaalilukuja, emmekä voi ottaa neliöjuurta negatiivisesta luvusta, koska jokaisen reaaliluvun neliö on positiivinen tai nolla. (Toisin sanoen ei ole esimerkiksi mitään sellaista reaalilukua $a$, että $a^2=-2$.) Huomionarvoista on tällöin, että parillisen juuren tuloksena ei voi myöskään olla negatiivinen luku.

\begin{esimerkki}
Laske
\alakohdat{
§ $\sqrt{4}$
§ $\sqrt{144}$
§ $\sqrt{4\,471}$.
}

\begin{esimratk}

\alakohdat{
§ Laskimella tai päässä laskemalla saadaan, että $\sqrt{4} = 2$, koska $2\geq0$ ja $2^2 =4$.
§ $\sqrt{144}=12$. Tämä voidaan vielä tarkistaa laskemalla $12^2 = 12\cdot 12=144$.
§ $\sqrt{4471}\approx 66,9$. Vertailun vuoksi laskimella saadaan myös $67\cdot 67=4\,489$.
}
\end{esimratk}
\begin{esimvast}
\alakohdat{
§ $2$
§ $12$
§ $66,9$
}
\end{esimvast}
\end{esimerkki}


\subsection*{Kuutiojuuri}

Kuution tilavuus on $a$. Halutaan tietää, mikä on kyseisen kuution sivun pituus. Vastausta tähän kysymykseen kutsutaan luvun $a\ge 0$ \termi{kuutiojuuri}{kuutiojuureksi} ja merkitään $\sqrt[3]{a}$. Luvun $a$ kuutiojuuri on myös yhtälön $x^3 = a$ vastaus.

\laatikko[Kuutiojuuri]{Luvun $a$ kuutiojuuri on luku, jonka kuutio on $a$. Tämä voidaan ilmaista myös $(\sqrt[3]{a})^3=a$.}

Jos $a<0$, niin kysymys kuutiosta jonka tilavuus on $a$, ei ole mielekäs. Tästä huolimatta kuutiojuuri määritellään myös negatiivisille luvuille. Syy tähän on, että yhtälöllä $x^3=a$ on tässäkin tapauksessa yksikäsitteinen ratkaisu. Positiivisen luvun kuutiojuuri on aina positiivinen ja negatiivisen luvun kuutiojuuri on aina negatiivinen luku. Kuutiojuuren voi siis ottaa mistä tahansa reaaliluvusta.


\begin{esimerkki}
Laske
\alakohdat{
§ $\sqrt[3]{27}$
§ $\sqrt[3]{1\,397}$
§ $\sqrt[3]{2\,197}$.
}

\begin{esimratk}
\alakohdat{
§ Laskimella tai päässä laskemalla saadaan, että $\sqrt[3]{27} = 3$, koska  $3^3 =3\cdot 3\cdot 3=27$.
§ $\sqrt[3]{1\,397}\approx 11,18$.
§ $\sqrt[3]{2\,197}=13$
}
Tämä voidaan vielä tarkistaa laskemalla $13^3 = 13\cdot 13\cdot 13=2\,197$.
\end{esimratk}

\begin{esimvast}
\alakohdat{
§ $3$
§ $11,18$
§ $13$
}
\end{esimvast}
\end{esimerkki}

\subsection*{Korkeammat juuret}

Yhtälön $x^n=a$ ratkaisujen avulla voidaan määritellä \termi{$n$:s juuri}{$n$:s juuri} mille tahansa positiiviselle kokonaisluvulle $n$. Neliö- ja kuutiojuurten tapauksessa voidaan kuitenkin huomata, että kuutiojuuri on määritelty kaikille luvuille, mutta neliöjuuri vain epänegatiivisille luvuille. Tämä toistuu myös korkeammissa juurissa: parilliset ja parittomat juuret on määritelty eri joukoissa.

Juurimerkinnällä $\sqrt[n]{a}$ (luetaan $n$:s juuri luvusta $a$) tarkoitetaan lukua, joka toteuttaa ehdon $(\sqrt[n]{a})^n = a$. Jotta juuri olisi yksikäsitteisesti määritelty, asetetaan lisäksi, että parillisessa tapauksessa $n$:s juuri tarkoittaa kyseisen yhtälön epänegatiivista ratkaisua.
%($\sqrt{a}, \sqrt[4]{a}, \sqrt[6]{a}$\ldots) vaadittava, että $b\ge0$.

\laatikko[Parillinen juuri]{
Kun $n$ on parillinen luku, luvun $a \ge 0$ $n$:s juuri on luku $b \ge 0$, jonka $n$:s potenssi on $a$, eli $b^n = a$. Lukua $b$ merkitään $\sqrt[n]{a}$, eli $\left(\sqrt[n]{a}\right)^n = a$. Edelleen, kun $n$ on parillinen luku, $n$:ttä juurta ei ole määritelty negatiivisille luvuille.
}

Luvun toista juurta, eli neliöjuurta $\sqrt[2]{a}$ merkitään yleensä yksinkertaisemmin $\sqrt{a}$.

\laatikko[Pariton juuri]{
Kun $n$ on pariton luku, luvun $a \in \rr$ $n$:s juuri on yksikäsitteinen luku $b \in \rr$, jonka $n$:s potenssi on $a$, eli $b^n = a$. Lukua $b$ merkitään $\sqrt[n]{a}$, eli $\left(\sqrt[n]{a}\right)^n = a$. Erityisesti $n$:s juuri on määritelty parittomilla $n$ myös negatiivisille luvuille toisin kuin $n$:s juuri parillisilla $n$.
}

Korkeampien juurten laskeminen tapahtuu tavallisesti laskimella kuin myös juurenotto suhteellisen monimutkaisista luvuista ja lausekkeista.

\begin{esimerkki}
Laske
\alakohdat{
§ $\sqrt[4]{256}$
§ $\sqrt[5]{-243}$
§ $\sqrt[4]{-8}$.
}

	\begin{esimratk}
	\alakohdat{
§ Laskimella saadaan $\sqrt[4]{256}=4$, koska $4^4=256$.
§ Laskimella $\sqrt[5]{-243}=-3$, koska $(-3)^5=-243$.
§ Luvun $-8$ neljäs juuri $\sqrt[4]{-8}$ ei ole määritelty, koska minkään luvun neljäs potenssi ei ole negatiivinen.
	}
	\end{esimratk}
	
	\begin{esimvast}
	\alakohdat{
	§ $4$
	§ $-3$
	§ ei määritelty
	}
	\end{esimvast}
\end{esimerkki}

\subsection*{Juurten laskusäännöt}
Juurille on olemassa laskusääntöjä, jotka voivat auttaa sieventämisessä ja ratkaisujen löytämisessä. 

%vaihda laatikkotyyppi
\numerointilaatikko{Laskusäännöt}{
	§ Ehdolla, että $x\geq0$, $y>0$ ja $m$ sekä $n$ ovat positiivisia kokonaislukuja.
	§§ $\sqrt[m]{x}\cdot{\sqrt[m]{y}}=\sqrt[m]{xy}$
	§§ $\frac{\sqrt[m]{x}}{\sqrt[m]{y}}=\sqrt[m]{\frac{x}{y}}$ 
	§§ $\sqrt[m]{\sqrt[n]{x}}=\sqrt[mn]{x}$
}
Lisäksi parillisilla $m$:n arvoilla $\sqrt[m]{x^m}=|x|$, missä merkintä $|x|$ tarkoittaa luvun $x$ itseisarvoa. Itseisarvo ilmaisee lukusuoralla luvun etäisyyttä nollasta, joten se on aina epänegatiivinen.

\begin{esimerkki}
\alakohdat{
§ $\sqrt[6]{4^6}=|4| = 4$
§ $\sqrt[2]{(-5)^2}=|-5| = 5$
§ $\sqrt[12]{0^{12}}=|0|=0$
}
\end{esimerkki}

Itseisarvo määritellään tarkemmin luvussa 3.1. %FIXME: viittausympäristö?

Laskusäännöt perustuvat juuren määritelmään ja potenssin laskusääntöihin. Esimerkiksi määritelmän mukaan $\sqrt[m]{xy}$ on sellainen luku, että $\left( \sqrt[m]{xy} \right)^m = xy$. Luku $\sqrt[m]{x} \cdot \sqrt[m]{y}$ täyttää edellisen vaatimuksen, sillä $\left( \sqrt[m]{x} \cdot \sqrt[m]{y} \right)^m = \left( \sqrt[m]{x} \right)^m \cdot \left( \sqrt[m]{y} \right)^m = xy$.

\begin{esimerkki}
Laske
\alakohdat{
§ $\sqrt{3}\cdot\sqrt{48}$
§ $\frac{\sqrt[4]{32}}{\sqrt[4]{2}}$
§ $\sqrt[x]{\sqrt{4^x}}$, missä $x\in\nn$
§ $\sqrt[8]{515^8}$.
}

\begin{esimratk}
\alakohdat{
§ $\sqrt{3}\cdot{\sqrt{48}}=\sqrt{{3}\cdot{48}}=\sqrt{144}=12$ 
§ $\frac{\sqrt[4]{32}}{\sqrt[4]{2}}=\sqrt[4]{\frac{32}{2}}=\sqrt[4]{16}=2$ 
§ $\sqrt[x]{\sqrt{4^x}}=\sqrt[x\cdot2]{4^x}=\sqrt{\sqrt[x]{4^x}}=\sqrt{4}=2$
§ $\sqrt[8]{515^8}=|515|=515$
}
\end{esimratk}

\begin{esimvast}
\alakohdat{§$12$
§ $2$
§ $2$
§ $515$}
\end{esimvast}
\end{esimerkki}

Sievennystapana on, että juuren sisälle jätetään mahdollisimman yksinkertainen (itseisarvoltaan pieni, vähän numeroita sisältävä) luku. Tällöin pyritään jakamaan juurrettava alkulukutekijöihin, joista yhden tai useamman potenssi vastaava otetun juuren kertalukua. Vaikeissa tapauksissa sopivan tulomuodon etsimiseen voi käyttää esimerkiksi laskinta. Huomaa, että oikeita lopullisia vastauksia on vain yksi oikea, eikä optimaalisinta tulomuotoa tarvitse keksiä heti ensimmäisenä.

%\begin{esimerkki}
%Esitä luvut tulomuodossa, jossa...
%\end{esimerkki}

\begin{esimerkki}
Sievennä juurilausekkeet. \alakohdat{§ $\sqrt{20}$ § $\sqrt{\frac{9}{16}}$ § $\sqrt[3]{120}$ § $\sqrt{10}$}

	\begin{esimratk}
	\alakohdat{
§ Koska $20$ voidaan kirjoittaa tulona $4\cdot5$, on $\sqrt{20}=\sqrt{4\cdot 5}$. Juurten tulosäännön nojalla tämä voidaan kirjoittaa kahden neliöjuuren tulona $\sqrt{4}\sqrt{5}$, joka sievenee muotoon $2\sqrt{5}$, koska luku neljä on neliöluku.
§ $\sqrt{\frac{9}{16}}=\frac{\sqrt{9}}{\sqrt{16}}=\frac{\sqrt{3^2}}{\sqrt{4^2}}=\frac{3}{4}$
§ Kirjoitetaan luku $120$ tulona, jossa esiintyisi jokin kuutioluku, eli jonkin kokonaisluvun kolmas potenssi. Esimerkiksi $120=8\cdot15$ kelpaa, sillä $8=2^3$. Tällöin $\sqrt[3]{120}=\sqrt[3]{8\cdot 15}=\sqrt[3]{2^3\cdot 15}=\sqrt[3]{2^3}\sqrt[3]{15}=2\sqrt[3]{15}$. (Luvun $15$ kuutiojuurta ei enää sievennettyä.)
§ Luvun $10$ alkulukukehitelmä on $10=2\cdot 5$, ja siinä ei ole yhtäkään toista potenssia, josta voisi päästä neliöjuurella eroon. $\sqrt{10}$ on siis jo mahdollisimman sievässä muodossa.
	}
	\end{esimratk}
	\begin{esimvast}
	\alakohdat{§ $2\sqrt{5}$ § $\frac{3}{4}$ § $2\sqrt[3]{15}$ § $\sqrt{10}$ (valmiiksi sievin muoto)}
	\end{esimvast}
	
	%+aitojen desimaalilukujen juuren sieventäminen
\end{esimerkki}