\subsection*{Neliöjuuri}
\label{neliojuuri}

Ajatellaan, että neliön pinta-ala on $a$. Halutaan tietää, mikä on kyseisen neliön sivun pituus. Vastausta tähän kysymykseen kutsutaan luvun $a\ge 0$ \termi{neliöjuuri}{neliöjuureksi} ja merkitään $\sqrt{a}$. Luvun $a$ neliöjuuri on myös yhtälön $x^2 = a$ ratkaisu. Tällöin täytyy kuitenkin huomata, että myös luku $x=-\sqrt{a}$ toteuttaa kyseisen yhtälön. Neliöjuurella tarkoitetaan kyseisen yhtälön epänegatiivista ratkaisua. Tämä on luonnollista, koska neliön sivun pituus ei voi olla negatiivinen luku.

\laatikko{Luvun $a$ neliöjuuri on epänegatiivinen luku, jonka neliö on $a$. Tämä voidaan ilmaista myös $(\sqrt{a})^2=a$.}

Neliöjuurta ei tällä kurssilla määritellä negatiivisille luvuille, koska jokaisen reaaliluvun neliö on positiivinen tai nolla.
(Toisin sanoen ei ole esimerkiksi mitään sellaista reaalilukua $a$, että $a^2 = -2$.)
Käytännössä lukujen neliöjuuria lasketaan usein laskimella.


\begin{esimerkki}
Laske
\begin{alakohdat}
\alakohta{$\sqrt{4}$}

\alakohta{$\sqrt{144}$}

\alakohta{$\sqrt{4471}$.}
\end{alakohdat}

{\bf Ratkaisut.}

a)
Laskimella tai päässä laskemalla saadaan, että $\sqrt{4} = 2$, koska $2\geq0$ ja $2^2 =4$.

b) 
$\sqrt{144}=12$. Tämä voidaan vielä tarkistaa laskemalla $12^2 = 12\cdot 12=144$.

c)
$\sqrt{4471}\approx 66,9$. Vertailun vuoksi laskimella saadaan myös $67\cdot 67=4489$.

{\bf Vastaukset.}
a) $2$, b) $12$, c) $66,9$.

\end{esimerkki}

\begin{esimerkki}
Taulutelevision kooksi (lävistäjäksi) on ilmoitettu mainoksessa $46,0$ tuumaa ($116,8$ cm) ja kuvasuhteeksi 16:9. Kuinka leveä televisio on (senttimetreinä)?

{\bf Ratkaisu.}

Taulutelevision halkaisija, alareuna ja toinen sivu muodostavat suorakulmaisen kolmion. Kolmion hypotenuusa ($c$) on television halkaisija ja kateetit ($a$ ja $b$) alareuna ja toinen sivu.

Kuvasuhteen perusteella kateettien pituuksia voidaan merkitä $16x$ ja $9x$. Pythagoraan lauseesta ($c^2 = a^2 + b^2$) saadaan
\[
(116,8)^2 = (16x)^2 + (9x)^2
\]
\[
13642,24 = (256+81)x^2.
\]
\[
x^2 = \frac{13642,24}{337}
\]
\[
x= \sqrt{\frac{13642,24}{337}} \approx 6,36.
\]
Television leveys on noin $16x = 16\cdot 6,36\approx 102$ cm.

{\bf Vastaus.} Noin $102$ cm.
\end{esimerkki}



\subsection*{Kuutiojuuri}

Kuution tilavuus on $a$. Halutaan tietää, mikä on kyseisen kuution sivun pituus. Vastausta tähän kysymykseen kutsutaan luvun $a\ge 0$ \termi{kuutiojuuri}{kuutiojuureksi} ja merkitään $\sqrt[3]{a}$. Luvun $a$ kuutiojuuri on myös yhtälön $x^3 = a$ vastaus.

\laatikko{Luvun $a$ kuutiojuuri on luku, jonka kuutio on $a$. Tämä voidaan ilmaista myös $(\sqrt[3]{a})^3=a$.}

Jos $a<0$, niin kysymys kuutiosta jonka tilavuus on $a$, ei ole mielekäs. Tästä huolimatta kuutiojuuri määritellään myös negatiivisille luvuille. Syy tähän on, että yhtälöllä $x^3=a$ on tässäkin tapauksessa yksikäsitteinen ratkaisu. Positiivisen luvun kuutiojuuri on aina positiivinen ja negatiivisen luvun kuutiojuuri on aina negatiivinen luku. Kuutiojuuren voi siis ottaa mistä tahansa reaaliluvusta.


\begin{esimerkki}
Laske
\begin{alakohdat}
\alakohta{$\sqrt[3]{27}$}

\alakohta{$\sqrt[3]{1397}$}

\alakohta{$\sqrt[3]{2197}$.}
\end{alakohdat}

{\bf Ratkaisut.}

a)
Laskimella tai päässä laskemalla saadaan, että $\sqrt[3]{27} = 3$, koska  $3^3 =3\cdot 3\cdot 3=27$.

b) 
$\sqrt[3]{1397}\approx 11,18$. 

c)
$\sqrt[3]{2197}=13$.
Tämä voidaan vielä tarkistaa laskemalla $13^3 = 13\cdot 13\cdot 13=2197$.

{\bf Vastaukset.}

a) $3$, b) $11,18$, c) $13$.
\end{esimerkki}


\subsection*{Korkeammat juuret}


Yhtälön $x^n=a$ ratkaisujen avulla voidaan määritellä \termi{$n$:s juuri}{$n$:s juuri} \index{juuri} mille tahansa positiiviselle kokonaisluvulle $n$. Neliö- ja kuutiojuurten tapauksessa voidaan kuitenkin huomata, että kuutiojuuri on määritelty kaikille luvuille, mutta neliöjuuri vain epänegatiivisille luvuille. Tämä toistuu myös korkeammissa juurissa: parilliset ja parittomat juuret on määritelty eri joukoissa.

Juurimerkinnällä $\sqrt[n]{a}$ (luetaan $n$:s juuri luvusta $a$) tarkoitetaan lukua, joka toteuttaa ehdon $(\sqrt[n]{a})^n = a$. Jotta juuri olisi yksikäsitteisesti määritelty, asetetaan lisäksi, että parillisessa tapauksessa $n$:s juuri tarkoittaa kyseisen yhtälön epänegatiivista ratkaisua.
%($\sqrt{a}, \sqrt[4]{a}, \sqrt[6]{a}$\ldots) vaadittava, että $b\ge0$.



\laatikko{
{\bf Parillinen juuri}

Kun $n$ on parillinen luku, luvun $a \ge 0$ $n$:s juuri on luku $b \ge 0$, jonka $n$:s potenssi on $a$, eli $b^n = a$. Lukua $b$ merkitään $\sqrt[n]{a}$, eli $\left(\sqrt[n]{a}\right)^n = a$. Edelleen, kun $n$ on parillinen luku, $n$:ttä juurta ei ole määritelty negatiivisille luvuille.
}

Luvun toista juurta, eli neliöjuurta $\sqrt[2]{a}$ merkitään yleensä yksinkertaisemmin $\sqrt{a}$.

\laatikko{
{\bf Pariton juuri}

Kun $n$ on pariton luku, luvun $a \in \rr$ $n$:s juuri on yksikäsitteinen luku $b \in \rr$, jonka $n$:s potenssi on $a$, eli $b^n = a$. Lukua $b$ merkitään $\sqrt[n]{a}$, eli $\left(\sqrt[n]{a}\right)^n = a$. Erityisesti $n$:s juuri on määritelty parittomilla $n$ myös negatiivisille luvuille toisin kuin $n$:s juuri parillisilla $n$.
}

Korkeampien juurten laskeminen tapahtuu tavallisesti laskimella.

\begin{esimerkki}
Laske
\begin{alakohdat}
\alakohta{$\sqrt[4]{256}$}
\alakohta{$\sqrt[5]{-243}$}
\alakohta{$\sqrt[4]{-8}$.}
\end{alakohdat}

{\bf Ratkaisut.}

a) Laskimella saadaan $\sqrt[4]{256}=4$, koska $4^4=256$.

b) Laskimella $\sqrt[5]{-243}=-3$, koska $(-3)^5=-243$.

c) Luvun $-8$ neljäs juuri $\sqrt[4]{-8}$ ei ole määritelty, koska minkään luvun neljäs potenssi ei ole negatiivinen.
\end{esimerkki}



%%Lisännyt Aleksi Sipola 09.11.2013
\subsection*{Juurten laskusäännöt}
Juurille on olemassa laskusääntöjä, jotka voivat auttaa sieventämisessä ja ratkaisujen löytämisessä. 


\laatikko{
{\bf Laskusäännöt}

Laskusäännöt pätevät, kun $x\geq0$, $y>0$ ja $m$ sekä $n$ positiivisia kokonaislukuja.
\begin{alakohdat}
\alakohta{$\sqrt[m]{x}\cdot{\sqrt[m]{y}}=\sqrt[m]{xy}$}
\alakohta{$\frac{\sqrt[m]{x}}{\sqrt[m]{y}}=\sqrt[m]{\frac{x}{y}}$ }
\alakohta{$\sqrt[m]{\sqrt[n]{x}}=\sqrt[mn]{x}$}
\end{alakohdat}

Lisäksi parillisilla $m$:n arvoilla $\sqrt[m]{x^m}=|x|$
 


}

Huomionarvoista on, että parillisen juuren tuloksena ei voi olla negatiivinen luku.

Laskusäännöt perustuvat juuren määritelmään ja potenssin laskusääntöihin. Esimerkiksi
määritelmän mukaan $\sqrt[m]{xy}$ on sellainen luku, että $\left( \sqrt[m]{xy} \right)^m = xy$.
Luku $\sqrt[m]{x} \cdot \sqrt[m]{y}$ täyttää edellisen vaatimuksen, sillä
$\left( \sqrt[m]{x} \cdot \sqrt[m]{y} \right)^m = \left( \sqrt[m]{x} \right)^m \cdot \left( \sqrt[m]{y} \right)^m = xy$.



\begin{esimerkki}
Laske
\begin{alakohdat}
\alakohta{$\sqrt{3}\cdot\sqrt{48}$}
\alakohta{$\frac{\sqrt[4]{32}}{\sqrt[4]{2}}$}
\alakohta{$\sqrt[x]{\sqrt{4^x}}$, $x$ on eräs luonnollinen nollaa suurempi luku}
\alakohta{$\sqrt[8]{515^8}$.}
\end{alakohdat}

{\bf Ratkaisut.}
\begin{alakohdat}
\alakohta{$\sqrt{3}\cdot{\sqrt{48}}=\sqrt{{3}\cdot{48}}=\sqrt{144}=12$ }
\alakohta{$\frac{\sqrt[4]{32}}{\sqrt[4]{2}}=\sqrt[4]{\frac{32}{2}}=\sqrt[4]{16}=2$ }
\alakohta{$\sqrt[x]{\sqrt{4^x}}=\sqrt[x\cdot2]{4^x}=\sqrt{\sqrt[x]{4^x}}=\sqrt{4}=2$}
\alakohta{$\sqrt[8]{515^8}=|515|=515$}
\end{alakohdat}

\end{esimerkki}



