Kvanttorit $\forall$ ja $\exists$ merkitsevät ''kaikilla'' ja ''on olemassa,'' vastaavasti.

Reaalilukujen aksiomaattinen määritelmä muodostuu kolmesta osasta:

\paragraph*{Kunta-aksioomat, kuntana reaaliluvut}
\begin{description}
\item[K1.] $\forall x, y \in \rr: \, x+(y+z) = (x+y)+z$ (summan liitäntälaki)
\item[K2.] $\exists 0 \in \rr: \, x+0 = x$ (summan neutraalialkio)
\item[K3.] $\forall x \in \rr: \, \exists (-x) \in \rr: \, x+(-x)=0$ (vasta-alkio)
\item[K4.] $\forall x, y \in \rr: \, x+y = y+x$ (summan vaihdantalaki)
\item[K5.] $\forall x, y, z \in \rr: \, x \cdot (y+z) = x \cdot y + x \cdot z$ (osittelulaki)
\item[K6.] $\forall x, y, z \in \rr: \, x \cdot (y \cdot
§ = (x \cdot
§ \cdot z$ (tulon liitäntälaki)
\item[K7.] $\exists 1 \in \rr, 1 \neq 0: \, 1 \cdot x = x$ (tulon neutraalialkio)
\item[K8.] $\forall x \in \rr \setminus \{0\}: \, \exists x^{-1} \in \rr \setminus \{0\}: \, x \cdot x^{-1}=1$ (tulon käänteisalkio)
\item[K9.] $\forall x, y \in \rr: \, x \cdot y = y \cdot x$ (tulon vaihdantalaki)
\end{description}

\paragraph*{Järjestysaksioomat}
\begin{description}
\item[J1.] $\forall x, y \in \rr: \, \text{täsmälleen yksi seuraavista:} \, (x > y), \, (x = y), \, (x < y)$
\item[J2.] $\forall x, y, z \in \rr: \, (x <
§ \land (y <
§ \Rightarrow (x < z)$
\item[J3.] $\forall x, y, z \in \rr: \, (x <
§ \Leftrightarrow (x + z < y + z)$
\item[J4.] $\forall x, y \in ]0,\infty[: \, x \cdot y \in ]0,\infty[$
\end{description}

\paragraph*{Täydellisyysaksiooma}
\begin{description}
\item[T1.] Jokaisella ylhäältä rajoitetulla epätyhjällä reaalilukujen osajoukolla on pienin yläraja.
\end{description}

\begin{tehtava}
	Todista aksioomista lähtien:
	\alakohdat{
		§ $\forall x \in \rr: 0 \cdot x = 0$
		§ $\forall x \in \rr: -1 \cdot x = -x$
		§ $\forall x, y \in ]-\infty,0[: x \cdot y \in ]0,\infty[$
		§ $1 > 0$
	}
\end{tehtava}
