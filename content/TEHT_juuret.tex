\begin{tehtavasivu}

\paragraph*{Opi perusteet}

Laske
\begin{tehtava}
§ $\sqrt{64}$ 
§ $\sqrt{-64}$ 
§ $\sqrt[3]{64}$ 
§ $\sqrt[3]{-64}$ \quad 
§ $\sqrt{100}$

\begin{vastaus}
§ $8$
§ Ei määritelty
§ $4$
§ $-4$ 
§ $10$
\end{vastaus}
\end{tehtava}

\begin{tehtava}
§ $\sqrt[4]{81}$ 
§ $\sqrt[4]{-81}$ 
§ $\sqrt[5]{32}$ 
§ $\sqrt[5]{-32}$ 
§ $\sqrt[3]{1\,000}$

\begin{vastaus}
§ $3$
§ Ei määritelty
§ $2$
§ $-2$ 
§ $10$
\end{vastaus}
\end{tehtava}

%Tehtävät laatinut Aleksi Sipola 10.11.2013
%Ratkaisut tehnyt Aleksi Sipola 10.11.2013
\begin{tehtava}
% Laske.
§ $\sqrt{81}$ 
§ $\sqrt[3]{27}$ 
§ $\sqrt[4]{-81}$ 
§ $\sqrt[8]{256}$ 

\begin{vastaus}
§ $9$
§ $3$
§ Ei määritelty
§ $2$
\end{vastaus}
\end{tehtava}

%Tehtävät laatinut Aleksi Sipola 09.11.2013
%Ratkaisut tehnyt Aleksi Sipola 09.11.2013
\begin{tehtava} %alakohtaympäristöön!
% Laske. 
§ $\sqrt{2}\sqrt{8}$  
§  $\sqrt{3}\sqrt{5}$ 
§  $\sqrt{3}\sqrt{27}$ \quad 
§ $\sqrt[3]{3}\sqrt[3]{25} $
\begin{vastaus}
§ $4$ 
§ $\sqrt{15}$ (Ei sievene enempää.)  
§ $9$ 
§ $5$
\end{vastaus}
\end{tehtava}


%Tehtävät laatinut Aleksi Sipola 09.11.2013
%Ratkaisut tehnyt Aleksi Sipola 09.11.2013
\begin{tehtava} 
% Laske. 
§ $ \frac{\sqrt{8}}{\sqrt{2}}$  
§   $ \frac{\sqrt[3]{3} \cdot \sqrt[3]{9}}{\sqrt[3]{8}}$   
§  $ \frac{\sqrt{3} \cdot \sqrt{6}}{\sqrt{2}}$ 
§ $ \frac {\sqrt[3]{2}}{\sqrt[3]{16}} \cdot \sqrt[3]{4}$ 
\begin{vastaus}
§ $2$ 
§ $\frac{3}{2}$  
§ $3$ 
§ $\frac{1}{\sqrt{2}}$
\end{vastaus}
\end{tehtava}


\begin{tehtava}
Laske $\sqrt{8\,100}$ päässä ajattelemalla juurrettava luku sopivana tulona.
\begin{vastaus}
$\sqrt{8100}=\sqrt{81}\sqrt{100}=9\cdot 10$.
\end{vastaus}
\end{tehtava}

%Tehtävät laatinut Aleksi Sipola 09.11.2013
%Ratkaisut tehnyt Aleksi Sipola 09.11.20133
\begin{tehtava}
Pienen kuution sivu on puolet ison kuution sivusta. Kuinka pitkä pienen kuution sivu on, jos ison kuution tilavuus on 64 yksikköä? \\
\begin{vastaus}
Pienen kuution sivu on $\sqrt[3]{64}/2=2$ 
\end{vastaus}
\end{tehtava}

\begin{tehtava}
%Laatinut Jaakko Viertiö 2013-11-10
 Sievennä
	\alakohdat{
		§ $\sqrt{\sqrt[3]{27}}$
		§ $\sqrt[3]{{\sqrt[16]{{\sqrt{2}}^{16}}}^2}$
		§ ${\sqrt{\sqrt[3]{{\sqrt[4]{\sqrt[5]{{\sqrt[6]{42}}^{22}}}}^2}}}^3$.
	}
	  \begin{vastaus}
		\alakohdat{
		 § $\sqrt{3}$
		 § $\sqrt[3]{2}$
		 § ${\sqrt[60]{42}}^{11}$
		}
	\end{vastaus}
\end{tehtava}


\paragraph*{Hallitse kokonaisuus}

Laske.

%Tehtävät laatinut Aleksi Sipola 09.11.2013
%Ratkaisut tehnyt Aleksi Sipola 09.11.2013
\begin{tehtava}
% Laske. 
§ $ \frac{\sqrt{15}}{\sqrt{7}} \cdot  \frac{\sqrt{27}}{\sqrt{35}}$  
§  $ \frac{\sqrt{64}-\sqrt{9}}{\sqrt{5}}$   
§  $ \frac{2 \cdot \sqrt[3]{32}}{\sqrt[3]{4}}$ \quad 
\begin{vastaus}
§ $9/7$ 
§ $\sqrt{5}$ 
§ $4$ 
\end{vastaus}
\end{tehtava}

%Tehtävät laatinut Aleksi Sipola ja Jaakko Viertiö 10.11.2013
%Ratkaisut tehnyt Aleksi Sipola ja Jaakko Viertiö 10.11.2013
\begin{tehtava} 
% Laske. 
§ $ \frac{3\sqrt{5}+5\sqrt{3}-\sqrt{15}}{\sqrt{15}}$  
§  $ \frac{2-\sqrt{2}}{\sqrt{2}-1}$   
\begin{vastaus}
§ $\sqrt{3}+\sqrt{5}-1$ 
§ $\sqrt{2}$ 
\end{vastaus}
\end{tehtava}

%Tehtävät laatinut Aleksi Sipola 09.11.2013
%Ratkaisut tehnyt Aleksi Sipola 09.11.2013
\begin{tehtava} Kumpi juurista on suurempi? Arvaa ja laske.
§ $\sqrt[3]{8}$ vai $\sqrt{16}$ 
§  $\sqrt{8}$ vai $2\sqrt{2}$  
§ $\sqrt[3]{64}$ vai $\sqrt[5]{32}$ 
§ $\sqrt[2]{121}$ vai $\sqrt[5]{243}$ 
\begin{vastaus}
§ $\sqrt[3]{8}=2<\sqrt{16}=4$ 
§ $\sqrt{8}=\sqrt{4}\sqrt{2}=2\sqrt{2} = 2\sqrt{2}$ 
§ $\sqrt[3]{64}=4>\sqrt[5]{32}=2$ 
§ $\sqrt[2]{121}=11 >\sqrt[5]{243}=3$ 
\end{vastaus}
\end{tehtava}


\begin{tehtava}
%Tehtävän laatinut Johanna Rämö 9.11.2013.
%Ratkaisun tehnyt Johanna Rämö 9.11.2013.
        Selvitä ilman laskinta, kumpi luvuista $3\sqrt{2}$ ja $2\sqrt{3}$ on suurempi. 
       
        \begin{vastaus}
        Korotetaan luvut toiseen potenssiin: $(3\sqrt{2})^2=3^2\cdot\sqrt{2}^2=9 \cdot 2=18$ ja $(2\sqrt{3})^2=2^2\cdot\sqrt{3}^2=4 \cdot 3=12$. Koska $3\sqrt{2}$ ja $2\sqrt{3}$ ovat molemmat lukua 1 suurempia, voidaan niiden keskinäinen suuruusjärjestys lukea neliöiden suuruusjärjestyksestä. Edellisten laskujen perusteella $(3\sqrt{2})^2 > (2\sqrt{3})^2$, joten $3\sqrt{2} > 2\sqrt{3}$.
        \end{vastaus}
\end{tehtava}

\begin{tehtava}
%Tehtävän laatinut Johanna Rämö 9.11.2013.
%Ratkaisun tehnyt Johanna Rämö 9.11.2013.
Mitä tapahtuu neliöjuuren arvolle, kun juurrettavaa kerrotaan luvulla $100$? Miten tulos yleistyy?
       
        \begin{vastaus}
        Neliöjuuri kasvaa 10-kertaiseksi. Reaalilukujen $a$ ja $100a$ neliöjuuret ovat nimitäin $\sqrt{a}$ ja $\sqrt{100a}=\sqrt{100}\sqrt{a}=10\sqrt{a}$.
        
        Yleisesti jos juurrettavaa kerrotaan reaaliluvulla $k$, neliöjuuri kasvaa $\sqrt{k}$-kertaiseksi.
        \end{vastaus}
\end{tehtava}

%Antti, lisää tämä
\begin{tehtava}
Laske seuraavat neliöjuuret laskimella\\
§ $\sqrt{320}$,\
§ $\sqrt{15}$,\
§ $\sqrt{71}$.
\begin{vastaus}
§ $17,89$
§ $3,87$
§ $8,43$
\end{vastaus}
\end{tehtava}

\begin{tehtava}
Oletetaan että suorakaiteen leveyden suhde korkeuteen on $2$ ja suorakaiteen pinta-ala on $10$. Mikä on suorakaiteen 
leveys ja korkeus?
\begin{vastaus}
Suorakaide muodostuu kahdesta vierekkäisestä neliöstä, joiden pinta-ala on $5$. Tämän neliön sivun pituus on $\sqrt{5}$. Siis suorakaiteen korkeus on $\sqrt{5}$ ja leveys $2\sqrt{5}$.
\end{vastaus}
\end{tehtava}

\begin{tehtava}
Etsi luku $a>0$ jolle $a^4=83\,521$.
\begin{vastaus}
$a=\sqrt{\sqrt{83\,521}}$.
\end{vastaus}
\end{tehtava}

\begin{tehtava}
Laske luvun $10$ potensseja: $10^1, 10^2, 10^3, 10^4, \ldots$ Kuinka monta nollaa on luvussa $10^n$? Laske sitten $\sqrt[6]{1\,000\,000}$ ja $\sqrt[10]{10\,000\,000\,000}$.

\begin{vastaus}
$10^1 = 10, 10^2 = 100, 10^3 = 1\,000, 10^4 = 10\,000$. Luvussa $10^n$ on $n$ kappaletta nollia. Niinpä $\sqrt[6]{1\,000\,000} = 10$ ja $\sqrt[10]{10\,000\,000\,000} = 10$.
\end{vastaus}
\end{tehtava}

\begin{tehtava}
Onko annettu juuri määritelty kaikilla luvuilla $a$? Millaisia arvoja juuri voi saada luvusta $a$ riippuen?\\
§ $\sqrt[4]{a^2}$ 
§ $\sqrt[4]{-a^2}$ 
§ $\sqrt[4]{(-a)^2}$ 
§ $- \sqrt[4]{a^2}$

\begin{vastaus}

\alakohdat{
	§ Juuri on määritelty kaikilla luvuilla $a$, koska kaikkien lukujen neliöt ovat vähintään nolla. Vastaus on aina epänegatiivinen.
	§ Juuri on määritelty vain luvulla $a = 0$. Muilla $a$:n arvoilla $-a^2$ on negatiivinen, jolloin parillinen juuri ei ole määritelty. Ainoa vastaus, joka voidaan saada, on siis $\sqrt[4]{0} = 0$.
	§ Juuri on määritelty kaikilla luvuilla $a$, koska $(-a)^2$ on aina vähintään nolla. Vastaus on aina epänegatiivinen.
	§ Juuri on määritelty kaikilla luvuilla $a$, koska kaikkien lukujen neliöt ovat vähintään nolla. Vastaus on aina ei-positiivinen, koska $\sqrt[4]{a^2}$ on aina epänegatiivinen.
}
\end{vastaus}
\end{tehtava}

\begin{tehtava}
Onko annettu juuri määritelty kaikilla luvuilla $a$? Millaisia arvoja juuri voi saada luvusta $a$ riippuen?\\
§ $\sqrt[5]{a^2}$ 
§ $\sqrt[5]{a^3}$ 
§ $\sqrt[5]{-a^2}$ 
§ $- \sqrt[5]{a^2}$

\begin{vastaus}

\alakohdat{
	§ Juuri on määritelty kaikilla luvuilla $a$. Vastaus on aina epänegatiivinen.
	§ Juuri on määritelty kaikilla luvuilla $a$. Vastaus voi olla mikä tahansa luku.
	§ Juuri on määritelty kaikilla luvuilla $a$. Vastaus on aina ei-positiivinen.
	§ Juuri on määritelty kaikilla luvuilla $a$. Vastaus on aina ei-positiivinen.
}
\end{vastaus}
\end{tehtava}

%Tehtävät laatinut Aleksi Sipola 09.11.2013
%Ratkaisut tehnyt Aleksi Sipola 09.11.2013
\begin{tehtava} Etsi positiivinen rationaaliluku a, jolla
§ $\sqrt{a} \sqrt{2} = 4$  
§   $ \sqrt{a}\cdot{\sqrt{3}} =9 $.   \quad 
\begin{vastaus}
§ $a=8$ 
§ $a=27$ \quad 
\end{vastaus}
\end{tehtava}


%Tehtävät laatinut Aleksi Sipola 09.11.2013
%Ratkaisut tehnyt Aleksi Sipola 09.11.2013
\begin{tehtava} Etsi jotkin erisuuret ykköstä suuremmat luonnolliset luvut a ja b, jolla 
§ $\frac{\sqrt{a}}{\sqrt{b}}$ on jokin luonnollinen luku  
§ $\sqrt[b]{a}=2$.   
\begin{vastaus}
§ Esimerkiksi $a=27$ \quad ja \quad $b=3$  
§ Esimerkiksi $a=16$ \quad ja \quad $b=4$ 
\end{vastaus}
\end{tehtava}



\paragraph*{Lisää tehtäviä}

\begin{tehtava}
Sievennä $(3+\sqrt{3}x)^4:(\sqrt{3}+x)^3$.
\begin{vastaus}
$9(3 + \sqrt{3}x)$
\end{vastaus}
\end{tehtava}

%Tehtävät laatinut Aleksi Sipola 09.11.2013
%Ratkaisut tehnyt Aleksi Sipola 09.11.2013
\begin{tehtava} Laske laskimella likiarvot viiden merkitsevän luvun tarkkuudella. Mikä yhteys luvuilla on?
§ $ \sqrt{1^2+1^2}$ 
§  $ \frac {\sqrt{3^2+3^2}}{3}$    
§  $ \frac {\sqrt{9^2+9^2}}{9}$  \quad 
\begin{vastaus}
Luku
§ $1,41421$ 
§ $1,41421$ 
§ $1,41421$ on tietysti $\sqrt{2}$ ja annetut laskut mukailevat neliön halkaisijan suhdetta sivuun. (Tästä lisää geometrian kurssilla.) %selvennä ratkaisua
\end{vastaus}
\end{tehtava}



%Tehtävät laatinut Aleksi Sipola 09.11.2013
%Ratkaisut tehnyt Aleksi Sipola 09.11.2013
\begin{tehtava} Etsi jotkin erisuuret luonnolliset luvut a ja b, jolla $\frac{\sqrt{\sqrt[a]{b}}}{\sqrt{a}}=1$.
\begin{vastaus}
Esimerkiksi $a=3$, \quad $b=27$ 
\end{vastaus}
\end{tehtava}

\begin{tehtava}
	Osoita esimerkin avulla, että seuraavat opiskelijat ovat erehtyneet:
	\alakohdat{
		§ Mikael: "Jos $n$:s juuri korotetaan johonkin muuhun potenssiin kuin $n$,
		myös tuloksessa esiintyy välttämättä juuri. Esimerkiksi
		$\sqrt[3]{5}^2 = \sqrt[3]{25}$ ja $\sqrt[5]{3}^6 = 3 \sqrt[5]{3}$."
		§ Raisa: "Neliöjuuria sisältävän summalausekkeen voi aina korottaa
		toiseen, jolloin neliöjuurista pääsee eroon. Esimerkiksi $\sqrt{5}^2 = 5$ ja
		$(\sqrt{2} + \sqrt{8})^2 = 18$."
	}
	\begin{vastaus}
		\alakohdat{
			§ Esimerkiksi $\sqrt[4]{3}^8 = 16$
			§ Esimerkiksi $(\sqrt{2} + \sqrt{3})^2 = 5 + 2 \sqrt{6}$
		}
	\end{vastaus}
\end{tehtava}

\end{tehtavasivu}
