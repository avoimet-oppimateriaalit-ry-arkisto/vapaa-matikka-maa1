\begin{tehtavasivu}

\subsubsection*{Opi perusteet}

\begin{tehtava}
Onko annettu suure skalaarisuure? Jos kyllä, mainitse esimerkkinä jokin yksikkö (koko nimi ja lyhenne), jota kyseisen ominaisuuden mittaamiseen käytetään.
\alakohdat{
§ energia
§ lämpötila
§ kiihtyvyys
§ massa
§ tiheys
}
\begin{vastaus}
\alakohdat{
§ On: joule, J tai kalori, cal
§ On: celsiusaste, \textdegree C tai kelvin, K
§ Ei ole.
§ On: gramma, g
§ On: kilogrammaa per kuutiometri, kg/m$^3$
}
\end{vastaus}
\end{tehtava}

\begin{tehtava}
Esitä luvut ilman kymmenpotenssia.
\alakohdat{
§ $3,2 \cdot 10^4$
§ $-7,03 \cdot 10^{-5}$
§ $10,005 \cdot 10^{-2}$
}
\begin{vastaus}
\alakohdat{
§ $32\,000$
§ $-0,0000703$
§ $0,10005$
}
\end{vastaus}
\end{tehtava}

\begin{tehtava}
Kirjoita luku numeroin (ilman kymmenenpotenssia).
\alakohdat{
§ viisimiljardia kuusimiljoonaa
§ miinus kolmebiljoonaa seitsemänsataatuhatta kolmekymmentäkuusi
§ kaksitriljoonaa kaksi
§ kolmekymmentäkuusi miljardisosaa
§ forty-three billion seventy-three thousand (amerikanenglanti)
}
	\begin{vastaus}
	\alakohdat{
	§ $5\,006\,000\,000$
	§ $-3\,000\,000\,700\,036$
	§ $2\,000\,000\,000\,000\,002$
	§ $0,000000036$
	§ $43\,000\,073\,000$
	}
	\end{vastaus}
\end{tehtava}

\begin{tehtava}
Kirjoita/lausu luku kirjaimin.
\alakohdat{
§ $76\,000\,000\,076$
§ $1\,002\,003\,004\,560$
}
	\begin{vastaus}
	\alakohdat{
	§ seitsemänkymmentäkuusimiljardia seitsemänkymmentäkuusi
	§ biljoona kaksimiljardia kolmemiljoonaa neljätuhatta viisisataakuusikymmentä
	}
	\end{vastaus}

\end{tehtava}


% FIXME \todo{TEHTÄVÄ: sanallinen tehtävä, jossa pitää laskea esim. km ja m yhteen}
\begin{tehtava}
Esitä luku ilman etuliitettä tai kymmenpotenssimuotoa.
\alakohdat{
§ $0,5$\,dl
§ $233$\,mm
§ $33$ senttimetriä
§ $16$\,kg
§ $2$\,MJ
§ $4$ kibitavua
§ $0,125$\,Mit
}
\begin{vastaus}
\alakohdat{
§ $0,05$\,l
§ $0,233$\,m
§ $0,33$ metriä
§ $16\,000$\,g
§ $2\,000\,000$\,J
§ $4\,096$ tavua
§ $131\,072$\,t
}
\end{vastaus}
\end{tehtava}

\begin{tehtava}
Maailman nopein kamera vuonna 2014 ottaa $4,4$ biljoonaa kuvaa sekunnissa.
\alakohdat{
§ Esitä kymmenpotenssimuodossa sievennettynä tarkkana arvona, kuinka kauan yhden kuvan ottaminen keskimäärin kestää.
§ Kuinka monta kokonaista nanosekuntia kestää ottaa kyseisellä kameralla miljoonaa kuvaa?
}
	\begin{vastaus}
	\alakohdat{
	§ $\frac{5}{22}\cdot 10^{-12}$
	§ $227$\,ns
	}
	\end{vastaus}
\end{tehtava}

\begin{tehtava}
Muuta
      \alakohdat{
    § $1$\,h $36$\,min tunneiksi
    § $2$\,h $45$\,min minuuteiksi
    § $1,5$\,h sekunneiksi.
      }
  \begin{vastaus}
    \alakohdat{
    § $1,6$\,h
    § $165$\,min
    § $5\,400$\,h
      }
  \end{vastaus}
\end{tehtava}

\begin{tehtava}
Muuta seuraavat ajat tunneiksi.
	\alakohdat{
		§ $73$ minuuttia
		§ $649$ sekuntia
		§ $15$ minuuttia ja $50$ sekuntia
		§ $42$ minuuttia ja $54$ sekuntia
	}
	\begin{vastaus}
		\alakohdat{
			§ $1,21666\ldots$ tuntia
			§ $0,1802777\ldots$ tuntia
			§ $0,263888\ldots$ tuntia
			§ $0,715$ tuntia
		}
	\end{vastaus}
\end{tehtava}

\begin{tehtava}
Elektroniikkayhtiö on ilmoittanut, että laitteen täyteen ladattu akku kestää käyttöä $450$ minuuttia. Laitetta on käytetty lataamisen jälkeen $3$\,h $30$\,min. Kuinka monta tuntia akun voi olettaa vielä kestävän?
\begin{vastaus}
$4$ tuntia
\end{vastaus}
\end{tehtava}

\begin{tehtava}
Elokuva kestää $142$ minuuttia ja näytös alkaa klo $17.30$. Jos elokuvan alussa on lisäksi (tasan) $15$ minuuttia mainoksia, monelta elokuva loppuu?
	\begin{vastaus}
	Näytös päättyy kello $21.07$.
	\end{vastaus}
\end{tehtava}

%teh: eistä kymmenpotenssimuodossa

%\begin{tehtava}
%Muuta \emph{järkevimpään} yksikköön %en taida ymmärtää tätä tehtävää :S T: JoonasD6
%\alakohdat{
%§ $0,3$\,km$ +200$\,m
%§ $0,04$\,m$ + 10$\,mm
%§ $0,4$\,km$ + 7\,000$\,m$ + 1\,000\,000$\,mm
%§ $0,2$\,cm$ + 4$\,cm.
%}
%\begin{vastaus}
%\alakohdat{
%§ $500$\,m
%§ $5$\,cm
%§ $8,4$\,km
%§ $4,2$\,cm
%}
%\end{vastaus}
%\end{tehtava}

\begin{tehtava}
Muuta seuraavat pituudet SI-muotoon ($1$ tuuma$= 2,54$\,cm, $1$ jaardi$=0,914$\,m, $1$ jalka$= 0,305$\,m, $1$ maili$ = 1,609$\,km).
\alakohdat{
§ $5$ tuumaa senttimetreiksi
§ $0,3$ tuumaa millimetreiksi
§ $79$ jaardia metreiksi
§ $80$ mailia kilometreiksi
§ $5$ jalkaa ja $7$ tuumaa senttimetreiksi
§ $330$ jalkaa kilometreiksi
}
\begin{vastaus}
\alakohdat{
§ $12,7$\,cm
§ $7,62$\,mm
§ $72,206$\,m
§ $128,72$\,km
§ $170,28$\,cm
§ $100,65$\,m
}
\end{vastaus}
\end{tehtava}

\begin{tehtava}
Laske
\alakohdat{
§ $3,0\,\textrm{cm} \cdot 2,5\,\textrm{\,cm} \cdot 6,0\,\textrm{cm}$
§ $0,4\,\textrm{km} : 8\,\textrm{m/s}$
§ $2\,\textrm{g} : 50\,\textrm{mg/ml}$.
}
\begin{vastaus}
\alakohdat{
§ $45$\,cm$^3$
§ $50$\,s
§ $40$\,ml
}
\end{vastaus}
\end{tehtava}

\begin{tehtava}
Laske
\alakohdat{
§ $3,0\,\textrm{ml} \cdot 25\,\textrm{mg/ml}$
§ $150\,\textrm{m} : 3\,\textrm{m/s}$
§ $3,0\,\textrm{g} : 2,0\,\textrm{dm}^2$.
}
\begin{vastaus}
\alakohdat{
§ $75$\,mg
§ $50$\,s
§ $1,5$\,g/dm$^2$
}
\end{vastaus}
\end{tehtava}

%binäärikertoimista tehtäviä, tiedonsiirtoaikoja!

\begin{tehtava}
Kuinka monta litraa maitoa mahtuu $1\,\textrm{m}\times 1\,\textrm{m}\times 1\,\textrm{m}$ -laatikkoon? Jos kuution särmän pituus on $a$, on kuution tilavuus $a^3$.
	\begin{vastaus}
	$1\,000$\,l
	\end{vastaus}
\end{tehtava}

\begin{tehtava}
Liuoksen pitoisuus voidaan laskea kaavalla $c=m/V$, missä $c$ on pitoisuus, $m$ on liuenneen aineen määrä ja $V$ on liuoksen tilavuus. Laske suolaliuoksen pitoisuus (yksikkönä mg/ml), kun $4,5$ grammaa suolaa on liuotettu $500$ millilitraan liuosta. %tarkennus pitoisuuksista (massapitoisuus, ei molaarinen)
\begin{vastaus}
$9$\,mg/ml
\end{vastaus}
\end{tehtava}

\begin{tehtava}
USB (engl. \textit{Universal Serial Bus} on yleinen tekniikka erilaisten lisälaitteiden yhdistämiseksi tietokoneisiin. Vuonna 2008 esitelty USB:n versio $3.0$ on yli kymmenen kertaa nopeampi kuin USB $2.0$, ja USB $3.0$:n suurin tiedonsiirtonopeus on jopa $5$\,Gb/s.
\alakohdat{
§ Kuinka paljon tämä on megatavuina sekunnissa?
§ Kuinka kauan kyseisellä nopeudella kestää siirtää $13$\,Git:n kokoinen Blu-ray-elokuva ulkoiselle kiintolevylle? Oletetaan, että mikään muu siirtovaihe ei toimi pullonkaulana.
}
	\begin{vastaus}
	\alakohdat{
	§ $625$\,Mt/s
	§ USB $3.0$:n tiedonsiirtonopeus gigatavuina sekunnissa on $\frac{5}{8}$\,Gb/s$=0,625\,$Gt/s. Elokuvan koko gigatavuina on $13$\,Git$=13\cdot 2^{30}$\,t$\approx 13,959\,$Gt. Jakamalla tämä koko siirtonopeudella saadaan vastaukseksi sekunteina $13,959\,\text{Gt}:0,625\,\text{Gt/s}\approx 22,3$ sekuntia.
	}
	\end{vastaus}
\end{tehtava}

\begin{tehtava}
(YO K00/3) Suorakulmaisen särmiön muotoinen hautakivi on $80$\,cm korkea, $2,10$\,m pitkä ja $32$\,cm leveä.
Voidaanko kivi nostaa nosturilla, joka pystyy nostamaan enintään kahden tonnin painoisen kuorman? Hautakiven tiheys on $2,7 \cdot 10^3$\,kg/m$^3$. [Suorakulmaisen särmiön tilavuus on sen kolmen pituusdimension eli pituuden, leveyden ja syvyyden tulo.]
\begin{vastaus}
Kyllä voidaan: hautakiven tilavuus on noin $0,54$\,m$^3$, joten se painaa noin $1,5$ tonnia.
\end{vastaus}
\end{tehtava}

%\begin{tehtava}
%Lääkehiiltä eli aktiivihiiltä käytetään muun muassa ripulin ja myrkytysten akuuttihoitoon, sillä se sitoo suuren pinta-alansa vuoksi tehokkaasti nesteitä ja useita lääkeaineita ja myrkkyjä. Eräässä $50,0$ gramman pakkauksessa ohjeena oli: ''käytetään $3$ ruokalusikallista hiiliraetta kymmentä painokiloa kohti''. Yksi ruokalusikallinen on tilavuudeltaan noin $15\,$ml, ja aktiivihiilen tiheys on $2,26$\,g\,cm$^{-1}$.
%\alakohdat{
%§ Kuinka monta ruokalusikallista lääkehiiltä $40$-kiloinen lapsi tarvitsee?
%§ Kuinka monta ruokalusikallista hiiltä pakkaukseen jää a-kohdan käytön jälkeen?
%}
%	\begin{vastaus}
%\alakohdat{
%§ ($3\,\textrm{rkl}):(10\,\textrm{kg})\cdot 40\,\textrm{kg}=12\,\textrm{rkl}$
%§
%}
%	\end{vastaus}
%\end{tehtava}

\subsubsection*{Lisää tehtäviä}

\begin{tehtava}
Muuta sekunneiksi
\alakohdat{
§ $1$\,h $42$\,min
§ $3$\,h $32$\,min
§ $1,25$\,h
§ $4,5$\,h.
}
\begin{vastaus}
\alakohdat{
§ $6\,120$\,s
§ $12\,720$\,s
§ $4\,500$\,s
§ $16\,200$\,s
}
\end{vastaus}
\end{tehtava}

\begin{tehtava}
Muuta minuuteiksi
\alakohdat{
§ $1$\,h $17$\,min
§ $2$\,h $45$\,min
§ $1,5$\,h
§ $1,75$\,h
}
\begin{vastaus}
\alakohdat{
§ $77$\,min
§ $165$\,min
§ $90$\,min
§ $105$\,min
}
\end{vastaus}
\end{tehtava}

\begin{tehtava}
Muuta tunneiksi ja minuuteiksi
\alakohdat{
§ $125$\,min
§ $667$\,min
§ $120$\,min
§ $194$\,min.
}
\begin{vastaus}
\alakohdat{
§ $2$\,h $5$\,min
§ $11$\,h $7$\,min
§ $2$\,h
§ $3$\,h $14$\,min
}
\end{vastaus}
\end{tehtava}

\begin{tehtava}
Jasper-Korianteri ja Kotivalo vertailivat keppejään. Jasper-Korianteri mittasi oman keppinsä $5,9$ tuumaa pitkäksi ja Kotivalo omansa $14,8$ senttimetriä pitkäksi. Kummalla on pidempi keppi?
\begin{vastaus}
Jasper-Korianterilla, sillä $5,9$ tuumaa = $14,986$\,cm
\end{vastaus}
\end{tehtava}

\begin{tehtava}
Liisa ohjelmoi tietokoneensa sammumaan $14\,400$ sekunnin kuluttua. Kuinka monen tunnin kuluttua Liisan tietokone sammuu?
\begin{vastaus}
$4$ tunnin kuluttua
\end{vastaus}
\end{tehtava}

\begin{tehtava}
Laske oma pituutesi ja painosi tuumissa ja paunoissa. %vai laitetaanko "yhdysvaltalaisten keskipituus on..." ja sitten vertaa itseesi :)
%\begin{vastaus}
%Esimerkiksi $168$\,cm...
%\end{vastaus}
\end{tehtava}

\begin{tehtava}
$\star$ (Muunnelma fyysikko Enrico Fermin esittämästä ongelmasta.) Chicagossa asuu osapuilleen $5\,000\,000$ asukasta. Kussakin kotitaloudessa asuu keskimäärin kaksi asukasta. Karkeasti joka kahdennessakymmenennessä kotitaloudessa on säännöllisesti viritettävä piano. Säännöllisesti viritettävät pianot viritetään keskimäärin kerran vuodessa. Pianonvirittäjällä kestää noin kaksi tuntia virittää piano, kun matkustusajat huomioidaan. Kukin pianonvirittäjä työskentelee kahdeksan tuntia päivässä, viisi päivää viikossa ja $50$ viikkoa vuodessa. Mikä arvio Chicagon pianonvirittäjien lukumäärälle saadaan näillä luvuarvoilla?
\begin{vastaus}
$125$ pianonvirittäjää
\end{vastaus}
\end{tehtava}

\end{tehtavasivu}