\begin{tehtavasivu}
\subsubsection*{Opi perusteet}
Ratkaise yhtälöt.
\begin{tehtava}
	\alakohdat{ 
		§ $ x^2 = 4 $
		§ $ x^3 = 27 $
		§ $ x^5 = -1 $
		§ $ x^2 - 3 = 0 $
		§ $ x^3 + 125 = 0 $
		§ $x^3 = 81$
		§ $x^5 = 10$
	§ $x^4 = 1$
	}
	\begin{vastaus}
	\alakohdat{ 
		§ $ x = \pm 2 $
		§ $ x = 3 $
		§ $ x = -1 $
		§ $ x = \pm\sqrt{3} $
		§ $ x = -5 $
		§ $x= 3 \sqrt[3]{3}$
		§ $x= \sqrt[5]{10}$
		§ $x= \pm 1$
	}
	\end{vastaus}
\end{tehtava}

%+tehtäviä, joissa ei reaaliratkaisuja

\begin{tehtava}
	\alakohdat{ 
		§ $ 5x^2 = 25 $
		§ $ (2x)^3 = 8 $
		§ $ x^4 = \frac{1}{4} $
		§ $ (3x)^2 = 36 $
		§ $ (4x)^2 + 16 = 0 $
	}
	
	\begin{vastaus}
	\alakohdat{ 
		§ $ x = \pm\sqrt{5} $
		§ $ x = 1 $
		§ $ x = \pm\frac{1}{\sqrt{2}} $
		§ $ x = \pm 2 $
		§ ei ratkaisua
	}
	\end{vastaus}	
\end{tehtava}

\begin{tehtava}
\alakohdat{
§ $x^4 - 8 = 0$
§ $2x^3 + 7 = 0$
§ $\frac{x^2}{4} - \frac{5}{2} = 1$
§ $1,51 x^4 - 1,2 = 7,5$
}
\begin{vastaus}
\alakohdat{
§ $x = \pm\sqrt[4]{8}$
§ $x= -\sqrt[3]{\frac{7}{2}}$
§ $x= \pm\sqrt{14}$
§ $x= \pm\sqrt[4]{\frac{870}{151}}$
}
\end{vastaus}
\end{tehtava}

\begin{tehtava}
\alakohdat{
§ $2,3x^{-1} = 7,9$
§ $\pi x^{-3} + 4 = -23$
§ $5x^{-4} = -16$
}
\begin{vastaus}
\alakohdat{
§ $x = \frac{23}{79}$
§ $x = -\frac{3}{\sqrt[3]{\pi}}$
§ ei ratkaisua
}
\end{vastaus}
\end{tehtava}

\begin{tehtava}
	Ratkaise yhtälö $7(x-3)+1=x^2-1-(x^2-1)$.
    \begin{vastaus}
	$x=\frac{20}{7}$
    \end{vastaus}
\end{tehtava}

\begin{tehtava}%Laati Henri Ruoho 10-22-2013
Viivi sijoitti $3\,100$ euroa pääomatilille. Yhdeksän vuoden kuluttua tilillä oli $5\,000$ euroa. Mikä oli tilin vuotuinen korkokanta prosentin sadaosien tarkkuudella, kun se oli pysynyt samana vuodesta toiseen? 
	\begin{vastaus}
$5,46\,\%$
	\end{vastaus}
\end{tehtava}

\begin{tehtava}%Laati Henri Ruoho 10-22-2013
Viivi haluaa sijoittamansa pääoman kaksinkertaistuvan seuraavassa kymmenessä vuodessa. Kuinka suurta korkoprosenttia hän esittää pankinjohtajalle?
	\begin{vastaus}
$7,1\,\%$
	\end{vastaus}
\end{tehtava}

\begin{tehtava}%Laati Henri Ruoho 10-22-2013
Torimyyjä tarvitsee kappoja eli kuution muotoisia mitta-astioita. Pienen kapan vetoisuus on $2$ litraa ja ison kapan vetoisuus $5$ litraa. Määritä astioiden mitat senttimetrin kymmenesosan tarkkuudella. Kuution tilavuus lasketaan kaavalla $V=a^3$, missä $a$ on kuution särmän pituus.
	\begin{vastaus}
Pienen kapan sivun pituus $12,6$\,cm, ison kapan $17,1$\,cm
	\end{vastaus}
\end{tehtava}

\begin{tehtava}%Laati Henri Ruoho 10-22-2013
Suomi sitoutui vähentämään kasvihuonepäästöjään vuoden 2005 alusta $20\,\%$ vuoteen 2020 mennessä. Kuinka paljon päästöjä oli tarkoitus vähentää vuosittain? Anna vastaus yhden desimaalin tarkkuudella. %vastaava esimerkki
\begin{vastaus}
$1,5\,\%$
\end{vastaus}
\end{tehtava}

\subsubsection*{Hallitse kokonaisuus}
\begin{tehtava}%Laati Henri Ruoho 10-11-2013
Syksyllä 2012 maapallon väkiluvun arveltiin olevan noin $7$ miljardia. Väkiluku oli kaksinkertaistunut arviolta $38$ vuodessa. Tutki laskimella, milloin seuraava miljardi saavutettaisiin väestönkasvun jatkuessa samalla tavalla..
\begin{vastaus}
Noin vuonna 2020
\end{vastaus}
\end{tehtava}

\begin{tehtava}
Ihmisen punasolun keskimääräinen tilavuus on noin $90$ femtolitraa. Jos punasolu oletetaan pallon muotoiseksi (oikeasti punasolu on kaksoiskupera), sen tilavuudella pätee kaava $V=\frac{4}{3}\pi r^3$, missä $r$ on pallon säde (eli puolet läpimitasta). Laske ja esitä mikrometreinä punasolun läpimitta.
	\begin{vastaus}
	$5,56$\,$\mu$m
	\end{vastaus}
\end{tehtava}

\begin{tehtava}
Kun pelaajahahmo kuolee \emph{Diablo 3} -pelissä, hänen varusteidensa kuntoluokitus pienenee $10$\,\%.
	\alakohdat{
		§ Kuinka monta kertaa pelaajahahmo voi kuolla peräjälkeen ennen kuin hänen varusteidensa kunto on pudonnut alle puoleen?
		§ Pelaajan taikamiekan kuntoluokitus on $90$ pistettä, ja kunto voi saada vain kokonaislukuarvoja. Jokaisen kuoleman jälkeen kuntolukitus pyöristetään lähimpään kokonaislukuun. Vaikuttaako pyöristäminen tehtävän a-kohdan tulokseen?
	}
	\begin{vastaus}
		\alakohdat{
		§ Kuusi kertaa
		§ Jokaisen kuntoluokan pudotuksen jälkeen pyöristämällä kuolemia voikin olla kuuden sijaan seitsemän ennen kuin kunto on pudonnut alle $50$ prosentin.
	}
	\end{vastaus}
\end{tehtava}

\begin{tehtava}
Muinainen hallitsija Tauno Alpakka rakennuttaa itselleen kuution muotoista palatsia.  Palatsin tilavuuden tulee olla $5\,000\,\mathrm{m}^3$. Kuution tilavuus lasketaan kaavalla $V=a^3$, missä $a$ on kuution särmän pituus.
\alakohdat{
§ Kuinka korkea palatsista tulee?
§ Palatsin ulkopuoli päällystetään $10\,$cm:n paksuisella kultakerroksella. Kuinka monta kiloa kultaa tarvitaan? (Kullan tiheys on $19,23 \cdot 10^3\,\mathrm{kg}/\mathrm{m}^3$.)
§ Kuinka monta kiloa kultaa tarvitaan, jos myös palatsin sisäpuoli päällystetään?
}
\begin{vastaus}
\alakohdat{
§ $10 \sqrt[3]{5} \approx 17\,\mathrm{m}$
§ $961\,500 \sqrt[3]{25} \approx 2\,800\,000\,\mathrm{kg}$
§ $2\,115\,300 \sqrt[3]{25} \approx 6\,200\,000\,\mathrm{kg}$
}
\end{vastaus}
\end{tehtava}

\begin{tehtava}
(YO S1978/2) Laske $k^\frac{5}{3}$, kun $k^\frac{-1}{3}=2$. \\
	\begin{vastaus}
		$k^\frac{5}{3}=1/32$
	\end{vastaus}
\end{tehtava}

\subsubsection*{Lisää tehtäviä}

\begin{tehtava}%Laati Henri Ruoho 10-11-2013
Tilan eläimille rakennetaan uutta aitausta. Käytettävissä on $100$ metriä sähköaitaan tarvittavaa metallijohtoa. Kuinka suuri pinta-ala kotieläimillä on käytettävissään, jos aitaus on muodoltaan
\alakohdat{
§ neliö?
§ ympyrä?
}
Neliön pinta-ala on sen sivun pituuden toinen potenssi, ja ympyrän pinta-ala on $\pi r^2$, missä $r$ on ympyrän säde. Saman ympyrän kehän pituus on $2\pi r$.
\begin{vastaus}
\alakohdat{
§ $625\,$m$^2$
§ $796\,$m$^2$
}
\end{vastaus}
\end{tehtava}

\begin{tehtava}
Kuution tilavuus särmän pituuden funktiona on $f(x) = x^3$. Jos kuution tilavuus on $64$, mikä on särmän pituus?
\begin{vastaus}
Särmän pituus on $4$.
\end{vastaus}
\end{tehtava}

\begin{tehtava}
Tietokonepelissä \emph{Tyrmiä ja Traakkeja} pelaajalla on käytettävissä useita eri hahmoja, kuten Santeri-soturi tai Valtteri-velho. Peli alkaa tasolta $1$ ja tason noustessa hahmoista tulee vahvempia. Pelin suunnittelija määrittää kaikille vihollisille voimaluvun ja arvioi, että tasolla $t$ Santeri-soturille keskimääräisen haastavan vihollisen on oltava voimaluvultaan $5t^{\frac{3}{2}}$ ja Valtteri-velholle $t^{\frac{5}{2}}$. Millä tasolla vihollinen on keskimääräisen haastava kullekin hahmolle, jos sen voimaluku on
\alakohdat{
§ $30$
§ $56$
§ $100$?
}
\begin{vastaus}
\alakohdat{
§ Santeri-soturille tasolla $3$, Valtteri-velholle tasolla $4$
§ Santeri-soturille tasolla $5$, Valtteri-velholle tasolla $5$
§ Santeri-soturille tasolla $7$, Valtteri-velholle tasolla $6$
}
\end{vastaus}
\end{tehtava}

\begin{tehtava}
$\star$ Kuvaruudun pituuksien suhde on $16:9$. Määritä ruudun mitat sentin tarkkuudella, kun sen pinta ala on $0,40\,\mathrm{m}^2$. Suorakulmion pinta-ala on sen erisuuntaisten sivujen pituuksien tulo.
\begin{vastaus}
$84\,$cm ja $47$\,cm
\end{vastaus}
\end{tehtava}

\begin{tehtava}
$\star$ $50$ tuuman Full HD -näytössä on $1\,920\times1\,080$ kuva-alkiota eli pikseliä. Kuinka monta pikseliä näytöllä on neliösenttimetriä kohden? $1$ tuuma on $2,54$ senttimetriä. (Tarvitset Pythagoraan lausetta.)
	\begin{vastaus}
Näyttö on suorakulmion muotoinen, $1\,920$ pikseliä leveä ja $1\,080$ pikseliä korkea. Kuvasuhteeksi saadaan $16:9$ supistamalla pituuksien osamäärä. Näytön lävistäjän pituus on $50$ tuumaa eli $127$ senttimetriä. Lävistäjä toimii hypotenuusana suorakulmaiselle kolmiolle, jonka kateettien suhde on mainittu $16:9$. Pythagoraan lauseella saadaan yhtälö $127^2=(16a)^2+(9a)^2$, missä $a$ on näytön leveyden ja korkeuden pienin yhteinen tekijä (senttimetreissä). Yhtälö saadaan potenssit laskemalla muotoon $16\,129=256a^2+81a^2$, mistä edelleen $337a^2=16\,129$. Potenssiyhtälön positiivisena ratkaisuna (koska pituudet ovat positiivisia) saadaan $a=\sqrt{\frac{16\,129}{337}}$\,cm.

Näytön kokonaispinta-ala on tällöin $16a\cdot 9a=144a^2=144\cdot (\sqrt{\frac{16\,129}{337}}$\,cm$)^2=144\cdot\frac{16\,129}{337}$\,cm$^2\approx 6\,891,92$cm$^2$. Näytöllä on pikseleitä kokonaisuudessaan $1\,920\cdot 1\,080=2\,073\,600$ (eli reilut kaksi megapikseliä). Neliösenttimetriä kohden pikseleitä on näytöllä $2\,073\,600/6\,891,92 \approx 301$. (Pikseli on perustavanlaatuinen kuva-alkio, ja niitä ei voi hajottaa osiin. Vain luonnollinen luku kelpaa vastaukseksi.)
	\end{vastaus}
\end{tehtava}

\end{tehtavasivu}