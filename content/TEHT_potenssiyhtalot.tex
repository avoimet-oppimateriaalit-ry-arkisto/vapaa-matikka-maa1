\begin{tehtavasivu}
\subsubsection*{Opi perusteet}
Ratkaise yhtälöt.

\begin{tehtava}
	\alakohdat{ 
		§ $ x^2 = 4 $
		§ $ x^3 = 27 $
		§ $ x^5 = -1 $
		§ $ x^2 - 3 = 0 $
		§ $ x^3 + 125 = 0 $
	}
	
	\begin{vastaus}
	\alakohdat{ 
		§ $ x = \pm 2 $
		§ $ x = 3 $
		§ $ x = -1 $
		§ $ x = \pm\sqrt{3} $
		§ $ x = -5 $
	}
\end{vastaus}
\end{tehtava}

\begin{tehtava}
% Ratkaise \\
	\alakohdat{ 
		§ $ 5x^2 = 25 $
		§ $ (2x)^3 = 8 $
		§ $ x^4 = \frac{1}{4} $
		§ $ (3x)^2 = 36 $
		§ $ (4x)^2 + 16 = 0 $
	}
	
	\begin{vastaus}
	\alakohdat{ 
		§ $ x = \pm\sqrt{5} $
		§ $ x = 1 $
		§ $ x = \pm\frac{1}{\sqrt{2}} $
		§ $ x = \pm 2 $
		§ ei ratkaisua
	}
	\end{vastaus}	
\end{tehtava}


\begin{tehtava}
% Ratkaise potenssiyhtälöt
\alakohdat{
§ $x^3 = 81$
§ $x^5 = 10$
§ $x^2 = 4$
§ $x^4 = 1$
}
\begin{vastaus}
\alakohdat{
§ $x= 3 \sqrt[3]{3}$
§ $x= \sqrt[5]{10}$
§ $x= \pm 2$
§ $x= \pm 1$
}
\end{vastaus}
\end{tehtava}

\begin{tehtava}
% Ratkaise potenssiyhtälöt
\alakohdat{
§ $x^4 - 8 = 0$
§ $2x^3 + 7 = 0$
§ $\frac{x^2}{4} - \frac{5}{2} = 1$
§ $1,51 x^4 - 1,2 = 7,5$
}
\begin{vastaus}
\alakohdat{
§ $x = \pm\sqrt[4]{8}$
§ $x= -\sqrt[3]{\frac{7}{2}}$
§ $x= \pm\sqrt{14}$
§ $x= \pm\sqrt[4]{\frac{870}{151}}$
}
\end{vastaus}
\end{tehtava}

\begin{tehtava}
\alakohdat{
§ $2,3x^{-1} = 7,9$
§ $\pi x^{-3} + 4 = -23$
§ $5x^{-4} = -16$
}
\begin{vastaus}
\alakohdat{
§ $x = \frac{23}{79}$
§ $x = -\frac{3}{\sqrt[3]{\pi}}$
§ ei ratkaisua
}
\end{vastaus}
\end{tehtava}

\begin{tehtava}
	Ratkaise yhtälö $7(x-3)+1=x^2-1-(x^2-1)$.
    \begin{vastaus}
	$x=\frac{20}{7}$
    \end{vastaus}
\end{tehtava}

\begin{tehtava}%Laati Henri Ruoho 10-22-2013
Viivi sijoitti 3\,100 euroa pääomatilille. Yhdeksän vuoden kuluttua tilillä oli 5\,000 euroa. Mikä oli tilin vuotuinen korkokanta prosentin sadaosien tarkkuudella, kun se oli pysynyt samana vuodesta toiseen? 
\begin{vastaus}
$5,46\,\%$
\end{vastaus}
\end{tehtava}
\begin{tehtava}%Laati Henri Ruoho 10-22-2013
Viivi haluaa sijoittamansa pääoman kaksinkertaistuvan seuraavassa kymmenessä vuodessa. Kuinka suurta korkoprosenttia hän esittää pankinjohtajalle?
\begin{vastaus}
$7,1\,\%$
\end{vastaus}
\end{tehtava}

\begin{tehtava}%Laati Henri Ruoho 10-22-2013
Torimyyjä tarvitsee kappoja eli kuution muotoisia mitta-astioita. Pienen kapan vetoisuus on $2$ litraa ja ison kapan vetoisuus $5$ litraa. Määritä astioiden mitat.
\begin{vastaus}
Pienen kapan sivun pituus $12,6$\,cm, ison kapan $17,1$\,cm
\end{vastaus}
\end{tehtava}

\begin{tehtava}%Laati Henri Ruoho 10-22-2013
Suomi sitoutui vähentämään kasvihuonepäästöjään vuoden 2005 alusta $20\,\%$ vuoteen 2020 mennessä. Kuinka paljon päästöjä oli tarkoitus vähentää vuosittain? Anna vastaus yhden desimaalin tarkkuudella. %vastaava esimerkki
\begin{vastaus}
$1,5\,\%$
\end{vastaus}
\end{tehtava}


\subsubsection*{Hallitse kokonaisuus}
\begin{tehtava}%Laati Henri Ruoho 10-11-2013
Syksyllä 2012 maapallon väkiluvun arveltiin olevan noin 7 miljardia. Väkiluku oli kaksinkertaistunut arviolta 38:ssa vuodessa. Tutki laskimella, milloin seuraava miljardi saavutettaisiin väestönkasvun jatkuessa tasaisesti.
\begin{vastaus}
Noin vuonna 2020
\end{vastaus}
\end{tehtava}

\begin{tehtava}
Muinainen hallitsija Tauno Alpakka rakennuttaa itselleen kuution muotoista palatsia.  Palatsin tilavuuden tulee olla $5\,000\,\mathrm{m}^3$. 
\alakohdat{
§ Kuinka korkea palatsista tulee?
§ Palatsin ulkopuoli päällystetään $10\,$cm:n paksuisella kultakerroksella. Kuinka monta kiloa kultaa tarvitaan? (Kullan tiheys on $19,23 \cdot 10^3\,\mathrm{kg}/\mathrm{m}^3$.)
§ Kuinka monta kiloa kultaa tarvitaan, jos myös palatsin sisäpuoli päällystetään?
}
\begin{vastaus}
\alakohdat{
§ $10 \sqrt[3]{5} \approx 17\,\mathrm{m}$
§ $961\,500 \sqrt[3]{25} \approx 2\,800\,000\,\mathrm{kg}$
§ $2\,115\,300 \sqrt[3]{25} \approx 6\,200\,000\,\mathrm{kg}$
}
\end{vastaus}
\end{tehtava}

\begin{tehtava}
Kuvaruudun pituuksien suhde on $16:9$. Määritä ruudun mitat sentin tarkkuudella, kun sen pinta ala on \(0,40\,\mathrm{m}^2\). (Muista, että suorakulmion pinta-ala on sen erisuuntaisten sivujen pituuksien tulo.)
\begin{vastaus}
$84\,$cm ja $47$\,cm
\end{vastaus}
\end{tehtava}

\begin{tehtava}
%Tehtävän laatinut Johanna Rämö 9.11.2013.
%Ratkaisun tehnyt Johanna Rämö 9.11.2013.
Suorakulmion muotoisen levyn mitat ovat $230\,\text{cm}\times 250\,\text{cm}$. Mahtuuko se sisään oviaukosta, joka on 90 cm leveä ja 205 cm korkea?
        \begin{vastaus}
        Ei mahdu. Suurin mahdollinen levy, joka mahtuu ovesta sisään on Pythagoraan lauseen perusteella leveydeltään $\sqrt{90^2+205^2}\approx 224$ cm.
        \end{vastaus}
\end{tehtava}

\begin{tehtava}
Ajatellaan suorakulmaista hiekkakenttää, jonka pinta-ala on 1 aari ($100\,\mathrm{m}^2$). Lyhyemmän ja pidemmän sivujen pituuksien suhde on $4:3$. Laske Pythagoraan lauseen avulla matka hiekkakentän kulmasta kauimmaisena olevaan kulmaan.
\begin{vastaus}
$\frac{4}{3}x^2=100$ joten $x = \sqrt{\frac{300}{4}}$. 
Hypotenuusa: $\sqrt{x^2 + (\frac{4}{3}x)^2}=\sqrt{\frac{300}{4}+\frac{16}{9}\cdot \frac{300}{4}}
=\frac{5}{3}\sqrt{\frac{300}{4}}\approx 14{,}4$.
\end{vastaus}
\end{tehtava}


\begin{tehtava}
(YO S1978/2) Laske $k^\frac{5}{3}$, kun $k^\frac{-1}{3}=2$. \\
	\begin{vastaus}
		$k^\frac{5}{3}=1/32$
	\end{vastaus}
\end{tehtava}

\subsubsection*{Lisää tehtäviä}

\begin{tehtava}%Laati Henri Ruoho 10-11-2013
Tilan eläimille rakennetaan uutta aitausta. Käytettävissä on 100 metriä sähköaitaan tarvittavaa metallijohtoa. Kuinka suuri pinta-ala kotieläimillä on käytettävissään, jos aitaus on muodoltaan
\alakohdat{
§ neliö?
§ ympyrä?
}
Muista, että neliön pinta-ala on sen sivun pituuden toinen potenssi ja ympyrän pinta-ala on $\pi r^2$, missä $r$ on ympyrän säde. Saman ympyrän kehän pituus on $2\pi r$.
\begin{vastaus}
\alakohdat{
§ $625\,$m$^2$
§ $796\,$m$^2$
}
\end{vastaus}
\end{tehtava}

\begin{tehtava}
Millä muutujan $x$ arvoilla funktio $ f(x)=x^5$ saa saman arvon kuin funktio $ g(x)=x^4$?
\begin{vastaus}
Muuttujan arvoilla $0$ ja $1$
\end{vastaus}
\end{tehtava}

\begin{tehtava}
Kuution tilavuus särmän pituuden funktiona on $f(x) = x^3$. Jos kuution tilavuus on $64$, mikä on särmän pituus?
\begin{vastaus}
Särmän pituus on $4$.
\end{vastaus}
\end{tehtava}

\begin{tehtava}
Tietokonepelissä \emph{Tyrmiä ja Traakkeja} pelaajalla on käytettävissä useita eri hahmoja, kuten Santeri-soturi tai Valtteri-velho. Peli alkaa tasolta 1 ja tason noustessa hahmoista tulee vahvempia. Pelin suunnittelija määrittää kaikille vihollisille voimaluvun ja arvioi, että tasolla $t$ Santeri-soturille keskimääräisen haastavan vihollisen on oltava voimaluvultaan $5t^{\frac{3}{2}}$ ja Valtteri-velholle $t^{\frac{5}{2}}$. Millä tasolla vihollinen on keskimääräisen haastava kullekin hahmolle, jos sen voimaluku on
\alakohdat{
§ $30$
§ $56$
§ $100$?
}
\begin{vastaus}
\alakohdat{
§ Santeri-soturille tasolla $3$, Valtteri-velholle tasolla $4$
§ Santeri-soturille tasolla $5$, Valtteri-velholle tasolla $5$
§ Santeri-soturille tasolla $7$, Valtteri-velholle tasolla $6$
}
\end{vastaus}
\end{tehtava}

\begin{tehtava}
Kun pelaajahahmo kuolee \emph{Diablo 3} -pelissä, hänen varusteidenta kuntoluokitus pienenee $10$\,\%.
	\alakohdat{
		§ Kuinka monta kertaa pelaajahahmo voi kuolla peräjälkeen ennen kuin hänen varusteidensa kunto on pudonnut alle puoleen?
		§ Pelaajan taikamiekan kuntoluokitus on $90$ pistettä, ja kunto voi saada vain kokonaislukuarvoja. Jokaisen kuoleman jälkeen kuntolukitus pyöristetään lähimpään kokonaislukuun. Vaikuttaako pyöristäminen a-kohdan vastaukseen?
		§ $100$?
	}
	\begin{vastaus}
		\alakohdat{
		§ Kuusi kertaa
		§ Jokaisen kuntoluokan pudotuksen jälkeen pyöristämällä kuolemia voikin olla kuuden sijaan seitsemän ennen kuin kunto on pudonnut 50 prosenttiin.
	}
	\end{vastaus}
\end{tehtava}

\end{tehtavasivu}
