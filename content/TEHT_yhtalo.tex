\begin{tehtavasivu}


%\subsubsection*{Opi perusteet}

\begin{tehtava}
Ovatko seuraavat yhtälöt tosia, epätosia vai ehdollisesti tosia?
  \begin{alakohdat}
    \alakohta{$1=1$}
    \alakohta{$\pi=3,14$}
    \alakohta{$\sqrt[3]{27}=3$}
    \alakohta{$t=t$}
    \alakohta{$t=t-1$}
    \alakohta{$x=\frac{x}{2}$}
    \alakohta{$3=\frac{6}{2}$}
    \alakohta{$\frac{2}{\sqrt{2}}=\sqrt{2}$}
    \alakohta{$1^{9001}=1$}
    \alakohta{$(-1)^{42}=-1$}
    \alakohta{$2y+1=5$}
  \end{alakohdat}

     \begin{vastaus}
	\begin{alakohdat}
    \alakohta{tosi}
    \alakohta{epätosi}
    \alakohta{tosi}
    \alakohta{tosi}
    \alakohta{epätosi}
    \alakohta{ehdollisesti tosi (tosi, kun $x=0$)}
    \alakohta{tosi}
    \alakohta{tosi}
    \alakohta{tosi}
    \alakohta{epätosi}
    \alakohta{ehdollisesti tosi (tosi, kun $y=2$)}
	\end{alakohdat}
    \end{vastaus}
\end{tehtava}

\begin{tehtava}
Onko $-2$ yhtälön $3x-4 = 7-2x$ ratkaisu?
\begin{vastaus}
Ei ole. $3(-2)-4 = -10 \ne 11=7-2(-2)$
\end{vastaus}
\end{tehtava}

\begin{tehtava}
Tarkista, voivatko seuraavat tuntemattoman $y$ arvot olla yhtälön $-y^3+y=0$ juuria.
  \begin{alakohdat}
   \alakohta{$y=0$}
   \alakohta{$y=2$}
   \alakohta{$y=-1$}
   \alakohta{$y=1$}
   \alakohta{$y=\sqrt[3]{2}$}
  \end{alakohdat}

  \begin{vastaus}
    \begin{alakohdat}
 \alakohta{kyllä}
   \alakohta{ei}
   \alakohta{kyllä}
   \alakohta{kyllä}
   \alakohta{ei}
    \end{alakohdat}
  \end{vastaus}
\end{tehtava}


\begin{tehtava}
Onko $x=\frac{1}{3}$ seuraavien yhtälöiden (eräs) ratkaisu?
  \begin{alakohdat}
   \alakohta{$\frac{1}{3}-x=0$}
   \alakohta{$x^2=\frac{1}{9}$}
   \alakohta{$\frac{x}{x+1}=\frac{9}{2}$}
   \alakohta{$(-x)^3=\frac{1}{27}$}
   \alakohta{$-x^3=-\frac{1}{27}$}
   \alakohta{$125^x=5$}
  \end{alakohdat}

  \begin{vastaus}
    \begin{alakohdat}
 \alakohta{kyllä}
   \alakohta{kyllä}
   \alakohta{ei}
   \alakohta{ei}
   \alakohta{kyllä}
   \alakohta{kyllä}
    \end{alakohdat}
  \end{vastaus}
\end{tehtava}


\begin{tehtava}
Kirjoita seuraavat tilanteet yhtälönä.
  \begin{alakohdat}
  \alakohta{Kun lukuun kolme lisätään luku yksi, summaksi tulee luku neljä.}
   \alakohta{Massa $m$ kaksinkertaistuu, minkä jälkeen uusi massa on $42$ kilogrammaa.}
   \alakohta{Tuntemattoman $x$ arvo on rationaalilukujen $\frac{1}{2}$ ja $\frac{1}{3}$ summa.}
   \alakohta{Ympyrän kehän pituus $p$ on $\pi$:n ja ympyrän halkaisijan $d$ tulo.}
   \alakohta{Luvun $a$ ja sen käänteisluvun tulo on $1$.}
   \alakohta{Kun lukuun $k$ lisätään sen vastaluku, summaksi saadaan nolla.}
   \alakohta{Kun lukuun $k$ lisätään sen vastaluku, summaksi saadaan yksi.}
  \end{alakohdat}

  \begin{vastaus}
    \begin{alakohdat}
        \alakohta{$3+1=4$}
   \alakohta{$2m=42$\,kg}
   \alakohta{$x=\frac{1}{2} + \frac{1}{3}$}
   \alakohta{$p=\pi d$}
   \alakohta{$a\cdot \frac{1}{a}=1$ (tai $a\cdot a^{-1}=1$)}
   \alakohta{$k+(-k)=0$}
   \alakohta{$k+(-k)=1$ (Huomaa, että tämä yhtälö on epätosi.)}
    \end{alakohdat}
  \end{vastaus}
\end{tehtava}


\begin{tehtava}
Päteekö yhtälö, kun $a=7$?
	\begin{alakohdat}
		\alakohta{$a^5-13a^2=a-3+16166$}
		\alakohta{$\sqrt{a}=\frac{529}{200}$}
	\end{alakohdat}
    \begin{vastaus}
	\begin{alakohdat}
		\alakohta{Pätee.}
		\alakohta{Ei päde.}
	\end{alakohdat}
    \end{vastaus}
\end{tehtava}

%laskintehtäviä! numeriikkaa! useita tuntemattomia

\end{tehtavasivu}