\begin{tehtavasivu}


\begin{tehtava}
Ovatko seuraavat yhtälöt tosia, epätosia vai ehdollisesti tosia?
  \alakohdat{
    § $1=1$
    § $\pi=3,14$
    § $\sqrt[3]{27}=3$
    § $t=t$
    § $t=t-1$
    § $x=\frac{x}{2}$
    § $3=\frac{6}{2}$
    § $\frac{2}{\sqrt{2}}=\sqrt{2}$
    § $1^{9\,001}=1$
    § $(-1)^{42}=-1$
    § $2y+1=5$
  }

     \begin{vastaus}
	\alakohdat{
    § tosi
    § epätosi
    § tosi
    § tosi
    § epätosi
    § ehdollisesti tosi (tosi, kun $x=0$)
    § tosi
    § tosi
    § tosi
    § epätosi
    § ehdollisesti tosi (tosi, kun $y=2$)
	}
    \end{vastaus}
\end{tehtava}

\begin{tehtava}
Onko $-2$ yhtälön $3x-4 = 7-2x$ ratkaisu?
\begin{vastaus}
Ei ole. $3(-2)-4 = -10 \ne 11=7-2(-2)$
\end{vastaus}
\end{tehtava}

\begin{tehtava}
Tarkista, voivatko seuraavat tuntemattoman $y$ arvot olla yhtälön $-y^3+y=0$ juuria.
  \alakohdat{
   § $y=0$
   § $y=2$
   § $y=-1$
   § $y=1$
   § $y=\sqrt[3]{2}$
  }

  \begin{vastaus}
    \alakohdat{
 § kyllä
   § ei
   § kyllä
   § kyllä
   § ei
    }
  \end{vastaus}
\end{tehtava}


\begin{tehtava}
Onko $x=\frac{1}{3}$ seuraavien yhtälöiden (eräs) ratkaisu?
  \alakohdat{
   § $\frac{1}{3}-x=0$
   § $x^2=\frac{1}{9}$
   § $\frac{x}{x+1}=\frac{9}{2}$
   § $(-x)^3=\frac{1}{27}$
   § $-x^3=-\frac{1}{27}$
   § $125^x=5$
  }

  \begin{vastaus}
    \alakohdat{
 § kyllä
   § kyllä
   § ei
   § ei
   § kyllä
   § kyllä
    }
  \end{vastaus}
\end{tehtava}

\begin{tehtava}
Päteekö yhtälö, kun $a=7$?
	\alakohdat{
		§ $a^5-13a^2=a-3+16\,166$
		§ $\sqrt{a}=\frac{529}{200}$
	}
    \begin{vastaus}
	\alakohdat{
		§ Pätee.
		§ Ei päde.
	}
    \end{vastaus}
\end{tehtava}

\begin{tehtava}
%Tehtävän laatinut Johanna Rämö 9.11.2013.
%Ratkaisun tehnyt Johanna Rämö 9.11.2013.
       \alakohdat{
       § Onko $x=4$ yhtälön $x^3=64$ ratkaisu?
       § Onko $x=2$ yhtälön $x^5=30$ ratkaisu?
       § Onko $x=-1$ yhtälön $x^4-3x+2=0$ ratkaisu?
       }
       \begin{vastaus}
       \alakohdat{
           § On, sillä $4^3=4 \cdot 4 \cdot 4=64$.
           § Ei, sillä $2^5=2 \cdot 2 \cdot 2 \cdot 2 \cdot 2=32$.
           § On, sillä $(-1)^4-3(-1)+2=1-3+2=0$.
       } % tämä oli ennen "alakohdatrivi"
       \end{vastaus}
\end{tehtava}

\begin{tehtava}
Kirjoita tilanne yhtälönä.
  \alakohdat{
  § Kun lukuun kolme lisätään luku yksi, summaksi tulee luku neljä.
   § Massa $m$ kaksinkertaistuu, minkä jälkeen uusi massa on $42$ kilogrammaa.
   § Tuntemattoman $x$ arvo on rationaalilukujen $\frac{1}{2}$ ja $\frac{1}{3}$ summa.
   § Ympyrän kehän pituus $p$ on $\pi$:n ja ympyrän halkaisijan $d$ tulo.
   § Luvun $a$ ja sen käänteisluvun tulo on $1$.
   § Kun lukuun $k$ lisätään sen vastaluku, summaksi saadaan nolla.
   § Kun lukuun $k$ lisätään sen vastaluku, summaksi saadaan yksi.
  }

  \begin{vastaus}
    \alakohdat{
   § $3+1=4$
   § $2m=42$\,kg
   § $x=\frac{1}{2} + \frac{1}{3}$
   § $p=\pi d$
   § $a\cdot \frac{1}{a}=1$ (tai $a\cdot a^{-1}=1$)
   § $k+(-k)=0$
   § $k+(-k)=1$ (Huomaa, että tämä yhtälö on aina epätosi!)
    }
  \end{vastaus}
\end{tehtava}
%lisää mallinatamistreeniä ja esimerkkejä!

\begin{tehtava}
Kirjoita tilanne yhtälöparina. (Ei tarvitse ratkaista -- tämä on mallinnusharjoitus)
  \alakohdat{
§ Lukujen $k$ ja $l$ summa on $10$ ja tulo $20$.
§ Suorakulmion muotoisen aitauksen, jonka leveys on $x$ ja pituus $y$, pinta-ala $30$ neliömetriä. Aitaa on käytettävissä yhteensä $150$ metriä.
§ Nuorisojärjestö teki ryhmätilauksen demopartyille: $35$ euron sisäänpääsylippuja ja $70$ euron konepaikkalippuja (sis. sisäänpääsyn) ostettiin yhteensä $15$ kappaletta, ja koko tilaus maksoi yhteensä $875$ euroa.
  }
  \begin{vastaus}
    \alakohdat{
    § $$\left\{    
    \begin{array}{rcl}
        k+l&=&10 \\
        kl&=&20 \\
    \end{array}
    \right. $$
    § $$\left\{    
    \begin{array}{rcl}
        xy&=&30\,\text{m}^2 \\
        2x+2y&=&150 \\
    \end{array}
    \right. $$
   §$$\left\{    
    \begin{array}{rcl}
		x+y&=&15 \\
		35x+70y&=&875 \\
    \end{array}
    \right. ,$$ missä $x$ on sisäänpääsylipullisten määrä ja $y$ konepaikallisten määrä
    }
  \end{vastaus}
\end{tehtava}

\begin{tehtava}
Pekka on melomassa joella. Hän kulkee ensiksi $3,7$\,km matkan vastavirtaan, ja palaa sitten samaa reittiä takaisin lähtöpisteeseen. Pekalla kuluu menomatkaan aikaa $42$ minuuttia, ja paluumatkaan $27$ minuuttia. Pekka meloo tasaisella nopeudella koko matkan. Laske joen virtausnopeus ja Pekan melomisnopeus. Ilmoita nopeudet yksikössä km/h kahden desimaalin tarkkuudella.
		\begin{vastaus}
			Joen virtausnopeus on noin $1,47$\,km/h ja Pekan melomisnopeus noin $6,75\,\frac{km}{h}$.
		\end{vastaus}
\end{tehtava}
%lisännyt jaakko viertiö 17.5.2014
%tarkistakaa vastaus, ei oo luottavainen olo sen suhteen -jv %tarkistetaan myös, voisiko siirtää parempaan paikkaan!
%laskintehtäviä! numeriikkaa! useita tuntemattomia

\end{tehtavasivu}