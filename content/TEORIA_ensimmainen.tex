Siirrymme nyt tarkastelemaan tärkeää yhtälöiden erikoistapausta, ensimmäisen asteen yhtälöitä.

\laatikko{Ensimmäisen asteen yhtälöksi kutsutaan yhtälöä, joka on esitettävissä muodossa $ax+b=0$, jossa $a \neq 0$.}

Yhtälötyypin nimi tulee siitä, että korkein potenssi, johon tuntematon $x$
yhtälössä korotetaan, on $1$. (Sillä $x^{1}=x$.)

\begin{esimerkki}
Muun muassa seuraavat yhtälöt ovat ensimmäisen asteen yhtälöitä:
\begin{alakohdat}
\alakohta{$2x = 4$}
\alakohta{$5x+3 = 0$}
\alakohta{$x+2 = 3x-4$}
\end{alakohdat}
\end{esimerkki}

%pitää työstää tätä selitystä...
%\laatikko[Yhtälöiden muokkaaminen]{
%	\begin{description}
%		\item[Laskutoimitukset], joita ovat
%		
%			\begin{itemize}
%				\item{Puolittain lisääminen ja vähentäminen: yhtälön molemmille puolille voidaan lisätä tai molemmilta puolilta voidaan vähentää luku tai muuttuja. 
%				Esimerkiksi yhtälö $3x+5 = 3$ saadaan näin muotoon $3x = -2$.}
%				\item{Puolittain kertominen tai jakaminen: yhtälön molemmat puolet voidaan kertoa tai jakaa nollasta poikkeavalla luvulla, muuttujalla tai lausekkeella. Muuttujalla tai lausekkeella jakaessa täytyy ottaa huomioon ja merkitä, että yhtälö ei ole määritelty, kun jakaja on nolla.
%				Esimerkiksi kertomalla yhtälön $2x = 4$ molemmat puolet luvulla $\frac{1}{2}$ saadaan yhtälö $x = 2$.}
%				\item{Huomaa, että itseasiassa vähennyslasku on negatiivisen luvun summaamista [$a-b=a+(-b)$], 
%				ja jakaminen on jakajan käänteisluvulla kertomista [$a:b=a\cdot{\frac{1}{b}}=\frac{a}{b}$]}				
%			\end{itemize}
%		\item[Sieventäminen laskusääntöjä käyttäen] Yhtälön lausekkeista voi esimerkiksi hakea yhteistä tekijää, avata sulkuja tai supistaa, 
%		ja pyrkiä täten saamaan lauseke yksinkertaisempaan muotoon. Esimerkiksi yhtälöä $3(x-2)=5(x+6)$ kannattaa lähteä ratkaisemaan avaamalla ensin 
%		sulut soveltamalla tuttuja laskusääntöjä. Näin yhtälö saadaan muotoon $3x-6=5x+30$, jonka jälkeen on helppoa jatkaa käyttämällä laskutoimituksia 
%		tuntemattoman $x$ ratkaisemiseksi. Yhtälön ratkaisu on $x=-18$.
%	\end{description}

Ensimmäisen asteen yhtälöt ratkaistaan kuten edellisessä kappaleessa esitettiin, eli tekemällä yhtälön molemmille puolille 
samoja toimenpiteitä, kunnes $x$ saadaan selvitettyä. 

\begin{esimerkki}
Yhtälön $7x+4=4x+7$ ratkaisu saadaan seuraavasti:
\begin{align*}
7x+4 &= 4x+7 & &| \, \text{Vähennetään molemmilta puolilta $4x$.} \\
3x+4 &= 7 & &| \, \text{Vähennetään molemmilta puolilta 4.} \\
3x &= 3 & &| \, \text{Jaetaan molemmat puolet luvulla 3.} \\
x &= 1 & & \\
\end{align*}

\textbf{Vastaus.} $x=1$
\end{esimerkki}

Ensimmäisen asteen yhtälöllä on aina täsmälleen yksi ratkaisu.

Kaikki muotoa $ax+b=cx+d$ olevat yhtälöt, joissa $a \neq c$, ovat ensimmäisen asteen yhtälöitä. Tämä voidaan todistaa seuraavasti:

\begin{align*}
ax+b &= cx+d & &| \, \text{Vähennetään molemmilta puolilta $cx+d$}. \\
ax+b - (cx+d) &= 0 & &| \, \text{Avataan sulut ja järjestellään termejä uudelleen.} \\
ax - cx + b - d &= 0 & &| \, \text{Otetaan yhteinen tekijä.} \\
(a-c)x + (b-d) &= 0 & &
\end{align*}

Tämä on määritelmän mukainen ensimmäisen asteen yhtälö, koska $a \neq c$, eli $a-c \neq 0$.

\begin{esimerkki}
Yleinen lähestymistapa muotoa $ax+b = cx+d$ olevien yhtälöiden ratkaisuun: \\
(1) Vähennetään molemmilta puolilta $cx$ ja otetaan $x$ yhteiseksi tekijäksi. Saadaan yhtälö $(a-c)x + b = d$. \\
(2) Vähennetään molemmilta puolilta $b$. Saadaan yhtälö $(a-c)x = d-b$. \\
(3) Jaetaan molemmat puolet lausekkeella $(a-c)$. Saadaan yhtälö ratkaistuun muotoon $x = \frac{d-b}{a-c}$. 
\end{esimerkki}

Ensimmäisen asteen yhtälöllä voidaan mallintaa joitakin yksinkertaisia käytännön tilanteita. 

\begin{esimerkki} 
Matti ja Teppo ovat veljeksiä. Teppo on 525 päivää vanhempi kuin Matti. Muodostetaan ensimmäisen asteen yhtälö, 
josta voidaan ratkaista Matin ikä sijoittamalla $x$:n paikalle Tepon ikä. Lasketaan saadulla yhtälöllä Matin ikä sinä päivänä, kun 
Teppo täyttää 50 vuotta. Oletetaan nyt, että vuodessa on 365 päivää.

Merkitään Tepon ikää päivissä $x$:llä ja Matin ikää päivissä $y$:llä. Tiedetään, että koska Teppo on 525 päivää vanhempi kuin 
Matti, niin $x=y+525$. Koska tehtävässä pyydettiin yhtälöä, josta voidaan ratkaista Matin ikä, muokataan saatua ensimmäisen asteen 
yhtälöä hieman:

      \begin{align*}
	   x &= y+525 & &| \, \text{Vähennetään puolittain 525.} \\
	   x-525 &= y & &| \, \text{Tiedetään, että jos $a=b$, niin $b=a$.} \\
	   y &= x-525 & &
      \end{align*}
      
Nyt on saatu ensimmäisen asteen yhtälö, josta saadaan ratkaistua Matin ikä, kunhan vain tiedetään Tepon ikä. Koska $x$ ja $y$ ovat 
veljesten iät päivinä, muunnetaan hieman yksiköitä. 50 vuotta päivissä on $50 \cdot 365 = 18\,250$. Nyt saadaan

      \begin{align*}
	  y &= x-525 & &| \, \text{Sijoitetaan $x$:n paikalle Tepon ikä $18\,250$ päivää.} \\
	  y &= 18\,250-525 & & \\
	  y &= 17\,725 & &| \, \text{Muokataan vastausta informatiivisempaan muotoon.} \\
	  y &= 17\,520+205 & & \\
	  y &= 48 \cdot 365 +205 & &
      \end{align*}
      
  Siis kysytty Matin ikä on 48 vuotta ja 205 päivää.
  
\end{esimerkki}
%räpelsi jaakko viertiö 23.3.2014 