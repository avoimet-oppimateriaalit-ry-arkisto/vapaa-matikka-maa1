Jos $a$ on rationaaliluku ja $n$ on positiivinen kokonaisluku, potenssimerkintää $a^n$ käytetään lyhennysmerkintänä tulolle
   
\laatikko{    \[
        a^n = \underbrace{a\cdot \ldots \cdot a}_{n\text{ kpl}}. 
    \]}
    
Lukua $a$ kutsutaan potenssin \termi{kantaluku}{kantaluvuksi} ja lukua $n$ \termi{eksponentti}{eksponentiksi}. Merkinnällä $2^4$ siis tarkoitetaan tuloa 
        \[
            2^4=2\cdot 2\cdot 2\cdot 2=16.
        \]
Luvun toista potenssia $a^2$ kutsutaan myös luvun $a$ \termi{neliö}{neliöksi} ja kolmatta potenssia $a^3$ sen \termi{kuutio}{kuutioksi}. Nimitysten taustalla on,
että luvun $a>0$ neliö on sellaisen neliön pinta-ala, jonka sivun pituus on $a$. Vastaavasti, jos kuution särmän pituus on $a$, on $a^3$ kyseisen kuution tilavuus.
Luvun ensimmäinen potenssi on luku itse eli $a^1 = a$.

\subsubsection*{Laskujärjestys}

Sopimus on, että potenssit lasketaan ennen kerto- ja jakolaskuja. Sulut lasketaan ensin tavalliseen tapaan.

    \begin{esimerkki}
      Tarkkaile laskujärjestystä
        \begin{alakohdat}
            \alakohta{$1+2\cdot 5^2 = 1+2\cdot 25 = 51$}
            \alakohta{$1+3^2 = 1+3\cdot 3 = 10$}
            \alakohta{$(1+3)^2=4^2=16$.}
        \end{alakohdat}
    \end{esimerkki}  

Mikäli potensseja on monta päällekkäin (eli eksponenttina on potenssilauseke),
laskeminen aloitetaan ylimmästä potenssista. Esimerkiksi
\[2^{3^2}= 2^9 = 512, \]
mutta
\[(2^3)^2=8^2=64. \]

   
%    \laatikko{
%        \begin{enumerate}
%            \alakohta{Sulut}
%            \alakohta{Potenssilaskut}
%            \alakohta{Kerto- ja jakolaskut vasemmalta oikealle}
%            \alakohta{Yhteen- ja jakolaskut vasemmalta oikealle}
%        \end{alakohdat}
%    }
%  

\subsubsection*{Negativinen kantaluku}

Jos kantaluku on negatiivinen, sen ympärille merkitään selkeyden vuoksi
aina sulkeet. Esimerkiksi

\[ (-2)^4 = -2 \cdot (-2)\cdot(-2)\cdot(-2) = 16. \]

Merkinnässä $-2^4$ puolestaan potenssin kantaluku on 2. Koska $2^4 = 16$,
$-2^4=-16$. Kerralla laskettuna:     

\[ -2^4 =-(2^4)= -(2 \cdot 2\cdot 2 \cdot 2) = -16. \]          
            
    \begin{esimerkki}
        Negatiivisella kantaluvulla lasketaan seuraavasti
        \begin{alakohdat}
            \alakohta{$(-5)^2 = -5\cdot (-5)= 25$}
            \alakohta{$-5^2 = -(5\cdot 5)= -25$}
            \alakohta{$10-5^2=10-25=-15$}
            \alakohta{$10+(-5)^2 = 10 + 25 = 35$.}
         \end{alakohdat}
    \end{esimerkki}
    
    
Huomaa seuraavat potenssien etumerkkeihin liittyvät ominaisuudet.

\laatikko{
	\begin{itemize}
		\item $x^a=0$ kun $x=0$
		\item Kun eksponentti $a$ on pariton kokonaisluku, niin tuloksen etumerkki on sama, kuin kantaluvun etumerkki.
		\item Jos eksponentti on taas parillinen, on tulos aina positiivinen. Siis, kun $a$ on parillinen kokonaisluku, niin $x^{a}\geq0$.
			 \begin{itemize}
				\item $x^{a}\geq0$ kun $x\geq0$
				\item $x^{a}\leq0$ kun $x\leq0$
			 \end{itemize}
	\end{itemize}
}


Parillinen potenssi on negatiivisellakin kantaluvulla positiivinen siksi koska tulossa on negatiivisia kertoimia parillinen määrä. Tällöin miinukset kumoavat toisensa, olipa kantaluku mikä tahansa. 
Esimerkiksi $(-1)^4=(-1) \cdot (-1) \cdot (-1) \cdot (-1)= 1\cdot 1=1$

Kun negatiivinen kantaluku korotetaan parittomaan potenssiin, tulee kertoimia pariton määrä, jolloin tulos on negatiivinen. 
Esimerkiksi $(-2)^5=(-2) \cdot (-2) \cdot (-2) \cdot (-2) \cdot (-2)=4\cdot4\cdot(-2)=16\cdot(-2)= -32$
    
    
\subsection*{Potenssien laskusäännöt}

Potenssilausekkeita voi usein sieventää paljonkin, vaikka kantaluku
olisi tuntematon. Tässä esimerkkejä.    
    
    Samankantaisten potenssien kertolasku
	\[
a^3\cdot a^4=\underbrace{a\cdot a\cdot a}_{\text{3 kpl}}\cdot \underbrace{a\cdot a\cdot a\cdot a}_{\text{4kpl}}=a^{\mathbf{3+4}}=a^7
    	\]
    Samankantaisten potenssien jakolasku
	\[
\frac{a^7}{a^4}=\frac{a\cdot a\cdot a\cdot \cancel{a}\cdot \cancel{a}\cdot \cancel{a}\cdot \cancel{a}}	{\cancel{a}\cdot \cancel{a}\cdot \cancel{a}\cdot \cancel{a}}=a^{\mathbf{7-4}}=a^3
    	\]
    Potenssin potenssi
	\[
(a^2)^3=\underbrace{a^2\cdot a^2\cdot a^2}_{3\text{ kpl}}=
\underbrace{(a\cdot a)\cdot (a\cdot a)\cdot (a\cdot a)}_{2\cdot 3=6\text{ kpl}}=a^{\boldsymbol{{2\cdot 3}}}=a^6
\]
    Tulon potenssi
	\[
(ab^5)^3=ab^5\cdot ab^5\cdot ab^5=a\cdot a\cdot a\cdot b^5\cdot b^5\cdot b^5=a^{\mathbf{1\cdot 3}}\cdot b^{\mathbf{5\cdot 3}}=a^3b^{15}
	\]
     Osamäärän potenssi
	\[
	\left(\frac{a^9}{b^7}\right)^3=\frac{a^9}{b^7}\cdot \frac{a^9}{b^7}\cdot \frac{a^9}{b^7}=\frac{a^9\cdot a^9\cdot a^9}{b^7\cdot b^7\cdot b^7}=\frac{a^{\mathbf{9\cdot 3}}}{b^{\mathbf{7\cdot 3}}}=\frac{a^{27}}{b^{21}}
	\]

 Edellä esitetyt ideat voi yleistää potenssien laskusäännöiksi:
    
    \laatikko{
        \textbf{Potenssien laskusääntöjä}
                  
        \begin{tabular}{ll}
            $a^m\cdot a^n            = a^{m+n}$ & Samakantaisten potenssien tulo\\ % Kielitoimiston sanakirjan mukaan samakantainen, ei samankantainen
            $\displaystyle \frac{a^m}{a^n}= a^{m-n}$ & Samakantaisten potenssien osamäärä ($a\neq 0$)\\
            $(a^m)^n                 = a^{m\cdot n}$ & Potenssin potenssi\\
            $(a\cdot b)^n            = a^n\cdot b^n$ & Tulon potenssi\\
            $\displaystyle \left (\frac{a}{b}\right)^n = \frac{a^n}{b^n}$ & Osamäärän potenssi ($b\neq 0$)\\ 
        \end{tabular}
    }
 
    \begin{esimerkki}
        Lasketaan potenssien laskusääntöjen avulla
        \begin{alakohdat}
            \alakohta{$a^3\cdot a^2 = a^{2+3}=a^5$}
            \alakohta{$(a^3)^2 = a^{2\cdot 3}=a^6$}
            \alakohta{$\frac{7^9}{7^7}=7^{9-7}=7^2=49$}
            \alakohta{$(ab)^5=a^5b^5$.}
         \end{alakohdat}
    \end{esimerkki} 
 
\subsection*{Negatiivinen luku ja nolla eksponenttina}

Eksponenttimerkintää voidaan laajentaa myös tapauksiin, joissa eksponentti on negatiivinen tai nolla, kunhan määritellään järkevästi, mitä sillä tarkoitetaan.

Oletetaan seuraavissa päättelyissä, että $a \neq 0$. Silloin voimme jakaa luvulla $a$.
    
Sievennetään osamäärä $\frac{a^3}{a^5}$ kahdella eri tavalla. Supistamalla yhteiset tekijät saadaan lopputulos
    
    \begin{equation*}
        \frac{a^3}{a^5} =
        \frac{\cancel{a}\cdot \cancel{a}\cdot \cancel{a}}{a\cdot a\cdot
        \cancel{a}\cdot \cancel{a}\cdot \cancel{a}} = 
        \frac{1}{a\cdot a}=\boldsymbol{\frac{1}{a^2}}.
    \end{equation*}
    
Potenssien laskusääntöjä käyttämällä saataisiin
    
    \begin{equation*}
        \frac{a^3}{a^5} = a^{3-5}= a^{-2}{.}
    \end{equation*}
    
Samasta laskusta saatiin kaksi eri tulosta, joten on johdonmukaista määritellä
    
    \begin{equation*}
        a^{-2} = \frac{1}{a^2}.
    \end{equation*}

Sama ajatus yleistyy: koska potenssien laskusääntöjen halutaan pätevän myös negatiivisille eksponenteille, ne voidaan määritellä vain yhdellä tavalla:
  
    \laatikko{
        \textbf {Negatiivinen eksponentti} (kun $a\neq 0$)
        \begin{equation*}
        	a^{-n} = \frac{1}{a^n}
        \end{equation*}
    }

Erityisesti $a^{-1}=\frac{1}{a}$ kaikilla $a \neq 0$, eli miinus yksi eksponenttina tarkoittaa käänteislukua.
    
\begin{esimerkki}

$4^{-1}=\frac{1}{4^1}=\frac{1}{4}$ \\
$8^{-1}=\frac{1}{8^1}=\frac{1}{8}$ \\
$(\frac{2}{3})^{-1}=\frac{1}{(\frac{2}{3})^{1}}=\frac{1}{\frac{2}{3}}=1:\frac{2}{3}=1\cdot \frac{3}{2}=\frac{3}{2}$ \\
$(\frac{5}{7})^{-1}=\frac{7}{5}$\\
$3^{-2}=\frac{1}{3^2}=\frac{1}{9}$\\
$(\frac{4}{5})^{-2}=(\frac{5}{4})^2=\frac{5^2}{4^2}=\frac{25}{16}$\\
 
\end{esimerkki}


Vastaavasti potenssin $a^0$ voi määritellä vain yhdellä tavalla, jos halutaan potenssien laskusääntöjen pätevän sillekin. Lasketaan $\frac{a^3}{a^3}$ kahdella eri tavalla:

    \[
        \frac{a^3}{a^3}=\frac{\cancel{a}\cdot \cancel{a}\cdot \cancel{a}}
        {\cancel{a}\cdot \cancel{a}\cdot \cancel{a}}=1,
    \]
	toisaalta
 \[ \frac{a^3}{a^3}=a^{3-3}=a^0. \]

Siis $a^0=1$, kunhan $a\neq 0$.  Laskua $0^0$ ei ole määritelty, koska
nollalla ei voi jakaa. 
 
  
    \laatikko{
        \textbf{Eksponenttina nolla} (kun $a\neq 0$)
        \begin{equation*}
            a^{0}=1
        \end{equation*}
    }

    \begin{esimerkki}
        Laske
        \begin{alakohdat}
            \alakohta{$2^{-3}=\frac{1}{2^3} = \frac{1}{8}$}
            \alakohta{$7^0=1$}
            \alakohta{$(-556)^0=1$}
            \alakohta{$(-2)^{-2}=\frac{1}{(-2)^{2}}=\frac{1}{4}$.}
         \end{alakohdat}
    \end{esimerkki}

Jos taulukoidaan esimerkiksi luvun 2 potensseja, huomataan että
yhtä pienempi potenssi tarkoittaa aina sitä, että luku on \emph{puolet}
alkuperäisestä. Alla olevassa taulukossa on muutamien lukujen potensseja:
   


\begin{tabular}{c|c c c c c c c}
Kantaluku & $a^{-3}$ & $a^{-2}$ & $a^{-1}$ & $a^0$ & $a^1$ & $a^2$ & $a^3$ \\
\hline
$2$ & $\frac{1}{8}$ & $\frac{1}{4}$ & $\frac{1}{2}$ & $1$ & $2$ & $4$ & $8$ \\
$3$ & $\frac{1}{27}$ & $\frac{1}{9}$ & $\frac{1}{3}$ & $1$ & $3$ & $9$ & $27$ \\
$4$ & $\frac{1}{64}$ & $\frac{1}{16}$ & $\frac{1}{4}$ & $1$ & $4$ & $16$ & $64$ \\
$\frac{1}{3}$ & $27$ & $9$ & $3$ & $1$ & $\frac{1}{3}$ & $\frac{1}{9}$ & $\frac{1}{27}$ \\
$-2$ & $-\frac{1}{8}$ & $\frac{1}{4}$ & $-\frac{1}{2}$ & $1$ & $-2$ & $4$ & $-8$
\end{tabular}
