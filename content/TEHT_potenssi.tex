\begin{tehtavasivu}

\paragraph*{Opi perusteet}

\begin{tehtava}
Mikä seuraavissa potenssimerkinnöissä on kantaluku, eksponentti ja potenssin arvo?
    \begin{alakohdat}
	\alakohta{$5^3$}
	\alakohta{$2^5$}
	\alakohta{$9$}
	\alakohta{$1$}
	\alakohta{$(-4)^3$}
	\alakohta{$(-1)^{10}$}
	\alakohta{$6^{-2}$}
    \end{alakohdat}
\begin{vastaus}
 \begin{alakohdat}
 	\alakohta{kantaluku $5$, eksponentti $3$, potenssin arvo $125$}
	\alakohta{kantaluku $2$, eksponentti $5$, potenssin arvo $32$}
	\alakohta{kantaluku $9$, eksponentti $1$, potenssin arvo $9$}
	\alakohta{kantaluku $1$, eksponentti $1$, potenssin arvo $1$}
	\alakohta{kantaluku $-4$, eksponentti $3$, potenssin arvo $-64$}
	\alakohta{kantaluku $-1$, eksponentti $10$, potenssin arvo $1$}
	\alakohta{kantaluku $6$, eksponentti $-1$, potenssin arvo $\frac{1}{36}$}
 \end{alakohdat}
\end{vastaus}
\end{tehtava}


 \begin{tehtava}
        Esitä potenssin avulla 
        \begin{alakohdat}
        \alakohta{$a\cdot a\cdot a$}
        \alakohta{$a\cdot a\cdot a\cdot b\cdot b\cdot b\cdot b$}
        \alakohta{$a\cdot b\cdot a\cdot b\cdot a\cdot b\cdot a$.}
        \end{alakohdat}
        
        \begin{vastaus}
        \begin{alakohdatrivi}
            \alakohta{$a^3$}
            \alakohta{$a^3b^4$}
            \alakohta{$a^4b^3$}
        \end{alakohdatrivi}
        \end{vastaus}
    \end{tehtava}
    \begin{tehtava}
    Laske
        \begin{alakohdatrivi}
        \alakohta{$2^4$ }
       	\alakohta{$(-2)^4$}
        \alakohta{$-2^4$.}
 		\end{alakohdatrivi}
        \begin{vastaus}
        \begin{alakohdatrivi}
            \alakohta{$16$}
            \alakohta{$16$}
            \alakohta{$-16$}
        \end{alakohdatrivi}
        \end{vastaus}
    \end{tehtava}


    \begin{tehtava}
    Sievennä
        \begin{alakohdatrivi}
        	\alakohta{$a^2\cdot a^3$ }
       		\alakohta{$(a^2)^3$}
        	\alakohta{$\frac{a^7}{a^5}$}
        	\alakohta{$a^0$.}
 		\end{alakohdatrivi}
        \begin{vastaus}
        \begin{alakohdatrivi}
            \alakohta{$a^5$}
            \alakohta{$a^6$}
            \alakohta{$a^2$}
            \alakohta{$1$, ei määritelty kun $a=0$}
        \end{alakohdatrivi}
        \end{vastaus}
    \end{tehtava}
    
\begin{tehtava}
  		Laske
        \begin{alakohdatrivi}
            \alakohta{$-7^{2}$}
            \alakohta{$(-7)^2$}
        	\alakohta{$7^{-2}$}
        	\alakohta{$(-7)^{-2}$.}
 		\end{alakohdatrivi}
        \begin{vastaus}
        \begin{alakohdatrivi}
            \alakohta{$-49$}
            \alakohta{$49$}
            \alakohta{$\frac{1}{7^2}=\frac{1}{49}$}
            \alakohta{$\frac{1}{(-7)^2}=\frac{1}{49}$}
        \end{alakohdatrivi}
        \end{vastaus}
\end{tehtava}

        \begin{tehtava}
    Sievennä muotoon, jossa ei ole sulkuja
        \begin{alakohdatrivi}
        	\alakohta{$(ab)^2$}
       		\alakohta{$\left( \frac{a}{b} \right)^3$.}
 		\end{alakohdatrivi}
        \begin{vastaus}
        \begin{alakohdatrivi}
            \alakohta{$a^2b^2$}
            \alakohta{$\frac{a^3}{b^3}$}
        \end{alakohdatrivi}
        \end{vastaus}
    \end{tehtava}

        \begin{tehtava}
  		Kirjoita murtolukuna
        \begin{alakohdatrivi}
        	\alakohta{$3^{-1}$}
      		\alakohta{$5^{-2}$}
        	\alakohta{$\left(\frac{3}{4}\right)^{-1}$.}
 		\end{alakohdatrivi}
       \begin{vastaus}
        \begin{alakohdatrivi}
            \alakohta{$\frac{1}{3}$}
            \alakohta{$\frac{1}{25}$}
            \alakohta{$\frac{4}{3}$}
        \end{alakohdatrivi}
        \end{vastaus}
    \end{tehtava}
    
    
    %siirretään yhtälölukuun
%\begin{tehtava}
%%Tehtävän laatinut Johanna Rämö 9.11.2013.
%%Ratkaisun tehnyt Johanna Rämö 9.11.2013.
%        \begin{alakohdat}
%        \alakohta{Onko $x=4$ yhtälön $x^3=64$ ratkaisu?}
%        \alakohta{Onko $x=2$ yhtälön $x^5=30$ ratkaisu?}
%        \alakohta{Onko $x=-1$ yhtälön $x^4-3x+2=0$ ratkaisu?}
%        \end{alakohdat}
%        
%        \begin{vastaus}
%        \begin{alakohdatrivi}
%            \alakohta{On, sillä $4^3=4 \cdot 4 \cdot 4=64$.}
%            \alakohta{Ei, sillä $2^5=2 \cdot 2 \cdot 2 \cdot 2 \cdot 2=32$.}
%            \alakohta{On, sillä $(-1)^4-3(-1)+2=1-3+2=0$.}
%        \end{alakohdatrivi}
%        \end{vastaus}
%\end{tehtava}

   \begin{tehtava}
        Laske \quad
        a) $2^3\cdot2^3$ \qquad
        b) $4^3$ \qquad
        c) $(2^2)^3$ \qquad
        d) $2^{2+2+2}$.

        \begin{vastaus}
            a) $64$ \qquad
            b) $64$ \qquad
            c) $64$ \qquad
            d) $64$
        \end{vastaus}
    \end{tehtava}
    
    \begin{tehtava}
        %Sievennä \quad
        a) $a^2\cdot a^3$ \qquad
        b) $a^3a^2$ \qquad
        c) $a^2 a$ \qquad
        d) $a a^2 a$ \qquad
        e) $a^2a^1a^3$
        
        \begin{vastaus}
            a) $a^5$ \qquad
            b) $a^5$ \qquad
            c) $a^3$ \qquad
            d) $a^4$ \qquad
            e) $a^6$
        \end{vastaus}
    \end{tehtava}

\begin{tehtava}
  		Sievennä
        \begin{alakohdatrivi}
        	\alakohta{$a^3\cdot a^2\cdot a^5$}
       		\alakohta{$(a^2a^3)^4$}
        	\alakohta{$\frac{k^3}{k^{-5}}$.}
 		\end{alakohdatrivi}
        \begin{vastaus}
        \begin{alakohdatrivi}
            \alakohta{$a^{10}$}
            \alakohta{$a^6$}
            \alakohta{$a^{20}$}
        \end{alakohdatrivi}
        \end{vastaus}
\end{tehtava}



\begin{tehtava}
  		Sievennä
        \begin{alakohdatrivi}
        	\alakohta{$b^3(ab^0)^2$}
       		\alakohta{$(ab^3)^0$}
        	\alakohta{$(aa^4)^3a^2$.}
 		\end{alakohdatrivi}
        \begin{vastaus}
        \begin{alakohdatrivi}
            \alakohta{$a^2b^3$}
            \alakohta{$1$ }
            \alakohta{$a^{17}$}
        \end{alakohdatrivi}
        \end{vastaus}
\end{tehtava}


    \begin{tehtava}
        %Sievennä \quad
        a) $a^0$ \qquad
        b) $a^0a^0$ \qquad
        c) $a a^1$ \qquad
        d) $aa^0$ \qquad
        e) $a^0a^1$
        
        \begin{vastaus}
            a) $1$ \quad ($a\neq0$, koska $0^0$ ei ole määritelty) \qquad
            b) $1$ \qquad
            c) $a$ \qquad
            d) $a^2$ \qquad
            e) $a$
        \end{vastaus}
    \end{tehtava}

    \begin{tehtava}
        Laske \quad
        a) $\displaystyle \frac{a^3}{a^2}$ \quad \
        b) $\displaystyle \frac{b^4}{b^2}$ \quad \
        c) $\displaystyle \frac{c^3}{c^1}$ \quad \
        d) $\displaystyle \frac{d^3}{d^0}$ \quad \
        e) $\displaystyle \frac{e^3}{e^4}$ \quad \
        f) $\displaystyle \frac{f^3}{f^5}$.
        
        \begin{vastaus}
            a) $a$ \qquad
            b) $b^2$ \qquad
            c) $c^2$ \qquad
            d) $d^3$ \qquad
            e) $e^{-1}$ \qquad
            f) $f^{-2}$
        \end{vastaus}
    \end{tehtava}
    
        \begin{tehtava}
     Sievennä.
        a) $(-a)\cdot(-a)$ \qquad
        b) $(-a)\cdot(-a)\cdot(-b)^3$ \qquad
        c) $(-a^2)\cdot(-a)^2$

        \begin{vastaus}
            a) $a^2$ \qquad
            b) $-a^2b^3$ \qquad
            c) $-a^4$
        \end{vastaus}
    \end{tehtava}
    
\paragraph*{Hallitse kokonaisuus}



\begin{tehtava}
  		Sievennä
        \begin{alakohdatrivi}
        	\alakohta{$(a^2)^{-2}$}
       		\alakohta{$(ab^{-1})^3$}
        	\alakohta{$\frac{k^3}{k^{-5}}$.}
 		\end{alakohdatrivi}
        \begin{vastaus}
        \begin{alakohdatrivi}
            \alakohta{$\frac{1}{a^4}$}
            \alakohta{$\frac{a^3}{b^3}$}
            \alakohta{$k^8$}
        \end{alakohdatrivi}
        \end{vastaus}
\end{tehtava}

\begin{tehtava}
  		Sievennä
        \begin{alakohdatrivi}
        	\alakohta{$\frac{10^{n+1}}{10\cdot 10^{n+1}} \cdot 10^{-1}$}
       		\alakohta{$\frac{125^3}{5^4}$}
        	\alakohta{$\frac{2\cdot 4^n}{4^2}$}
        	\alakohta{$\frac{8^{2i}}{16^{-3}}$.}
 		\end{alakohdatrivi}
        \begin{vastaus}
        \begin{alakohdatrivi}
            \alakohta{$\frac{1}{10^{2}}$}
            \alakohta{$5^5 = 3125$}
            \alakohta{$2^{2n-3}$}
            \alakohta{$2^{6i+12}$}
        \end{alakohdatrivi}
        \end{vastaus}
\end{tehtava}

\begin{tehtava}
  		Sievennä.
        \begin{alakohdat}
       		\alakohta{$\left(\frac{a^2b^{-2}}{a^2b}\right)^{-3}$}
        	\alakohta{$\frac{5a^2}{-15a}$}
		\alakohta{$\left(-(\frac{10^3}{100b})^2 b^{-1} \right )^2$.}
 		\end{alakohdat}
        \begin{vastaus}
        \begin{alakohdatrivi}
            \alakohta{$b^9$ }
            \alakohta{$-\frac{1}{3}a = -\frac{a}{3}$}
            \alakohta{$\frac{10\,000}{b^6}$}
        \end{alakohdatrivi}
        \end{vastaus}
\end{tehtava}

    \begin{tehtava}
    Sievennä.
        \begin{alakohdatrivi}
        	\alakohta{$a^2(-a^4) $ }
        	\alakohta{$(ab^2)^0$ }
        	\alakohta{$(3a)^3$ }
        	\alakohta{$(a^5b^3)^3$}
		\end{alakohdatrivi}        
        \begin{vastaus}
        \begin{alakohdat}
            \alakohta{$-a^6$ }
            \alakohta{$1$ }
            \alakohta{$27a^3$ }
            \alakohta{$a^{15}b^9$}
        \end{alakohdat}
        \end{vastaus}
    \end{tehtava}

    \begin{tehtava}
        Sievennä. \qquad
        a) $a^1 a a^2$ \qquad
        b) $aaaa$ \qquad
        c) $a^3ba^2$ \qquad
        d) $aba^0ba^1$
        
        \begin{vastaus}
            a) $ a^4$ \qquad
            b) $a^4$ \qquad
            c) $a^5b$ \qquad
            d) $a^2b^2$
        \end{vastaus}
    \end{tehtava}

    \begin{tehtava}
Laske $x^{\frac{4}{3}}$, kun tiedetään että
    \begin{alakohdat}
      \alakohta{$X=27$}
      \alakohta{$x^{\frac{3}{4}}=27$}
      \alakohta{$x^{- \frac{3}{4}}=27$}
      \alakohta{$x^{- \frac{4}{3}}=27$}
     \end{alakohdat}
\begin{vastaus}
\begin{alakohdat}
 \alakohta{$81$}
 \alakohta{$350,48$}
 \alakohta{$0,037$}
 \alakohta{$0,0029$}
\end{alakohdat}
\end{vastaus}
    \end{tehtava}
 
 \begin{tehtava}
Sievennä \quad
$4 \cdot \sqrt[x]{\frac{2^{x^2}}{(2^x)^2}}$
\begin{vastaus}
$2^x$
\end{vastaus}

 \end{tehtava}

 \begin{tehtava}
        Sievennä. \quad
        a) $(\frac{1}{2})\cdot(\frac{1}{2})$ \qquad
        b) $(-\frac{ab^2}{a^2b})^3$ \qquad
        c) $(-a^4b^4)^2$ \qquad
        d) $\left((\frac{a}{b})^4\right)^2$
        
        \begin{vastaus}
            a) $\frac{1}{4}$ \qquad
            b) $-\frac{b^3}{a^3}$ \qquad
            c) $a^8b^8$ \qquad
            d) $\frac{a^8}{b^8}$
        \end{vastaus}
    \end{tehtava}
\begin{tehtava}
Sievennä.
	\begin{enumerate}[a)]
	\item $\frac{2x^3}{x}$
	\item $\frac{3x^3y^2}{xy}$
	\item $\frac{x^2yz}{xy^2}$
	\item $\frac{6xy^3z^2}{2xz}$
	\end{enumerate}

\begin{vastaus}
	\begin{enumerate}[a)]
	\item $2x^2$
	\item $\frac{x}{y}$
	\item $\frac{xz}{y}$
	\item $3y^3z$
	\end{enumerate}
\end{vastaus}
\end{tehtava}

\begin{tehtava}
Sievennä.
	\begin{enumerate}[a)]
	\item $\frac{2x^5+3x^3}{x^2}$
	\item $\frac{6x^2+8y}{2x^2}$
	\item $\frac{3x-2x^2y^3}{xy}$
	\item $\frac{2x^2+3xy^2z-4xz}{2xy^2z}$
	\end{enumerate}

\begin{vastaus}
	\begin{enumerate}[a)]
	\item $2x^3+3x$
	\item $3+4 \frac{y}{x^2}$
	\item $\frac{3}{y} - 2xy^2$
	\item $\frac{x}{y^2z} + \frac{3}{2} + \frac{2}{y^2}$
	\end{enumerate}
\end{vastaus}
\end{tehtava}

\begin{tehtava}
%Tehtävän laatinut Johanna Rämö 9.11.2013.
%Ratkaisun tehnyt Johanna Rämö 9.11.2013.
        Pudotat pallon kädestäsi lattialle. Pallo pomppaa ensin metrin korkeudelle ja sen jälkeen jokaisen pompun korkeus on aina puolet edellisestä korkeudesta. Kuinka korkea on pallon 5. pomppu? Entä 13. pomppu?     
        \begin{vastaus}
        Viidennen pompun korkeus on $(1/2)^4=1/16 \approx 0,06$ metriä. Pallon 13. pompun korkeus on $(1/2)^{12} \approx 0{,}0002$ metriä eli $0,2$ millimetriä.
        \end{vastaus}
\end{tehtava}

\begin{tehtava}
Sievennä
$( \frac{2^{\frac{x}{y}}}{2^{\frac{y}{x}}} )^{xy}$
\begin{vastaus}
 $2^{x^2 - y^2}$
\end{vastaus}


\end{tehtava}

\begin{tehtava}
Sievennä.
$\sqrt[3]{ x^{\frac{1}{x}} \cdot x^{\frac{2}{x}} }$
 \begin{vastaus}
  $\sqrt[x]{x}$
 \end{vastaus}
\end{tehtava}
 
\paragraph*{Lisää tehtäviä}

    \begin{tehtava}
%Tehtävän laatinut Johanna Rämö 9.11.2013.
%Ratkaisun tehnyt Johanna Rämö 9.11.2013.
        Ympyrän mallisen pöytäliinan halkaisija on $1,2$ m.  Mikä on pöytäliinan pinta-ala? (Ympyrän pinta-ala A voidaan laskea kaavalla $A=\pi\text{r}^2$, missä r on pöytäliinan säde)
        \begin{vastaus}
        Pinta-ala on $\pi \cdot (0{,}6)^2 \approx 1,1$ $cm^2$.
        \end{vastaus}
\end{tehtava}


%%%%%%%%%%%%%%%%%%%%%%%%%%%%%%%%%%

 \begin{tehtava}
        Sievennä 
        a) $(1\cdot a)^3$ \qquad
        b) $(a\cdot 2)^2$ \qquad
        c) $(-2abc)^3$ \qquad
        d) $(3a)^4$.

        \begin{vastaus}
            a) $a^3$ \qquad
            b) $4a^2$ \qquad
            c) $-8a^3b^3c^3$ \qquad
            d) $91a^4$
        \end{vastaus}
    \end{tehtava}
    \begin{tehtava}
        Sievennä. \quad
        a) $-a^3\cdot(-a^2)$ \qquad
        b) $a\cdot(-a)\cdot(-b)$ \qquad
        c) $a^2\cdot(-a^2)$
        
        \begin{vastaus}
            a) $a^5$ \qquad
            b) $a^2b$ \qquad
            c) $-a^4$
        \end{vastaus}
    \end{tehtava}

    \begin{tehtava}
        Sievennä. \quad
        a) $(a^3b^2)^2$ \qquad
        b) $a(a^2b^3)^4$ \qquad
        c) $(b^2a^4)^5$ \qquad
        d) $b(2ab^2)^3$
        
        \begin{vastaus}
            a) $a^6b^4$ \qquad
            b) $a^9b^{12}$ \qquad
            c) $a^{20}b^{10}$ \qquad
            d) $8a^3b^7$
        \end{vastaus}
    \end{tehtava}
      
    
    %teht. 20
    \begin{tehtava}
        Sievennä. \quad
        a) $\frac{a^2b^2}{ab}$ \qquad
        b) $\frac{a^2b}{a^2}$ \qquad
        c) $\frac{a^3}{a^3}$ \qquad
        d) $\frac{1}{a^0}$ \qquad
        e) $\frac{ab^3}{-b^4}$
        
        \begin{vastaus}
            a) $ab$ \qquad
            b) $b$ \qquad
            c) $1$ \qquad
            d) $1$ \qquad
            e) $-\frac{a}{b}$
        \end{vastaus}
    \end{tehtava}
    
   
    
    \begin{tehtava}
         Sievennä ja kirjoita potenssiksi, jonka eksponentti on positiivinen\\
        a) $a^{-3}$ \qquad
        b) $\frac{a}{a^3}$ \qquad
        c) $a^{-2}\cdot a^5$ \qquad
        d) $\frac{b}{a^4}b^{-4}$ \qquad
        e) $\frac{a^3}{a^{-5}}$.
        
        \begin{vastaus}
            a) $\frac{1}{a^3}$ \qquad
            b) $\frac{1}{a^2}$ \qquad
            c) $a^3$ \qquad
            d) $\frac{}{a^4b^3}$ \qquad
            e) $a^8$
        \end{vastaus}
    \end{tehtava}
    
    
    
    \begin{tehtava}
        Esitä ilman sulkuja ja sievennä \\
        a) $(\frac{1}{2})^2$ \qquad
        b) $(\frac{1}{3})^3$ \qquad
        c) $(\frac{a}{b})^4$ \qquad
        d) $(\frac{a^2}{b^3})^2$ \qquad
        e) $\left(\frac{a^2}{ab^2}\right)^2$.
        
        \begin{vastaus}
            a) $\frac{1}{4}$ \qquad
            b) $\frac{1}{27}$ \qquad
            c) $\frac{a^4}{b^4}$ \qquad
            d) $\frac{a^4}{b^6}$ \qquad
            e) $\frac{a^2}{b^4}$
        \end{vastaus}
    \end{tehtava}
    % FIXME alla olevat on muotoiltu erilailla (yksi alikohta per rivi).
% Mahtunee samalle riville.
% Lisäksi tässä on duplikaatteja (esim. aaaa:n sievennys on sekä yllä että alla). 

Laske tai sievennä.

    \begin{tehtava}%perteht
        %Sievennä
        \begin{alakohdatrivi}
        	\alakohta{$a^2a^5$}
        	\alakohta{$\frac{a^5}{a^3}$}
        	\alakohta{$(a^3)^2$ }
        	\alakohta{$12^0$}
		\end{alakohdatrivi}        
        \begin{vastaus}
        \begin{alakohdat}
            \alakohta{$a^7$ }
            \alakohta{$a^2$ }
            \alakohta{$a^6$ }
            \alakohta{$1$}
        \end{alakohdat}
        \end{vastaus}
    \end{tehtava}    
    
    %soveltavia tehtäviä
         
    
    \begin{tehtava}%sovteht
        %Sievennä
        \begin{alakohdatrivi}
        	\alakohta{$\frac{2^7}{2^9}$ }
        	\alakohta{$\frac{a^3}{a}$ }
        	\alakohta{$\left(\frac{1}{3}\right)^2$ }
        	\alakohta{$\left(\frac{a^{-2}}{ab^4}\right)^4$}
		\end{alakohdatrivi}        
        \begin{vastaus}
        \begin{alakohdat}
            \alakohta{$\frac{1}{4}$ }
            \alakohta{$a^2$ }
            \alakohta{$\frac{1}{9} $ }
            \alakohta{$ \left(\frac{1}{a^{12}b^{16}}\right)$ tai $a^{-12}b^{-16}$}
        \end{alakohdat}
        \end{vastaus}
    \end{tehtava}     


\begin{tehtava}
        Sievennä \quad
        a) $a^3\cdot b^2\cdot a^5$ \qquad 
        b) $(-ab^3)^2$ \qquad 
        c) $(a^5a^4)^3$ \qquad 
        d) $10^{2^3}$.

        \begin{vastaus}
            a) $a^8b^2$ \qquad
            b) $a^2b^6$ \qquad
            c) $a^{15}b^{12}$ \qquad
            d) $10^8 = 100\,000\,000$
        \end{vastaus}
    \end{tehtava}


\begin{tehtava}
	Laske
	\begin{alakohdat}
		\alakohta{$\left(\frac{4}{8}\right)^{1543} 2^{1546}$}
		\alakohta{$\left(\frac{28}{15}\right)^{214} \left(\frac{45}{98}\right)^{109} \left(\frac{5}{8}\right)^{105}$.}
	\end{alakohdat}
	
	\begin{vastaus}
		\begin{alakohdat}
			\alakohta{$8$}
			\alakohta{$\left(\frac{2\cdot 3}{7}\right)^4 = \frac{1296}{2401}$}
		\end{alakohdat}
	\end{vastaus}
\end{tehtava}

\begin{tehtava}
	Millä kokonaisluvun $n$ arvoilla
	\begin{alakohdatrivi}
		\alakohta{$2^n$}
		\alakohta{$(-3)^n$}
		\alakohta{$(-1)^{n-1}$}
		\alakohta{$(-1)^{n-1} (-2)^n$}
	\end{alakohdatrivi}
	on positiivinen?
	
	\begin{vastaus}
		\begin{alakohdat}
			\alakohta{kaikilla kokonaisluvuilla}
			\alakohta{parillisilla kokonaisluvuilla}
			\alakohta{parittomilla kokonaisluvuilla}
			\alakohta{ei millään kokonaisluvulla}
		\end{alakohdat}
	\end{vastaus}
\end{tehtava}
\begin{tehtava}
%Laatinut ja ratkaissut Matias Jalkanen 10.11.2013
Tarinan mukaan muuan intialainen ruhtinas pyysi erästä matemaatikkoa kehittämään uuden strategisen lautapelin ja luvannut hänelle pelin keksimisestä suuren palkkion. Tällöin matemaatikko keksi shakin. Ruhtinas ihastui peliin ja kysyi keksijältä, mitä tämä halusi palkkioksi. Keksijä pyysi palkkioksi niin monta vehnän jyvää kuin saadaan koko shakkilaudalta, jos niitä asetetaan sen ensimmäiselle ruudulle yksi, toiselle ruudulle kaksi, kolmannelle neljä, neljännelle kahdeksan ja edelleen jokaiselle ruudulle kaksi kertaa niin monta kuin edelliselle ruudulle.
	\begin{alakohdat}
		\alakohta{Kuinka monta vehnän jyvää tulee viidennelle ruudulle?}
		\alakohta{Kuinka monta vehnän jyvää viimeiselle ruudulle tulee?}
		\alakohta{$\star$ Kuinka montaa vehnän jyvää keksijä pyysi yhteensä?}
	\end{alakohdat}
	
	\begin{vastaus}
	      \begin{alakohdat}
		    \alakohta{$2^{5-1} = 16$}
		    \alakohta{$2^{64-1}$}
		    \alakohta{$2^64 - 1 = 18 446 744 073 709 551 615$}
	      \end{alakohdat}
	\end{vastaus}

\end{tehtava}

\begin{tehtava}
  		$\star$ Sievennä lauseke
$$\left[ \frac{(a^2b^{-2}c)^{-3}:\left(c^2\cdot (ab^{-2})^0 \cdot a^{-4}\right)}
{\left((c^{-1}\cdot a^3)^{-1}:a^2\right)^3(ab^2c^{-3})^3} \right]^2.$$
\begin{vastaus}
%tarkistettu ja korjattu, Matias Jalkanen 10.11.2013
%http://www.wolframalpha.com/input/?i=[%28a^2b^{-2}c%29^{-3}\%28c^2*%28ab^{-2}%29^0*a^{-4}%29}%2F{%28%28c^{-1}*a^3%29^{-1}*a^-2%29^3%28ab^2c^{-3}%29^3}]^2
$a^20c^2$
\end{vastaus}
\end{tehtava} 
 
\begin{tehtava}
%Tarkistettu ja korjattu, Matias Jalkanen 10.11.2013
$\star$ Tetraatio on lyhennysmerkintä ''potenssitornille'',
jossa esiintyy vain yhtä lukua. Se määritellään seuraavasti
\[^na = \underbrace{{a^{a^{a^{\mathstrut^{.^{.^{.^{a}}}}}}}}}_{n\textrm{ kpl }}. \]
Laske \quad a) $^42$  \quad b) $^35$. \\ c) Ratkaise yhtälö $^x2= 16$.
\begin{vastaus}
a) $^42 = 2^{2^{2^2}}=2^{2^4}=2^{16}=65\ 536$ \
b) $^35 = 4^{4^4} = 4^{256} \approx 1,34 \cdot 10^{154}$. \\
c) Kokeilemalla $x =3$.
\end{vastaus}
\end{tehtava}

\end{tehtavasivu}
