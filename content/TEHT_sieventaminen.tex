\begin{tehtavasivu}

\begin{tehtava}
Kirjoita seuraavat laskutoimitukset uudestaan käyttäen annettua laskulakia. Tarkista laskemalla, että tulos säilyy samana.
    %toisen laskutoimituksen tulos tosiaan on sama.

    \alakohdat{
        § $3\cdot (-6)$ \quad(vaihdantalaki)
        § $5\cdot (7+6)$ \quad(yhteenlaskun vaihdantalaki)
        § $5\cdot (7+6)$ \quad(kertolaskun vaihdantalaki)
        § $5\cdot (7+6)$ \quad (osittelulaki)
        § $-8\cdot (-5)\cdot 2$ \quad (liitäntälaki) %FIXME: check
    }
    \begin{vastaus}
	\alakohdat{
	    § $(-6)\cdot 3$
	    § $5\cdot (6+7)$
	    § $(7+6)\cdot 5$
	    § $5\cdot 7 + 5\cdot 6)$
	    § $-8\cdot ((-5)\cdot 2)$ (Vastaukseksi ei kelpaa $(-8\cdot (-5))\cdot 2$, koska siinä on sama laskujärjestys kuin alkuperäisessä laskussa.)
	}
    \end{vastaus}
\end{tehtava}

\begin{tehtava}
Laske seuraavat laskut ilman laskinta soveltamalla vaihdantalakia, liitäntälakia ja osittelulakia. Kerro, mitä laskulakeja sovelsit.

    \alakohdat{
        § $350\cdot 271-272\cdot 350$
        § $370\cdot 1010$
        § $594+368+3-368$
    }
    \begin{vastaus}
    	\alakohdat{
        §
            \begin{align*}
	    & 350\cdot 271-272\cdot 350 \\
	    =& 350\cdot 271-350\cdot 272 &\text{(vaihdantalaki)} \\
	    =& 350\cdot (271-272) &\text{(osittelulaki)} \\
	    =& 350\cdot (-1) \\
	    =& -350
	    \end{align*}
        § 
            \begin{align*}
	    & 370\cdot 1\,010 \\
	    =& 370\cdot (1\,000+10) \\
	    =& 370\cdot 1\,000 + 370\cdot 10 &\text{(osittelulaki)} \\
	    =& 370\,000 + 3\,700 \\
	    =& 373\,700
	    \end{align*}
		§
            \begin{align*}
	    & 594+368+3-368 \\
	    =& 594+368+(3-368) &\text{(liitäntälaki)} \\
	    =& 594+368+(-368+1) &\text{(vaihdantalaki)} \\
	    =& 594+(368+(-368))+1 &\text{(liitäntälaki)} \\
	    =& 594+0+1 \\
	    =& 595
	    \end{align*}
		}
    \end{vastaus}
\end{tehtava}

\begin{tehtava}
%Laatinut Henri Ruoho 9.11.2013
Sievennä
	\alakohdat{
		§ $2(a+b)-a$
		§ $3(2a+b)+(2a+b)$
		§ $-(-(-a)-b)$ 
		§ $-(-a-(-a-b)-b)\cdot(1+2b)$.
	}
\begin{vastaus}
	\alakohdat{
		§ $a+2b$
		§ $8a+4b$
		§ $-a+b$ 
		§ $0$ 
	}
\end{vastaus}
\end{tehtava}


\begin{tehtava}
%Laatinut Henri Ruoho 9.11.2013
Perustele annettujen määritelmien avulla, miksi
	\alakohdat{
		§ $3\cdot 2=6$
		§ $6-4=2$
		§ $3x+x=4x $
		§ $xy+yx=2xy$
	}
	kun \(x\) ja \(y\) ovat kokonaislukuja?
\begin{vastaus}
	\alakohdat{
		§ $3\cdot2 = 6$, koska $2+2+2=2+4=6$
		§ $6-4=2$, koska $2+4=6$
		§ $3x+x=3\cdot x+1\cdot x=(3+1)\cdot x=4x$
		§ $xy+yx=xy+xy=1\cdot (xy)+1\cdot (xy)=(1+1)\cdot (xy)=2(xy)=2xy$
	} %FIXME: Lisää tallaisia
\end{vastaus}
\end{tehtava}

\end{tehtavasivu}
