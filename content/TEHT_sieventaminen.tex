\begin{tehtavasivu}

\begin{tehtava}
Kirjoita seuraavat laskutoimitukset uudestaan käyttäen annettua laskulakia. Tarkista laskemalla, että tulos säilyy samana.
    %toisen laskutoimituksen tulos tosiaan on sama.

    \begin{alakohdat}
        \alakohta{$3\cdot (-6)$ \quad(vaihdantalaki)}
        \alakohta{$5\cdot (7+6)$ \quad(yhteenlaskun vaihdantalaki)}
        \alakohta{$5\cdot (7+6)$ \quad(kertolaskun vaihdantalaki)}
        \alakohta{$5\cdot (7+6)$ \quad (osittelulaki)}
        \alakohta{$-8\cdot (-5)\cdot 2$ \quad (liitäntälaki)}
    \end{alakohdat}
    \begin{vastaus}
	\begin{alakohdat}
	    \alakohta{$(-6)\cdot 3$}
	    \alakohta{$5\cdot (6+7)$}
	    \alakohta{$(7+6)\cdot 5$}
	    \alakohta{$5\cdot 7 + 5\cdot 6)$}
	    \alakohta{$-8\cdot ((-5)\cdot 2)$
	    (Vastaukseksi ei kelpaa $(-8\cdot (-5))\cdot 2$, koska siinä on sama laskujärjestys kuin alkuperäisessä laskussa.)}
	\end{alakohdat}
    \end{vastaus}
\end{tehtava}

\begin{tehtava}
Laske seuraavat laskut ilman laskinta soveltamalla vaihdantalakia, liitäntälakia ja osittelulakia.
Kerro, mitä laskulakeja sovelsit.

    \begin{alakohdat}
        \alakohta{$350\cdot 271-272\cdot 350$}
        \alakohta{$370\cdot 1010$}
        \alakohta{$594+368+3-368$}
    \end{alakohdat}
    \begin{vastaus}
    	\begin{alakohdat}
        \alakohta{
            \begin{align*}
	    & 350\cdot 271-272\cdot 350 \\
	    =& 350\cdot 271-350\cdot 272 &\text{(vaihdantalaki)} \\
	    =& 350\cdot (271-272) &\text{(osittelulaki)} \\
	    =& 350\cdot (-1) \\
	    =& -350
	    \end{align*}
	}
        \alakohta{
            \begin{align*}
	    & 370\cdot 1010 \\
	    =& 370\cdot (1000+10) \\
	    =& 370\cdot 1000 + 370\cdot 10 &\text{(osittelulaki)} \\
	    =& 370000 + 3700 \\
	    =& 373700
	    \end{align*}
	}
	\alakohta{
            \begin{align*}
	    & 594+368+3-368 \\
	    =& 594+368+(3-368) &\text{(liitäntälaki)} \\
	    =& 594+368+(-368+1) &\text{(vaihdantalaki)} \\
	    =& 594+(368+(-368))+1 &\text{(liitäntälaki)} \\
	    =& 594+0+1 \\
	    =& 595
	    \end{align*}
	}
	\end{alakohdat}
    \end{vastaus}
\end{tehtava}

\begin{tehtava}
%Laatinut Henri Ruoho 9.11.2013
Sievennä
	\begin{alakohdat}
		\alakohta{$2(a+b)-a$}
		\alakohta{$3(2a+b)+(2a+b)$}
		\alakohta{$-(-(-a)-b)$ }
		\alakohta{$-(-a-(-a-b)-b)\cdot(1+2b)$.}
	\end{alakohdat}
\begin{vastaus}
	\begin{alakohdat}
		\alakohta{$a+2b$}
		\alakohta{$8a+4b$}
		\alakohta{$-a+b$ }
		\alakohta{$0$ }
	\end{alakohdat}
\end{vastaus}
\end{tehtava}


\begin{tehtava}
%Laatinut Henri Ruoho 9.11.2013
Perustele annettujen määritelmien avulla, miksi
	\begin{alakohdat}
		\alakohta{$3\cdot 2=6$}
		\alakohta{$6-4=2$}
		\alakohta{$3x+x=4x $}
		\alakohta{$xy+yx=2xy$}
	\end{alakohdat}
	kun \(x\) ja \(y\) ovat kokonaislukuja?
\begin{vastaus}
	\begin{alakohdat}
		\alakohta{$3\cdot2 = 6$, koska $2+2+2=2+4=6$}
		\alakohta{$6-4=2$ koska $2+4=6$}
		\alakohta{$3x+x=3\cdot x+1\cdot x=(3+1)\cdot x=4x$ }
		\alakohta{$xy+yx=xy+xy=1\cdot (xy)+1\cdot (xy)=(1+1)\cdot (xy)=2(xy)=2xy$}
	\end{alakohdat}
\end{vastaus}
\end{tehtava}

\end{tehtavasivu}
