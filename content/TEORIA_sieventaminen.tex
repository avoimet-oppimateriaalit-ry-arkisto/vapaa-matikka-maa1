Mitä tahansa lukua tai kirjoitettua laskutoimitusta (kuten $9$ tai $\frac{x}{3}-9$) kutsutaan \termi{lauseke}{lausekkeeksi}. Lausekkeen \termi{arvo}{arvoksi} kutsutaan sitä lukua, joka saadaan, kun lausekkeen laskutoimitukset lasketaan. Esimerkiksi lausekkeen $5-3$ arvo on $2$. Samoin lausekkeen $2$ arvo on myös $2$.

Lausekkeiden sanotaan olevan yhtäsuuria, kun niiden arvot ovat yhtä suuria. Tällöin voidaan merkitä esimerkiksi $2+3=10:2$, koska lausekkeen $2+3$ arvo on $5$ ja lausekkeen $10:2$ arvo on myös $5$.

Usein matematikkassa on tapana, että luku ilmaistaan kirjoittamalla lauseke, jonka arvo kyseinen luku on. Tällöin tietyn luvun voi itse asiassa ilmaista hyvin monella eri tavalla. Esimerkiksi seuraavat merkinnät tarkoittavat kaikki samaa lukua:

\begin{align*}
                & 42 \\ & -(-42) \\ & 6 \cdot 7 \\ & (50-29) \cdot 2
\end{align*}

Niinpä matematiikassa voi luvun $42$ paikalle kirjoittaa merkinnän $(50-29)\cdot 2$, sillä se tarkoittaa samaa lukua kuin merkintä $42$.

Kun lauseke muodostuu yhteenlaskusta, käytetään nimitystä \termi{termi}{termi} tarkoittamaan yhtä yhteenlaskettavaa osaa lausekkeesta. Termissä voi olla yhdistelmä luvuista, vakioista ja muuttujista tulona tai se voi myös olla vain yksittäinen luku. Esimerkiksi lausekkeen $-4\sqrt{2}ax^2+ax+1$ termit ovat $-4\sqrt{2}ax^2$, $ax$ ja $1$.

Usein lausekkeessa esiintyy myös kirjaimia. Niillä merkitään lukuja, joiden arvoa ei tiedetä tai joiden arvo voi vaihdella tilanteesta riippuen ja niitä kutsutaan muuttujiksi, tuntemattomiksi tai vakioiksi (ks. luku \ref{muuttuja}). Yleensä lausekkeella on myös tällöin lukuarvo, mutta tätä lukuarvoa ei välttämättä voi laskea, jos muuttujan, tuntemattoman tai vakion lukuarvoa ei tiedetä.

Kun lausekkeissa esiintyy tuntemattomia lukuja, on tapana sanoa, että kaksi lauseketta ovat yhtä suuria silloin, kun ne ovat yhtä suuria riippumatta siitä, mikä lukuarvo tuntemattomilla on. Esimerkiksi lausekkeet $x+y+2$ ja $y+2+x$ ovat yhtä suuria, koska riippumatta siitä, mitä $x$:n ja $y$:n lukuarvot ovat, kummastakin lausekkeesta saadaan sama tulos.

\subsection*{Laskujärjestys}

Kun lauseke sisältää monta eri laskutoimitusta, on laskutoimitusten suoritusjärjestyksellä usein vaikutusta lausekkeen arvoon. Tällöin käytetään sulkumerkkejä ilmaisemaan haluttua laskujärjestystä. Tämän lisäksi on olemassa yleinen sopimus siitä, missä järjestyksessä laskutoimitukset lasketaan silloin, kun suluilla ei ole muuta ilmaistu.

Jos lausekkeessa on useita eri laskutoimituksia, ne suoritetaan seuraavassa järjestyksessä:

\numerointilaatikko{Suoritusjärjestys}{
§ Suluissa olevat lausekkeet sisemmistä suluista ulompiin sulkuihin
§ Potenssilaskut (potenssia käsitellään tarkemmin myöhemmin tässä kirjassa)
§ Kerto- ja jakolaskut vasemmalta oikealle
§ Yhteen- ja vähennyslaskut vasemmalta oikealle
}

Käytännössä laskutoimituksen voi laskea, jos sen kummallakaan puolella ei ole laskutoimitusta, joka pitäisi laskea ensin.

Tässä kannattaa huomata lisäksi, että vaikka laskujärjestyssääntöjen mukaan kerto- ja jako- sekä yhteen- ja vähennyslaskut lasketaan vasemmalta oikealle, on tiettyjä tilanteita, joissa järjestys ei vaikuta laskun lopputulokseen ja laskut voidaan laskea myös vaikka oikealta vasemmalle. Tästä puhutaan tarkemmin vähän myöhemmin vaihdanta- ja liitäntälain yhteydessä.

\begin{esimerkki}
Lasketaan lausekkeen $2+(3+1)\cdot 5$ arvo.

Koska lausekkeessa on useita eri laskutoimituksia, joudumme laskemaan ne annetun järjestyksen mukaisesti.

Ensimmäiseksi laskemme lausekkeen $(3+1)$ arvon, koska se on sulkujen sisällä. Tämän jälkeen laskemme tulon, koska kertolasku lasketaan ennen yhteenlaskua. Lopuksi summaamme saadun tulon luvun $2$ kanssa.

\begin{align*}
   &2+(3+1)\cdot 5&\textrm{Lasketaan ensin } (3+1)
\\= &2+4\cdot5&\qquad\textrm{sen jälkeen tulo } 4\cdot 5
\\= &2+20& \textrm{lopuksi vielä summa}
\\= &22&
\end{align*}
\end{esimerkki}

%\begin{esimerkki}
%Lasketaan laskutoimitus $-(8+3)+3+1-2\cdot 5$.
%
%Suluissa olevan laskutoimituksen voi laskea heti. Sen sijaan laskua $1-2$ ei voi laskea vielä,
%koska vieressä oleva kertolasku on laskujärjestyksessä sitä ennen.
%%"laskua ei voida laskea" olettaa, että siinä on lasku 1-2., mitä siinä EI OLE! Mur. Tämä pitää selittää kunnolla! :C T: Joonas
%
%Laskun $3+1$ sen sijaan voi laskea. Oikealla puolella oleva vähennyslasku on
%laskujärjestyksessä vasta sen jälkeen, ja vasemmalla puolella oleva yhteenlasku ei haittaa,
%koska yhteenlaskut voi laskea missä järjestyksessä vain.
%
%\begin{align*}
%   &-(8+3)+4-2\cdot 5&\textrm{Lasketaan ensin kertolasku ja suluissa oleva lasku.}
%\\=&-11+4-10&\qquad\textrm{Loput laskut lasketaan laskujärjestyksen mukaan vasemmalta oikealle.}
%\\=&-7-10&
%\\=&-17&
%\end{align*}
%\end{esimerkki}

\subsection*{Lausekkeiden sieventäminen}

Matemaattisia ongelmia ratkaistaessa kannattaa usein etsiä vaihtoehtoisia tapoja jonkin laskutoimituksen, lausekkeen tai luvun ilmaisemiseksi. Tällöin usein korvataan esimerkiksi jokin laskutoimitus toisella laskutoimituksella, josta tulee sama tulos. Näin lauseke saadaan sellaiseen muotoon, jonka avulla ratkaisussa päästään eteenpäin. Kun merkitsemme monimutkaisen lausekkeen lyhyemmin, sitä kutsutaan \termi{sieventäminen}{sieventämiseksi}. Sieventäminen on ikään kuin sotkuisen kaavan siistimistä selkeämmäksi. Tähän lukuun on koottu sääntöjä, joiden avulla lauseketta voi muutaa niin, että sen arvo ei muutu.

Tässä luvussa on esitelty tärkeimpiä sääntöjä, joita voidaan käyttää lausekkeiden sieventämisessä. Näitä sääntöjä tulee myöhemmin paljon lisää. Monet myöhemmin vastaan tulevista säännöistä voidaan myös perustella tässä luvussa esitettyjen sääntöjen avulla.

\subsubsection*{Vaihdantalaki ja liitäntälaki}

\laatikko[Yhteenlaskujen järjestys]{
Yhteenlaskut voi laskea missä järjestyksessä tahansa

\begin{tabular}{ll}
  $a+b=b+a$\qquad\qquad&(vaihdantalaki)\\
  \\
  $a+(b+c)=(a+b)+c=a+b+c$\qquad\qquad&(liitäntälaki)
\end{tabular}
}

Esimerkiksi laskemalla voidaan tarkistaa, että $5+7=7+5$ ja että $(2+3)+5=2+(3+5)$.

Nämä säännöt voidaan yhdistää yleiseksi säännöksi, jonka mukaan yhteenlaskun sisällä laskujärjestystä voi vaihtaa miten tahansa.

Tämä sääntö voidaan yleistää koskemaan myös vähennyslaskua, kun muistetaan, että vähennyslasku tarkoittaa oikeastaan vastaluvun lisäämistä. $5-8$ tarkoittaa siis samaa kuin $5+(-8)$, joka voidaan nyt kirjoittaa yhteenlaskun vaihdantalain perusteella muotoon $(-8)+5$ eli $-8+5$ ilman, että laskun lopputulos muuttuu.
Tästä seuraa seuraava sääntö:

\laatikko[Yhteen- ja vähennyslaskujen järjestys]{
Pelkästään yhteen- ja vähennyslaskua sisältävässä lausekkeessa laskujärjestystä voi vaihtaa vapaasti, kun ajattelee miinusmerkin kuuluvan sitä seuraavaan lukuun ja liikkuvan sen mukana.
}

\begin{esimerkki}
$5-8+7-2=5+(-8)+7+(-2)=(-2)+(-8)+5+7=-2-8+5+7$
\end{esimerkki}

Vastaavat säännöt pätevät kerto- ja jakolaskulle samoista syistä.

\laatikko[Kertolaskujen järjestys]{
Kertolaskut voi laskea missä järjestyksessä tahansa

\begin{tabular}{ll}
  $a\cdot b=b\cdot a$\qquad\qquad&(vaihdantalaki)\\
  \\
  $a\cdot (b\cdot c)=(a\cdot b)\cdot c=a\cdot b\cdot c$\qquad\qquad&(liitäntälaki)
\end{tabular}
}

\begin{esimerkki}

$5 \cdot 6 = 6 \cdot 5$
 
 $2 \cdot (1+2) = 2 \cdot 1 + 2 \cdot 2$
\end{esimerkki}

%$2 \cdot (1+2) = 2 \cdot 1 + 2 \cdot 2$

\laatikko[Kerto- ja jakolaskujen järjestys]{
Pelkästään kerto- ja jakolaskua sisältävässä lausekkeessa laskujärjestystä voi vaihtaa vapaasti, kun ajattelee jakolaskun käänteisluvulla kertomisena.
}

\begin{esimerkki}
$5:8\cdot 7:2=5\cdot\frac18\cdot 7\cdot\frac12=7\cdot \frac12\cdot\frac18\cdot 5=7:2:8\cdot 5$
\end{esimerkki}

\subsubsection*{Osittelulaki}

Yhteen- ja kertolaskua sisältävälle lausekkeelle pätee seuraava sääntö:

\laatikko[Osittelulaki]{
$a(b+c)=ab+ac$

Vasemmalta oikealle luettaessa puhutaan \termi{sulkeiden avaaminen}{sulkujen avaamisesta}. Oikealta vasemmalle päin mentäessä puhutaan \termi{yhteinen tekijä}{yhteisen tekijän ottamisesta}.
}

Osittelulakia voidaan käyttää muiden laskutoimituksiin liittyvien lakien rinnalla. Kun vähennyslasku ajatellaan vastaluvun lisäämisenä ja jakolasku ajatellaan käänteisluvulla kertomisena, voidaan osittelulaki laajentaa koskemaan myös jakolaskua. Seuraavassa on esimerkkejä tällaisista osittelulain sovelluksista:

\begin{esimerkki}
\begin{align*}
&(b+c)a = a(b+c) = ab+ac = ba+ca \\
&\text{ (Sovellettu vaihdantalakia)} \\
&a(b+c+d) = a((b+c)+d) = a(b+c)+ad = ab+ac+ad \\
&\text{ (Sovellettu liitäntälakia)} \\
&a(b-c) = a(b+(-c))=ab+a\cdot(-c)=ab-ac \\
&\text{ (Sovellettu kertolaskun merkkisääntöä)} \\
&(b+c):a = (b+c)\cdot\frac1a = b\cdot\frac1a+c\cdot\frac1a = b:a+c:a \\
&\text{ (Sovellettu jakolaskun ilmaisemista käänteisluvun avulla.) }
\end{align*}
\end{esimerkki}

Osittelulakia voidaan käyttää myös tuntemattomia lukuja sisältävien lausekkeiden muokkaamisessa. Esim. $2(x+5)=2x+10$. Sievennyksen lisäksi osittelulaki on kätevä työkalu luvuilla laskemisessa ja päässälaskussa, kuten seuraavat esimerkit osoittavat:

\begin{esimerkki}
     \begin{align*}
7\,777\cdot 542-7\,777\cdot 541 &= 7\,777\cdot (542-541) \\ &= 7\,777\cdot 1 \\ &= 7\,777
     \end{align*}
\end{esimerkki}

\begin{esimerkki}
    \begin{align*}
46\cdot 9 &= 46\cdot (10-1) \\ &= 46\cdot 10 - 46\cdot 1 \\ &= 460-46 \\ &= 414
    \end{align*}
\end{esimerkki}
%toinen esimerkki

\subsubsection*{Käänteistoimitukset}

Luvun ja sen vastaluvun summa on nolla. Yhteen- ja vähennyslasku ovat toistensa käänteistoimituksia, joten saman luvun lisääminen ja vähentäminen peräkkäin kumoavat toisensa:

\luettelolaatikko{Käänteistoimitukset I}{
§ $a-a=0$
§ $-a+a=0$
§ $a+b-b=a$
§ $a-b+b=a$
}

Vastaavasti luvun ja sen käänteisluvun tulo on nolla. Kerto- ja jakolasku ovat toistensa käänteistoimituksia, joten peräkkäin kertominen ja jakaminen samalla luvulla kumoavat toisensa:

\luettelolaatikko{Käänteistoimitukset II}{
§ $a\cdot \frac1a=1$
§ $\frac1a\cdot a=1$
§ $a\cdot b:b=a$
§ $a:b\cdot b=a$
}
s
Nämä säännöt ovat sieventämisen työkaluja. Voimme poistaa niiden avulla ylimääräisiä termejä ja tulontekijöitä lausekkeista.

\begin{esimerkki} 
    Sievennetään $2+c-c$, missä $c$ on mikä tahansa kokonaisluku.\vspace*{11pt}
   
    Koska $-c$ on $c$:n vastaluku, niin niiden summa $c+(-c)$ on $0$. Tällöin lausekkeen arvo on $2+0$. Huomaa, että vastaus on $2$ riippumatta $c$:n arvosta.
%    § Esimerkki kerto- ja jakolaskun keskinäisestä kumoamisvaikutuksesta
%    § Esimerkki sekasupistuksesta. Vaikkapa $8+((x-x)+z)/z=8$. 
\end{esimerkki}

