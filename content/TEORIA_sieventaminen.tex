\subsection*{Laskujärjestys}



Yhteenlaskun sisällä laskujärjestystä voi vaihtaa miten tahansa. Esimerkiksi laskemalla voidaan tarkistaa, että $5+7=7+5$ ja että $(2+3)+5=2+(3+5)$.

Tämä sääntö voidaan yleistää koskemaan myös vähennyslaskua, kun muistetaan, että vähennyslasku tarkoittaa oikeastaan vastaluvun lisäämistä. $5-8$ tarkoittaa siis samaa kuin $5+(-8)$, joka voidaan nyt kirjoittaa yhteenlaskun vaihdantalain perusteella muotoon $(-8)+5$ eli $-8+5$ ilman, että laskun lopputulos muuttuu. Tästä seuraa seuraava sääntö:
\laatikko{
Yhteenlaskut voi laskea missä järjestyksessä tahansa

\begin{tabular}{ll}
  $a+b=b+a$\qquad\qquad&(vaihdantalaki)\\
  $a-b=-b+a$\\
  \\  
  $a+(b+c)=(a+b)+c=a+b+c$\qquad\qquad&(liitäntälaki)
\end{tabular} 
}

\begin{esimerkki} 
$5-8+7-2=5+(-8)+7+(-2)=(-2)+(-8)+5+7=-2-8+5+7$ 
\end{esimerkki}

Vastaavat säännöt pätevät kerto- ja jakolaskulle samoista syistä.

\laatikko{
Kertolaskut voi laskea missä järjestyksessä tahansa

\begin{tabular}{ll}
  $a\cdot b=b\cdot a$\qquad\qquad&(vaihdantalaki)\\
  \\
  $a\cdot (b\cdot c)=(a\cdot b)\cdot c=a\cdot b\cdot c$\qquad\qquad&(liitäntälaki)
\end{tabular} 
}



\begin{esimerkki}

$5 \cdot 6 = 6 \cdot 5$
 
 $2 \cdot (1+2) = 2 \cdot 1 + 2 \cdot 2$
\end{esimerkki} 

%$2 \cdot (1+2) = 2 \cdot 1 + 2 \cdot 2$

\laatikko{
Pelkästään kerto- ja jakolaskua sisältävässä lausekkeessa laskujärjestystä voi vaihtaa vapaasti, kun ajattelee jakolaskun käänteisluvulla kertomisena.
}

\begin{esimerkki}
$5:8\cdot 7:2=5\cdot\frac18\cdot 7\cdot\frac12=7\cdot \frac12\cdot\frac18\cdot 5=7:2:8\cdot 5$
\end{esimerkki} 

Jos lausekkeessa on useita eri laskutoimituksia, ne suoritetaan seuraavassa järjestyksessä:

\laatikko{
\begin{enumerate}
\item{Suluissa olevat lausekkeet}
\item{Potenssilaskut (potenssia käsitellään tarkemmin myöhemmin tässä luvussa)}
\item{Kerto- ja jakolaskut}
\item{Yhteen- ja vähennyslaskut}
\end{enumerate}
}

Käytännössä laskutoimituksen voi laskea, jos sen kummallakaan puolella ei ole laskutoimitusta, joka pitäisi laskea ensin.



Lisäksi yhteenlaskulla on sellainen erityisominaisuus, että pelkästään yhteenlaskua sisältävässä
lausekkeessa laskujärjestys ei vaikuta tulokseen. Sen vuoksi yhteenlaskut voi laskea myös oikealta
vasemmalle kunhan välissä ei ole esimerkiksi vähennyslaskua tai muita laskutoimituksia.

Samoin kertolaskut voi laskea missä järjestyksessä tahansa kunhan välissä ei ole muita laskutoimituksia.

Vähennyslaskussa ja jakolaskussa laskujärjestyksellä sen sijaan on merkitystä.

\begin{esimerkki}
Lasketaan lausekkeen $2+(3+1)\cdot 5$ arvo.

Koska lausekkeessa on useita eri laskutoimituksia, joudumme laskemaan ne annetun järjestyksen mukaisesti.

Ensimmäiseksi laskemme lausekkeen $(3+1)$ arvon, koska se on sulkujen sisällä. Tämän jälkeen laskemme tulon, koska kertolasku lasketaan ennen yhteenlaskua. Lopuksi summaamme saadun tulon kahden kanssa.

\begin{align*}
   &2+(3+1)\cdot 5&\textrm{Lasketaan ensin } (3+1)
\\= &2+4\cdot5&\qquad\textrm{sen jälkeen tulo } 4\cdot 5 
\\= &2+20& \textrm{lopuksi vielä summa}
\\= &22&
\end{align*}
\end{esimerkki}

Yhteen- ja kertolaskua sisältävälle lausekkeelle pätee seuraava sääntö:

\laatikko{
$a(b+c)=ab+ac$\qquad\qquad(osittelulaki)

Vasemmalta oikealle luettaessa puhutaan \termi{sulkeiden avaaminen}{sulkujen avaamisesta}. Oikealta vasemmalle päin mentäessä puhutaan \termi{yhteinen tekijä}{yhteisen tekijän ottamisesta}.
}

Osittelulakia voidaan käyttää muiden laskutoimituksiin liittyvien lakien rinnalla.

\begin{esimerkki}
\begin{align*}
&(b+c)a = a(b+c) = ab+ac = ba+ca \text{ (Sovellettu vaihdantalakia)} \\
&a(b+c+d) = a((b+c)+d) = a(b+c)+ad = ab+ac+ad \text{ (Sovellettu liitäntälakia)} \\
&a(b-c) = a(b+(-c))=ab+a\cdot(-c)=ab-ac \text{ (Sovellettu kertolaskun merkkisääntöä)} \\
&(b+c):a = (b+c)\cdot\dfrac1a = b\cdot\dfrac1a+c\cdot\dfrac1a = b:a+c:a \text{ (Sovellettu jakolaskun ilmaisemista käänteisluvun avulla.) }
\end{align*}

Esimerkiksi seuraava laskutoimitus on helppo laskea osittelulain avulla: 
     \begin{align*}
	  7777\cdot 542-7777\cdot 541 &= 7777\cdot (542-541)  \\ &= 7777\cdot 1 \\ &= 7777
     \end{align*}
\end{esimerkki}

Osittelulakia voidaan käyttää myös tuntemattomia lukuja sisältävien lausekkeiden muokkaamisessa. Esim. $2(x+5)=2x+10$.

\subsection*{Lausekkeiden sieventäminen}

Mitä tahansa lukua tai kirjoitettua laskutoimitusta (kuten $9$ tai $\frac{x}{3}-9$) kutsutaan \termi{lauseke}{lausekkeeksi}. \termi{termi}{Termi} puolestaan on yksi yhteenlaskettava osa lausekkeesta. Termissä voi olla yhdistelmä luvuista, vakioista ja muuttujista tulona, kuten $-4\sqrt{2}ax^2$ tai se voi olla jokin näistä yksittäisenä kuten pelkkä $1$.

Matemaattisia ongelmia ratkaistaessa kannattaa usein etsiä vaihtoehtoisia tapoja jonkin laskutoimituksen, lausekkeen tai luvun ilmaisemiseksi. Tällöin usein korvataan esimerkiksi jokin laskutoimitus toisella laskutoimituksella, josta tulee sama tulos. Näin lauseke saadaan sellaiseen muotoon, jonka avulla ratkaisussa päästään eteenpäin. Kun merkitsemme monimutkaisen lausekkeen lyhyemmin, sitä kutsutaan \termi{sieventäminen}{sieventämiseksi}. Sieventäminen on ikään kuin sotkuisen kaavan siistimistä selkeämmäksi.

Matematiikassa on tapana ajatella niin, että saman luvun voi kirjoittaa monella eri tavalla. Esimerkiksi merkinnät \begin{align*}
                & 42 \\ & -(-42) \\ & 6 \cdot 7 \\ & (50-29) \cdot 2                                                                                                      
                                                                                                                 \end{align*}
tarkoittavat kaikki samaa lukua. Niinpä missä tahansa lausekkeessa voi luvun $42$ paikalle kirjoittaa merkinnän $(50-29)\cdot 2$, sillä ne tarkoittavat samaa lukua. Tähän lukuun on koottu sääntöjä, joiden avulla laskutoimituksia voi vaihtaa niin, että lopputulos ei muutu.

Yhteen- ja vähennyslasku ovat toistensa käänteistoimituksia, joten saman luvun lisääminen ja vähentäminen peräkkäin kumoavat toisensa:

\laatikko{
\begin{alakohdat}
\alakohta{$a+b-b=a$}
\alakohta{$a-b+b=a$}
\end{alakohdat}
}

Kerto- ja jakolasku ovat toistensa käänteistoimituksia, joten peräkkäin kertominen ja jakaminen samalla luvulla kumoavat toisensa:

\laatikko{
\begin{alakohdat}
\alakohta{$a\cdot b:b=a$}
\alakohta{$a:b\cdot b=a$}
\end{alakohdat}
}

Nämä säännöt ovat sieventämisen työkaluja. Voimme poistaa niiden avulla ylimääräisiä termejä ja tulontekijöitä lausekkeesta. 

\begin{esimerkki} 

 \begin{alakohdat}
   \alakohta{sievennetään $2+c-c$, missä c on mikä tahansa kokonaisluku.\vspace*{11pt}
   
   Koska -c on c:n vastaluku, niin niiden summa $c+(-c)$ on $0$. Tällöin lausekkeen arvo on $2+0$. Huomaa, että vastaus on 2 riippumatta c:n arvosta.}
   \alakohta{Esimerkki kerto- ja jakolaskun keskinäisestä kumoamisvaikutuksesta}
   \alakohta{Esimerkki sekasupistuksesta. Vaikkapa $8+((x-x)+z)/z=8$. }
\end{alakohdat}
\end{esimerkki}
% === SIISTITTÄVÄ PÄTKÄ LOPPUU ===
