\subsection*{Luvut ja laskutoimitukset}

\begin{tehtava}
Mitä eroa on luvulla ja numerolla?
\end{tehtava}

\begin{tehtava}
Miten vähennyslasku ja jakolasku määritellään?
\end{tehtava}

\begin{tehtava}
Minkälainen ristiriita seuraisi, jos nollalla jakaminen olisi määritelty reaaliluvuille?
\end{tehtava}

\begin{tehtava}
Miksei reaalilukuja käytettäessä negatiivisesta luvusta voi ottaa neliöjuurta?
\end{tehtava}

\begin{tehtava}
Miksi mielivaltaisen, nollasta poikkeavan luvun nollas potenssi on arvoltaan yksi?
\end{tehtava}

\begin{tehtava}
Mitä reaalilukujen laskusääntöjä käytetään, kun johdetaan 
\begin{alakohdat}
\alakohta{vaihdannaisuus}
\alakohta{osittelulaki}
\end{alakohdat}
\end{tehtava}

\begin{tehtava}
Mitä tarkoitetaan johdannaissuureella?
\end{tehtava}

\begin{tehtava}
Mikä nyanssiero on saman luvun esitysmuodoilla 9,1 ja 9,10?
\end{tehtava}

\begin{tehtava}
Tutkit luvun desimaalikehitelmää ja huomaat siinä äärettömästi toistuvan jakson 2104. Minkälainen luku on kyseessä?
\end{tehtava}

Monivalinta, valitse yksi paras vastaus
\begin{tehtava}
Kuinka monta merkitsevää numeroa on luvussa 1,0240?
\begin{alakohdat}
\alakohta{ei yhtään}
\alakohta{yksi}
\alakohta{kolme}
\alakohta{neljä}
\alakohta{viisi}
\end{alakohdat}
\end{tehtava}

\begin{tehtava}
Kuinka monta nollaa on triljoonassa?
\begin{alakohdat}
\alakohta{3}
\alakohta{12}
\alakohta{13}
\alakohta{18}
\alakohta{30}
\end{alakohdat}
\end{tehtava}

\begin{tehtava}
Mitä on 0,34 nanometriä esitettynä kymmenpotenssimuodossa, kun yksikkönä on metri?
\begin{alakohdat}
\alakohta{$0,34 \cdot 10^{-9}$ \, m}
\alakohta{$3,4 \cdot 10^{-10}$ \, m}
\alakohta{$34 \cdot 10^{-9}$ \, m}
\alakohta{$3,4 \cdot 10^{-9}$ \, m}
\alakohta{$34 \cdot 10^{-10}$ \, m}
\end{alakohdat}
\begin{vastaus}
a) $0,34 \cdot 10^{-9}$ \, m
\end{vastaus}
\end{tehtava}
%vastauksvaihtoehdot lisäsi jaakkoviertiö 22.3.2014 

\begin{tehtava}
Kun lausekkeeseen $bc-2ac+3b-6a$ sovelletaan osittelulakia, saadaan sille siistimpi esitys, joka on jokin allaolevista. Mikä?
	\begin{alakohdat}
		\alakohta{$(b-3a)(c+3)$}
		\alakohta{$(b-2a)(c+3)$}
		\alakohta{$2(b-a)(c+3)$}
		\alakohta{$(b-a)(2c+3$}
	\end{alakohdat}
    \begin{vastaus}
	b) $(b-2a)(c+3)$
    \end{vastaus}
\end{tehtava}
%muokannut jaakko viertiö 22.3.2014

\begin{tehtava}
Nopeus 60 km/h ilmaistuna yksikössä m/s on suurinpiirtein jokin seuraavista. Mikä?
\begin{alakohdat}
\alakohta{10}
\alakohta{17}
\alakohta{30}
\alakohta{36}
\alakohta{216}
\end{alakohdat}
	\begin{vastaus}
	 b) $17$
	\end{vastaus}
\end{tehtava}
%muokannutn jaakko viertiö 22.3.2014

\begin{tehtava}
Potenssi $\left( \dfrac{2}{3} \right)^{-2}$ sievennetynä on jokin seuraavista. Mikä?
\begin{alakohdat}
\alakohta{$36$}
\alakohta{$\frac{1}{36}$}
\alakohta{$\frac{4}{9}$}
\alakohta{$-\frac{4}{9}$}
\alakohta{$\frac{9}{4}$}
\alakohta{$-\frac{9}{4}$}
\end{alakohdat}
\begin{vastaus}
e) $\frac{9}{4}$
\end{vastaus}
\end{tehtava}
%muokannut jaakko viertiö 22.3.2014

\begin{tehtava}
Sekaluku $7\,\frac{3}{8}$ ilmaistuna sekalukuna on..
\begin{alakohdat}
\alakohta{$\frac{31}{4}$}
\alakohta{$\frac{21}{8}$}
\alakohta{$\frac{59}{8}$}
\end{alakohdat}
    \begin{vastaus}
	 c) $\frac{59}{8}$
    \end{vastaus}
\end{tehtava}
%muokannut jaakko viertiö 22.3.2014

\begin{tehtava}
Neliöjuuri 8 sievennettynä on
\begin{alakohdat}
\alakohta{$2\sqrt{2}$}
\alakohta{$4$}
\alakohta{$\sqrt{8}$}
\alakohta{$64$}
\end{alakohdat}
\end{tehtava}
%muokamnnut jaakkoviertiö 22.3.2014

\subsection*{Yhtälöt}

\subsection*{Funktiot}
