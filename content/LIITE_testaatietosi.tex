\subsection*{Luvut ja laskutoimitukset}

\begin{tehtava}
Mitä eroa on luvulla ja numerolla?
\end{tehtava}

\begin{tehtava}
Miten vähennyslasku sekä jakolasku määritellään?
\end{tehtava}

\begin{tehtava}
Minkälainen ristiriita seuraisi, jos nollalla jakaminen olisi määritelty reaaliluvuille?
\end{tehtava}

\begin{tehtava}
Mitä eroa on kertomisella ja laventamisella?
\end{tehtava}

\begin{tehtava}
Mitä tarkoitetaan alkuluvulla?
\end{tehtava}

\begin{tehtava}
Mitä tarkoitetaan luvun alkulukukehitelmällä?
\end{tehtava}

\begin{tehtava}
Miksei reaalilukuja käytettäessä negatiivisesta luvusta voi ottaa neliöjuurta?
\end{tehtava}

\begin{tehtava}
Miksi mielivaltaisen, nollasta poikkeavan luvun nollas potenssi on arvoltaan yksi?
\end{tehtava}

\begin{tehtava}
Mitä tarkoitetaan johdannaissuureella?
\end{tehtava}

\begin{tehtava}
Mikä nyanssiero on saman luvun esitysmuodoilla $9,1$ ja $9,10$?
\end{tehtava}

\begin{tehtava}
Tutkit luvun desimaalikehitelmää ja huomaat siinä äärettömästi toistuvan jakson 2104. Minkälainen luku on kyseessä?
\end{tehtava}

\begin{tehtava}
Mikä yhteys murtopotensseilla ja juurilla on?
\end{tehtava}

Monivalinta, valitse yksi paras vastaus
\begin{tehtava}
Kuinka monta merkitsevää numeroa on luvussa 1,0240?
\begin{alakohdat}
\alakohta{ei yhtään}
\alakohta{yksi}
\alakohta{kolme}
\alakohta{neljä}
\alakohta{viisi}
\end{alakohdat}
\end{tehtava}

\begin{tehtava}
Kuinka monta nollaa on miljardissa?
\begin{alakohdat}
\alakohta{3}
\alakohta{12}
\alakohta{13}
\alakohta{18}
\alakohta{30}
\end{alakohdat}
\end{tehtava}

\begin{tehtava}
Mitä on 0,34 nanometriä esitettynä standardikymmenpotenssimuodossa, kun yksikkönä on metri?
\begin{alakohdat}
\alakohta{$0,34 \cdot 10^{-9}$\,m}
\alakohta{$3,4 \cdot 10^{-10}$\,m}
\alakohta{$34 \cdot 10^{-9}$\,m}
\alakohta{$3,4 \cdot 10^{-9}$\,m}
\alakohta{$34 \cdot 10^{-10}$\,m}
\end{alakohdat}
\begin{vastaus}
b) $3,4 \cdot 10^{-10}$\,m
\end{vastaus}
\end{tehtava}

\begin{tehtava}
Kun lausekkeeseen $bc-2ac+3b-6a$ sovelletaan osittelulakia, saadaan sille siistimpi esitys, joka on jokin allaolevista. Mikä?
	\begin{alakohdat}
		\alakohta{$(b-3a)(c+3)$}
		\alakohta{$(b-2a)(c+3)$}
		\alakohta{$2(b-a)(c+3)$}
		\alakohta{$(b-a)(2c+3$}
	\end{alakohdat}
    \begin{vastaus}
	b) $(b-2a)(c+3)$
    \end{vastaus}
\end{tehtava}
%muokannut jaakko viertiö 22.3.2014

\begin{tehtava}
Luvun 180 alkulukukehitelmä on
\begin{alakohdat}
\alakohta{$2\cdot3\cdot5$}
\alakohta{$2^2\cdot3^2\cdot5^2$}
\alakohta{$4\cdot3^2\cdot5$}
\alakohta{$4\cdot9\cdot5$}
\alakohta{$2^2\cdot3^2\cdot5$}
\end{alakohdat}
	\begin{vastaus}
	 e) $2^2\cdot3^2\cdot5$
	\end{vastaus}
\end{tehtava}

\begin{tehtava}
Nopeus 60 km/h ilmaistuna yksikössä m/s on suurinpiirtein jokin seuraavista. Mikä?
\begin{alakohdat}
\alakohta{10}
\alakohta{17}
\alakohta{30}
\alakohta{36}
\alakohta{216}
\end{alakohdat}
	\begin{vastaus}
	 b) $17$
	\end{vastaus}
\end{tehtava}
%muokannut jaakko viertiö 22.3.2014

\begin{tehtava}
Potenssi $\left( \dfrac{2}{3} \right)^{-2}$ sievennettynä on jokin seuraavista. Mikä?
\begin{alakohdat}
\alakohta{$36$}
\alakohta{$\frac{1}{36}$}
\alakohta{$\frac{4}{9}$}
\alakohta{$-\frac{4}{9}$}
\alakohta{$\frac{9}{4}$}
\alakohta{$-\frac{9}{4}$}
\end{alakohdat}
\begin{vastaus}
e) $\frac{9}{4}$
\end{vastaus}
\end{tehtava}
%muokannut jaakko viertiö 22.3.2014

\begin{tehtava}
Sekaluku $7\,\frac{3}{8}$ ilmaistuna murtolukuna on
\begin{alakohdat}
\alakohta{$\frac{31}{4}$}
\alakohta{$\frac{21}{8}$}
\alakohta{$\frac{59}{8}$}
\end{alakohdat}
    \begin{vastaus}
	 c) $\frac{59}{8}$
    \end{vastaus}
\end{tehtava}
%muokannut jaakko viertiö 22.3.2014

\begin{tehtava}
Neliöjuuri 8 mahdollisimman pitkälle sievennettynä on
\begin{alakohdat}
\alakohta{$2\sqrt{2}$}
\alakohta{$4$}
\alakohta{$\sqrt{8}$}
\alakohta{$64$}
\alakohta{$16$}
\end{alakohdat}
\end{tehtava}
%muokannut jaakkoviertiö 22.3.2014

\subsection*{Yhtälöt}

\subsection*{Funktiot}

\begin{tehtava}
	Määritellään funktio $f$ lausekkeella \[f(x)=\frac{1}{2}x^2+2x-1.\] Funktion $f$ kuvaaja löytyy alta. 
	Oletetaan lisäksi, että $a>1$. Päättele lukujen suuruusjärjestys.
	\begin{alakohdat}
		\alakohta{$f(a)$ ja $f(a+2)$}
		\alakohta{$f(-a)$ ja $f(a)$}
	\end{alakohdat}
    \begin{vastaus}
	\begin{alakohdat}
		\alakohta{$f(a)>f(a+2)$}
		\alakohta{$f(-a)<f(a)$}
	\end{alakohdat}
    \end{vastaus}
\end{tehtava}
\vspace{1cm}
\begin{center}
	\begin{kuvaajapohja}{0.7}{-3}{7}{-5}{3}
		\kuvaaja{-0.5*x**2+2*x-1}{\qquad$f(x)=-\frac{1}{2}x^2+2x-1$}{black}
	\end{kuvaajapohja}
\end{center}
%lisännyt jaakko viertiö 22.3.2014

\begin{tehtava} Keksi jokin sellainen funktio, joka toteuttaa annetun ehdon. Jonkinlaisen kuvan hahmottelusta on apua.
	\begin{alakohdat}
		\alakohta{$f(3)=2$ ja $f(24)=16$}
		\alakohta{Jos $f(a)=b$, niin $f(a+1)=b+1$}
		\alakohta{Jos $f(a)=b$, niin $f(a+1)=b-1$}
		\alakohta{Jos $f(0)=0$ ja $z>0$, niin $f(\pm z)=z$. (Vinkki: $|\pm a|=a$)}
	\end{alakohdat}
    \begin{vastaus}
	\begin{alakohdat}
		\alakohta{$f(x)=\frac{2}{3}x$}
		\alakohta{$f(x)=x$}
		\alakohta{$f(x)=-x$}
		\alakohta{$f(x)=|x|$}
	\end{alakohdat}
    \end{vastaus}

\end{tehtava}

%lisännyt jaakko viertiö 22.3.2014








\newpage