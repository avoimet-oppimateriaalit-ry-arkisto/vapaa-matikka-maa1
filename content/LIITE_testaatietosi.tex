\subsection*{Luvut ja laskutoimitukset}

\begin{tehtava}
Mitä eroa on luvulla ja numerolla?
\begin{vastaus}
Numero on merkki 0--9. Luvulla on suuruus, ja niitä esitetään numeroiden avulla. %kenties ei tuota ajatusviivaesitystä...
\end{vastaus}
\end{tehtava}

\begin{tehtava}
Miten vähennyslasku sekä jakolasku määritellään?
\begin{vastaus}
Kahden luvun $a$ ja $b$ vähennyslasku (erotus) $a-b$ määritellään summana $a+(-b)$, eli $a$:han on lisätty $b$:n vastaluku. Jakolasku (osamäärä) $\frac{a}{b}$ määritellään tulona $a\cdot \frac1b$, eli $a$ on kerrottu $b$:n käänteisluvulla.
\end{vastaus}
\end{tehtava}

\begin{tehtava}
Minkälainen ristiriita seuraisi, jos nollalla jakaminen olisi määritelty reaaliluvuille?
\begin{vastaus}
Kaikilla luvuilla $x$ pätee $0 \cdot x = 0$. Kuitenkin, jos nollalla voisi jakaa (eli sille olisi määritelty käänteisluku $\frac10$), olisi voimassa $0 \cdot\frac10 = 1$, mikä ei kuitenkaan samanaikaisesti voi pitää paikkansa sen kanssa, että kun mikä tahansa luku kerrotaan nollalla, saadaan tuloksi nolla.
\end{vastaus}
\end{tehtava}

\begin{tehtava}
Mitä eroa on kertomisella ja laventamisella?
\begin{vastaus}
Laventaminen muuttaa luvun eri muotoon, mutta pitää sen suuruuden ennallaan. Kertominen on laskutoimitus, joka voi muuttaa luvun suuruutta. Esimerkiksi $frac34$ lavennettuna viidellä on $frac{15}{20}$ mutta sama luku kerrottuna viidellä on $\frac{15}{4}$.
\end{vastaus}
\end{tehtava}

\begin{tehtava}
Mitä tarkoitetaan alkuluvulla?
\begin{vastaus}
Alkuluku on positiivinen kokonaisluku, jota ei voi esittää kahden pienemmän positiivisen kokonaisluvun tulona.
\end{vastaus}
\end{tehtava}

\begin{tehtava}
Mitä tarkoitetaan luvun alkulukukehitelmällä?
\begin{vastaus}
Alkutekijähajotelma (ja muut sanan variantit) on luvun esitys alkulukujen tulona, esimerkiksi $45=3\cdot 3 \cdot 5$.
\end{vastaus} %FIXME termikoherenssi
\end{tehtava}

\begin{tehtava}
Miksei reaalilukuja käytettäessä negatiivisesta luvusta voi ottaa neliöjuurta?
\begin{vastaus}
Koska $x^2 \geq 0$ kaikilla reaaliluvuilla $x$. Ei ole olemassa reaalilukua, joka kerrottuna itsellään tuottaisi negatiivisen luvun.
\end{vastaus}
\end{tehtava}

\begin{tehtava}
Miksi mielivaltaisen, nollasta poikkeavan luvun nollas potenssi on arvoltaan yksi?
\begin{vastaus}
Koska $x=x^1=x^{0+1}=x^0 \cdot x^1 = x^0 \cdot x$.
\end{vastaus}
\end{tehtava}

\begin{tehtava}
Mitä tarkoitetaan johdannaissuureella?
\begin{vastaus}
Johdannaissuure on suure, joka voidaan ilmaista perussuureiden avulla.
\end{vastaus}
\end{tehtava}

\begin{tehtava}
Mikä nyanssiero on saman luvun esitysmuodoilla $9,1$ ja $9,10$?
\begin{vastaus}
Jälkimmäisessä esityksessä on kolme merkitsevää numeroa -- ensimmäisessä vain kaksi. Jos kyseessä on mittaustulos, jälkimmäistä lukua voidaan pitää tarkempana.
\end{vastaus}
\end{tehtava}

\begin{tehtava}
Tutkit luvun desimaalikehitelmää ja huomaat siinä äärettömästi toistuvan jakson 2104. Minkälainen luku on kyseessä?
\begin{vastaus}
Rationaaliluku.
\end{vastaus}
\end{tehtava}

\begin{tehtava}
Mikä yhteys murtopotensseilla ja juurilla on?
\begin{vastaus}
Positiivisilla reaaliluvuilla $x$ pätee $\sqrt[n]{x} = x^{\frac{1}{n}}$.
\end{vastaus}
\end{tehtava}

Valitse monivalintatehtävissä yksi paras vastaus.

\begin{tehtava}
Kuinka monta merkitsevää numeroa on luvussa $1,0240$?
\begin{alakohdat}
\alakohta{ei yhtään}
\alakohta{yksi}
\alakohta{kolme}
\alakohta{neljä}
\alakohta{viisi}
\end{alakohdat}
\begin{vastaus}
e) Viisi
\end{vastaus}
\end{tehtava}

\begin{tehtava}
Luku $-78,2449$ pyöristettynä sadasosien tarkkuuteen on
\begin{alakohdat}
\alakohta{$-78,245$}
\alakohta{$-78,25$}
\alakohta{$-78,24$}
\alakohta{$-78,2$}
\end{alakohdat}
\begin{vastaus}
c) $-78,24$
\end{vastaus}
\end{tehtava}

\begin{tehtava}
Kuinka monta nollaa on miljardissa?
\begin{alakohdat}
\alakohta{3}
\alakohta{6}
\alakohta{9}
\alakohta{12}
\alakohta{15}
\end{alakohdat}
\begin{vastaus}
c) 9
\end{vastaus}
\end{tehtava}

\begin{tehtava}
Mitä on 0,34 nanometriä esitettynä standardikymmenpotenssimuodossa, kun yksikkönä on metri?
\begin{alakohdat}
\alakohta{$0,34 \cdot 10^{-9}$\,m}
\alakohta{$3,4 \cdot 10^{-10}$\,m}
\alakohta{$34 \cdot 10^{-9}$\,m}
\alakohta{$3,4 \cdot 10^{-9}$\,m}
\alakohta{$34 \cdot 10^{-10}$\,m}
\end{alakohdat}
\begin{vastaus}
b) $3,4 \cdot 10^{-10}$\,m
\end{vastaus}
\end{tehtava}

\begin{tehtava}
Kun lausekkeeseen $bc-2ac+3b-6a$ sovelletaan osittelulakia, saadaan sille siistimpi esitys, joka on jokin allaolevista. Mikä?
	\begin{alakohdat}
		\alakohta{$(b-3a)(c+3)$}
		\alakohta{$(b-2a)(c+3)$}
		\alakohta{$2(b-a)(c+3)$}
		\alakohta{$(b-a)(2c+3)$}
	\end{alakohdat}
    \begin{vastaus}
	b) $(b-2a)(c+3)$
    \end{vastaus}
\end{tehtava}
%muokannut jaakko viertiö 22.3.2014

\begin{tehtava}
Luvun 180 alkulukukehitelmä on
\begin{alakohdat}
\alakohta{$2\cdot3\cdot5$}
\alakohta{$2^2\cdot3^2\cdot5^2$}
\alakohta{$4\cdot3^2\cdot5$}
\alakohta{$4\cdot9\cdot5$}
\alakohta{$2^2\cdot3^2\cdot5$.}
\end{alakohdat}
	\begin{vastaus}
	 e) $2^2\cdot3^2\cdot5$
	\end{vastaus}
\end{tehtava}

\begin{tehtava}
Nopeus 60 km/h ilmaistuna yksikössä m/s on suurinpiirtein jokin seuraavista. Mikä?
\begin{alakohdat}
\alakohta{10}
\alakohta{17}
\alakohta{30}
\alakohta{36}
\alakohta{216}
\end{alakohdat}
	\begin{vastaus}
	 b) $17$
	\end{vastaus}
\end{tehtava}
%muokannut jaakko viertiö 22.3.2014

\begin{tehtava}
Potenssi $\left( \dfrac{2}{3} \right)^{-2}$ sievennettynä on jokin seuraavista. Mikä?
\begin{alakohdat}
\alakohta{$36$}
\alakohta{$\frac{1}{36}$}
\alakohta{$\frac{4}{9}$}
\alakohta{$-\frac{4}{9}$}
\alakohta{$\frac{9}{4}$}
\alakohta{$-\frac{9}{4}$}
\end{alakohdat}
\begin{vastaus}
e) $\frac{9}{4}$
\end{vastaus}
\end{tehtava}
%muokannut jaakko viertiö 22.3.2014

\begin{tehtava}
Sekaluku $7\,\frac{3}{8}$ ilmaistuna murtolukuna on
\begin{alakohdat}
\alakohta{$\frac{31}{4}$}
\alakohta{$\frac{21}{8}$}
\alakohta{$\frac{10}{8}$}
\alakohta{$\frac{59}{8}$}
\alakohta{$\frac{53}{8}$}
\end{alakohdat}
    \begin{vastaus}
	 d) $\frac{59}{8}$
    \end{vastaus}
\end{tehtava}
%muokannut jaakko viertiö 22.3.2014

\begin{tehtava}
Neliöjuuri 8 mahdollisimman pitkälle sievennettynä on
\begin{alakohdat}
\alakohta{$\sqrt{8}$}
\alakohta{$2\sqrt{2}$}
\alakohta{$4$}
\alakohta{$16$}
\alakohta{$64$}
\end{alakohdat}

  \begin{vastaus}
	 b) $2\sqrt{2}$
    \end{vastaus}
\end{tehtava}
%muokannut jaakkoviertiö 22.3.2014

\subsection*{Yhtälöt}

\begin{tehtava}
Mitä tarkoitetaan yhtälön ratkaisulla?
\begin{vastaus}
Muuttujan arvoa, jolla yhtälö on tosi.
\end{vastaus}
\end{tehtava}

\begin{tehtava}
Milloin yhtälön molemmat puolet voi kertoa samalla luvulla niin, että yhtälö säilyy yhtäpitävänä?
\begin{vastaus}
Milloin vain, kunhan kertojana ei ole nolla. Nollalla kertominen kadottaa informaatiota, eikä alkuperäistä yhtälöä enää saa ratkaistua.
\end{vastaus}
\end{tehtava}

\begin{tehtava}
Milloin yhtälön molemmat puolet voi kertoa samalla luvulla niin, että tosi yhtälö säilyy totena?
\begin{vastaus}
Aina ja millä tahansa.
\end{vastaus}
\end{tehtava}

\begin{tehtava}
Jos muuttujat $x$ ja $y$ ovat suoraan verrannolliset ja muuttuja $x$ kasvaa, mitä tapahtuu muuttujalle $y$?
\begin{vastaus}
Muuttuja $y$ kasvaa myös -- ja samassa suhteessa.
\end{vastaus}
\end{tehtava}

\begin{tehtava}
Jos muuttujat $x$ ja $y$ ovat suoraan verrannolliset, ja muuttujat $y$ ja $z$ ovat kääntäen verrannolliset, millainen on muuttujien $x$ ja $z$ välinen suhde?
\begin{vastaus}
$x$ ja $z$ ovat kääntäen verrannolliset.
\end{vastaus}
\end{tehtava}

% prosentti: miksi +10 % - 10 % ei alkuperäinen?

Valitse monivalintatehtävissä yksi paras vastaus.

\begin{tehtava}
Yhtälö $x = x-1$ on
\begin{alakohdat}
\alakohta{aina tosi}
\alakohta{joskus tosi}
\alakohta{ei koskaan tosi}
\end{alakohdat}
\begin{vastaus}
c) ei koskaan tosi
\end{vastaus}
\end{tehtava}

\begin{tehtava}
Yhtälö $x = 4$ on
\begin{alakohdat}
\alakohta{aina tosi}
\alakohta{joskus tosi}
\alakohta{ei koskaan tosi}
\end{alakohdat}
\begin{vastaus}
b) joskus tosi
\end{vastaus}
\end{tehtava}

\begin{tehtava}
Yhtälö $x = x$ on
\begin{alakohdat}
\alakohta{aina tosi}
\alakohta{joskus tosi}
\alakohta{ei koskaan tosi}
\end{alakohdat}
\begin{vastaus}
a) aina tosi
\end{vastaus}
\end{tehtava}

%verrannollisusmonivalintoja
%prosenttilaskentaa

\subsection*{Funktiot}

\begin{tehtava}
Mitä eroa on funktion maali- ja arvojoukolla?
\begin{vastaus}

\end{vastaus}
\end{tehtava}

Miten funktion määrittelyjoukon voi selvittää?

moni: Jos funktion arvot saadaan lausekkeella ..., niin määrittelyehto...

\newpage