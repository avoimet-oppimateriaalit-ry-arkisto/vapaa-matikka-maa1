\subsection*{Luvut ja laskutoimitukset}

\begin{tehtava}
Mitä eroa on luvulla ja numerolla?
\end{tehtava}

\begin{tehtava}
Miten vähennyslasku ja jakolasku määritellään?
\end{tehtava}

\begin{tehtava}
Minkälainen ristiriita seuraisi, jos nollalla jakaminen olisi määritelty reaaliluvuille?
\end{tehtava}

\begin{tehtava}
Miksei reaalilukuja käytettäessä negatiivisesta luvusta voi ottaa neliöjuurta?
\end{tehtava}

\begin{tehtava}
Miksi mielivaltaisen, nollasta poikkeavan luvun nollas potenssi on arvoltaan yksi?
\end{tehtava}

\begin{tehtava}
Mitä reaalilukujen laskusääntöjä käytetään, kun johdetaan 
\begin{alakohdat}
\alakohta{vaihdannaisuus}
\alakohta{osittelulaki}
\end{alakohdat}
\end{tehtava}

\begin{tehtava}
Mitä tarkoitetaan johdannaissuureella?
\end{tehtava}

\begin{tehtava}
Mikä nyanssiero on saman luvun esitysmuodoilla 9,1 ja 9,10?
\end{tehtava}

\begin{tehtava}
Tutkit luvun desimaalikehitelmää ja huomaat siinä äärettömästi toistuvan jakson 2104. Minkälainen luku on kyseessä?
\end{tehtava}

Monivalinta, valitse yksi paras vastaus
\begin{tehtava}
Kuinka monta merkitsevää numeroa on luvussa 1,0240?
\begin{alakohdat}
\alakohta{ei yhtään}
\alakohta{yksi}
\alakohta{kolme}
\alakohta{neljä}
\alakohta{viisi}
\end{alakohdat}
\end{tehtava}

\begin{tehtava}
Kuinka monta nollaa on triljoonassa?
\begin{alakohdat}
\alakohta{3}
\alakohta{12}
\alakohta{13}
\alakohta{18}
\alakohta{30}
\end{alakohdat}
\end{tehtava}

\begin{tehtava}
Mitä on 0,34 nanometriä esitettynä kymmenpotenssimuodossa?
\begin{alakohdat}
\alakohta{sd}
\alakohta{sds}
\alakohta{dsd}
\alakohta{sd}
\alakohta{sds}
\end{alakohdat}
\end{tehtava}

\begin{tehtava}
Kun lausekkeeseen … käytetään osittelulakaía,s saadaan...osittelulaki,,,
\end{tehtava}

\begin{tehtava}
Mitä on nopeus … 60 km/h ilmaistuna yksiköissä m/s? (noin)
\begin{alakohdat}
\alakohta{10}
\alakohta{16}
\alakohta{30}
\alakohta{36}
\alakohta{216}
\end{alakohdat}
\end{tehtava}

\begin{tehtava}
Potenssi… sievennetynä on..
\begin{alakohdat}
\alakohta{d}
\alakohta{d}
\alakohta{d}
\alakohta{d}
\alakohta{d}
\alakohta{d}
\end{alakohdat}
\end{tehtava}

\begin{tehtava}
Murtoluku 9/8 ilmaistuna sekalukuna on..
\begin{alakohdat}
\alakohta{d}
\alakohta{d}
\alakohta{d}
\alakohta{d}
\alakohta{d}
\end{alakohdat}
\end{tehtava}

\begin{tehtava}
Neliöjuuri 8 sievennttynä on
\begin{alakohdat}
\alakohta{$2\sqrt{2}$}
\alakohta{$4$}
\alakohta{dfd}
\alakohta{fdfd}
\end{alakohdat}
\end{tehtava}

\subsection*{Yhtälöt}

\subsection*{Funktiot}
