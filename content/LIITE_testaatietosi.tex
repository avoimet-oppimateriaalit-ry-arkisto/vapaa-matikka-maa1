\subsection{Osaatko selittää?} % muuta niin, että kaikki selitykset yhdessä ja kaikki monivalinnat yhdessä

\subsubsection*{Luvut ja laskutoimitukset}

\begin{tehtava}
Mitä eroa on luvulla ja numerolla?
\begin{vastaus}
Luvulla on suuruus, ja niitä esitetään numeroiden eli numeromerkkien avulla. Kymmenjärjestelmässä käytetyt numeromerkit ovat 0, 1, 2, 3, 4, 5, 6, 7, 8, ja 9. (Huomaa myös vakiintuneet nimitykset kuten postinumero, puhelinnumero, \ldots)
\end{vastaus}
\end{tehtava}

\begin{tehtava}
Miten vähennyslasku ja jakolasku määritellään?
\begin{vastaus}
Kahden luvun $a$ ja $b$ vähennyslasku (erotus) $a-b$ määritellään summana $a+(-b)$, eli $a$:han on lisätty $b$:n vastaluku. Jakolasku (osamäärä) $\frac{a}{b}$ määritellään tulona $a\cdot \frac1b$, eli $a$ on kerrottu $b$:n käänteisluvulla.
\end{vastaus}
\end{tehtava}

\begin{tehtava}
Minkälainen ristiriita seuraisi, jos nollalla jakaminen olisi määritelty reaaliluvuille?
\begin{vastaus}
Kaikilla luvuilla $x$ pätee $0 \cdot x = 0$, eli mikä tahansa luku kerrottuna nollalla on aina nolla. Kuitenkin, jos nollalla voisi jakaa (eli sille olisi määritelty käänteisluku $0^{-1}=\frac10$), olisi voimassa $0 \cdot\frac10 = 1$, mikä ei kuitenkaan samanaikaisesti voi pitää paikkansa sen kanssa, että, kun mikä tahansa luku kerrotaan nollalla, saadaan tuloksi nolla.
\end{vastaus}
\end{tehtava}

\begin{tehtava}
Mitä eroa on kertomisella ja laventamisella?
\begin{vastaus}
Laventaminen muuttaa lausekkeen (tavallisesti rationaaliluvun) eri muotoon, mutta pitää sen suuruuden ennallaan. Kertominen on laskutoimitus, joka voi muuttaa luvun suuruutta. Esimerkiksi $\frac34$ lavennettuna viidellä on $\frac{15}{20}$ mutta sama luku kerrottuna viidellä on $\frac{15}{4}$.
\end{vastaus}
\end{tehtava}

\begin{tehtava}
Mitä tarkoitetaan alkuluvulla?
\begin{vastaus}
Alkuluku on luonnollinen luku, jota ei voi esittää kahden pienemmän luonnollisen tulona. Pienin alkuluku on $2$. %parempi määritelmä?
\end{vastaus}
\end{tehtava}

%FIXME termikoherenssi
\begin{tehtava}
Mitä tarkoitetaan luvun alkulukukehitelmällä (tai alkutekijäkehitelmällä)?
\begin{vastaus}
Alkutekijäkehitelmä on luvun esitys alkulukujen tulona, esimerkiksi $45=3\cdot 3 \cdot 5$ tai potenssien avulla kirjoitettuna $45=3^2\cdot 5$.
\end{vastaus}
\end{tehtava}

\begin{tehtava}
Miksei reaalilukuja käytettäessä negatiivisesta luvusta voi ottaa neliöjuurta?
\begin{vastaus}
Koska $x^2 \geq 0$ kaikilla reaaliluvuilla $x$, eli ei ole olemassa reaalilukua, joka kerrottuna itsellään tuottaisi negatiivisen luvun.
\end{vastaus}
\end{tehtava}

\begin{tehtava}
Mitä tarkoitetaan, kun sanotaan luvun olevan epänegatiivinen?
\begin{vastaus}
Kyseessä \textit{ei} ole vaikea tapa ilmaista sana positiivinen. Reaaliluvut voivat olla joko negatiivisia, positiivisia tai nolla (joka ei ole kumpaakaan), joten epänegatiivinen-ilmaisu sulkee pois ainoastaan negatiivisen -- luku voi olla nolla tai positiivinen. Ilmaisuun törmää usein määrittelyehtojen yhteydessä. Esimerkiksi parillisia juuria voi ottaa reaalilukuja käytettäessä vain epänegatiivisista luvuista.
\end{vastaus}
\end{tehtava}

\begin{tehtava}
Miksi mielivaltaisen, nollasta poikkeavan luvun nollas potenssi on arvoltaan yksi?
\begin{vastaus}
Koska käyttämällä potenssisääntöjä saadaan: $x^0=x^{1-1}=\frac{x^1}{x^1}=\frac{x}{x}=1$, mikä pätee kaikille luvuille paitsi $x=0$. Tämä on ainut tapa määritellä nollas potenssi niin, että se ei olisi ristiriidassa muiden potenssin ominaisuuksien (potenssisääntöjen) kanssa. (Toinen variantti: $x=x^1=x^{0+1}=x^0 \cdot x^1 = x^0 \cdot x$, mistä tulos seuraa.)
\end{vastaus}
\end{tehtava}

\begin{tehtava}
Mitä tarkoitetaan johdannaissuureella?
\begin{vastaus}
Johdannaissuure on suure, joka voidaan ilmaista (SI-järjestelmän) perussuureiden avulla niiden tuloina tai osamäärinä.
\end{vastaus}
\end{tehtava}

\begin{tehtava}
Mikä nyanssiero on saman luvun esitysmuodoilla $9,1$ ja $9,10$?
\begin{vastaus}
Jälkimmäisessä esityksessä on kolme merkitsevää numeroa -- ensimmäisessä vain kaksi. Jos kyseessä on mittaustulos, jälkimmäistä lukua voidaan pitää tarkempana.
\end{vastaus}
\end{tehtava}

\begin{tehtava}
Tutkit luvun desimaalikehitelmää ja huomaat siinä äärettömästi toistuvan jakson 2104. Minkälainen luku on kyseessä?
\begin{vastaus}
Kyseessä on rationaaliluku, sillä kaikkien rationaalilukujen (ja ainoastaan rationaalilukujen) desimaaliesitys sisältää äärettömästi toistuvaan jaksoon.
\end{vastaus}
\end{tehtava}

\begin{tehtava}
Mikä yhteys murtopotensseilla ja juurilla on?
\begin{vastaus}
Positiivisilla reaaliluvuilla $x$ pätee $\sqrt[n]{x^m} = x^{\frac{m}{n}}$. Tämä mahdollistaa potenssisääntöjen käyttämisen juurilausekkeille.
\end{vastaus}
\end{tehtava}
%tehtäviä liitännäisyydestä, vaihdannaisuudesta, refleksiivisyydestä, transitiivisuudesta ja symmetrisyydestä

\subsubsection*{Yhtälöt}

\begin{tehtava}
Mitä tarkoitetaan yhtälön ratkaisulla?
\begin{vastaus}
Muuttujan arvoa (tai muuttujien arvoja), jolla yhtälö on tosi
\end{vastaus}
\end{tehtava}

\begin{tehtava}
Jos lukua kasvatetaan $p$:llä prosentilla ja sen jälkeen pienennetään $p$:llä prosentilla, miksi ei päädytä takaisin alkuperäiseen lukuun?
	\begin{vastaus}
	Kyse on absoluuttisista, ei suhteellisista muutoksista. Jos lisätään jokin luku ja sitten vähennetään sama luku, laskutoimitukset kumoavat aina toisensa. Sen sijaan jos mielivaltaista lukua kasvatetaan suhtellisella osuudella, kasvun suuruus riippuu alkuperäisestä luvusta. Kun kasvatetusta luvusta otetaan sama suhteellinen osa pois, kyseinen osa onkin nyt suurempi, joten päädytään alkuperäistä pienempään lukuun.
	\end{vastaus}
\end{tehtava}

\subsubsection*{Funktiot}

\begin{tehtava}
Mitä eroa on funktion määrittelyjoukolla ja määrittelyehdolla?
\begin{vastaus}
Määrittelyjoukko on, kuten nimikin sanoo, joukko. Se siis esittää kaikki mahdolliset luvut (tai muunlaiset syötteet), jotka funktion muuttuja voi saada arvokseen, niin että funktion arvon laskeminen on mielekästä. Määrittelyehto kertoo määrittelyjoukon epäsuorasti antamalla jonkinlaisen ehdon muuttujalle, yleensä epäyhtälön muodossa. Esimerkiksi, jos määrittelyjoukko olisi $\mathbb{R}_+$ (positiiviset reaaliluvut), niin sitä vastaava määrittelyehto olisi $x>0$, missä $x$ on kyseisen funktion muuttuja.
\end{vastaus}
\end{tehtava}

\begin{tehtava}
Mitä eroa on funktion maali- ja arvojoukolla?
	\begin{vastaus}
Maalijoukko voi olla laajempi kuin arvojoukko, mutta sisältää aina arvojoukon kokonaisuudessaan. Arvojoukko on niiden alkioiden (yleensä lukujen) joukko, jotka funktio voi arvokseen saada. Maalijoukko koostuu alkioista, jotka funktio \textit{saattaa} saada arvokseen. Maalijoukkoa ei voi päätellä funktion mahdollisesta lausekkeesta, mutta arvojoukon voi. (Maalijoukon tärkeys ei tule kunnolla esille lukion matematiikassa -- ei huolta, vaikkei sen merkitys kunnolla aukeaisikaan. Arvojoukko sen sijaan tulee ymmärtää, ja sitä käsitellään lisää tulevilla kursseilla.)
	\end{vastaus}
\end{tehtava}

\newpage
\subsection{Monivalinta}

Valitse kussakin monivalintatehtävässä yksi paras vastaus.

\subsubsection*{Luvut ja laskutoimitukset}

\begin{tehtava}
Luvun $180$ alkulukukehitelmä on
\alakohdat{
§ $2\cdot3\cdot5$
§ $2^2\cdot3^2\cdot5^2$
§ $4\cdot3^2\cdot5$
§ $4\cdot9\cdot5$
§ $2^2\cdot3^2\cdot5$.
}
	\begin{vastaus}
	 e) $2^2\cdot3^2\cdot5$
	\end{vastaus}
\end{tehtava}

\begin{tehtava}
Kun lausekkeeseen $bc-2ac+3b-6a$ sovelletaan osittelulakia, saadaan sille siistimpi esitys, joka on jokin allaolevista. Mikä?
	\alakohdat{
		§ $(b-3a)(c+3)$
		§ $(b-2a)(c+3)$
		§ $2(b-a)(c+3)$
		§ $(b-a)(2c+3)$
	}
    \begin{vastaus}
	b) $(b-2a)(c+3)$
    \end{vastaus}
\end{tehtava}

\begin{tehtava}
Sekaluku $7\,\frac{3}{8}$ ilmaistuna murtolukuna on
\alakohdat{
§ $\frac{31}{4}$
§ $\frac{21}{8}$
§ $\frac{10}{8}$
§ $\frac{59}{8}$
§ $\frac{53}{8}$
}
    \begin{vastaus}
	 d) $\frac{59}{8}$
    \end{vastaus}
\end{tehtava}
%muokannut jaakko viertiö 22.3.2014

\begin{tehtava}
Potenssi $\left( \dfrac{2}{3} \right)^{-2}$ sievennettynä on jokin seuraavista. Mikä?
\alakohdat{
§ $36$
§ $\frac{1}{36}$
§ $\frac{4}{9}$
§ $-\frac{4}{9}$
§ $\frac{9}{4}$
§ $-\frac{9}{4}$
}
\begin{vastaus}
e) $\frac{9}{4}$
\end{vastaus}
\end{tehtava}
%muokannut jaakko viertiö 22.3.2014

\begin{tehtava}
Kuinka monta nollaa on biljoonassa?
\alakohdat{
§ $6$
§ $9$
§ $12$
§ $15$
§ ei mikään annetuista vaihtoehdoista
}
\begin{vastaus}
c) $12$
\end{vastaus}
\end{tehtava}

\begin{tehtava}
Mitä on $0,34$\,nm esitettynä metreinä kymmenpotenssimuodossa (oikeaoppisesti)?
\alakohdat{
§ $0,34 \cdot 10^{-9}$\,m
§ $3,4 \cdot 10^{-10}$\,m
§ $34 \cdot 10^{-9}$\,m
§ $3,4 \cdot 10^{-9}$\,m
§ $34 \cdot 10^{-10}$\,m
}
\begin{vastaus}
b) $3,4 \cdot 10^{-10}$\,m (standardimuodossa kymmenen potenssin kertoimena on luku itseisarvoltaan nollan ja kymmenen väliltä)
\end{vastaus}
\end{tehtava}

\begin{tehtava}
Kuinka monta merkitsevää numeroa on luvussa $1,0240$?
\alakohdat{
§ ei yhtään
§ yksi
§ kolme
§ neljä
§ viisi
}
\begin{vastaus}
e) Viisi
\end{vastaus}
\end{tehtava}

\begin{tehtava}
Luku $-78,2449$ pyöristettynä sadasosien tarkkuuteen on
\alakohdat{
§ $-78,245$
§ $-78,25$
§ $-78,24$
§ $-78,2$
}
\begin{vastaus}
c) $-78,24$ (tuhannesosia pienempiä desimaaleja ei huomioida lainkaan; pyöristys nollaan päin)
\end{vastaus}
\end{tehtava}

\begin{tehtava}
Kaksi litraa kuutiosenttimetreinä on	
\alakohdat{
§ $2$\,cm$^3$
§ $20$\,cm$^3$
§ $200$\,cm$^3$
§ $2\,000$\,cm$^3$.
§ Ei mikään annetuista vaihtoehdoista.
}
	\begin{vastaus}
	d) $2\,000$\,cm$^3$
	\end{vastaus}
\end{tehtava}

\begin{tehtava}
Nopeus $60$\,km/h ilmaistuna yksiköissä m/s on suunnilleen
\alakohdat{
§ $10$
§ $17$
§ $30$
§ $36$
§ $216$.
}
	\begin{vastaus}
	b) $17$
	\end{vastaus}
\end{tehtava}

\begin{tehtava}
Neliöjuuri luvusta $8$ mahdollisimman pitkälle sievennettynä on
\alakohdat{
§ $\sqrt{8}$
§ $2\sqrt{2}$
§ $4$
§ $16$
§ $64$.
}

  \begin{vastaus}
	 b) $2\sqrt{2}$
    \end{vastaus}
\end{tehtava}
%muokannut jaakkoviertiö 22.3.2014

\subsubsection*{Yhtälöt}

\begin{tehtava}
Yhtälö $x = x-1$ on
\alakohdat{
§ aina tosi
§ joskus tosi
§ ei koskaan tosi
}
\begin{vastaus}
c) ei koskaan tosi
\end{vastaus}
\end{tehtava}

\begin{tehtava}
Yhtälö $x = 4$ on
\alakohdat{
§ aina tosi
§ joskus tosi
§ ei koskaan tosi
}
\begin{vastaus}
b) joskus tosi
\end{vastaus}
\end{tehtava}

\begin{tehtava}
Yhtälö $x = x$ on
\alakohdat{
§ aina tosi
§ joskus tosi
§ ei koskaan tosi
}
\begin{vastaus}
a) aina tosi
\end{vastaus}
\end{tehtava}

\begin{tehtava}
Milloin yhtälön molemmat puolet voi kertoa samalla luvulla niin, että tosi yhtälö säilyy totena?
\alakohdat{
§ Aina muttei millä tahansa luvulla
§ Aina ja millä tahansa luvulla
§ Vain joskus tietyissä tilanteissa
§ Ei koskaan
}
	\begin{vastaus}
b) Aina ja millä tahansa luvulla
	\end{vastaus}
\end{tehtava}

\begin{tehtava}
Milloin yhtälön molemmat puolet voi kertoa samalla luvulla niin, että yhtälö säilyy yhtäpitävänä?
\alakohdat{
§ Aina muttei nollalla
§ Aina ja millä tahansa luvulla
§ Vain joskus tietyissä tilanteissa
§ Ei koskaan
}
\begin{vastaus}
a) Aina muttei nollalla (nollalla kertominen kadottaa informaatiota, eikä alkuperäistä yhtälöä enää saa ratkaistua)
\end{vastaus}
\end{tehtava}

\begin{tehtava}
Jos muuttujat $x$ ja $y$ ovat suoraan verrannolliset, ja muuttuja $x$ kasvaa, mitä tapahtuu muuttujalle $y$?
\alakohdat{
§ Muuttuja $y$ kasvaa myös mutta suhteessa vähemmän.
§ Muuttuja $y$ kasvaa myös mutta suhteessa enemmän.
§ Muuttuja $y$ kasvaa myös -- ja samassa suhteessa.
§ Muuttuja $y$ pienee -- ja samassa suhteessa.
§ Muuttujan $y$ arvo pysyy vakiona.
}
\begin{vastaus}
e) Muuttuja $y$ kasvaa myös -- ja samassa suhteessa.
\end{vastaus}
\end{tehtava}

\begin{tehtava}
Jos muuttujat $x$ ja $y$ ovat suoraan verrannolliset, ja muuttujat $y$ ja $z$ ovat kääntäen verrannolliset, millainen on muuttujien $x$ ja $z$ välinen suhde?
\alakohdat{
§ $x$ ja $z$ ovat suoraan verrannolliset.
§ $x$ ja $z$ ovat kääntäen verrannolliset.
§ $x$ on verrannollinen $z$:n neliöön.
§ $z$ on verrannollinen $x$:n neliöön.
§ Mikään annetuista vaihtoehdoista ei ole oikein.
}
	\begin{vastaus}
b) $x$ ja $z$ ovat kääntäen verrannolliset.
	\end{vastaus}
\end{tehtava}

\begin{tehtava} %FIXME: virkekoherenssi
Yhtälön $a^2+a^{-2}=0$ (eräs) juuri on
\alakohdat{
§ $-1$
§ $0$
§ $1$
§ $\sqrt{2}$
§ ei mikään mainituista vaihtoehdoista.
}
	\begin{vastaus}
e) ei mikään mainituista vaihtoehdoista
	\end{vastaus}
\end{tehtava}

\begin{tehtava}
Kuinka monta eri reaaliratkaisua yhtälöllä $x^{10}=-2$ on?
\alakohdat{
§ ei yhtään
§ yksi
§ kaksi
§ kymmenen
§ ei mikään mainituista vaihtoehdoista
}
	\begin{vastaus}
a) nolla
	\end{vastaus}
\end{tehtava}

\begin{tehtava}
Yhtälönratkaisun vaiheessa
\begin{align*}
2x+x&=4 \\
3x&=4
\end{align*}
käytettiin avuksi
\alakohdat{
§ reaalilukujen vaihdantalakia
§ reaalilukujen liitäntälakia
§ reaalilukujen osittelulakia
§ kaikkia edellä mainittuja
§ ei mitään edellä mainituista.
}
	\begin{vastaus}
	c) reaalilukujen osittelulakia
	\end{vastaus}
\end{tehtava}

%\begin{tehtava} %verrannollisuus
%
%\alakohdat{
%§
%§
%§
%§
%}
%	\begin{vastaus}
%	
%	\end{vastaus}
%\end{tehtava}
%
%\begin{tehtava}%verrannollisusmonivalintoja
%
%\alakohdat{
%§
%§
%§
%§
%}
%	\begin{vastaus}
%	
%	\end{vastaus}
%\end{tehtava}
%
%\begin{tehtava}%verrannollisusmonivalintoja
%
%\alakohdat{
%§
%§
%§
%§
%}
%	\begin{vastaus}
%	
%	\end{vastaus}
%\end{tehtava}

\begin{tehtava}
$0,46$\,\permil esitettynä desimaalilukuna on
	\alakohdat{
	§ $0,046$
	§ $0,0046$
	§ $0,00046$
	§ $0,000046$
	§ Ei mikään annetuista vaihtoehdoista.
	}
	\begin{vastaus}
	c) $0,00046$
	\end{vastaus}
\end{tehtava}

\begin{tehtava}
Jos lukua $a$ vähennetään $3$\,\%, laskutoimitusta vastaava lauseke on
\alakohdat{
§ $0,03a$
§ $0,97a$
§ $a-0,03$
§ $a-0,97$
§ $\frac{a}{0,03}$
§ $\frac{a}{1,03}$.
}
	\begin{vastaus}
b) $0,97a$
	\end{vastaus}
\end{tehtava}

\subsubsection*{Funktiot}

\begin{tehtava}
Funktio $f$ määritellään yhtälöllä $f(x)=\frac{1}{x-1}$. Funktion määrittelyjoukko on tällöin
\alakohdat{
§ $\mathbb{R}$
§ $\mathbb{R}_+$
§ $\mathbb{R}_-$
§ $\mathbb{R}\setminus \lbrace 0 \rbrace$
§ $\mathbb{R}\setminus \lbrace 1 \rbrace$
§ $\mathbb{R}\setminus \lbrace -1 \rbrace$.
}
    \begin{vastaus}
	 e) $\mathbb{R}\setminus \lbrace 1 \rbrace$
    \end{vastaus}
\end{tehtava}

\begin{tehtava}
Funktio $g$ määritellään yhtälöllä $g(y)=\sqrt{y}$. Funktion määrittelyehto on tällöin
\alakohdat{
§ $y<0$
§ $y\leq 0$
§ $y\in \mathbb{R}$
§ $y\geq 0$
§ $y>0$.
}
    \begin{vastaus}
	 d) $y\geq 0$
    \end{vastaus}
\end{tehtava}

\begin{tehtava}
Pallon tilavuus lasketaan pallon säteen funktiona kaavalla $V(r)=\frac{4}{3}\pi r^3$. Funktiolle $V$ sopivin arvojoukko on tällöin
\alakohdat{
§ $\mathbb{N}$
§ $\mathbb{Q}$
§ $\mathbb{R}$
§ $\mathbb{R}_-$
§ $\mathbb{R}_+$.
}
	\begin{vastaus}
	e) $\mathbb{R}_+$ (tilavuus ei voi olla negatiivinen)
	\end{vastaus}
\end{tehtava}

	\begin{tehtava}
Arvosanojen $6$, $7$ ja $10$ aritmeettinen keskiarvo on
	\alakohdat{
	§ $7$
	§ $7,25$
	§ $7\,\frac{2}{3}$
	§ $8,25$
	§ Ei mikään annetuista vaihtoehdoista.
	}
	\begin{vastaus}
	c) $7\,\frac{2}{3}$
	\end{vastaus}
\end{tehtava}

\newpage
\subsection{Tosi vai epätosi?}

Pitävätkö väitteet paikkansa?

\begin{tehtava}
\alakohdat{
§ Kahden eri alkuluvun osamäärä ei voi olla kokonaisluku.
§ Jos keskinopeus nousee $5$\,\%, matka-aika lyhenee $5$\,\%.
§ Jos funktio on relaationa symmetrinen, sen määrittely- ja arvojoukko ovat samat.
§ Triljoona kirjoitetaan kymmenpotessimuodossa $10^{15}$.
§ Lukujen osamäärä on laskutoimituksena liitännäinen.
§ Erotus on laskutoimituksena vaihdannainen.
§ Kaikki kokonaisluvut ovat rationaalilukuja.
§ Jos $a$ on $p$ prosenttia suurempi kuin $b$, niin $b$ on $p$ prosenttia pienempi kuin $a$.
§ Lämpötila on skalaarisuure.
§ Jos tiheys kerrotaan tilavuudella, saadaan massa.
§ Yhtälö $2(3x^2-x-1)=3(2x^2-x-1)$ on ensimmäisen asteen yhtälö.
§ Jos eksponenttifunktion kantalukuna on positiivinen reaaliluku, ja funktion määrittelyjoukko on $\mathbb{R}$, tällöin funktion arvojoukko on myös $\mathbb{R}$.
§ Suurin osa luonnollisista luvuista on suurempia kuin $10^{100}$.
§
§
§
§
§
§
§
}
	\begin{vastaus}
	\alakohdat{
	§ Tosi
	§ Epätosi
	§ Tosi 
	§ Tosi
	§ Epätosi
	§ Epätosi
	§ Tosi
	§ Epätosi
	§ Tosi
	§ Tosi
	§ Tosi
	§ Epätosi
	§ Tosi
	§
	§
	§
	§
	§
	§
	§
	}
	\end{vastaus}
\end{tehtava}

\newpage