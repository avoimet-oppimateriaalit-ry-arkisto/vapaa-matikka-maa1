 \subsection{Suureet ja yksiköt}

Monissa sovelluksissa (ja erityisesti fysiikassa) luvut eivät esitä vain matemaattista lukuarvoa, vaan niitä käytetään yhdessä jonkin \termi{yksikkö}{yksikön} kanssa. Tällöin luku ja yksikkö yhdessä esittävät tietoa \termi{suure}{suureesta} eli jostakin mitattavasta ominaisuudesta.
%selitä vielä, mitä yksikkö tekee?

\begin{esimerkki}

Aika on suure, jota voidaan mitata käyttämällä yksikkönä esimerkiksi sekunteja tai vuosia.

Pituus on suure, ja sen arvo voidaan mitata ja esittää käyttäen yksikkönä esimerkiksi metrejä ja tuumia. Itse asiassa myös matkaa ja etäisyyttä mitataan samoissa yksiköissä; näillä suureilla on sovellusriippuvainen, erityisesti geometriassa ja fysiikassa esille tuleva nyanssiero.

Energia on suure, jota mitataan esimerkiksi yksiköissä joule tai kalori.
\end{esimerkki}

Jotkin suureet ovat \termi{skalaarisuure}{skalaarisuureita} ja toiset \termi{vektorisuure}{vektorisuureita}. Skalaarisuureet ovat sellaisia mitattavia ominaisuuksia, joiden kuvaamiseen riittää yksi luku. Vektorisuureet ovat ominaisuuksia, joilla on sekä suuruus että suunta, eikä yksi luku riitä kertomaan kaikkea. (Vektorilaskentaan tutustutaan tarkemmin MAA5-kurssilla sekä fysiikan mekaniikan kursseilla.)

\begin{esimerkki}
Lämpötilan esittämiseen riittää yksi luku, eli lämpötila on skalaarisuure. Käytetystä yksiköstä riippuen kyseinen luku saattaa olla joko negatiivinen, nolla tai positiivinen.

Nopeus on vektorisuure, koska nopeuteen liittyy sen suuruuden lisäksi myös suunta. Esimerkiksi puhuttaessa tuulen nopeudesta kerrotaan myös, mistä suunnasta tuuli puhaltaa.
%kuva nopeusvektorinuolesta (tuulennopeus)
\end{esimerkki}

Matemaattisesti suureita käsitellään aivan kuin luku ja yksikkö olisivat kerrotut keskenään:
\laatikko[Suure]{luku $\cdot$ yksikkö}

Tehtävissä yksiköistä käytetään standardilyhenteitä, eikä yksiköiden nimiä tarvitse kirjoittaa kokonaan. Käytäntö on jättää kertolaskun sijasta luvun ja yksikön väliin tyhjää (tämä on ISO-standardin mukainen merkintätapa). %Ehkä myös Kotuksen ja Jukka Korpelan suosittelema. Aiemmin tätä perusteltiin tässä Suomen kieliopilla. Suomen kielioppi ei ole kuitenkaan mikään yksittäinen monoliittinen virallisesti olemassa oleva standardi vaan pikemminkin (normatiivisessakin muodossaan) kokoelma eri tavoin perusteltuja näkemyksiä ja suosituksia. -Jouni
Tähän on muutama poikkeus: kulman suuruutta ilmaisevat asteet, minuutit ja sekunnit, joita merkitään symboleilla $\textdegree$, ' ja '', sekä jotkin pituusyksiköt kuten tuuma, jota merkitään ${\verb+"+}$. Yksiköt kirjoitetaan konekirjoitetussa tekstissä pystyyn, ei \textit{kursiivilla}.

%Sivunlaitahuomio:: jos yksikkö on nimetty henkilön nimen mukaan, niin yksikkö kirjoitettuna lyhenteellään alkaa isolla kirjaimella, esim. N, J tai Hz. Auki kirjoitettuna kuitenkin "kolme newtonia", kymmenen kilojoulea" ja "5,7 hertziä".


\begin{esimerkki}
Seuraava taulukko selventää oikeinkirjoitusta:
\begin{center}
\begin{tabular}{c|c}
Oikein & Väärin \\
\hline
$3$\,m & $3$m 	\\
$100$\,$\frac{\text{km}}{\text{h}}$ & $100\frac{km}{h}$	\\
$2$\,\% & $2$\% \\
$3\textdegree$ & $3$ $\textdegree$\\
$3,16\textdegree$ $5$'\,$13$'' & $3,16$ $\textdegree$ $5$\,' $13$\,'' 	\\
$6$\,$\textdegree$C & $6\textdegree$C 	\\
\end{tabular}
\end{center}
\end{esimerkki}

\newpage
\subsection{SI-järjestelmä}

SI-järjestelmä eli kansainvälinen yksikköjärjestelmä (ransk. \textit{Système international d'unités}) on maailman käytetyin mittayksikköjärjestelmä. Siihen kuuluvat seuraavat \termi{perussuure}{perussuureet} ja vastaavat perusyksiköt:

\laatikko[SI-järjestelmän perussuureet ja yksiköt]{
\begin{center}
\begin{tabular}{c|c|c|c}
Suure & Suureen tunnus & Yksikkö & Yksikön tunnus\\
\hline
pituus & $l$ & metri &	m\\
massa & $m$ &kilogramma & kg 	\\
aika & $t$ & sekunti & s \\
sähkövirta & $I$ & ampeeri & A \\
(termodynaaminen) lämpötila & $T$ & kelvin & K \\
ainemäärä & $n$ & mooli & mol \\
valovoima & $I$ & kandela & cd
\end{tabular}
\end{center}}

Toisin kuin yksiköt, suureiden kirjainlyhenteet kirjoitetaan konekirjoituksessa oikeaoppisesti \textit{kursiivilla}.

Jokaisella perussuureen yksiköllä on oma tarkkaa määritelmänsä. Suureiden tunnuksia käytetään yleensä muuttujan symbolina kaavoissa. Itse suureita käsitellään tarkemmin fysiikassa ja kemiassa, mutta matematiikan kursseilla tulee tuntea ainakin pituus, massa ja aika. (Huomaa, että arkikielessä käytetään usein sanaa paino, vaikka tarkoitetaan massaa.) Tämän lisäksi matematiikan tehtävissä esiintyy usein joitakin näistä perussuureistayhdistämällä saatuja \termi{johdannaissuure}{johdannaissuureita} kuten nopeus, pinta-ala, tilavuus ja tiheys.
\newpage
\luettelolaatikko{Tärkeitä johdannaissuureita}{

§ Nopeus (tunnus $v$, engl. \textit{velocity}) on johdannaissuure, joka saadaan kuljetun matkan ja matkan kulkemiseen käytetyn ajan osamäärästä: $\text{nopeus}=\frac{\text{matka}}{\text{aika}}$. Vastaava nopeuden yksikkö saadaan jakamalla matkan yksikkö ajan yksiköllä, käytettiin mitä yksiköitä hyvänsä – esimerkiksi $\frac{\text{m}}{\text{s}}$ ja $\frac{\text{km}}{\text{h}}$.

§ Pinta-ala (tunnus $A$, engl. \textit{area}) saadaan kertomalla pituus toisella pituudella. Vastaavasti pinta-alan yksiköt saadaan pituuden yksiköistä. Jos esimerkiksi pituutta on mitattu metreissä, niin vastaava pinta-alan yksikkö on $\text{m}\cdot \text{m}=\text{m}^2$ eli neliömetri.

§ Tilavuus (tunnus $V$, engl. \textit{volume}) on edelleen johdettavissa kolmen pituuden tulona. Jos pituutta on mitattu metreissä, niin vastaava tilavuuden yksikkö on $\text{m}\cdot \text{m} \cdot \text{m}=\text{m}^3$ eli kuutiometri.

§ Tiheys (tunnus $\rho$, kreikkalainen pieni kirjain \textit{roo}) on johdannaissuure, joka määritellään massan ja tilavuuden osamääränä: $\text{tiheys}=\frac{\text{massa}}{\text{tilavuus}}$. Vastaavasti tiheyden yksiköt saadaan jakolaskulla massan ja tilavuuden yksiköistä: jos kappaleen tilavuus on mitattu kuutiometreissä ja massa kilogrammoina, voidaan kappaleen tiheys esittää yksiköissä $\frac{\text{kg}}{\text{m}^3}$. Tiheydestä kannattaa muistaa nyrkkisääntönä, että veden tiheys on noin yksi kilogramma per litra.
}

Kun johdannaisyksiköissä esiintyy jakolaskua, yksikön voi kirjoittaa useassa eri muodossa potenssisääntöjen avulla.

\begin{esimerkki}
Nopeuden yksikkö metriä sekunnissa voidaan kirjoittaa murtolausekkeena $\frac{\text{m}}{\text{s}}$, samalla rivillä m/s tai negatiivisen eksponentin avulla m\,s$^{-1}$.
\end{esimerkki}

Joillekin johdannaisyksiköille on annettu oma nimensä ja lyhenteensä, eikä yksikössä esiinny alkuperäisiä perusyksiköitä. 

\begin{esimerkki}
\alakohdat{
§ Tuhatta kilogrammaa kutsutaan tonniksi.
§ Yksi litra vastaa yhtä kuutiodesimetriä, eli $1\,\text{l}=1\,\text{dm}^3$ (ks. etuliitteet edellä).
§ Energian yksikkö joule ilmaistuna vain SI-järjestelmän perusyksiköitä käyttämällä: $\text{J}=\frac{\text{m}^2\,\text{kg}}{\text{s}^2}$.
}
\end{esimerkki}

\subsection{Suuruusluokat ja kymmenpotenssimuoto}

Suuruusluokkien ymmärtäminen (kuvaajien ja tilastojen tulkitsemisen ohella) on tärkeä osa \textit{luku}taitoa -- näiden ymmärtäminen estää sinua tulemasta huijatuksi ja harhaanjohdetuksi.\footnote[1]{Kirjallisuutta aiheesta tarjoaa muun muassa John Allen Paulosin teos Numerotaidottomuus. (Kirja on nimetty hassusti, ottaen huomioon, että siinä erityisesti puhutaan \textit{luvuista}, ei numeroista -- oletamme tämän huonoksi käännökseksi, koska englanninkielen sana \textit{number} tarkoittaa lukua.}
%monilla menee milligrammat ja mikrogrammat samaan kastiin "vähän"

Kymmenpotenssimuoto eli eksponenttimuoto on merkintätapa, jossa luku ilmoitetaan kertoimen sekä jonkin kymmenen kokonaislukupotenssin tulona. Kymmenpotenssimuodosta on hyötyä, kun halutaan kirjoittaa suuria tai pieniä lukuja lyhyesti.Myös mittaustarkkuuden ilmoittamiseksi tarvitaan joskus kymmenpotenssimuotoa. On yleinen tapa (standardi) valita esitysmuoto niin, että potenssin kerroin on vähintään yksi mutta alle kymmenen. Kymmenpotenssimuotoa $a\cdot 10^n$, missä $a$ on jotain ydehn ja kymmenen väliltä, kutsutaan joskus myös tieteelliseksi merkintätavaksi, engl. \termi{scientific notation}{scientific notation}. Sama luku voidaan kuitenkin esittää kymmenpotensismuodossa äärettömän monella tavalla, jos luovutaan kertoimen suuruusehdosta.

Yleisellä ilmaisulla ``\termi{pilkun siirtäminen}{pilkun siirtäminen}'' tarkoitetaan käytännössä luvun kymmenpotenssiesityksen muokkaamista. Pilkku siirtyy oikealle yhtä monta pykälää kuin kymmenen potenssi pienenee ja päinvastoin. Pilkuttoman luvun tapauksessa pilkku ``kuvitellaan luvun perään''. Koska $10^0 = 1$, voidaan ilman kymmenpotenssia esitetyn luvun perään tarvittaessa lisätä tämä kerroin mielessä. $10$ tarkoittaa luonnollisesti samaa asiaa kuin $10^1$. 

\laatikko[Joitakin esitysmuotoja samoille luvuille]{
\begin{center}
\begin{tabular}{c|c|c|c|c|c|c}
\hline
$1\,234\,000\cdot10^{-3}$ & $14\cdot10^{-3}$ & $130\,000\,000\,000\cdot10^{-3}$ & $1\cdot10^{-5}$ \\
$123\,400\cdot10^{-2}$ & $1,4\cdot10^{-2}$ & $13\,000\,000\,000\cdot10^{-2}$ & $0,1\cdot10^{-4}$ \\
$12\,340\cdot10^{-1}$ & $0,14\cdot10^{-1}$ &  $1\,300\,000\,000\cdot10^{-1}$ &  $0,001\cdot10^{-2}$ \\
$1\,234$ & $0,014$ & $130,000\,000$ & $0,00001$ \\
$123,4\cdot10$ & $0,0014\cdot10$ & $130\cdot10^{6}$ & $0,000001\cdot10$ \\
$12,34\cdot10^{2}$ & $0,00014\cdot10^{2}$ & $1,3\cdot10^{2}$ & $0,0000001\cdot10^{2}$ \\
$1,234\cdot10^{3}$ & $0,000014\cdot10^{3}$ & $0,13\cdot10^{9}$ & $0,00000001\cdot10^{3}$ \\
    	 \end{tabular}
  \end{center}
}
\begin{esimerkki}
\alakohdat{
§ Maan massa on noin $5\,974\,000\,000\,000\,000\,000\,000\,000$\,kg. Luvussa on ensimmäisen numeron (5) jälkeen vielä $24$ numeroa, eli kymmenpotenssimuodoksi saadaan $5,974\cdot10^{24}$\,kg.
§ Vetymolekyylissä H$_2$ ytimien välinen etäisyys on noin $0,000000000074$\,m eli $7,4\cdot10^{11}$\,m.  
§ Jos juoksuradan pituudeksi ilmoitetaan $1\,000$\,m, ei lukija voi tietää, millä tarkkuudella mittaus on tehty. Mittaustarkkuus välittyy kymmenpotenssin avulla: jos etäisyys ilmoitetaan muodossa $1,00\cdot10^{3}$\,m, on mitattu kymmenen metrin tarkkuudella, ja jos muodossa $1,0\cdot10^{3}$\,m, on mitattu sadan metrin tarkkuudella jne.
}
\end{esimerkki}

\subsection{Etuliitteet ja suurten lukujen nimet}

Monesti suureita kuvataan \termi{kerrannaisyksikkö}{kerrannaisyksiköiden} avulla. Tällöin yksikön suuruutta muokataan etuliitteellä, joka toimii suuruusluokkaa muuttavana kertoimena.

\laatikko[Tavallisimmat kerrannaisyksiköiden etuliitteet]{
\begin{center}
\begin{tabular}{c|c|c|c}
Nimi & Kerroin & Tunnus & Arvo suomeksi \\
\hline
deka & $10^{1}$ & da  & kymmenen	\\
hehto & $10^{2}$ & h & sata 	\\
kilo & $10^{3}$ & k & tuhat	\\
mega & $10^{6}$ & M & miljoona 	\\
giga & $10^{9}$ & G & miljardi		\\
tera & $10^{12}$ & T & biljoona \\
peta & $10^{15}$ & P & tuhat biljoonaa \\
\end{tabular}

\begin{tabular}{c|c|c|c}
Nimi & Kerroin & Tunnus & Arvo suomeksi \\
\hline
desi & $10^{-1}$ & d & kymmenesosa	\\
sentti & $10^{-2}$ & c & sadasosa\\
milli & $10^{-3}$ & m	& tuhannesosa\\
mikro & $10^{-6}$ & $\upmu$ & miljoonasosa\\
nano & $10^{-9}$ & n & miljardisosa\\
piko & $10^{-12}$ & p & biljoonasosa\\
femto & $10^{-15}$ & f & tuhatbiljoonasosa
\end{tabular}
\end{center}
}

%toi taulukko pitää tasata! 

Etuliitteitä käytetään siis muuttamaan yksikköä pienemmäksi tai suuremmaksi. Tarkoituksena on yleensä valita sovelluskohtaisesti sopivan kokoinen yksikkö niin, että käytetyn luvun ei tarvitse olla valtavan suuri tai pieni. Mitä suuremman yksikön valitsee, sitä pienemmäksi sen kanssa käytettävä luku tulee -- ja toisinpäin.

\begin{esimerkki}
\alakohdat{
§ Kilogramma tarkoittaa tuhatta grammaa. Kilogramma on valittu SI-järjestelmässä massan perusyksiköksi gramman asemesta käytännöllisyyssyistä, sillä yksi gramma on moniin sovelluksiin turhan pieni yksikkö.
§ Ilmaisussa $0,000046$ metriä luku on niin pieni, että on käytännöllisempää ja usein helppotajuisempaa kirjoittaa $46$ mikrometriä eli $46$\,$\upmu$m.
§ Hyvin pienistä massoista puhuttaessa voidaan puhua esimerkiksi nanogrammoista. Esimerkiksi $1\,\textrm{ng} = 1 \cdot 10^{-9}\,\textrm{g} = 0,000000009\,\textrm{g}$.
}
\end{esimerkki}

Taulukoitujen suurten lukujen nimet kannattaa opetella, ja myös huomata, että eri kielten välillä on joitakin poikkeuksia.
\begin{esimerkki}
Amerikanenglanniksi miljardi on billion, ja biljoona on trillion.
\end{esimerkki}
% Suomessa on käytössä niin sanottu lyhyt asteikko...

%long ja short scale -selitys! +harjoitustehtäviä englannista suomeen ja toisinpäin

\subsubsection*{Huomautuksia tietotekniikasta}

Datan tai informaation määrää mitataan bitteinä ja tavuina -- yksi tavu on kahdeksan bittiä. Tietotekniikassa datan määrä on suureena dikstreetti eli bittejä voi olla vain jokin luonnollisen luvun osoittama määrä. Bittejä ei siis voi jakaa mielivaltaisen pieniin palasiin toisin kuin esimerkiksi sekunteja ja metrejä. Yhteen kahdeksan bitin mittaiseen tavuun voidaan tallettaa kokonaisluku väliltä 0--255. Tavun sisällön merkitys voidan tulkita usealla tavalla, esimerkiksi yhtenä kirjaimena jossain merkistössä. Suomen kielessä bitin lyhenne on b ja tavun t. Usein käytettävät englanninkieliset lyhenteet bitille (engl. \textit{bit}) ja tavulle (engl. \textit{byte}) ovat vastaavasti b ja B.

\begin{esimerkki} %tarkennus tähän
Pelkästään muotoilematonta tekstiä kirjoittaessa yksi tavallinen englanninkielen kirjoitusmerkki vie tietokoneella tallennustilaa yhden tavun. % tulee huono arvio, koskaerikoismerkit :(  Englanninkielensanoissa on keskimäärin $5,1$ kirjainta. \textit{Taru sormusten herrasta} -kirjatrilogian alkuperäistekstissä on yhteensä noin $473$ tuhatta sanaa, joten merkkejä trilogiassa on yhteensä noin

Internetpalveluntarjoajat mainostavat yhteysnopeuksiaan yleensä muodossa $X$/$Y$, missä $X$ on luvattu teoreettinen maksiminopeus datan vastaanottamiselle (lataamiselle, engl. \textit{download}) ja $Y$ teoreettinen maksiminopeus datan lähettämiselle (engl. \textit{upload}). Yksikkönä mainosteksteissä on tavallisilla kuluttajanopeuksilla usein englanninkielinen Mbps eli megabittiä sekunnissa (\textit{megabits per second}, oikeaoppisesti kirjoitettuna Mb/s). Koska ilmaisu on biteissä käytännössä enemmän käytettyjen tavujen sijaan, nopeus näyttää suurelta. Oma yhteysnopeus kannattaa testata jossain siihen tarkoitetussa verkkopalvelussa -- palveluntarjoajalle eli operaattorille kannattaa valittaa, jos yhteysnopeus jää paljon luvatusta.
\end{esimerkki}

Tietokoneiden rakenteen kaksikantaisuuden (ks. liite lukujärjestelmistä) johdosta tavujen kerrannaisyksiköllä tarkoitetaan yleensä kymmenen potenssien sijaan usein lähimpiä kahden potensseja. SI-järjestelmä ei tue kaksikantaista käytäntöä, vaan sen mukaan kt on tasan $1\,000$\,t ja niin edelleen. Tarvitsemme datan käsittelyyn uudet binääriset etuliitteet:
%sisältyvät IEC:n standardiin (\textit{International Electrotechnical Commission}

\begin{center}
\begin{tabular}{c|c|c|c}
Etuliite & Nimi & Arvo & Lähin kymmenenpotenssi \\
\hline
ki & kibi & $2^{10}$ & $10^3$	\\
Mi & mebi & $2^{20}$ & $10^6$\\
Gi & gibi & $2^{30}$ & $10^9$\\
Ti & tebi & $2^{40}$ & $10^{12}$ \\
Pi & pebi & $2^{50}$ & $10^{15}$\\
\end{tabular}
\end{center}

Kyseisten etuliitteiden käyttö on kuitenkin (edelleen) harvinaista. Kun vuosikymmeniä sitten datan tallennuskapasiteetti oli pieni, ei kilotavun ja kibitavun (suhteellinen) ero ollut suuri, mutta nykyään ero on suurempien datamäärien käsittelyssä hyvin merkityksellinen. 

\begin{esimerkki}
Lasketaan SI-etuliiteiden ja binääristen eli kaksikantaisten etuliitteiden erotuksia tavujen tilanteessa.
\alakohdat{
§ $1$\,kit$-1$\,kt$=1\,024$\,t$-1\,000$\,t$=24$\,t
§ $1$\,Mit$-1$\,Mt$=2^{20}-10^6=48\,576$\,t
§ $1$\,Git$-1$\,Gt$=2^{30}-10^9=73\,741\,824$\,t
}
Huomataan, että suuruusluokkien kasvaessa esitystapojen välinen virhe kasvaa merkittävästi.
\end{esimerkki}

Arkikielessä ja markkinoinnissa nykyään käytetään kuitenkin lähinnä SI-järjestelmän etuliitteitä, ja on ymmärrettävä kontekstista, että ''kilotavulla'' tarkoittaakin $1\,024$:ää tavua ($2^{10}$), ''megatavulla'' tarkoitetaankin $1\,024$:ta ''kilotavua'', ja niin edelleen.

Valitettavasti monessa yhteydessä ei ole selvää, tarkoitetaanko kymmenen vai kahden potensseja. Jotkin käyttöjärjestelmät ja tietokoneohjelmat sanovat toista ja tarkoittavat toista. Kuluttajan hämmennykseksi usein esimerkiksi keskusmuistien koosta puhuttaessa käytetään binäärikerrannaisyksikköjä mutta kiintolevyjen koosta puhuttaessa kymmenen potensseja. Siirtonopeudet sen sijaan ovat yleensä \textit{aivan oikein} ilmoitettu SI-etuliitteillä.

\begin{esimerkki}
Kun ääntä (esimerkiksi musiikkia) pakataan MP3-muotoon, keskiverron ääneenlaadun saavuttamiseksi bittivirta (engl. \textit{bitrate}) voi olla $128$\,kb/s. Tämä tarkoittaa oikeasti tasan $128\,000$ bitin siirtoa sekuntia kohden. Koska tavu on kahdeksan bittiä, vastaa tämä tavuissa tiedonsiirtonopeutta $\frac{128}{8}$\,kb/s$=16$\,kt/s.
\end{esimerkki}

\begin{esimerkki}
Ostat kahden ''teran'' kiintolevyn pöytäkoneeseesi. Tuotteessa itsessään lukee tällöin yleensä englanniksi $2$\,TB ja suomeksi puhumme teratavuista (merkitään $2$\,Tt). Ilmaisu tuo kuitenkin mukanaan yllätyksen, sillä kun levyn asentaa koneeseen, voi käyttöjärjestelmä ilmoittaa vapaan tilan olevan huomattavasti kahta teratavua pienempi. Ero johtuu siitä, että levyllä on tilaa $2\,\textrm{Tt}=2\cdot 10^{12}$ tavua, mutta ohjelmat, käyttöjärjestelmä (ja myös koneen käyttäjä) voi olettaa koon olevan ilmoitettu kaksikantaisena niin, että kokonaistila olisikin $2\,\textrm{Tit}=2\cdot 2^{40}$ tavua. Jos käyttöjärjestelmä tämän lisäksi ilmoittaa levytilan SI-etuliitteiden avulla (vaikka tarkoittaiskin kaksikantaisia), voi levyn ostajalle tulla huijattu olo. Eroa näillä kahdella ilmaisulla on $2\cdot 2^{40}\,$t$-2\cdot 10^{12}\,$t$=99\,511\,627\,776\textrm{t}\approx 100 gigatavua \approx 93$ gigitavua! , eli käytettävissä olevaa tilaa onkin melkein kymmenesosa vähemmän kuin mitä on mielestää ostanut! (Monta kiintolevyvalmistajaa on haastettu oikeuteen näistä epämääräisyyksistä.)
\end{esimerkki}

\subsection{Yksikkömuunnokset}

Kerrannaisyksiköiden tapauksessa tulee vain huomata, että etuliitteet ovat matemaattisesti vain kertoimia. Kaikki ilmaisut voidaan aina halutessa esittää luvun ja yksikön tulona ilman etuliitteitä.

\begin{esimerkki}
$270\,\text{mm}=270\cdot\text{mm}=270\cdot10^{-3}\,\text{m}=\frac{270}{1\,000}\,\text{m}=0,27\,\text{m}$
\end{esimerkki}

\begin{esimerkki}
Erään solun tilavuus on $9,0 \cdot 10^{-12}$\,l. Ilmoita tilavuus
\alakohdat{
§ pikolitroina
§ nanolitroina
}

\begin{esimratk}
\alakohdat{
§ $9,0 \cdot 10^{-12}\,\textrm{l} = 9,0\,\textrm{pl}$.
§ Kymmenen eksponentiksi halutaan $-9$. Koska $10^{-12}$ on pienempi kuin $10^{-9}$, sitä tulee kasvattaa -- samalla kerroin pienenee. Kymmenenpotenssi kerrotaan tuhannella, ja kerroin jaetaan samaten tuhannella, jotta itse lukuarvo ei muutu: $9,0 \cdot 10^{-12}=9,0\cdot 10^{-3} \cdot 10^{-9}=0,009\,\textrm{nl}$.
}
\end{esimratk}
\end{esimerkki}

\laatikko[Erittäin tärkeää kerrannaisyksiköiden potensseista!]{Edellä esitetty taulukko etuliitteistä pätee vain ensimmäisen asteen yksiköihin! Desimetri on kymmenesosa metristä, mutta neliödesimetri \emph{ei} ole kymmenesosa neliömetristä. (Vrt. neliödesimetri ja desineliömetri)

Avain tämän ymmärtämiseen on tietää seuraava kirjoitustavan sopimus:

\begin{center}
dm$^2$ tarkoittaa oikeasti $(\text{dm})^2$!
\end{center}

Potenssisääntöjä käyttämällä huomataan: $(\text{dm})^2=\text{d}^2\text{m}^2=(10^{-1})^2\, \text{m}^2=10^{-2}\,\text{m}^2=\frac{1}{100}\, \text{m}^2$. Eli yksi neliödesimetri onkin \textit{sadaosa} neliömetristä!}

\begin{esimerkki}
%yksiköiden käyttö yksikseen ilman lukua?
Kuinka mones osa kuutiodesimetri on kuutiometristä?

	\begin{esimratk}
	Puretaan auki käyttäen mainittua kirjoituskonventiota, potenssisääntöjä ja desi-etuliitteen määritelmää:
	\begin{align*}
	1\,\text{dm}^3&=1\,(\text{dm})^3\\
	&=1\,\text{d}^3\,\text{m}^3 \\
	&=1\,(10^{-1})^3\,\text{m}^3 \\
	&=10^{-1\cdot3}\,\text{m}^3 \\
	&=10^{-3}\,\text{m}^3 \\
	&=\frac{1}{1\,000}\,\text{m}^3
	\end{align*}
	Nähdään, että yksi kuutiodesimetri on tuhannesosa kuutiometristä.
	\end{esimratk}

\end{esimerkki}

%\begin{esimerkki}

%Kuinka monta desilitraa \ldots 

%\end{esimerkki}

%toine ja ehkä kolmaskin esimerkki

%ESIMERKKI! Vertailu: Jenkeissä autoista kerrotaan, kuinka monta mailia gallonalla – Suomessa, kuinka monta litraa satasella

Matkan (pituuden), pinta-alan ja tilavuuden yksikkömuunnoksia käsitellään tarkemmin ja lisää geometrian MAA3-kurssilla. 

Ajan yksiköissä on jäänteitä $60$-järjestelmästä. Yksikköjä ei jaetakaan kymmeneen pienempään osaan vaan kuuteenkymmeneen:

\luettelolaatikko{Ajan yksiköt}{
  § Yksi tunti on $60$ minuuttia: $1\,\text{h} = 60\,\text{min}$
  § Yksi minuutti on $60$ sekuntia: $1\,\text{min} = 60\,\text{s}$
  § Yksi tunti on $3\,600$ sekuntia: $1\,\text{h} = 60\,\text{min} = 60 \cdot 60\,\text{s} = 3\,600\,\text{s}$ 
}

%millisekunnit
%\begin{esimerkki}
%
%Kuinka monta sekuntia on vuodessa?
%
%	\begin{esimratk}
%
%	\end{esimratk}
%
%\end{esimerkki}

\begin{esimerkki}
Kuinka monta minuuttia on $1,25$\,h?

\begin{esimratk}
$1,25\,\text{h} = 1,25 \cdot 60\,\text{min} = 75\,\text{min}$. $1,25$ tuntia on siis $75$ minuuttia. Huomaa, että voit laskuissasi esittää desimaaliluvun $1,25$ sekamurtolukuna $1\frac{25}{100}$ eli $1\frac{1}{4}$, mikä saattaa helpottaa laskemista.
\end{esimratk}
\end{esimerkki}

\begin{esimerkki}
Pikajuoksija Usain Boltin huippunopeudeksi $100$ metrin juoksukilpailussa mitattiin $12,2$\,m/s. Janin polkupyörän huippunopeus alamäessä oli $42,5$\, km/h. Kumman huippunopeus oli suurempi?

\begin{esimratk}
Muunnetaan yksiköt vastaamaan toisiaan, jolloin luvuista tulee keskenään vertailukelpoisia. Janin polkupyörän huippunopeus oli $42,5\,\textrm{km/h} =42,5\cdot \textrm{(1\,000\,m)/h} = 42\,500\,\textrm{m/h} = 42\,500\,\textrm{m/(3\,600\,s)}= \frac{42\,500}{3\,600}\,\textrm{m/s} \approx 11,8\,\textrm{m/s}$. Bolt oli siis pyöräilevää Jania nopeampi.
\end{esimratk}
\end{esimerkki}

%Lasketaan, kuinka nopeasti $100$/$10$-yhteydellä ...

SI-järjestelmä ei ole ainoa käytetty mittayksikköjärjestelmä. Esimerkiksi niin kutsutussa brittiläisessä mittayksikköjärjestelmässä (imperial) pituutta voidaan mitata esimerkiksi tuumissa. Yksi tuuma (merkitään $1^{\prime \prime}$) vastaa $2,54$ senttimetriä. Massan perusyksikkönä brittiläisessä mittayksikköjärjestelmässä käytetään paunaa. %mihin pauna perustuu?

\begin{tabular}{c|c|c|c}
Suure & Yksikkö & Yksikön tunnus & SI-järjestelmässä\\
\hline
pituus & tuuma & $^{\prime \prime}$ (tai in) & $2,54$\,cm \\
massa & pauna & lb & $453,59237$\,g \\
\end{tabular}

Tuumien tai paunojen avulla ilmaistut suureet voidaan muuntaa SI-järjestelmään yllä olevassa taulukossa näkyvien suhdelukujen avulla.

\begin{esimerkki}
Kuinka monta senttimetriä on $3,5$ tuumaa? Entä kuinka monta tuumaa on $7,2$ senttimetriä?
	\begin{esimratk}
Yksi tuuma vastaa $2,54$ senttimetriä, joten kaksi tuumaa vastaa $5,08$ senttimetriä jne.
	
\begin{kuva}
	lukusuora.pohja(0,9,9)
	lukusuora.piste(0, "$0^{\prime \prime} = 0 $ cm")
	lukusuora.piste(2.54,"$1^{\prime \prime} = 2,54$ cm")
	lukusuora.piste(5.08,"$2^{\prime \prime} = 5,08$ cm")
	lukusuora.piste(7.62,"$3^{\prime \prime} = 7,62$ cm")
	with vari("red"):
	  lukusuora.piste(7.2)
	  lukusuora.piste(8.9)
\end{kuva}

Tuumat saadaan siis senttimetreiksi kertomalla luvulla $2,54$. Näin ollen $3,5^{\prime \prime} = 3,5 \cdot 2,54\,\textrm{cm} \approx 8,9\,$cm. Senttimetrien muuntaminen tuumiksi onnistuu vastaavasti \emph{jakamalla} luvulla $2,54$. Näin ollen $7,2\,\textrm{cm} \approx 2,8^{\prime \prime}$, koska $7,2:2,54 \approx 2,8$.
	\end{esimratk}
\end{esimerkki}

\subsection{Yksiköiden yhdistäminen laskuissa}

Jos lausekkeessa on useita samanlaisia yksiköitä, näitä voidaan usein yhdistää. Suureen määritelmän luvun ja yksikön kertolaskukäsityksestä seuraa, että sovelluksissa ja sievennystehtävissä yksiköillä voidaan laskea kuin luvuilla. Käytännössä yksiköitä kohdellaan kuin lukuja, ja niitä voi kertoa ja jakaa keskenään. Tutut potenssisäännöt, rationaalilukujen laskusäännöt ja kertolaskun vaihdannaisuuslaki pätevät –- lopullisessa sievennetyssä muodossa vain merkitään taas luvun ja yksikön väliin lyhyt väli kertolaskun asemesta. Kerrannaisyksiköiden etuliitteet voi aina muuttaa tarvittaessa kymmenpotenssimuotoon tai murtolausekkeiksi.

\begin{esimerkki}
$2$ metriä kertaa $3$ metriä = $2\,\text{m} \cdot 3\,\text{m} = 2 \cdot \text{m} \cdot 3 \cdot \text{m}= 2 \cdot 3 \cdot \text{m} \cdot \text{m} =6\,\text{m}^2$ eli kuusi neliömetriä.
\end{esimerkki}

%\begin{esimerkki}
%
%Vetyatomin halkaisija on noin 0,1 nanometriä. Kuinka monta vetyatomia tarvitaan peräkkäin, jotta saataisiin yhden senttimetrin pituinen atomijono?
%
%	\begin{esimratk}
%	
%	\end{esimratk}
%
%\end{esimerkki}

%\begin{esimerkki}
%
%Paperiarkkipinossa, jonka korkeus on 10 senttimetriä, on yksi riisi eli\ldots arkkia. Mikä on yhden paperiarkin paksuus? Kuinka monta arkkia on pinossa, jonka korkeus on metri? Tehtävässä ei huomioida paperiarkkien puristumista muiden papereiden painon vuoksi.
%
%	\begin{esimratk}
%	
%	\end{esimratk}
%\end{esimerkki}


\begin{esimerkki}
Keskinopeus $v$ voidaan laskea kaavasta $\frac{\text{s}}{\text{t}}$, missä $s$ on kuljettu matka ja $t$ matkaan käytetty aika. Tästä voidaan johtaa (ks. luku Yhtälöt) kaava, jolla voidaan laskea matkan kulkemiseen kulunut aika: $t=s/v$, missä $s$ on kuljettu matka ja $v$ on nopeus. Lasketaan, kuinka kauan kestää $300$ metrin matka nopeudella $7$\,m/s.

Sijoitetaan kaavaan yksikköineen $s=300$\,m ja $v= 7$\,m/s ja käytetään rationaalilukujen laskusääntöjä, jolloin saadaan:
\[t=\frac{100\,\textrm{m}}{7\,\textrm{m/s}} = \frac{100}{7} \cdot \frac{\textrm{m}}{\frac{\textrm{m}}{\textrm{s}}} 
= \frac{100}{7} \cdot \frac{\textrm{m}}{1} \cdot \frac{\textrm{s}}{\textrm{m}}
= \frac{100}{7} \cdot \frac{\cancel{m}\textrm{s}}{\cancel{m}}
=\frac{100}{7}\,\textrm{s} \approx 14\,\textrm{s}.\]
\end{esimerkki}

%\begin{esimerkki}
%Kuinka kauan valolta kestää matkata Auringosta Maahan? Valon nopeus on noin $3\cdot 10^8$\,m/s
%\end{esimerkki}

%hemoglobiiniviitearvot g/dl -> g/l

%Oisko pitänyt laittaa vaiheittain tuo vielä, että voi merkata välivaiheissa tehdyt jutut\ldots olisi hyvä. Saisi vielä kerran kerrattua sanallisesti, mitä rationaalilukujen ominaisuuksia käytettiin ja mitä potenssisääntöjä käytettiin\ldots :) T: JoonasD6

%tiheysesimerkki
%valonnopeus, kuina pitkän matkan kulkee... + kuinka kauankestää matkata...

%jokin esimerkki, missä potensseja\ldots

Huomaa, että eri suureita voi kertoa ja jakaa keskenään, mutta yhteen- ja vähennyslasku ei ole määritelty. (Mitä tarkoittaisikaan "kolme metriä plus kaksi grammaa"?)