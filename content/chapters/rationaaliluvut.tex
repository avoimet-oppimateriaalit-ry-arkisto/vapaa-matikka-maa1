% engl. \emph{natural numbers, counting numbers} ruots. \emph{naturliga tal}

% Luonnollisia lukuja käytetään kolmeen eri tarkoitukseen:
% Lukumäärien ilmoittamiseen (kardinaaliluvut)
% Järjestyksen ilmoittamiseen (ordinaaliluvut)
% Indeksointiin ja asioiden nimeämiseen

% FIXME: tähän voisi lisätä lyhyen, selkeän johdannon muuttujien käytön etuihin

% === SIISTITTÄVÄ PÄTKÄ ALKAA ===

Luonnollisille luvuille $m$ ja $n$ on määritelty yhteenlasku $m + n$, esimerkiksi $5 + 3 = 8$.

Luonnollisten lukujen $m$ ja $n$ kertolasku määritellään peräkkäisinä yhteenlaskuina
\laatikko{ \[m \cdot n = \underbrace{m + m + \ldots + m}_{n\text{ kpl}} = \underbrace{n + n + \ldots + n}_{m\text{ kpl}}.\] }

Nollalla kertomisen ajatellaan olevan ''tyhjä yhteenlasku'' eli nolla:
\laatikko{ \[0 \cdot m = 0\] }

Luonnollisten lukujen $m$ ja $n$ erotus määritellään yhteenlaskun avulla:
$m-n$ on luku $k$, jolle $k + n = m$. Kahden luonnollisen luvun erotus
ei kuitenkaan aina ole luonnollinen luku, esimerkkinä $3 - 5$.
Ratkaisemme ongelman määrittelemällä kullekin luonnolliselle
luvulle vastaluvun.

\laatikko{ Jokaisella luvulla $n$ on vastaluku $-n$, jolle pätee $n+(-n)=0$. }

Luonnolliset luvut ja niiden vastaluvut muodostavat yhdessä kokonaislukujen joukon
\[\zz = \{\ldots, -2, -1, 0, 1, 2, \ldots\},\] jota voidaan havainnollistaa \emph{lukusuoran} avulla

\begin{center}
\begin{lukusuora}{-4}{4}{10}
\lukusuorapiste{-3}{}
\lukusuorapiste{-2}{}
\lukusuorapiste{-1}{}
\lukusuorapiste{0}{}
\lukusuorapiste{1}{}
\lukusuorapiste{2}{}
\lukusuorapiste{3}{}
\lukusuoraalanimi{-3}{$-3$}
\lukusuoraalanimi{-2}{$-2$}
\lukusuoraalanimi{-1}{$-1$}
\lukusuoraalanimi{0}{$0$}
\lukusuoraalanimi{1}{$1$}
\lukusuoraalanimi{2}{$2$}
\lukusuoraalanimi{3}{$3$}

\end{lukusuora}
\end{center}

Kun käytämme kokonaislukuja, voidaan kahden luvun erotus määritellä
yhteenlaskun ja vastaluvun avulla yksinkertaisesti:

\laatikko{
\[m-n = m+(-n)\]
}

    Esimerkiksi luvun $2$ vastalukua merkitään $-2$, ja sille pätee $2+(-2)=0$. Vastaavasti luvun $-2$ vastaluku on sellainen luku, joka laskettuna yhteen luvun $-2$ kanssa antaa luvun $0$. Tämä on tietysti $2$, koska $-2+2=0$. Näin voidaan huomata, että $-(-2)=2$.
    
\laatikko{$$-(-a)=a$$}

\subsection*{Yhteen- ja vähennyslasku}

    Kysymys: Mitä saadaan, kun luvusta $5$ vähennetään luku $-8$?
    
    Negatiivisten ja positiivisten lukujen yhteen- ja vähennyslaskut voidaan helposti tulkita lukusuoran avulla.
    
    % tässä on vähän kyseenalaista käyttää sekaisin sanallista ja numeerista esitystä
    
    $5+8$ ''viiteen lisätään kahdeksan''
    \begin{center}
      \begin{lukusuora}{-1}{14}{14}
        %\lukusuoranuolialas{5}{13}
        %\lukusuoranuolialas{0}{8}
        {\color{red} \lukusuoravaliss{0}{5}{$0$}{$5$}}
        \lukusuorauusi
        {\color{red} \lukusuoravaliss{8}{13}{$8$}{${\color{black}13}$}}
        {\color{blue} \lukusuoravaliss{0}{8}{$0$}{$8$}}
       \end{lukusuora}
       ${\color{red}5}+{\color{blue}8}=13$
    \end{center}
    
    $5+(+8)$ ''viiteen lisätään plus kahdeksan''
    
    $+8$ tarkoittaa samaa kuin $8$. '$+$'-merkkiä käytetään luvun edessä silloin, kun halutaan korostaa, että kyseessä on nimenomaan positiivinen luku.
    
%säädetään kuva vähän irti tuosta edeltävästä tektistä
%\vspace{0.3cm}     
    
    \begin{center}
          \begin{lukusuora}{-1}{14}{14}
        {\color{red} \lukusuoravaliss{0}{5}{$0$}{$5$}}
        \lukusuorauusi
        {\color{red} \lukusuoravaliss{8}{13}{$8$}{${\color{black}13}$}}
        {\color{blue} \lukusuoravaliss{0}{8}{$0$}{$8$}}
       \end{lukusuora}
       ${\color{red}5}+({\color{blue}+8})=13$
    \end{center}
    
    $5-(+8)$ ''viidestä vähennetään $+8$''
    
    Tämä tarkoittaa samaa kuin $5-8$. Lukusuoralla siis liikutaan 8 pykälää taaksepäin.

%säädetään kuva vähän irti tuosta edeltävästä tektistä
\vspace{0.3cm}     
    
    \begin{center}
              \begin{lukusuora}{-4}{8}{14}
        {\color{blue} \lukusuoravaliss{-3}{5}{\color{black}$-3$}{$5$}}
        \lukusuorauusi
%        {\color{red} \lukusuoravaliss{8}{13}{$8$}{${\color{black}13}$}}
        {\color{red} \lukusuoravaliss{0}{5}{$0$}{$5$}}
       \end{lukusuora}
       ${\color{red}5}-({\color{blue}+8})=-3$
    \end{center}


%lukiolaiset pitivät näitä epäselvinä kuvina
    
Mitä tapahtuu, kun lisätään negatiivinen luku? Kun lukuun lisätään $1$, se kasvaa yhdellä. Kun lukuun lisätään $0$, se ei kasva lainkaan. Kun lukuun lisätään negatiivinen luku, esimerkiksi $-1$, on luonnollista ajatella, että se pienenee. Tällä logiikalla negatiivisen luvun lisäämisen pitäisi siis pienentää alkuperäistä lukua. Juuri näin vähennyslasku määritellään: $5+(-8)$ on yhtä suuri kuin $5-8$.
    
\vspace{0.3cm}     
    \begin{center}
                 \begin{lukusuora}{-4}{8}{14}
        {\color{blue} \lukusuoravaliss{-3}{5}{\color{black}$-3$}{$5$}}
        \lukusuorauusi
%        {\color{red} \lukusuoravaliss{8}{13}{$8$}{${\color{black}13}$}}
        {\color{red} \lukusuoravaliss{0}{5}{$0$}{$5$}}
       \end{lukusuora}
       ${\color{red}5}+({\color{blue}-8})=-3$
    \end{center}
    
    
    $5-(-8)$ ''viidestä vähennetään miinus kahdeksan''
    
    Negatiivisen luvun lisääminen on vastakohta positiivisen luvun lisäämiselle. Tällöin on luonnollista, että negatiivisen luvun vähentäminen on vastakohta positiivisen luvun vähentämiselle. Koska positiivisen luvun vähentäminen pienentää lukua, pitäisi negatiivisen luvun vähentämisen kasvattaa lukua. Lasku $5-(-8)$ tarkoittaa siis samaa kuin $5+8$.
\vspace{0.3cm}     
        
    \begin{center}
    \begin{lukusuora}{-1}{14}{14}
        {\color{red} \lukusuoravaliss{0}{5}{$0$}{$5$}}
        \lukusuorauusi
        {\color{red} \lukusuoravaliss{8}{13}{$8$}{${\color{black}13}$}}
        {\color{blue} \lukusuoravaliss{0}{8}{$0$}{$8$}}
       \end{lukusuora}
       ${\color{red}5}-({\color{blue}-8})=13$
    \end{center}


\laatikko{
Yhteen- ja vähennyslaskun merkkisäännöt

\begin{alakohdat}
\alakohta{$a+(+b)=a+b$}
\alakohta{$a+(-b)=a-b$}
\alakohta{$a-(+b)=a-b$}
\alakohta{$a-(-b)=a+b$}
\end{alakohdat}
}

%Koska $+b-b=0$, yhteen- ja vähennyslasku kumoavat toisensa, eli
%
%\begin{itemize}
%\alakohta{$a+b-b=a$
%\alakohta{$a-b+b=a$
%\end{itemize}



\subsection*{Kertolasku}

    Samaan logiikkaan perustuen on sovittu myös merkkisäännöt positiivisten ja negatiivisten lukujen kertolaskuissa. Kun negatiivinen ja positiivinen luku kerrotaan keskenään, saadaan negatiivinen luku, mutta kun kaksi negatiivista lukua kerrotaan keskenään, saadaan positiivinen luku.

    $3 \cdot 4$ ''kolme kappaletta nelosia''
    
   \begin{center}
    \begin{lukusuora}{-1}{14}{14}
	\color{red} \lukusuoravaliss{0}{4}{$0$}{$4$}
	\color{red} \lukusuoravaliss{4}{8}{$4$}{$8$}
	\color{red} \lukusuoravaliss{8}{12}{$8$}{$12$}

      \end{lukusuora}
      $3\cdot {\color{red}4}=12$
    \end{center}
    
    $3 \cdot (-4)$ ''kolme kappaletta miinus-nelosia''
    
    
    \begin{center}
    \begin{lukusuora}{-13}{2}{14}
	\color{red} \lukusuoravaliss{-12}{-8}{$-12$}{$-8$}
	\color{red} \lukusuoravaliss{-8}{-4}{$-8$}{$-4$}
	\color{red} \lukusuoravaliss{-4}{0}{$-4$}{$0$}

      \end{lukusuora}
      $3\cdot ({\color{red}-4})=-12$
    \end{center}
    
    $-3 \cdot 4$ ''miinus-kolme nelinkertaistetaan''
    
    \begin{center}
    \begin{lukusuora}{-13}{2}{14}
	\color{red} \lukusuoravaliss{-12}{-9}{$-12$}{$-9$}
	\color{red} \lukusuoravaliss{-9}{-6}{$-9$}{$-6$}
	\color{red} \lukusuoravaliss{-6}{-3}{$-6$}{$-3$}
	\color{red} \lukusuoravaliss{-3}{0}{$-3$}{$0$}

      \end{lukusuora}
      ${\color{red}-3}\cdot 4=-12$
    \end{center}
    
    $-3 \cdot (-4)$ ''miinus-kolme miinus-nelinkertaistetaan''
    
    \begin{center}
    \begin{lukusuora}{-1}{14}{14}
	\color{red} \lukusuoravaliss{12}{9}{$12$}{$9$}
	\color{red} \lukusuoravaliss{9}{6}{$9$}{$6$}
	\color{red} \lukusuoravaliss{6}{3}{$6$}{$3$}
	\color{red} \lukusuoravaliss{3}{0}{$3$}{$0$}

      \end{lukusuora}
      ${\color{red}-3}\cdot (-4)=12$
    \end{center}

\subsection*{Jakolasku}

        Kun ensin kerrotaan jollain ja sitten jaetaan samalla luvulla, päädytään takaisin samaan, mistä lähdettiin.  Jakolaskujen merkkisäännöt on sovittu niin, että tämä ominaisuus säilyy. Ne ovat siis samat kuin kertolaskujen merkkisäännöt.
    
    Esimerkiksi haluamme, että $(-12):(-3)\cdot (-3)=-12$. Nyt voimme kysyä, mitä laskun $(-12):(-3)$ tulokseksi pitäisi tulla, jotta jakolasku ja kertolasku säilyvät toisilleen käänteisinä, eli mikä luku kerrottuna luvulla $-3$ on $-12$. Kertolaskun merkkisäännöistä nähdään helposti, että tämän luvun täytyy olla $+4$ eli $4$. Niinpä on sovittu, että $(-12):(-3)=12:3=4$.

\laatikko{
Kerto- ja jakolaskun merkkisäännöt
\begin{alakohdat}
\alakohta{$a\cdot (-b)=(-a)\cdot b=-(ab)$}
\alakohta{$(-a)\cdot (-b)=a\cdot b=ab$}
\alakohta{$(-a):b=a:(-b)=-\dfrac{a}{b}$}
\alakohta{$(-a):(-b)=a: b=\dfrac{a}{b}$}
\end{alakohdat}
}

\laatikko{
Kerto- ja jakolasku kumoavat toisensa
\begin{alakohdat}
\alakohta{$a\cdot b:b=a$}
\alakohta{$a:b\cdot b=a$}
\end{alakohdat}
}




\subsection*{Lausekkeiden sieventäminen}

Matemaattisia ongelmia ratkaistaessa kannattaa usein etsiä vaihtoehtoisia tapoja jonkin laskutoimituksen, lausekkeen tai luvun ilmaisemiseksi. Tällöin usein korvataan esimerkiksi jokin laskutoimitus toisella laskutoimituksella, josta tulee sama tulos. Näin lauseke saadaan sellaiseen muotoon, jonka avulla ratkaisussa päästään eteenpäin. Kun merkitsemme monimutkaisen lausekkeen lyhyemmin, sitä kutsutaan \emph{sieventämiseksi}. Sieventäminen on ikään kuin sotkuisen kaavan siistimistä selkeämmäksi.

Matematiikassa on tapana ajatella niin, että saman luvun voi kirjoittaa monella eri tavalla. Esimerkiksi merkinnät \begin{align*}
                & 42 \\ & -(-42) \\ & 6 \cdot 7 \\ & (50-29) \cdot 2                                                                                                      
                                                                                                                 \end{align*}
tarkoittavat kaikki samaa lukua. Niinpä missä tahansa lausekkeessa voi luvun $42$ paikalle kirjoittaa merkinnän $(50-29)\cdot 2$, sillä ne tarkoittavat samaa lukua. Tähän lukuun on koottu sääntöjä, joiden avulla laskutoimituksia voi vaihtaa niin, että lopputulos ei muutu.

\laatikko{
Yhteenlaskut voi laskea missä järjestyksessä tahansa

\begin{tabular}{ll}
  $a+b=b+a$\qquad\qquad&(vaihdantalaki)\\
  \\
  $a+(b+c)=(a+b)+c=a+b+c$\qquad\qquad&(liitäntälaki)
\end{tabular} 
}

Esimerkiksi laskemalla voidaan tarkistaa, että $5+7=7+5$ ja että $(2+3)+5=2+(3+5)$.

Nämä säännöt voidaan yhdistää yleiseksi säännöksi, jonka mukaan yhteenlaskun sisällä laskujärjestystä voi vaihtaa miten tahansa.

Tämä sääntö voidaan yleistää koskemaan myös vähennyslaskua, kun muistetaan, että vähennyslasku tarkoittaa oikeastaan vastaluvun lisäämistä. $5-8$ tarkoittaa siis samaa kuin $5+(-8)$, joka voidaan nyt kirjoittaa yhteenlaskun vaihdantalain perusteella muotoon $(-8)+5$ eli $-8+5$ ilman, että laskun lopputulos muuttuu. Tästä seuraa seuraava sääntö:

\laatikko{
Pelkästään yhteen- ja vähennyslaskua sisältävässä lausekkeessa laskujärjestystä voi vaihtaa vapaasti, kun ajattelee miinusmerkin kuuluvan sitä seuraavaan lukuun ja liikkuvan sen mukana.
}

\begin{esimerkki} 
$5-8+7-2=5+(-8)+7+(-2)=(-2)+(-8)+5+7=-2-8+5+7$ 
\end{esimerkki}

Vastaavat säännöt pätevät kerto- ja jakolaskulle samoista syistä.

\laatikko{
Kertolaskut voi laskea missä järjestyksessä tahansa

\begin{tabular}{ll}
  $a\cdot b=b\cdot a$\qquad\qquad&(vaihdantalaki)\\
  \\
  $a\cdot (b\cdot c)=(a\cdot b)\cdot c=a\cdot b\cdot c$\qquad\qquad&(liitäntälaki)
\end{tabular} 
}



\begin{esimerkki}

$5 \cdot 6 = 6 \cdot 5$
 
 $2 \cdot (1+2) = 2 \cdot 1 + 2 \cdot 2$
\end{esimerkki} 

%$2 \cdot (1+2) = 2 \cdot 1 + 2 \cdot 2$

\laatikko{
Pelkästään kerto- ja jakolaskua sisältävässä lausekkeessa laskujärjestystä voi vaihtaa vapaasti, kun ajattelee jakolaskun käänteisluvulla kertomisena.
}

\begin{esimerkki}
$5:8\cdot 7:2=5\cdot\frac18\cdot 7\cdot\frac12=7\cdot \frac12\cdot\frac18\cdot 5=7:2:8\cdot 5$
\end{esimerkki} 

Lisäksi yhteen- ja kertolaskua sisältävällä lausekkeelle pätee seuraava erittäin tärkeä sääntö:

\laatikko{
$a(b+c)=ab+ac$\qquad\qquad(osittelulaki)

Vasemmalta oikealle luettaessa puhutaan sulkujen avaamisesta. Oikealta vasemmalle päin mentäessä puhutaan \emph{yhteisen tekijän ottamisesta}.
}

Aikaisemmin mainittujen laskulakien perusteella osittelulaki voidaan yhdistää koskemaan myös toisin päin olevaa kertolaskun ja yhteenlaskun yhdistelmää, useamman luvun yhteenlaskua, vähennyslaskua ja jakolaskua:

\begin{align*}
&(b+c)a = a(b+c) = ab+ac = ba+ca \text{ (Sovellettu vaihdantalakia)} \\
&a(b+c+d) = a((b+c)+d) = a(b+c)+ad = ab+ac+ad \text{ (Sovellettu liitäntälakia)} \\
&a(b-c) = a(b+(-c))=ab+a\cdot(-c)=ab-ac \text{ (Sovellettu vähennyslaskun määritelmää vastaluvun avulla)} \\
&(b+c):a = (b+c)\cdot\dfrac1a = b\cdot\dfrac1a+c\cdot\dfrac1a = b:a+c:a \text{ (Sovellettu jakolaskun ilmaisemista käänteisluvun avulla. Tämä ominaisuus esitellään myöhemmin rationaalilukujen yhteydessä.) }
\end{align*}

Esimerkiksi seuraava laskutoimitus on helppo laskea osittelulain avulla: 
     \begin{align*}
	  7777\cdot 542-7777\cdot 541 &= 7777\cdot (542-541)  \\ &= 7777\cdot 1 \\ &= 7777
     \end{align*}


Osittelulakia voidaan käyttää myös tuntemattomia lukuja sisältävien lausekkeiden muokkaamisessa. Esim. $2(x+5)=2x+10$.

% === SIISTITTÄVÄ PÄTKÄ LOPPUU ===

Lukuja on tapana luokitella seuraavasti.

Lukumääriä tai järjestystä esittävät luvut $0, 1, 2, 3, \ldots$ muodostavat
\termi{luonnollinen luku}{luonnollisten lukujen} joukon $\nn$.
Toisinaan nollaa ei pidetä luonnollisena lukuna.
Joukkoja merkitään listaamalla niiden alkiot aaltosuluissa, eli merkitään
\[\nn=\{0, 1, 2, 3, \ldots \} \]

Kun luonnollisten lukujen joukkoa täydennetään luonnollisten lukujen vastaluvuilla, saadaan \termi{kokonaisluku}{kokonaislukujen} joukko $\zz$.
\[\zz=\{\ldots, -3, -2, -1, 0, 1, 2, 3, \ldots \} \]

\termi{rationaaliluku}{Rationaaliluvulla} tarkoitetaan lukua, joka voidaan esittää kahden kokonaisluvun osamääränä. Esimerkiksi $\frac{2}{3}$ on rationaaluku samoin $0,25=\frac{1}{4}$.
Myös kaikki kokonaisluvut ovat rationaalilukuja, sillä ne voidaan esittää osamäärinä:
esimerksi $5=\frac{5}{1}$. Rationaalilukujen joukkoa merkitään symbolilla $\qq$.
\[\qq= \text{ rationaalilukujen joukko} \]    

Rationaaliluvun esitystä kokonaislukujen osamääränä
$\frac{a}{b}$ kutsutaan \termi{murtoluku}{murtoluvuksi}. Luku $a$ on murtoluvun
\termi{osoittaja}{osoittaja} ja luku $b$ on
\termi{nimittäjä}{nimittäjä}. Määritelmän mukaan kaikki rationaaliluvut
voidaan esittää murtolukuina. Osoittajan ja nimittäjän yhtaikaista kertomista kokonaisluvulla kutsutaan
\termi{laventaminen}{laventamiseksi} ja jakamista kokonaisluvulla \termi{supistaminen}{supistamiseksi}.
Koska murtolukuesitystä voidaan laventaa millä tahansa kokonaisluvulla, esitystapoja on useita; esimerkiksi $\frac{1}{2}=\frac{3}{6}$.

Luvun kuulumista johonkin joukkoon voidaan merkitä symbolilla $\in$,
esimerkiksi $-2 \in \zz$. Jos luku ei kuulu johonkin joukkoon, merkitään vastaavasti $\notin$, esimerkiksi $-2 \notin \nn$.

Edellä esitellyt joukot $\nn$, $\zz$ ja 
$\qq$ ovat sisäkkäisiä: kaikki luonnolliset luvut ovat kokonaislukuja, ja kaikki kokonaisluvut rationaalilukuja.

\subsection*{Murtolukujen laskusäännöt}

\begin{esimerkki}
        Murtolukujen yhteenlasku. Laske
        \[
        \frac{1}{2} + \frac{1}{6} + \frac{2}{6}.
        \]
        
        \textbf{Ratkaisu.}
        Lavennetaan nimittäjät samannimisiksi ja lasketaan osoittajat yhteen:
        %lisätäänkö lavennusmerkki? teknisesti hankala? käytetäänkö maailmalla? opiskelijoille kuitenkin tuttu
        \begin{align*}
            \frac{1}{2} + \frac{1}{6} + \frac{2}{6} &=\frac{3\cdot 1}{3\cdot 2} + \frac{1}{6} + \frac{2}{6}\\
            										&=\frac{3}{6} + \frac{1}{6} + \frac{2}{6}\\
           											&= \frac{3+1+2}{6}\\
           											&= \frac{6}{6} = 1.
        \end{align*}
    \end{esimerkki}

\laatikko{
    Jos murtolukujen
    nimittäjät ovat samat, voidaan murtoluvut laskea yhteen laskemalla
    osoittajat yhteen.
    \[
    \frac{a}{c} + \frac{b}{c} = \frac{a+b}{c}
    \]
}

    Murtolukuja, joiden nimittäjät ovat samat, sanotaan \emph{samannimisiksi}.
    Jos yhteenlaskettavien murtolukujen nimittäjät eivät ole samat, murtoluvut
    \emph{lavennetaan} ensin samannimisiksi ja sitten osoittajat lasketaan yhteen.
    Jos siis $\frac{a}{b}$ ja $\frac{c}{d}$ ovat murtolukuja, lasketaan

\laatikko{
    \[
    \frac{a}{b} + \frac{c}{d} = \frac{ad}{bd} + \frac{bc}{bd} = \frac{ad+bc}{bd}
    \]
    Tässä $\frac{a}{b}$ lavennetaan luvulla $d$ ja $\frac{c}{d}$ lavennetaan
    luvulla $b$, jolloin saadaan kaksi samannimistä murtolukua, joiden kummankin
    nimittäjä on yhteenlaskettavien nimittäjien tulo $bd$.
 }    

\begin{esimerkki}
        Murtolukujen kertolaskussa osoittajat ja nimittäjät kerrotaan keskenään.
      \[
        \frac{3}{4}\cdot \frac{6}{5}= \frac{3\cdot 6}{4\cdot 5}= \frac{18}{20}=\frac{9}{10}
        \]
    \end{esimerkki}
\laatikko{
    Murtolukujen $\frac{a}{b}$ ja $\frac{c}{d}$ tulo lasketaan kertomalla lukujen osoittajat ja nimittäjät keskenään:
    \[
    \frac{a}{b}\cdot \frac{c}{d} = \frac{a\cdot c}{b\cdot d} = \frac{ac}{bd}
    \]
}

%\missingfigure{tähän Sampon paperille suunnittelema havainnollistus kertolaskusäännöstä}
\includegraphics[scale=0.4]{pictures/Kuva3-1-1.pdf}
\includegraphics[scale=0.4]{pictures/Kuva3-1-2.pdf}
\includegraphics[scale=0.4]{pictures/Kuva3-1-3.pdf}
\includegraphics[scale=0.4]{pictures/Kuva3-1-4.pdf}
\includegraphics[scale=0.4]{pictures/Kuva3-1-5.pdf}

\begin{esimerkki}
	Luvun $5$ \emph{käänteisluku} on $\frac{1}{5}$, koska
	\[
	 5\cdot \frac{1}{5}=1.
	\]
	Vastaavasti luvun $-\frac{2}{3}$ käänteisluku on $-\frac{3}{2}$, koska
	\[
	 -\frac{2}{3}\cdot (-\frac{3}{2})=1.
	\]

\end{esimerkki}
\laatikko{
    Rationaaliluvun $a$ ($a\neq 0$) \emph{käänteisluku} on  $\frac{1}{a}$, sillä
    \[
    a\cdot \frac{1}{a} = 1.
    \]
    Vastaavasti rationaaliluvun $\frac{a}{b}$ ($a\neq 0$ ja $b\neq 0$) käänteisluku on $\frac{b}{a}$, sillä
    \[
    \frac{a}{b}\cdot \frac{b}{a} = 1.
    \]    

  %  Murtolukujen $p=\frac{a}{b}$ ja $q=\frac{c}{d}\neq 0$ \emph{osamäärä} $p : q$ saadaan, kun kerrotaan luku $p$ luvun $q$ käänteisluvulla,
 %   \[
 %\frac{p}{q} = p\cdot q^{-1} = \frac{a}{b}\cdot\Big(\frac{c}{d}\Big)^{-1} = \frac{a}{b}\cdot \frac{d}{c}
 %   = \frac{ad}{bc}.
 %  \]
 }

\begin{esimerkki}
Kuinka lasketaan murtolukujen jakolasku $\frac 3 5 : \frac 2 7$? Jakolaskun määritelmän mukaan osamäärän tulisi olla sellainen luku, joka kerrottuna jakajalla antaa tulokseksi jaettavan. Jos merkitään laskun $\frac 3 5 : \frac 2 7$ vastausta kirjaimella $x$, pitää siis olla $x \cdot \frac 2 7 = \frac 3 5$.  Tätä yhtälöä kutsutaan jakolaskun $\frac 3 5 : \frac 2 7$ \emph{jakoyhtälöksi.}

Kerrotaan jakoyhtälön molemmat puolet luvun $\frac 2 7$ käänteisluvulla $\frac 7 2$. Koska käänteislukujen tulo on $1$, saadaan
\[
	\text{vasen puoli} = x \cdot \underbrace{\frac 2 7 \cdot \frac 7 2}_{= 1} = x \quad \text{ja} \quad \text{oikea puoli} = \frac 3 5 \cdot \frac 7 2.
\]

Kun yhtälön molemmat puolet kerrotaan samalla luvulla, ovat myös näin saadut luvut yhtä suuria. Siis on saatu
\[
	x = \frac 3 5 \cdot \frac 7 2.
\]
Koska $x$:llä merkittiin alkuperäistä jakolaskua, on nyt onnistuttu muuttamaan jakolasku kertolaskuksi:
\[
	\frac 3 5 : \frac 2 7 = \frac 3 5 \cdot \frac 7 2 = \frac{3 \cdot 7}{5 \cdot 2} = \frac{21}{10} = 2 \frac{1}{10}.
\]
Siis jakolasku laskettiin kertomalla jaettava jakajan käänteisluvulla. Näin toimitaan yleisestikin.
 \end{esimerkki}

\laatikko{
Olkoon $b \neq 0$, $c \neq 0$ ja $d \neq 0$. Murtolukujen osamäärä $\frac a b : \frac c d$ lasketaan kertomalla jaettava jakajan käänteisluvulla:
\[
	\frac a b : \frac c d = \frac a b \cdot \frac d c = \frac{ad}{bc}.
\]
}

    \textbf{Kun vertailet kahta murtolukua, lavenna ne ensin samannimisiksi.}
    
    \begin{esimerkki}
        Salamipizza jaetaan kuuteen ja tonnikalapizza neljään yhtä suureen
        siivuun. Vesa saa kaksi siivua salamipizzaa ja yhden siivun tonnikalapizzaa.
        Minttu saa kaksi siivua tonnikalapizzaa. Kumpi saa enemmän pizzaa, jos
        molemmat pizzat ovat saman kokoisia?
        
        \begin{center}        
          \includegraphics[scale=1.0]{pictures/Kuva3-1-6-pizzat.pdf}
        \end{center}

        \textbf{Ratkaisu.}
        
        Huomataan, että $12 = 3\cdot 4 = 2\cdot 6$. Luvut kannattaa
        pizzan kokonaismäärän laskemista varten laventaa niin, että
        nimittäjänä on luku $12$.
        Vesan saama määrä pizzaa on
        \begin{align*}
           \frac{2}{6} + \frac{1}{4} &= \frac{2\cdot 2}{2\cdot 6} + \frac{3\cdot 1}{3\cdot 4} \\ 
	       							 &= \frac{4}{12}+\frac{3}{12} \\ 
	       							 &= \frac{7}{12}.
        \end{align*}
        
        Mintun saama määrä pizzaa on
        \[
            \frac{2}{4} =
            \frac{3\cdot 2}{3\cdot 4} =
            \frac{6}{12}.
        \]
        Koska $6/12 < 7/12$, Vesa saa enemmän.
    \end{esimerkki}
    
    Kaikki rationaaliluvut voidaan esittää murtolukumuodossa, mutta myös
    kokonaisluvut voidaan esittää murtolukuina asettamalla murtoluvun
    nimittäjäksi yksi. Tätä voidaan käyttää, kun lasketaan yhteen
    kokonaislukuja ja murtolukuja.
    
    \begin{esimerkki}
        Laske
        \[
            2 + \frac{1}{3}.
        \]
        
        \textbf{Ratkaisu.}
        
%        Kirjoitetaan aluksi
%        \[
%            2=\frac{2}{1}.
%        \]
		Kirjoitetaan lausekkeen kokonaisluku $2$ murtolukuna, jonka
		jälkeen voidaan murtoluvut voidaan laventaa samannimisiksi
		ja laskea yhteen:
        \begin{align*}
           2 + \frac{1}{3} &= \frac{2}{1} + \frac{1}{3}  \\ 
	       				   &= \frac{3 \cdot 2}{3 \cdot 1} + \frac{1}{3} \\ 
	       				   &= \frac{6+1}{3} \\ 
	       				   &= \frac{7}{3}.
        \end{align*}
    \end{esimerkki}

\begin{tehtavasivu}

\subsubsection*{Opi perusteet}

%     vanha sivulaatikko
%    \laatikko{
%	Laskujärjestys:
%        \begin{enumerate}
%            \item Sulut
%            \item Potenssilaskut
%            \item Kerto- ja jakolaskut vasemmalta oikealle
%            \item Yhteen- ja jakolaskut vasemmalta oikealle
%        \end{enumerate}
%    }

	\begin{tehtava}
		Kirjoita laskutoimitukseksi. (Laskuun ei tarvitse merkitä yksikköjä, eli celsiusasteita tai euroja.)

	\begin{alakohdat}
            \alakohta{Pakkasta on aluksi $-10~^{\circ}$C, ja sitten se lisääntyy kahdella pakkasasteella.}
            \alakohta{Pakkasta on aluksi $-20~^{\circ}$C, ja sitten se hellittää (vähentyy) kolme (pakkas)astetta.}
            \alakohta{Lämpötila on aluksi $17~^{\circ}$C, ja sitten se vähentyy viisi astetta.}
            \alakohta{Lämpötila on aluksi $5~^{\circ}$C, ja sitten se kasvaa kuusi astetta.}
            \alakohta{Mies on mafialle $30~000$ euroa velkaa ja menehtyy. Hänen kolme 
                poikaansa jakavat velan tasan keskenään. Kuinka paljon kukin on
                velkaa mafialle? Merkitse velkaa negatiivisella luvulla.}
        \end{alakohdat}
        
        \begin{vastaus}
            \begin{alakohdat}
                \alakohta{$-10+(-2)=-12$}
                \alakohta{$-20-(-3)=-17$}
                \alakohta{$17-5=12$}
                \alakohta{$5+6=11$}
                \alakohta{$\dfrac{-30~000}{3}=10~000$}
            \end{alakohdat}
        \end{vastaus}
    \end{tehtava}

\begin{tehtava}
Supista. \quad
a) $\frac{15}{20}$ \qquad b) $\frac{14}{21}$ \qquad c) $\frac{12}{20}$
\begin{vastaus}
a) $\frac{3}{4}$ \qquad b) $\frac{2}{3}$\qquad c) $\frac{3}{5}$
\end{vastaus}
\end{tehtava}

\begin{tehtava}
Lavenna samannimisiksi
\begin{alakohdat}
\alakohta{$\frac{2}{3}$ ja $\frac{4}{5}$} 
\alakohta{$\frac{5}{6}$ ja $\frac{7}{9}$}
\alakohta{$\frac{2}{3}$ ja $\frac{7}{2}$}
\end{alakohdat}
\begin{vastaus}
\begin{alakohdat}
\alakohta{$\frac{10}{15}$ ja $\frac{12}{15}$}
\alakohta{$\frac{15}{18}$ ja $\frac{14}{18}$}
\alakohta{$\frac{4}{6}$ ja $\frac{21}{6}$}
\end{alakohdat}
\end{vastaus}
\end{tehtava}

\begin{tehtava}
Laske.
	\begin{alakohdat}
		\alakohta{$\frac{3}{11}+\frac{5}{11}$}
		\alakohta{$\frac{4}{5}-\frac{1}{5}$}
		\alakohta{$\frac{2}{3}+\frac{1}{6}$}
		\alakohta{$ \frac{11}{12}-\frac{5}{6}$}
	\end{alakohdat}
	\begin{vastaus}
		\begin{alakohdat}
			\alakohta{$\frac{8}{11}$}
			\alakohta{$\frac{3}{5}$}
			\alakohta{$\frac{5}{6}$}
			\alakohta{$\frac{1}{12}$}
		\end{alakohdat}
	\end{vastaus}
\end{tehtava}

\begin{tehtava}
Muuta sekamurtoluvuksi 
%täsmällisemmin sekamurtolukumuotoon, mutta pienellä piirillä 
% ajateltiin, että tämä epätäsmällinen muotoilu parempi
\begin{alakohdatrivi}
\alakohta{$\frac{15}{2}$} 
\alakohta{$\frac{9}{4}$}
\alakohta{$\frac{23}{7}$}
\end{alakohdatrivi}
\begin{vastaus}
\begin{alakohdat}
\alakohta{$7\frac{1}{2}$}
\alakohta{$2\frac{1}{4}$}
\alakohta{$3\frac{2}{7}$}
\end{alakohdat}
\end{vastaus}
\end{tehtava}

\begin{tehtava}
Muunna murtoluvuksi 
\begin{alakohdat}
\alakohta{$3\frac{2}{5}$}
\alakohta{$4\frac{1}{3}$}
\alakohta{$2\frac{6}{7}$}
\end{alakohdat}
\begin{vastaus}
\begin{alakohdat}
\alakohta{$\frac{17}{5}$}
\alakohta{$\frac{13}{12}$}
\alakohta{$\frac{20}{7}$}
\end{alakohdat}
\end{vastaus}
\end{tehtava}

\begin{tehtava}
Laske.
	\begin{alakohdat}
		\alakohta{$1\frac{2}{9}+\frac{5}{9}$}
		\alakohta{$\frac{1}{3}+2\frac{1}{3}$}
		\alakohta{$2+\frac{5}{4}$}
		\alakohta{$ \frac{3}{2}-\frac{5}{6}$}
	\end{alakohdat}
	\begin{vastaus}
		\begin{alakohdat}
			\alakohta{$\frac{16}{9}$}
			\alakohta{$\frac{8}{3}$}
			\alakohta{$\frac{13}{4}$}
			\alakohta{$\frac{2}{3}$}
		\end{alakohdat}
	\end{vastaus}
\end{tehtava}

\begin{tehtava}
Laske.
	\begin{alakohdat}
		\alakohta{$\frac{2}{3}\cdot \frac{4}{5}$}
		\alakohta{$\frac{3}{5} \cdot \frac{5}{4}$}
		\alakohta{$3\cdot \frac{2}{7}$}
		\alakohta{$\frac{4}{5}\cdot 5$}
	\end{alakohdat}
	\begin{vastaus}
		\begin{alakohdat}
			\alakohta{$\frac{8}{15}$}
			\alakohta{$\frac{15}{20}=\frac{3}{4}$}
			\alakohta{$\frac{6}{7}$}
			\alakohta{$4$ (koska $\frac{20}{5}=4$)}
		\end{alakohdat}
	\end{vastaus}
\end{tehtava}


\begin{tehtava}
a) $\frac{2}{3} : \frac{7}{11}$ \qquad b) $\frac{4}{3}:(\frac{-13}{4})$ \qquad c) $\frac{7}{8}:4$
\begin{vastaus}
a) $1\frac{1}{21}$ \qquad b) $-\frac{16}{39}$ \qquad c) $\frac{7}{32}$
\end{vastaus}
\end{tehtava}

\subsubsection*{Hallitse kokonaisuus}

\begin{tehtava}
Laske murtolukujen $\frac{5}{6}$ ja $-\frac{2}{15}$ \\ a) summa \qquad b) erotus \qquad c) tulo \qquad d) osamäärä.
\begin{vastaus}
a) $\frac{7}{10}$ \qquad b) $\frac{29}{30}$ \qquad c) $-\frac{1}{9}$ \qquad d) $-6\frac{1}{4}$
\end{vastaus}
\end{tehtava}

\begin{tehtava}
a) $\frac{5}{8}\cdot(\frac{3}{5}+\frac{2}{5})$ \qquad b) $\frac{1}{3}+\frac{1}{4}\cdot\frac{6}{5}$
\begin{vastaus}
a) $\frac{5}{8}$ \qquad b) $\frac{19}{30}$
\end{vastaus}
\end{tehtava}

\begin{tehtava}
a) $\dfrac{\frac{1}{2}:\frac{3}{2}}{\frac{3}{2}+\frac{1}{3}}$ \qquad b) $\dfrac{\frac{2}{3}+\frac{3}{4}}{\frac{5}{6}-\frac{7}{12}}$.
\begin{vastaus}
a) $\frac{2}{11}$ \qquad b) $5\frac{2}{3}$
\end{vastaus}
\end{tehtava}

\begin{tehtava} 
        Laatikossa on palloja, joista kolmasosa on mustia, neljäsosa
        valkoisia ja viidesosa harmaita. Loput palloista ovat 		 	punaisia.
        Kuinka suuri osuus palloista on punaisia?
        
        \begin{vastaus}
            $1-(\frac{1}{3}+\frac{1}{4}+\frac{1}{5})
            = \frac{60}{60}-\frac{20}{60}-\frac{15}{60}-\frac{12}{60}
            = \frac{60}{60}-\frac{47}{60}
            = \frac{13}{60}$
        \end{vastaus}
    \end{tehtava}

\begin{tehtava}
Laske lausekkeen $\frac{x}{2-3x}$ arvo, kun $x$ on \\ a) 4 \qquad b) $-\frac{1}{2}$ \qquad c) $\frac{7}{10}$.
\begin{vastaus}
a) $-\frac{2}{5}$ \qquad b) $-\frac{1}{7}$ \qquad c) $-7$
\end{vastaus}
\end{tehtava}


\subsubsection*{Lisää tehtäviä}

\begin{tehtava}
Laske lausekkeen $\frac{x+y}{2x-y}$ arvo, kun \\ a) $x=\frac{1}{2}$ ja $y= \frac{1}{4}$ \qquad b) $x=\frac{1}{4}$ ja $y= -\frac{3}{8}$.
\begin{vastaus}
a) $1$ \qquad b) $-\frac{1}{7}$
\end{vastaus}
\end{tehtava}


 %   Yksi prosentti tarkoittaa yhtä sadasosaa: $1~\% = \frac{1}{100}$
 
 %TODO pitäisikö selittää leipätekstissä ja antaa joku esimerkki?
    
        \begin{tehtava}
            \begin{alakohdat}
        	\alakohta{$\frac{3}{5} + \frac{1}{5}$}
        	\alakohta{$\frac{5}{7} + \frac{4}{7}$}
        	\alakohta{$2 + \frac{2}{3}$}
        	\alakohta{$3 + \frac{3}{5} + \frac{2}{5}$}
            \end{alakohdat}
            \begin{vastaus}
        		\begin{alakohdat}
        			\alakohta{$\frac{4}{5}$}
        			\alakohta{$\frac{9}{7} = 1 \frac{2}{7}$}
        			\alakohta{$2 \frac{2}{3} = \frac{8}{3}$}
        			\alakohta{$4$}
        		\end{alakohdat}
            \end{vastaus}
        \end{tehtava}
        
        \begin{tehtava}
        
        \begin{alakohdat}
        	\alakohta{$\frac{6}{2} + \frac{3}{5}$}
        	\alakohta{$\frac{7}{8} - \frac{1}{4}$}
        	\alakohta{$2 \frac{1}{3} + \frac{4}{6}$}
        	\alakohta{$4 \frac{7}{2} - 6 \frac{5}{4}$}
        \end{alakohdat}
            \begin{vastaus}		
        		\begin{alakohdat}
        			\alakohta{$\frac{18}{5}$}
        			\alakohta{$\frac{5}{8}$}
        			\alakohta{$3$}
        			\alakohta{$-\frac{41}{6}$ }
        		\end{alakohdat}
            \end{vastaus}
        \end{tehtava}
        
        \begin{tehtava}
        
        \begin{alakohdat}
        	\alakohta{$2 \cdot \frac{2}{5}$}
        	\alakohta{$2 \cdot \frac{2}{3}$}
        	\alakohta{$\frac{5}{4} \cdot 2 \cdot 3$}
        	\alakohta{$\frac{\frac{3}{7}}{4}$ }
        \end{alakohdat}
            \begin{vastaus}
        		\begin{alakohdat}
        			\alakohta{$\frac{4}{5}$}
        			\alakohta{$\frac{4}{3} = 1 \frac{1}{3}$}
        			\alakohta{$\frac{15}{2} = 7 \frac{1}{2}$}
        			\alakohta{$\frac{3}{28}$}
        		\end{alakohdat}
            \end{vastaus}
        \end{tehtava}
        
        \begin{tehtava}
        
        \begin{alakohdat}
        	\alakohta{$\frac{1}{3} \cdot \frac{6}{5}$}
        	\alakohta{$\frac{5}{4} \cdot (-\frac{2}{3})$ }
        	\alakohta{$\frac{2}{5} (2 - \frac{3}{4})$}
        	\alakohta{$(\frac{5}{6} - \frac{1}{3})(\frac{7}{4} - \frac{3}{2})$}
        \end{alakohdat}
            \begin{vastaus}		
        		\begin{alakohdat}
        			\alakohta{$\frac{2}{5}$}
        			\alakohta{$-\frac{5}{6}$}
        			\alakohta{$\frac{1}{2}$}
        			\alakohta{$\frac{1}{8}$ }
        		\end{alakohdat}
            \end{vastaus}
        \end{tehtava}
        
        \begin{tehtava}
        
        \begin{alakohdat}
        	\alakohta{$\displaystyle \frac{\frac{3}{7} + \frac{5}{4}}{3}$}
        	\alakohta{$\displaystyle \frac{\frac{10}{8}}{\frac{5}{2}}$}
        	\alakohta{$\displaystyle \frac{\frac{1}{3} - \frac{5}{10}}{\frac{3}{4} + \frac{1}{2}}$}
        	\alakohta{$\displaystyle 3\frac{\frac{4}{2} + \frac{10}{4}}{\frac{3}{2} - \frac{2}{3}}$}
        \end{alakohdat}
            \begin{vastaus}		
        		\begin{alakohdat}
        			\alakohta{$\frac{47}{28}$}
        			\alakohta{$\frac{1}{2}$}
        			\alakohta{$-\frac{1}{3}$}
        			\alakohta{$\frac{54}{5}$}
        		\end{alakohdat}
            \end{vastaus}
        \end{tehtava}
    
    \begin{tehtava} %syvteht
        Pontus, Viljami, Jarkko-Kaaleppi, Simo ja Milla leipoivat lanttuvompattipiirakkaa.
        Pontus kuitenkin söi piirakasta kolmanneksen ennen muita, ja loput piirakasta
        jaettiin muiden kanssa tasan. Kuinka suuren osan muut saivat?
        
        \begin{vastaus}
            Muut saivat piirakasta kuudesosan.
        \end{vastaus}
    \end{tehtava}
    
\begin{tehtava} %perusteht
    Huvipuiston sisäänpääsylippu maksaa 20 euroa, ja lapset pääsevät sisään puoleen hintaan.
	\begin{alakohdat}
		\alakohta{Kuinka paljon kolmen lapsen yksinhuoltajaperheelle maksaa päästä sisään?}
		\alakohta{Kuinka paljon sisäänpääsy maksaa perheelle avajaispäivänä,		kun silloin sisään pääsee 25~\% halvemmalla?}
    \end{alakohdat}
    \begin{vastaus}
		\begin{alakohdat}
			\alakohta{50 euroa }
			\alakohta{50 euroa }
			\alakohta{37,50 euroa}
		\end{alakohdat} 
    \end{vastaus}
\end{tehtava}  
  
\begin{tehtava}
    Laske 
    \[ \frac{10}{9}\cdot \frac{9}{8}\cdot \frac{8}{7}\cdot \frac{7}{6}\cdot \frac{6}{5}
    \cdot \frac{5}{4}\cdot \frac{4}{3}\cdot \frac{3}{2}. \]
    \begin{vastaus}
		$\frac{10}{2}=5$.
    \end{vastaus}        
\end{tehtava}
    
\begin{tehtava}
	Eräässä kaupassa on käynnissä loppuunmyynti, ja kaikki tuotteet
    myydään puoleen hintaan. Lisäksi kanta-asiakkaat saavat aina
    viidenneksen alennusta tuotteiden senhetkisestä hinnasta.
	Paljonko kanta-asiakas maksaa nyt tuotteesta, joka normaalisti
    maksaisi 40 euroa?
    \begin{vastaus}
		$40\cdot \frac{1}{2} \cdot \frac{4}{5}=40\cdot \frac{4}{10}= 16$. 
	\end{vastaus}
\end{tehtava}
    
\begin{tehtava}
	Kokonaisesta kakusta syödään maanantaina iltapäivällä puolet, ja jäljelle
	jääneestä palasta syödään tiistaina iltapäivällä taas puolet.
	Jos kakun jakamista ja syömistä jatketaan samalla tavalla koko viikko,
	kuinka suuri osa alkuperäisestä kakusta on
	jäljellä seuraavana maanantaiaamuna?
	\begin{vastaus}
		Toisena päivänä aamulla kakkua on jäljellä puolet, kolmantena
		päivänä aamulla
		$1-\left(\frac{1}{2} + \frac{1}{4}\right) = \frac{1}{4}$, 
		neljäntenä päivänä
		$1-\left(\frac{1}{2} + \frac{1}{4} + \frac{1}{8}\right)
		= \frac{1}{8}$, jne.
		Siis seitsemän päivän jälkeen kakkua on jäljellä
		$1-\left(\frac{1}{2} + \frac{1}{4} + \frac{1}{8} +
		\frac{1}{16} + \frac{1}{32} + \frac{1}{64} + \frac{1}{128}\right)
		= \frac{1}{128}$.  
	\end{vastaus}
\end{tehtava}

\begin{tehtava}
	Ratkaise lausekkeen $\frac{1}{n}-\frac{1}{m}$ arvo, kun tiedetään, että $n = \frac{1}{9}$ ja $m=n+1$.
	\begin{vastaus}
		$\frac{81}{10}$
	\end{vastaus}
\end{tehtava}

\begin{tehtava}
	Ratkaise lausekkeen $\frac{1}{n}-\frac{1}{2n}+\frac{1}{3n}$ arvo, kun tiedetään, että $n = 10$.
	\begin{vastaus}
		$\frac{1}{12}$
	\end{vastaus}
\end{tehtava}

\begin{tehtava}
	\begin{alakohdatrivi}
		\alakohta{$\frac{4}{9} : \frac{1}{5}$}
		\alakohta{$\frac{2}{7} : \frac{5}{9}$}
		\alakohta{$\frac{2}{3}:\frac{4}{3}$}
	\end{alakohdatrivi}
	\begin{vastaus}
		\begin{alakohdatrivi}
			\alakohta{$\frac{20}{9}$}
			\alakohta{$\frac{18}{35}$}
			\alakohta{$\frac{1}{2}$}
		\end{alakohdatrivi}
	\end{vastaus}
\end{tehtava}

\begin{tehtava}
	Vanhalla matemaatikolla oli kolme lasta. Eräänä päivänä hän antoi lapsilleen laatikon
	ja kertoi, että sen sisällä oli erilaisia palkkioita, joita hän oli saanut ratkottuaan
	pulmatilanteita ympäri maailmaa. Hän kertoi antavansa vanhimmalle lapselleen puolet 
	saamistaan arvoesineistä, keskimmäiselle neljäsosan ja nuorimmalle kuudesosan. Avattuaan 
	laatikon lapset näkivät 11 erilaista esinettä, ja matemaatikon suureksi iloksi osasivat 
	jakaa esineet oikein. Kuinka monta esinettä kukin sai?
	\begin{vastaus}
		Vanhin sai 6 esinettä, keskimmäinen 3 esinettä ja nuorin 2 esinettä. Vanha
		matemaatikko oli pitänyt yhden esineen itsellään, sillä $\frac{1}{2} + \frac{1}{4}
		+ \frac{1}{6} = \frac{11}{12}$.
	\end{vastaus}
\end{tehtava}

\begin{tehtava}
	$\star$ Fibonaccin luvut 0, 1, 1, 2, 3, 5, 8, 13, 21, $\ldots$ määritellään seuraavasti: Kaksi ensimmäistä
	Fibonaccin lukua ovat 0 ja 1, ja siitä seuraavat saadaan kahden
	edellisen summana: \[ 0+1=1, \quad 1+1=2, \quad 1+2 = 3, \quad 2+3=5 \] 
	ja niin edelleen. 
	Tutki, miten Fibonaccin luvut liittyvät lukuihin
	\[ \frac{1}{1+1}, \quad \frac{1}{1+\frac{1}{1+1}}, \quad
	\frac{1}{1+\frac{1}{1+\frac{1}{1+1}}}, \quad 
	\frac{1}{1+\frac{1}{1+\frac{1}{1+\frac{1}{1+1}}}}, \quad \ldots\]
	\begin{vastaus}
		Luvut ovat sievennettynä peräkkäisten Fibonaccin
		lukujen osamääriä:
		\[\frac{1}{2}, \ \frac{2}{3}, \ \frac{3}{5}, \frac{5}{8} \ldots  \]
	\end{vastaus}
\end{tehtava}

\begin{tehtava}
	\begin{alakohdat}
		\alakohta{Jos $n$ on positiivinen kokonaisluku, laske $n$:n ja $(n+1)$:n 		käänteislukujen erotus.}
		\alakohta{Laske summa \[ \frac{1}{1\cdot 2}+\frac{1}{2 \cdot 3}+ \ldots + \frac{1}{(n-1)n} \]}
	\end{alakohdat}
	\begin{vastaus}
		\begin{alakohdat}
			\alakohta{$\frac{1}{n(n+1)}$}
			\alakohta{$1-\frac{1}{n}$}
		\end{alakohdat}
	\end{vastaus}
\end{tehtava}

\end{tehtavasivu}
