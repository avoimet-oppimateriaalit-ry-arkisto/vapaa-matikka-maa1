Sana \termi{prosentti}{prosentti} tulee latinan kielen ilmaisusta \termi{pro centum}{pro centum},
joka tarkoittaa kirjaimellisesti sataa kohden. 
Prosentit ovat siis sadasosia jostain.
Prosentteja käytetään ilmaisemaan suhteellista osuutta.
Prosentin merkki on \%.

\laatikko{$1~\textnormal{prosentti} \; = 1~\% = \frac{1}{100} = 0,01$}

\begin{esimerkki}
Eräs kauppa myy tuotteensa $5$~prosentin alennuksella. Tämä tarkoittaa, että jokaisesta hinnasta vähennetään $\frac{5}{100}$ hinnasta. Jos tuotteen alkuperäinen hinta on esimerkiksi $100$~euroa, on alennettu hinta $95$~euroa. Jos tuotteen alkuperäinen hinta on $14,50$~euroa, on alennuksen määrä
\[
	\frac{5}{100} \cdot 14,50~\text{euroa} = 0,725~\text{euroa}.
\]
Alennettu hinta on tällöin $(14,50 - 0,725)~\text{euroa} \approx 13,78~\text{euroa}$.

Samaan tulokseen pääsee vielä kätevämmin, kun ajattelee alennuksen määrän sijaan sitä, kuinka suuri osa hinnasta jää jäljelle. Jos alennus on $5~\%$, jää hinnasta jäljelle $95~\%$. Jos alkuperäinen hinta on $14,50$~euroa, on alennettu hinta 
\[
	\frac{95}{100} \cdot 14,50~\text{euroa} \approx 13,78~\text{euroa}.
\]
\end{esimerkki}

\begin{esimerkki}
    Prosenttiluvut voidaan esittää myös muilla tavoin.
    \begin{alakohdat}
        \alakohta{$6~\% = \frac{6}{100} = 0,06$}
        \alakohta{$48,2~\% = \frac{48,2}{100} = 0,482$}
        \alakohta{$140~\% = \frac{140}{100} = 1,40$ }
    \end{alakohdat}
\end{esimerkki}

% PERUSARVO
\laatikko{Lukua, josta suhde lasketaan, kutsutaan \termi{perusarvo}{perusarvoksi}.}

\begin{esimerkki}
    Jos sadan euron hintaisen tuotteen hintaa on alennettu $25$ prosenttia,
    niin alennettu hinta on $75$ euroa. Jos sen sijaan alkuperäinen
    hinta nousee $15$ prosenttia, niin tuotteen uusi hinta on $115$ euroa.
    Perusarvo on molemmissa tapauksissa $100$ euroa.
    
    \begin{center}
        \includegraphics[scale=.25]{pictures/Kuva13-1-100.pdf}
        \includegraphics[scale=.25]{pictures/Kuva13-2-75.pdf}
        \includegraphics[scale=.25]{pictures/Kuva13-3-115.pdf}
    \end{center}
\end{esimerkki}

% MUUTOSPROSENTTI
\laatikko{
    Prosentteja käytetään usein ilmaisemaan suureiden muutoksia, esimerkiksi luku $a$ kasvaa luvuksi $b$.
    \termi{muutosprosentti}{Muutosprosenttia} laskettaessa muutoksen suuruutta verrataan alkuperäiseen lukuun.
    Perusarvona on siis alkuperäinen arvo, johon nähden muutos on tapahtunut. Muutosta merkitään yleensä symbolilla
    $\Delta$.
    
    \termi{absoluuttinen muutos}{Absoluuttinen muutos} luvusta $a$ lukuun $b$ on $b-a$.
    Suhteellinen muutos saadaan suhteuttamalla muutos absoluuttinen muutos alkuperäiseen lukuun $a$ eli laskemalla
    
    \[ \Delta_{\text{suhteellinen}} = \frac{\Delta_{\text{absoluuttinen}}}{a} = \frac{b-a}{a} \]
    
    Muutosprosentti saadaan suhteellisesta muutoksesta muuttamalla se prosenttiluvuksi:
    
    \[ \Delta_{\text{prosentti}} = \Delta_{\text{suhteellinen}} \cdot 100~\% = \frac{b-a}{a} \cdot 100~\% \]
    
    Jos muutosprosenttia lasketaan vastakkaiseen suuntaan, saadaan eri muutosprosentti.
}

\begin{esimerkki}
    Vesan paino on tammikuussa $68$~kg ja kesäkuussa $64$~kg. Kuinka monta prosenttia Vesa on laihtunut?
    
    \textbf{Ratkaisu.}
    
    Halutaan tietää Vesan painon muutos prosentteina tammikuusta kesäkuuhun.
    
    \[
        \Delta_{\text{prosentti}}
        = \frac{b-a}{a} \cdot 100~\%
        = \frac{64-68}{68} \cdot 100~\%
        = \frac{-4}{68} \cdot 100~\%
        \approx -0,06 \cdot 100~\%
        = -6~\% 
    \]
    
    Vesan paino on muuttunut kuudella prosentilla negatiiviseen suuntaan,
    eli Vesa on laihtunut kuusi prosenttia.
    
    \textbf{Vastaus.}
    
    Vesa on laihtunut $6~\%$.
\end{esimerkki}

% VERTAILUPROSENTTI
\laatikko{
    Muutosprosentille läheinen käsite on \termi{vertailuprosentti}{vertailuprosentti}.
    Vertailuprosentilla tarkoitetaan sitä, kuinka paljon jokin on jostakin.
    
    Vertailuprosentilla vastataan siis kysymykseen ''kuinka monta prosenttia luku $a$ on luvusta $b$?''
    Vertailuprosentti on tässä tapauksessa $\frac{a}{b} \cdot 100~\%$.
    
    Vertailuprosentista saamme myös uuden esitystavan muutosprosentille, sillä samaan suuntaan laskettujen
    vertailu- ja muutosprosenttien erotus on aina $100~\%$.
    
    Jos vertailuprosenttia lasketaan vastakkaiseen suuntaan, saadaan eri vertailuprosentti.
}

\begin{esimerkki}
    Vesa ansaitsee kuukaudessa ${3~200}$ euroa ja Antero ${2~300}$ euroa.
    Kuinka monta prosenttia Anteron tulot ovat Vesan tuloista? 
    
    \textbf{Ratkaisu.}
    
    Lasketaan vertailuprosentti. Perusarvo on tehtävänannon mukaisesti
    Vesan palkka eli ${3~200}$ euroa.
    
    \[
        \frac{2300}{3200} \cdot 100~\%
        \approx 0,72\cdot 100~\%
        = 72~\%.
    \]
    
    \textbf{Vastaus.}
    
    Anteron tulot ovat $72~\%$ Vesan tuloista.
\end{esimerkki}

Joissain tilanteissa perusarvo on tuntematon. Se ratkeaa usein mukavimmin yhtälöllä.

\begin{esimerkki}
Tuotteen hintaa korotettiin 15~\%, jolloin hinnaksi muodostui 175~euroa. Kuinka suuri oli alkuperäinen korottamaton hinta?

\textbf{Ratkaisu.} 
Olkoon alkuperäinen hinta $x$~euroa. Koska hintaa on korotettu 15~\%, on uusi hinta 115~\% alkuperäisestä. Käytetään vertailuprosentin laskukaavaa yhtälönä ja ratkaistaan se:
\begin{align*}
	\frac{175}{x} \cdot 100~\%	&= 115~\%	&	&|\, \text{Jaetaan $100~\%$:lla.} \\
	\frac{175}{x}	&= 1,15	&	&|\, \text{Kerrotaan $x$:llä.} \\
	175	&= 1,15x	&	&|\, \text{Jaetaan $1,15$:llä.} \\
	x	&= \frac{175}{1,15} \approx 152,17.	&	& \\
\end{align*}
    \textbf{Ratkaisu.}
    Tuotteen korottamaton hinta oli $152,17$~euroa.
\end{esimerkki}

Käytännössä tuntemattoman perusarvon selvittämisessä pääsee vielä vähemmällä, jos ajattelee, että $15$~prosentin korotuksen takia $x$ pitää kertoa $1,15$:llä, jotta saa korotetun hinnan. Näin pääsee suoraan yhtälöön $1,15x = 175$.


% PROSENTTIYKSIKKÖ
\laatikko{
    \termi{prosenttiyksikkö}{Prosenttiyksikkö} mittaa prosenttiosuuksien välisiä eroja.
    Esimerkiksi $4~\%$ on $2$ prosenttiyksikköä suurempi kuin $2~\%$, mutta $100~\%$
    suurempi kuin $2~\%$. Jos prosenttiluku muuttuu, muutos voidaan ilmaista joko
    prosentteina tai prosenttiyksikköinä.
    
    Prosentin ja prosenttiyksikön merkitysero on keskeinen esimerkiksi
    talousuutisten tulkinnassa.
}

\begin{esimerkki}
    Tuotteen markkinaosuus on vuoden tammikuussa $10$~\% ja kesäkuussa $15$~\%.
    \begin{alakohdat}
        \alakohta{Kuinka monta prosenttia tuotteen markkinaosuus on noussut?}
        \alakohta{Kuinka monta prosenttiyksikköä tuotteen markkinaosuus on noussut?}
    \end{alakohdat}
    
    \textbf{Ratkaisu.}
    
    \begin{alakohdat}
        \alakohta{Tuotteen markkinaosuus on noussut}
            \[
                \frac{15-10}{10} \cdot 100~\%= \frac{5}{10}\cdot 100~\% = 50~\%.
            \]
        
        \alakohta{Tuotteen markkinaosuus on noussut $15-10=5$ prosenttiyksikköä.}
    \end{alakohdat}
    
    \textbf{Vastaus.}
    
    \begin{alakohdat}
        \alakohta{$50$ prosenttia}
        \alakohta{$5$ prosenttiyksikköä.}
    \end{alakohdat}
\end{esimerkki}
