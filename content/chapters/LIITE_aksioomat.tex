Reaaliluvut ovat kunta, eräs algebrallinen rakenne. Myös esimerkiksi rationaaliluvut ja seuraavassa liitteessä esiteltävät kompleksiluvut muodostavat kunnan. Sen sijaan luonnolliset luvut ja kokonaisluvut eivät ole kuntia.

Reaalilukujen aksiomaattinen määritelmä muodostuu kolmesta osasta:

\begin{flalign*}
&\textbf{Kunta-aksioomat, kuntana reaaliluvut} &\\
&\textbf{K1.} \, \forall x, y \in \rr: & &x+(y+z) = (x+y)+z & &| \, \text{summan liitäntälaki} &\\
&\textbf{K2.} \, \exists 0 \in \rr: & &x+0 = x & &| \, \text{summan neutraalialkio} &\\
&\textbf{K3.} \, \forall x \in \rr & &\exists (-x) \in \rr: \quad x+(-x)=0 & &| \, \text{vasta-alkio} &\\
&\textbf{K4.} \, \forall x, y \in \rr: & &x+y = y+x & &| \, \text{summan vaihdantalaki} &\\
&\textbf{K5.} \, \forall x, y, z \in \rr: & &x \cdot (y+z) = x \cdot y + x \cdot z & &| \, \text{osittelulaki} &\\
&\textbf{K6.} \, \forall x, y, z \in \rr: & &x \cdot (y \cdot z) = (x \cdot y) \cdot z & &| \, \text{tulon liitäntälaki} &\\
&\textbf{K7.} \, \exists 1 \in \rr, 1 \neq 0: & &1 \cdot x = x & &| \, \text{tulon neutraalialkio} &\\
&\textbf{K8.} \, \forall x \in \rr \setminus \{0\} & &\exists x^{-1} \in \rr \setminus \{0\}: \quad x \cdot x^{-1}=1 & &| \, \text{tulon käänteisalkio} &\\
&\textbf{K9.} \, \forall x, y \in \rr: & &x \cdot y = y \cdot x & &| \, \text{tulon vaihdantalaki} \\
&\textbf{Järjestysaksioomat} &\\
&\textbf{J1.} \, \forall x, y \in \rr: & &\text{täsmälleen yksi seuraavista:} & \\
& & &(x > y), \, (x = y), \, (x < y) & &\\
&\textbf{J2.} \, \forall x, y, z \in \rr: & &(x < y) \land (y < z) \Rightarrow (x < z) & &\\
&\textbf{J3.} \, \forall x, y, z \in \rr: & &(x < y) \Leftrightarrow (x + z < y + z) & &\\
&\textbf{J4.} \, \forall x, y \in ]0,\infty[: & &x \cdot y \in ]0,\infty[ & &\\
&\textbf{Täydellisyysaksiooma} &\\
\shortintertext{\textbf{T1.} Jokaisella ylhäältä rajoitetulla epätyhjällä reaalilukujen osajoukolla on pienin yläraja.} 
\end{flalign*}

\begin{tehtava}
	Todista aksioomista lähtien:
	\begin{alakohdat}
		\alakohta{$\forall x \in \rr: 0 \cdot x = 0$}
		\alakohta{$\forall x \in \rr: -1 \cdot x = -x$}
		\alakohta{$\forall x, y \in ]-\infty,0[: x \cdot y \in ]0,\infty[$}
		\alakohta{$1 > 0$}
		% lisää
	\end{alakohdat}
	% ratkaisut
\end{tehtava}
