\begin{tehtavasivu}

\subsubsection*{Opi perusteet}

%     vanha sivulaatikko
%    \laatikko{
%	Laskujärjestys:
%        \begin{enumerate}
%            \item Sulut
%            \item Potenssilaskut
%            \item Kerto- ja jakolaskut vasemmalta oikealle
%            \item Yhteen- ja jakolaskut vasemmalta oikealle
%        \end{enumerate}
%    }

\begin{tehtava}
    Kirjoita laskutoimitukseksi. (Laskuun ei tarvitse merkitä yksikköjä, eli celsiusasteita %tai euroja.)
    )

    \begin{alakohdat}
        \alakohta{Pakkasta on aluksi $-10~^{\circ}$C, ja sitten se lisääntyy kahdella pakkasasteella.}
        \alakohta{Pakkasta on aluksi $-20~^{\circ}$C, ja sitten se hellittää (vähentyy) kolme (pakkas)astetta.}
        \alakohta{Lämpötila on aluksi $17~^{\circ}$C, ja sitten se vähentyy viisi astetta.}
        \alakohta{Lämpötila on aluksi $5~^{\circ}$C, ja sitten se kasvaa kuusi astetta.}
%        \alakohta{Mies on mafialle $30~000$ euroa velkaa ja menehtyy. Hänen kolme 
%            poikaansa jakavat velan tasan keskenään. Kuinka paljon kukin on
%            velkaa mafialle? Merkitse velkaa negatiivisella luvulla.}
    \end{alakohdat}

    \begin{vastaus}
        \begin{alakohdat}
            \alakohta{$-10+(-2)=-12$}
            \alakohta{$-20-(-3)=-17$}
            \alakohta{$17-5=12$}
            \alakohta{$5+6=11$}
           % \alakohta{$\dfrac{-30~000}{3}=10~000$}
        \end{alakohdat}
    \end{vastaus}
\end{tehtava}

\begin{tehtava}
   	Lukusuoralla on saavuttu pisteeseen $-2$. Mihin pisteeseen päädytään, kun liikutaan
    \begin{alakohdat}
        \alakohta{3 askelta negatiiviseen suuntaan?}
        \alakohta{6 askelta positiiviseen ja sitten 2 negatiiviisen suuntaan?}
        \alakohta{2 askelta positiiviseen ja sitten 6 negatiiviseen suuntaan?}
    \end{alakohdat}
Kirjoita laskut näkyviin.
    \begin{vastaus}
    	Pisteeseen 
        \begin{alakohdat}
            \alakohta{$-2+(-3)=-5$}
            \alakohta{$-2+6+(-2)=2$}
            \alakohta{$-2+2+(-6)=-6$}
        \end{alakohdat}
    \end{vastaus}
\end{tehtava}

\begin{tehtava}
Laske

    \begin{alakohdat}
        \alakohta{$3+(-8)$}
        \alakohta{$5+(+7)$}
        \alakohta{$-8-(-5)$}
        \alakohta{$+(+8)-(+5)$}
        \alakohta{$-(-8)-(+8)$}
    \end{alakohdat}
\begin{vastaus}
    \begin{alakohdat}
        \alakohta{$-5$}
        \alakohta{$12$}
        \alakohta{$-3$}
        \alakohta{$3$}
        \alakohta{$0$}
    \end{alakohdat}
\end{vastaus}
\end{tehtava}

\begin{tehtava}
Laske

    \begin{alakohdat}
        \alakohta{$3\cdot (-6)$}
        \alakohta{$-7\cdot (+7)$}
        \alakohta{$-8\cdot (-5)$}
        \alakohta{$-(-8)\cdot (-8)$}
        \alakohta{$-(-8)\cdot (-(+5))$}
        \alakohta{$-(-8)\cdot (-5)\cdot (-1)\cdot (-2)$}
    \end{alakohdat}
\begin{vastaus}
    \begin{alakohdat}
        \alakohta{$-36$}
        \alakohta{$-49$}
        \alakohta{$40$}
        \alakohta{$-64$}
        \alakohta{$-40$}
        \alakohta{$-80$}
    \end{alakohdat}
\end{vastaus}
\end{tehtava}

\begin{tehtava}
Kirjoita summana seuraavat erotukset:
	\begin{alakohdat}
		\alakohta{$130-(-5)$}
		\alakohta{$-4-7$}
		\alakohta{$2a-b$, kun $a$ ja $b$ ovat kokonaislukuja}
	\end{alakohdat}
\begin{vastaus}
	\begin{alakohdat}
		\alakohta{$130+5$}
		\alakohta{$-4+(-7)$}
		\alakohta{$2a+(-b)$}
	\end{alakohdat}
\end{vastaus}
\end{tehtava}

\begin{tehtava}
Kirjoita erotuksena seuraavat summat:
	\begin{alakohdat}
		\alakohta{$31+4$}
		\alakohta{$-15+(-92)$}
		\alakohta{$-a+2b$, kun $a$ ja $b$ ovat kokonaislukuja}
	\end{alakohdat}
\begin{vastaus}
	\begin{alakohdat}
		\alakohta{$31-(-4)$}
		\alakohta{$-15-92$}
		\alakohta{$-a-(-2b)$}
	\end{alakohdat}
\end{vastaus}
\end{tehtava}

\begin{tehtava}
Kirjoita seuraavat laskutoimitukset uudestaan käyttäen annettua laskulakia. 
%    Etsi toinen laskutoimitus, josta tulee sama tulos soveltamalla annettua lakia.
    Tarkista laskemalla, että tulos säilyy samana.
    %toisen laskutoimituksen tulos tosiaan on sama.

    \begin{alakohdat}
        \alakohta{$3\cdot (-6)$ \quad(vaihdantalaki)}
        \alakohta{$5\cdot (7+6)$ \quad(vaihdantalaki)}
        \alakohta{$5\cdot (7+6)$ \quad (osittelulaki)}
        \alakohta{$-8\cdot (-5)\cdot 2$ \quad (liitäntälaki)}
    \end{alakohdat}
    \begin{vastaus}
	\begin{alakohdat}
	    \alakohta{$(-6)\cdot 3$}
	    \alakohta{$5\cdot (6+7)$ tai $(7+6)\cdot 5$}
	    \alakohta{$5\cdot 7 + 5\cdot 6)$}
	    \alakohta{$-8\cdot ((-5)\cdot 2)$
	    (Vastaukseksi ei kelpaa $(-8\cdot (-5))\cdot 2$), koska siinä on sama laskujärjestys kuin alkuperäisessä laskussa.}
	\end{alakohdat}
    \end{vastaus}
\end{tehtava}

\begin{tehtava}
Supista. \quad
a) $\frac{15}{20}$ \qquad b) $\frac{14}{21}$ \qquad c) $\frac{12}{20}$
\begin{vastaus}
a) $\frac{3}{4}$ \qquad b) $\frac{2}{3}$\qquad c) $\frac{3}{5}$
\end{vastaus}
\end{tehtava}

\begin{tehtava}
Supista. \quad
a) $\frac{3\cdot 5 \cdot6}{4\cdot5\cdot 6}$ \qquad b) $\frac{4\cdot 8}{32}$ \qquad c) $\frac{7}{15}$
\begin{vastaus}
a) $\frac{3}{4}$ \qquad b) $\frac{1}{1}=1$\qquad c) $\frac{7}{15}$
\end{vastaus}
\end{tehtava}

\begin{tehtava}
Lavenna samannimisiksi.
\begin{alakohdat}
\alakohta{$\frac{2}{3}$ ja $\frac{4}{5}$} 
\alakohta{$\frac{5}{6}$ ja $\frac{7}{9}$}
\alakohta{$\frac{2}{3}$ ja $\frac{7}{2}$}
\end{alakohdat}
\begin{vastaus}
\begin{alakohdat}
\alakohta{$\frac{10}{15}$ ja $\frac{12}{15}$}
\alakohta{$\frac{15}{18}$ ja $\frac{14}{18}$}
\alakohta{$\frac{4}{6}$ ja $\frac{21}{6}$}
\end{alakohdat}
\end{vastaus}
\end{tehtava}

\begin{tehtava}
Laske.
	\begin{alakohdat}
		\alakohta{$\frac{3}{11}+\frac{5}{11}$}
		\alakohta{$\frac{4}{5}-\frac{1}{5}$}
		\alakohta{$\frac{2}{3}+\frac{1}{6}+\frac{2}{6}$}
		\alakohta{$ \frac{11}{12}-\frac{5}{6}+\frac{3}{2}$}
	\end{alakohdat}
	\begin{vastaus}
		\begin{alakohdat}
			\alakohta{$\frac{8}{11}$}
			\alakohta{$\frac{3}{5}$}
			\alakohta{$\frac{7}{6}$}
			\alakohta{$\frac{19}{12}$}
		\end{alakohdat}
	\end{vastaus}
\end{tehtava}

\begin{tehtava}
Muuta sekamurtoluvuksi 
%täsmällisemmin sekamurtolukumuotoon, mutta pienellä piirillä 
% ajateltiin, että tämä epätäsmällinen muotoilu parempi
\begin{alakohdatrivi}
\alakohta{$\frac{15}{2}$} 
\alakohta{$\frac{9}{4}$}
\alakohta{$\frac{23}{7}$}
\end{alakohdatrivi}
\begin{vastaus}
\begin{alakohdat}
\alakohta{$7\frac{1}{2}$}
\alakohta{$2\frac{1}{4}$}
\alakohta{$3\frac{2}{7}$}
\end{alakohdat}
\end{vastaus}
\end{tehtava}

\begin{tehtava}
Muunna murtoluvuksi 
\begin{alakohdatrivi}
\alakohta{$3\frac{2}{5}$}
\alakohta{$4\frac{1}{3}$}
\alakohta{$2\frac{6}{7}$}
\end{alakohdatrivi}
\begin{vastaus}
\begin{alakohdat}
\alakohta{$\frac{17}{5}$}
\alakohta{$\frac{13}{12}$}
\alakohta{$\frac{20}{7}$}
\end{alakohdat}
\end{vastaus}
\end{tehtava}

\begin{tehtava}
Laske.
	\begin{alakohdat}
		\alakohta{$1\frac{2}{9}+\frac{5}{9}$}
		\alakohta{$\frac{1}{3}+2\frac{1}{3}$}
		\alakohta{$2+\frac{5}{4}+3\frac{3}{4}$}
		\alakohta{$\frac{3}{2}+1-\frac{5}{6}$}
	\end{alakohdat}
	\begin{vastaus}
		\begin{alakohdat}
			\alakohta{$\frac{16}{9}$}
			\alakohta{$\frac{8}{3}$}
			\alakohta{$7$}
			\alakohta{$1\frac{2}{3}$}
		\end{alakohdat}
	\end{vastaus}
\end{tehtava}

\begin{tehtava}
Laske.
	\begin{alakohdat}
		\alakohta{$\frac{2}{3}\cdot \frac{4}{5}$}
		\alakohta{$\frac{3}{5} \cdot \frac{5}{4}$}
		\alakohta{$2\cdot \frac{2}{7}\cdot\frac{3}{4}$}
		\alakohta{$\frac{3}{2}\cdot\frac{4}{5}\cdot 10$}
	\end{alakohdat}
	\begin{vastaus}
		\begin{alakohdat}
			\alakohta{$\frac{8}{15}$}
			\alakohta{$\frac{15}{20}=\frac{3}{4}$}
			\alakohta{$\frac{3}{7}$}
			\alakohta{$12$}
		\end{alakohdat}
	\end{vastaus}
\end{tehtava}


\begin{tehtava} Laske:
a) $\frac{2}{3} : \frac{7}{11}$ \qquad b) $\frac{4}{3}:\left(\frac{-13}{4}\right)$ \qquad c) $\frac{7}{8}:4$
\begin{vastaus}
a) $1\frac{1}{21}$ \qquad b) $-\frac{16}{39}$ \qquad c) $\frac{7}{32}$
\end{vastaus}
\end{tehtava}

\begin{tehtava}
Aseta luvut suuruusjärjestykseen:
\mbox{$\frac{-15}{11}, \frac{-19}{9}, \frac{11}{12}, \frac{3}{4}, -7.$}
\begin{vastaus}
$-7, \frac{-19}{9}, \frac{-15}{11}, \frac{3}{4}, \frac{11}{12}.$
\end{vastaus}
\end{tehtava}

\begin{tehtava}
Aseta luvut suuruusjärjestykseen:
\mbox{$\frac{15}{21}, \frac{7}{-3}, \frac{-3}{-4}, \frac{8}{9}, \frac{5}{-2}.$}
\begin{vastaus}
$\frac{5}{-2}, \frac{7}{-3}, \frac{15}{21}, \frac{-3}{-4}, \frac{8}{9}.$
\end{vastaus}
\end{tehtava}

\begin{tehtava}
Laske luvun $\frac{3}{4}$
	\begin{alakohdat}
		\alakohta{vastaluku}
		\alakohta{käänteisluku}
		\alakohta{vastaluvun vastaluku}
		\alakohta{vastaluvun käänteisluku}
		\alakohta{käänteisluvun vastaluku}
		\alakohta{käänteisluvun käänteisluku.}
	\end{alakohdat}
	\begin{vastaus}
		\begin{alakohdat}
			\alakohta{$\frac{-3}{4}$}
			\alakohta{$\frac{4}{3}$}
			\alakohta{$\frac{3}{4}$}
			\alakohta{$-\frac{4}{3}$}
			\alakohta{$-\frac{4}{3}$}
			\alakohta{$\frac{3}{4}$}
		\end{alakohdat}
	\end{vastaus}
\end{tehtava}

\begin{tehtava}
Mitkä edellisen tehtävän luvuista ovat samoja kuin alkuperäinen luku? Entä samoja keskenään?
	\begin{vastaus}
		Luvun vastaluvun vastaluku on sama kuin luku itse, samoin käänteisluvun käänteisluku. Lisäksi käänteisluvun vastaluku on sama kuin vastaluvun käänteisluku. 
	\end{vastaus}
\end{tehtava}

\subsubsection*{Hallitse kokonaisuus}

\begin{tehtava}
Laske murtolukujen $\frac{5}{6}$ ja $-\frac{2}{15}$ \\ a) summa \qquad b) erotus \qquad c) tulo \qquad d) osamäärä.
\begin{vastaus}
a) $\frac{7}{10}$ \qquad b) $\frac{29}{30}$ \qquad c) $-\frac{1}{9}$ \qquad d) $-6\frac{1}{4}$
\end{vastaus}
\end{tehtava}

\begin{tehtava}
a) $\frac{5}{8}\cdot(\frac{3}{5}+\frac{2}{5})$ \qquad b) $\frac{1}{3}+\frac{1}{4}\cdot\frac{6}{5}$
\begin{vastaus}
a) $\frac{5}{8}$ \qquad b) $\frac{19}{30}$
\end{vastaus}
\end{tehtava}

\begin{tehtava}
a) $\dfrac{\frac{1}{2}:\frac{3}{2}}{\frac{3}{2}+\frac{1}{3}}$ \qquad b) $\dfrac{\frac{2}{3}+\frac{3}{4}}{\frac{5}{6}-\frac{7}{12}}$.
\begin{vastaus}
a) $\frac{2}{11}$ \qquad b) $5\frac{2}{3}$
\end{vastaus}
\end{tehtava}

\begin{tehtava}
Millä luvun $x$ arvolla lukujen $x$, $\frac{7}{91}$ ja $\frac{23}{2}$ tulo on $1$?
\begin{vastaus}
$\frac{182}{161}$
\end{vastaus}
\end{tehtava}

\begin{tehtava} 
        Laatikossa on palloja, joista kolmasosa on mustia, neljäsosa
        valkoisia ja viidesosa harmaita. Loput palloista ovat 		 	punaisia.
        Kuinka suuri osuus palloista on punaisia?
        
        \begin{vastaus}
            $1-(\frac{1}{3}+\frac{1}{4}+\frac{1}{5})
            = \frac{60}{60}-\frac{20}{60}-\frac{15}{60}-\frac{12}{60}
            = \frac{60}{60}-\frac{47}{60}
            = \frac{13}{60}$
        \end{vastaus}
    \end{tehtava}
    
\begin{tehtava} 
Eräässä pitkän matematiikan ensimmäisen kurssin ryhmässä on 16 opiskelijaa. Heistä 8 on tyttöjä ja tytöistä neljänneksellä on siniset silmät. Kuinka suuri osa luokan oppilaista on sinisilmäisiä tyttöjä?
        \begin{vastaus}
			Sinisilmäisiä tyttöjä on 2, joten luokan oppilaista heitä on $2/16=1/8$ eli yksi kahdeksasosa.
        \end{vastaus}
\end{tehtava}
%Laatinut HR 9.11.2013

\begin{tehtava}
Laske lausekkeen $\frac{x}{2-3x}$ arvo, kun $x$ on \\ a) 4 \qquad b) $-\frac{1}{2}$ \qquad c) $\frac{7}{10}$.
\begin{vastaus}
a) $-\frac{2}{5}$ \qquad b) $-\frac{1}{7}$ \qquad c) $-7$
\end{vastaus}
\end{tehtava}

\begin{tehtava}
Selvitä, millä nollasta poikkeavilla luvuilla pätee, että luvun vastaluvun käänteisluku on yhtäsuuri luvun käänteisluvun vastaluvun kanssa.
\begin{vastaus}
Kaikilla, jos $a \neq 0$, $-\frac{1}{a} = \frac{1}{-a}$.
\end{vastaus}
\end{tehtava}

\begin{tehtava}
Laske seuraavat laskut ilman laskinta soveltamalla vaihdantalakia, liitäntälakia ja osittelulakia.
Kerro, mitä laskulakeja sovelsit.

    \begin{alakohdat}
        \alakohta{$350\cdot 271-272\cdot 350$}
        \alakohta{$370\cdot 1010$}
        \alakohta{$594+368+3-368$}
    \end{alakohdat}
    \begin{vastaus}
    	\begin{alakohdat}
        \alakohta{
            \begin{align*}
	    & 350\cdot 271-272\cdot 350 \\
	    =& 350\cdot 271-350\cdot 272 &\text{\qquad (vaihdantalaki)} \\
	    =& 350\cdot (271-272) &\text{\qquad (osittelulaki)} \\
	    =& 350\cdot (-1) \\
	    =& -350
	    \end{align*}
	}
        \alakohta{
            \begin{align*}
	    & 370\cdot 1010 \\
	    =& 370\cdot (1000+10) \\
	    =& 370\cdot 1000 + 370\cdot 10 &\text{\qquad (osittelulaki)} \\
	    =& 370000 + 3700 \\
	    =& 373700
	    \end{align*}
	}
	\alakohta{
            \begin{align*}
	    & 594+368+3-368 \\
	    =& 594+368+(3-368) &\text{\qquad (liitäntälaki)} \\
	    =& 594+368+(-368+1) &\text{\qquad (vaihdantalaki)} \\
	    =& 594+(368+(-368))+1 &\text{\qquad (liitäntälaki)} \\
	    =& 594+0+1 \\
	    =& 595
	    \end{align*}
	}
	\end{alakohdat}
    \end{vastaus}
\end{tehtava}

\begin{tehtava}
Sievennä.
	\begin{alakohdat}
		\alakohta{$2(a+b)-a$}
		\alakohta{$3(2a+b)+(2a+b)$}
		\alakohta{$-(-(-a)-b)$ }
		\alakohta{$-(-a-(-a-b)-b)\cdot(1+2b)$ }
	\end{alakohdat}
\begin{vastaus}
	\begin{alakohdat}
		\alakohta{$a+2b$}
		\alakohta{$8a+4b$}
		\alakohta{$-a+b$ }
		\alakohta{$0$ }
	\end{alakohdat}
\end{vastaus}
\end{tehtava}
%Laatinut HR 9.11.2013


\begin{tehtava}
Perustele annettujen määritelmien avulla, miksi
	\begin{alakohdat}
		\alakohta{$3\cdot 2=6$}
		\alakohta{$6-4=2$}
		\alakohta{$3x+x=4x $}
		\alakohta{$xy+yx=2xy$}
	\end{alakohdat}
	kun \(x\) ja \(y\) ovat kokonaislukuja.
\begin{vastaus}
	\begin{alakohdat}
		\alakohta{$3\cdot2 = 6$, koska $2+2+2=2+4=6$}
		\alakohta{$6-4=2$ koska $2+4=6$}
		\alakohta{$3x+x=3\cdot x+1\cdot x=(3+1)\cdot x=4x$ }
		\alakohta{$xy+yx=xy+xy=1\cdot (xy)+1\cdot (xy)=(1+1)\cdot (xy)=2(xy)=2xy$ }
	\end{alakohdat}
\end{vastaus}
\end{tehtava}
%Laatinut HR 9.11.2013

\subsubsection*{Lisää tehtäviä}

\begin{tehtava}
Laske lausekkeen $\frac{x+y}{2x-y}$ arvo, kun \\ a) $x=\frac{1}{2}$ ja $y= \frac{1}{4}$ \qquad b) $x=\frac{1}{4}$ ja $y= -\frac{3}{8}$.
\begin{vastaus}
a) $1$ \qquad b) $-\frac{1}{7}$
\end{vastaus}
\end{tehtava}


 %   Yksi prosentti tarkoittaa yhtä sadasosaa: $1~\% = \frac{1}{100}$
 
 %TODO pitäisikö selittää leipätekstissä ja antaa joku esimerkki?
    
        \begin{tehtava} Laske:
            \begin{alakohdat}
        	\alakohta{$\frac{3}{5} + \frac{1}{5}$}
        	\alakohta{$\frac{5}{7} + \frac{4}{7}$}
        	\alakohta{$2 + \frac{2}{3}$}
        	\alakohta{$3 + \frac{3}{5} + \frac{2}{5}$}
            \end{alakohdat}
            \begin{vastaus}
        		\begin{alakohdat}
        			\alakohta{$\frac{4}{5}$}
        			\alakohta{$\frac{9}{7} = 1 \frac{2}{7}$}
        			\alakohta{$2 \frac{2}{3} = \frac{8}{3}$}
        			\alakohta{$4$}
        		\end{alakohdat}
            \end{vastaus}
        \end{tehtava}
        
        \begin{tehtava}
        
        \begin{alakohdat}
        	\alakohta{$\frac{6}{2} + \frac{3}{5}$}
        	\alakohta{$\frac{7}{8} - \frac{1}{4}$}
        	\alakohta{$2 \frac{1}{3} + \frac{4}{6}$}
        	\alakohta{$4 \frac{7}{2} - 6 \frac{5}{4}$}
        \end{alakohdat}
            \begin{vastaus}		
        		\begin{alakohdat}
        			\alakohta{$\frac{18}{5}$}
        			\alakohta{$\frac{5}{8}$}
        			\alakohta{$3$}
        			\alakohta{$-\frac{41}{6}$ }
        		\end{alakohdat}
            \end{vastaus}
        \end{tehtava}
        
        \begin{tehtava}
        
        \begin{alakohdat}
        	\alakohta{$2 \cdot \frac{2}{5}$}
        	\alakohta{$2 \cdot \frac{2}{3}$}
        	\alakohta{$\frac{5}{4} \cdot 2 \cdot 3$}
        	\alakohta{$\frac{\frac{3}{7}}{4}$ }
        \end{alakohdat}
            \begin{vastaus}
        		\begin{alakohdat}
        			\alakohta{$\frac{4}{5}$}
        			\alakohta{$\frac{4}{3} = 1 \frac{1}{3}$}
        			\alakohta{$\frac{15}{2} = 7 \frac{1}{2}$}
        			\alakohta{$\frac{3}{28}$}
        		\end{alakohdat}
            \end{vastaus}
        \end{tehtava}
        
        \begin{tehtava}
        
        \begin{alakohdat}
        	\alakohta{$\frac{1}{3} \cdot \frac{6}{5}$}
        	\alakohta{$\frac{5}{4} \cdot (-\frac{2}{3})$ }
        	\alakohta{$\frac{2}{5} (2 - \frac{3}{4})$}
        	\alakohta{$(\frac{5}{6} - \frac{1}{3})(\frac{7}{4} - \frac{3}{2})$}
        \end{alakohdat}
            \begin{vastaus}		
        		\begin{alakohdat}
        			\alakohta{$\frac{2}{5}$}
        			\alakohta{$-\frac{5}{6}$}
        			\alakohta{$\frac{1}{2}$}
        			\alakohta{$\frac{1}{8}$ }
        		\end{alakohdat}
            \end{vastaus}
        \end{tehtava}
        
        \begin{tehtava}
        
        \begin{alakohdat}
        	\alakohta{$\displaystyle \frac{\frac{3}{7} + \frac{5}{4}}{3}$}
        	\alakohta{$\displaystyle \frac{\frac{10}{8}}{\frac{5}{2}}$}
        	\alakohta{$\displaystyle \frac{\frac{1}{3} - \frac{5}{10}}{\frac{3}{4} + \frac{1}{2}}$}
        	\alakohta{$\displaystyle 3\frac{\frac{4}{2} + \frac{10}{4}}{\frac{3}{2} - \frac{2}{3}}$}
        \end{alakohdat}
            \begin{vastaus}		
        		\begin{alakohdat}
        			\alakohta{$\frac{47}{28}$}
        			\alakohta{$\frac{1}{2}$}
        			\alakohta{$-\frac{1}{3}$}
        			\alakohta{$\frac{54}{5}$}
        		\end{alakohdat}
            \end{vastaus}
        \end{tehtava}
    
    \begin{tehtava} %syvteht
        Pontus, Viljami, Jarkko-Kaaleppi, Simo ja Milla leipoivat lanttuvompattipiirakkaa.
        Pontus kuitenkin söi piirakasta kolmanneksen ennen muita, ja loput piirakasta
        jaettiin muiden kanssa tasan. Kuinka suuren osan muut saivat?
        
        \begin{vastaus}
            Muut saivat piirakasta kuudesosan.
        \end{vastaus}
    \end{tehtava}
    
\begin{tehtava} %perusteht
    Huvipuiston sisäänpääsylippu maksaa 20 euroa, ja lapset pääsevät sisään puoleen hintaan.
	\begin{alakohdat}
		\alakohta{Kuinka paljon kolmen lapsen yksinhuoltajaperheelle maksaa päästä sisään?}
		\alakohta{Kuinka paljon sisäänpääsy maksaa perheelle avajaispäivänä,		kun silloin sisään pääsee neljänneksen (25~\%) halvemmalla?}
    \end{alakohdat}
    \begin{vastaus}
		\begin{alakohdat}
			\alakohta{50 euroa }
			\alakohta{50 euroa }
			\alakohta{37,50 euroa}
		\end{alakohdat} 
    \end{vastaus}
\end{tehtava}  
  
\begin{tehtava}
    Laske 
    \[ \frac{10}{9}\cdot \frac{9}{8}\cdot \frac{8}{7}\cdot \frac{7}{6}\cdot \frac{6}{5}
    \cdot \frac{5}{4}\cdot \frac{4}{3}\cdot \frac{3}{2}. \]
    \begin{vastaus}
		$\frac{10}{2}=5$.
    \end{vastaus}        
\end{tehtava}
    
\begin{tehtava}
	Eräässä kaupassa on käynnissä loppuunmyynti, ja kaikki tuotteet
    myydään puoleen hintaan. Lisäksi kanta-asiakkaat saavat aina
    viidenneksen alennusta ostoksistaan.
	Paljonko kanta-asiakas maksaa nyt tuotteesta, joka normaalisti
    maksaisi 40 euroa?
    \begin{vastaus}
		$40\cdot \frac{1}{2} \cdot \frac{4}{5}=40\cdot \frac{4}{10}= 16$. 
	\end{vastaus}
\end{tehtava}
    
\begin{tehtava}
	Kokonaisesta kakusta syödään maanantaina iltapäivällä puolet, ja jäljelle
	jääneestä palasta syödään tiistaina iltapäivällä taas puolet.
	Jos kakun jakamista ja syömistä jatketaan samalla tavalla koko viikko,
	kuinka suuri osa alkuperäisestä kakusta on
	jäljellä seuraavana maanantaiaamuna?
	\begin{vastaus}
		Toisena päivänä aamulla kakkua on jäljellä puolet, kolmantena
		päivänä aamulla
		$1-\left(\frac{1}{2} + \frac{1}{4}\right) = \frac{1}{4}$, 
		neljäntenä päivänä
		$1-\left(\frac{1}{2} + \frac{1}{4} + \frac{1}{8}\right)
		= \frac{1}{8}$, jne.
		Siis seitsemän päivän jälkeen kakkua on jäljellä
		$1-\left(\frac{1}{2} + \frac{1}{4} + \frac{1}{8} +
		\frac{1}{16} + \frac{1}{32} + \frac{1}{64} + \frac{1}{128}\right)
		= \frac{1}{128}$.  
	\end{vastaus}
\end{tehtava}

\begin{tehtava}
	Laske lausekkeen $\frac{1}{n}-\frac{1}{m}$ arvo, kun tiedetään, että $n = \frac{1}{9}$ ja $m=n+1$.
	\begin{vastaus}
		$\frac{81}{10}$
	\end{vastaus}
\end{tehtava}

\begin{tehtava}
	Laske lausekkeen $\frac{1}{n}-\frac{1}{2n}+\frac{1}{3n}$ arvo, kun tiedetään, että $n = 10$.
	\begin{vastaus}
		$\frac{1}{12}$
	\end{vastaus}
\end{tehtava}
\begin{tehtava}
	Määritä positiivisten kokonaislukujen \(m\) ja \(n\) käänteislukujen summan käänteisluku. 
	\begin{vastaus}
		$\frac{mn}{m+n}$
	\end{vastaus}
\end{tehtava}
%Laatinut HR 9.11.2013
\begin{tehtava}
	\begin{alakohdatrivi}
		\alakohta{$\frac{4}{9} : \frac{1}{5}$}
		\alakohta{$\frac{2}{7} : \frac{5}{9}$}
		\alakohta{$\frac{2}{3}:\frac{4}{3}$}
	\end{alakohdatrivi}
	\begin{vastaus}
		\begin{alakohdatrivi}
			\alakohta{$\frac{20}{9}$}
			\alakohta{$\frac{18}{35}$}
			\alakohta{$\frac{1}{2}$}
		\end{alakohdatrivi}
	\end{vastaus}
\end{tehtava}

\begin{tehtava}
	Vanhalla matemaatikolla oli kolme lasta. Eräänä päivänä hän antoi lapsilleen laatikon
	ja kertoi, että sen sisällä oli erilaisia palkintoja, joita hän oli saanut ratkottuaan
	pulmatilanteita ympäri maailmaa. Hän kertoi antavansa vanhimmalle lapselleen puolet 
	saamistaan arvoesineistä, keskimmäiselle neljäsosan ja nuorimmalle kuudesosan. Avattuaan 
	laatikon lapset näkivät 11 erilaista esinettä, ja matemaatikon suureksi iloksi osasivat 
	jakaa esineet oikein. Kuinka monta esinettä kukin sai?
	\begin{vastaus}
		Vanhin sai 6 esinettä, keskimmäinen 3 esinettä ja nuorin 2 esinettä. Vanha
		matemaatikko oli pitänyt yhden esineen itsellään, sillä $\frac{1}{2} + \frac{1}{4}
		+ \frac{1}{6} = \frac{11}{12}$.
	\end{vastaus}
\end{tehtava}

\begin{tehtava}
	$\star$ Fibonaccin luvut 0, 1, 1, 2, 3, 5, 8, 13, 21, $\ldots$ määritellään seuraavasti: Kaksi ensimmäistä
	Fibonaccin lukua ovat 0 ja 1, ja siitä seuraavat saadaan kahden
	edellisen summana: \[ 0+1=1, \quad 1+1=2, \quad 1+2 = 3, \quad 2+3=5 \] 
	ja niin edelleen. 
	Tutki, miten Fibonaccin luvut liittyvät lukuihin
	\[ \frac{1}{1+1}, \quad \frac{1}{1+\frac{1}{1+1}}, \quad
	\frac{1}{1+\frac{1}{1+\frac{1}{1+1}}}, \quad 
	\frac{1}{1+\frac{1}{1+\frac{1}{1+\frac{1}{1+1}}}}, \quad \ldots\]
	\begin{vastaus}
		Luvut ovat sievennettynä peräkkäisten Fibonaccin
		lukujen osamääriä:
		\[\frac{1}{2}, \ \frac{2}{3}, \ \frac{3}{5}, \frac{5}{8} \ldots  \]
	\end{vastaus}
\end{tehtava}

\begin{tehtava}
	\begin{alakohdat}
		\alakohta{Jos $n$ on positiivinen kokonaisluku, laske lukujen $n$ ja $(n+1)$ käänteislukujen erotus.}
		\alakohta{Laske summa \[ \frac{1}{1\cdot 2}+\frac{1}{2 \cdot 3}+ \ldots + \frac{1}{(n-1)n} \]}
	\end{alakohdat}
	\begin{vastaus}
		\begin{alakohdat}
			\alakohta{$\frac{1}{n(n+1)}$}
			\alakohta{$1-\frac{1}{n}$}
		\end{alakohdat}
	\end{vastaus}
\end{tehtava}

\begin{tehtava}
%Laatinut Jaakko Viertiö 2013-11-9
%Vaativa, periaatteessa epäyhtälön ratkaisua, mutta myös ajateltavissa ilman epäyhtälöitä.
	Määritä ne positiiviset reaaliluvut x, jotka ovat käänteislukuaan $\frac{1}{x}$ suurempia.
	\begin{vastaus}
	 Kun $x>1$.
	\end{vastaus}
\end{tehtava}


\end{tehtavasivu}
