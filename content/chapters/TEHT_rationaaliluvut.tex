\begin{tehtavasivu}

\begin{tehtava}
Supista. \quad
a) $\frac{15}{20}$ \qquad b) $\frac{144}{243}$ \qquad c) $\frac{12a+2ab}{4a}$ \qquad d) $\frac{a\cdot a\cdot (3b)}{3a}$
\begin{vastaus}
a) $\frac{3}{4}$ \qquad b) $\frac{16}{27}$\qquad c) $\frac{6+b}{2}$\qquad d) $ab$
\end{vastaus}
\end{tehtava}

%\begin{tehtava}
%Supista. \quad
%a) $\frac{3\cdot 5 \cdot6}{4\cdot5\cdot 6}$ \qquad b) $\frac{3a+ab}{a}$ \qquad c) $\frac{3a\cdot a\cdot(4b)\cdot b}{(4a)b}$
%\begin{vastaus}
%a) $\frac{3}{4}$ \qquad b) $3+b$\qquad c) $3ab$
%\end{vastaus}
%\end{tehtava}

\begin{tehtava}
Lavenna samannimisiksi.
\begin{alakohdat}
\alakohta{$\frac{2}{3}$ ja $\frac{4}{5}$} 
\alakohta{$\frac{5a}{6}$ ja $\frac{7b}{9}$}
\alakohta{$\frac{2}{3a}$ ja $\frac{5b}{2}$}
\end{alakohdat}
\begin{vastaus}
\begin{alakohdat}
\alakohta{$\frac{10}{15}$ ja $\frac{12}{15}$}
\alakohta{$\frac{15a}{18}$ ja $\frac{14b}{18}$}
\alakohta{$\frac{4}{6a}$ ja $\frac{15ab}{6a}$}
\end{alakohdat}
\end{vastaus}
\end{tehtava}

\begin{tehtava}
Laske.
	\begin{alakohdat}
		\alakohta{$\frac{3}{11}+\frac{5}{11}$}
		\alakohta{$\frac{4}{5}-\frac{1}{5}$}
		\alakohta{$\frac{2}{3}+\frac{1}{6}+\frac{2}{6}$}
		\alakohta{$ \frac{11}{12}-\frac{5}{6}+\frac{3}{2}$}
	\end{alakohdat}
	\begin{vastaus}
		\begin{alakohdat}
			\alakohta{$\frac{8}{11}$}
			\alakohta{$\frac{3}{5}$}
			\alakohta{$\frac{7}{6}$}
			\alakohta{$\frac{19}{12}$}
		\end{alakohdat}
	\end{vastaus}
\end{tehtava}

\begin{tehtava}
Muuta sekamurtoluvuksi 
%täsmällisemmin sekamurtolukumuotoon, mutta pienellä piirillä 
% ajateltiin, että tämä epätäsmällinen muotoilu parempi
\begin{alakohdatrivi}
\alakohta{$\frac{15}{2}$} 
\alakohta{$\frac{177}{16}$}
\alakohta{$\frac{2013}{15}$}
\end{alakohdatrivi}
\begin{vastaus}
\begin{alakohdat}
\alakohta{$7\frac{1}{2}$}
\alakohta{$11\frac{1}{16}$}
\alakohta{$134\frac{3}{15}$}
\end{alakohdat}
\end{vastaus}
\end{tehtava}
%muokannut Henri Ruoho 9.11.2013

\begin{tehtava}
Muunna murtoluvuksi 
\begin{alakohdatrivi}
\alakohta{$3\frac{2}{5}$}
\alakohta{$4\frac{1}{34}$}
\alakohta{$2\frac{6}{3145}$}
\end{alakohdatrivi}
\begin{vastaus}
\begin{alakohdat}
\alakohta{$\frac{17}{5}$}
\alakohta{$\frac{137}{34}$}
\alakohta{$\frac{6296}{3145}$}
\end{alakohdat}
\end{vastaus}
\end{tehtava}

\begin{tehtava}
Laske.
	\begin{alakohdat}
		\alakohta{$1\frac{2}{9}+\frac{5}{9}$}
		\alakohta{$\frac{1}{3}+2\frac{1}{3}$}
		\alakohta{$2+\frac{5}{4}+3\frac{3}{4}$}
		\alakohta{$\frac{3}{2}+1-\frac{5}{6}$}
	\end{alakohdat}
	\begin{vastaus}
		\begin{alakohdat}
			\alakohta{$\frac{16}{9}$}
			\alakohta{$\frac{8}{3}$}
			\alakohta{$7$}
			\alakohta{$1\frac{2}{3}$}
		\end{alakohdat}
	\end{vastaus}
\end{tehtava}

\begin{tehtava}
Laske.
	\begin{alakohdat}
		\alakohta{$\frac{2}{3}\cdot \frac{4}{5}$}
		\alakohta{$\frac{3}{5} \cdot \frac{5}{4}$}
		\alakohta{$2\cdot \frac{2}{7}\cdot\frac{3}{4}$}
		\alakohta{$\frac{3}{2}\cdot\frac{4}{5}\cdot 10$}
	\end{alakohdat}
	\begin{vastaus}
		\begin{alakohdat}
			\alakohta{$\frac{8}{15}$}
			\alakohta{$\frac{15}{20}=\frac{3}{4}$}
			\alakohta{$\frac{3}{7}$}
			\alakohta{$12$}
		\end{alakohdat}
	\end{vastaus}
\end{tehtava}


\begin{tehtava} Laske:

a) $\frac{2}{3} : \frac{7}{11}$ \qquad b) $\frac{4}{3}:\left(\frac{-13}{4}\right)$ \qquad c) $\frac{7}{8}:4$  \qquad d) $\frac{1}{\frac{3}{7}-\frac{9}{21}}$
\begin{vastaus}
a) $1\frac{1}{21}$ \qquad b) $-\frac{16}{39}$ \qquad c) $\frac{7}{32}$ \qquad d) Ei määritelty
\end{vastaus}
\end{tehtava}

\begin{tehtava}
Aseta luvut suuruusjärjestykseen:
\begin{alakohdat}
\alakohta{$\frac{-15}{11}, \frac{-19}{9}, \frac{11}{12}, \frac{3}{4}, -7.$}
\alakohta{$\frac{15}{21}, \frac{7}{-3}, \frac{-3}{-4}, \frac{8}{9}, \frac{5}{-2}.$}
\end{alakohdat}
\begin{vastaus}
\begin{alakohdat}
\alakohta{$-7, \frac{-19}{9}, \frac{-15}{11}, \frac{3}{4}, \frac{11}{12}.$}
\alakohta{$\frac{5}{-2}, \frac{7}{-3}, \frac{15}{21}, \frac{-3}{-4}, \frac{8}{9}.$}
\end{alakohdat}
\end{vastaus}
\end{tehtava}

\begin{tehtava}
%Laatinut Jaakko Viertiö 2013-11-9
 Tarkastellaan lukuja $\frac{3+a}{a}$ ja $\frac{\frac{5}{3}+a}{a}$. Kumpi luvuista on suurempi, kun \(a\) on kokonaisluku ja suurempi tai yhtäsuuri kuin 1?
 \begin{vastaus}
  Luku $\frac{3+a}{a}$, sillä $3>\frac{5}{3}$. Vakio a on molemmissa sama luku, joten se vaikuttaa samalla tavalla molempien lukujen suuruuteen.
 \end{vastaus}
\end{tehtava}


\begin{tehtava}
Laske luvun $\frac{3}{4}$
	\begin{alakohdat}
		\alakohta{vastaluku}
		\alakohta{käänteisluku}
		\alakohta{vastaluvun vastaluku}
		\alakohta{vastaluvun käänteisluku}
		\alakohta{käänteisluvun vastaluku}
		\alakohta{käänteisluvun käänteisluku.}
	\end{alakohdat}
	\begin{vastaus}
		\begin{alakohdat}
			\alakohta{$\frac{-3}{4}$}
			\alakohta{$\frac{4}{3}$}
			\alakohta{$\frac{3}{4}$}
			\alakohta{$-\frac{4}{3}$}
			\alakohta{$-\frac{4}{3}$}
			\alakohta{$\frac{3}{4}$}
		\end{alakohdat}
	\end{vastaus}
\end{tehtava}


\begin{tehtava}
Mitkä edellisen tehtävän luvuista ovat samoja kuin alkuperäinen luku? Entä samoja keskenään?
	\begin{vastaus}
		Luvun vastaluvun vastaluku on sama kuin luku itse, samoin käänteisluvun käänteisluku. Lisäksi käänteisluvun vastaluku on sama kuin vastaluvun käänteisluku. 
	\end{vastaus}
\end{tehtava}

\begin{tehtava}
Laske lukujen $\frac{5}{4}$ ja $\frac{1}{8}$ 
	\begin{alakohdat}
		\alakohta{vastalukujen summa}
		\alakohta{käänteislukujen summa}
		\alakohta{summan vastaluku}
		\alakohta{summan käänteisluku}
	\end{alakohdat}
Huomaa, mitkä luvuista ovat samoja.
	\begin{vastaus}
		\begin{alakohdat}
			\alakohta{$\frac{-11}{8}$}
			\alakohta{$8\frac{5}{4}=10$}
			\alakohta{$-\frac{11}{8}$}
			\alakohta{$-\frac{8}{11}$}
		\end{alakohdat}
		
	\end{vastaus}
\end{tehtava}

\subsubsection*{Hallitse kokonaisuus}

\begin{tehtava}
Laske murtolukujen $\frac{5}{6}$ ja $-\frac{2}{15}$ \\ a) summa \qquad b) erotus \qquad c) tulo \qquad d) osamäärä.
\begin{vastaus}
a) $\frac{7}{10}$ \qquad b) $\frac{29}{30}$ \qquad c) $-\frac{1}{9}$ \qquad d) $-6\frac{1}{4}$
\end{vastaus}
\end{tehtava}

\begin{tehtava} Laske
a) $\frac{5}{8}\cdot(\frac{3}{5}+\frac{2}{5})$ \qquad b) $\frac{1}{3}+\frac{1}{4}\cdot\frac{6}{5}$
\begin{vastaus}
a) $\frac{5}{8}$ \qquad b) $\frac{19}{30}$
\end{vastaus}
\end{tehtava}

\begin{tehtava}Laske
a) $\dfrac{\frac{1}{2}:\frac{3}{2}}{\frac{3}{2}+\frac{1}{3}}$ \qquad b) $\dfrac{\frac{2}{3}+\frac{3}{4}}{\frac{5}{6}-\frac{7}{12}}$.
\begin{vastaus}
a) $\frac{2}{11}$ \qquad b) $5\frac{2}{3}$
\end{vastaus}
\end{tehtava}

\begin{tehtava}
Millä luvun $x$ arvolla lukujen $x$, $\frac{7}{91}$ ja $\frac{23}{2}$ tulo on $1$?
\begin{vastaus}
$\frac{182}{161}$
\end{vastaus}
\end{tehtava}

\begin{tehtava} 
        Laatikossa on palloja, joista kolmasosa on mustia, neljäsosa
        valkoisia ja viidesosa harmaita. Loput palloista ovat 		 	punaisia.
        Kuinka suuri osuus palloista on punaisia?
        
        \begin{vastaus}
            $1-(\frac{1}{3}+\frac{1}{4}+\frac{1}{5})
            = \frac{60}{60}-\frac{20}{60}-\frac{15}{60}-\frac{12}{60}
            = \frac{60}{60}-\frac{47}{60}
            = \frac{13}{60}$
        \end{vastaus}
    \end{tehtava}
    
\begin{tehtava} 
Eräässä pitkän matematiikan ensimmäisen kurssin ryhmässä on 16 opiskelijaa. Heistä 8 on tyttöjä ja tytöistä neljänneksellä on siniset silmät. Kuinka suuri osa luokan oppilaista on sinisilmäisiä tyttöjä?
        \begin{vastaus}
			Yksi kahdeksasosa.
        \end{vastaus}
\end{tehtava}
%Laatinut Henri Ruoho 9.11.2013

\begin{tehtava}
Laske lausekkeen $\frac{x}{2-3x}$ arvo, kun $x$ on \\ a) 4 \qquad b) $-\frac{1}{2}$ \qquad c) $\frac{7}{10}$.
\begin{vastaus}
a) $-\frac{2}{5}$ \qquad b) $-\frac{1}{7}$ \qquad c) $-7$
\end{vastaus}
\end{tehtava}

\begin{tehtava}
Selvitä, millä nollasta poikkeavilla luvuilla on se ominaisuus, että luvun vastaluvun käänteisluku on yhtäsuuri kuin luvun käänteisluvun vastaluku.
\begin{vastaus}
Kaikilla, jos $a \neq 0$, $-\frac{1}{a} = \frac{1}{-a}$.
\end{vastaus}
\end{tehtava}

\subsubsection*{Lisää tehtäviä}

\begin{tehtava}
Laske lausekkeen $\frac{x+y}{2x-y}$ arvo, kun \\ a) $x=\frac{1}{2}$ ja $y= \frac{1}{4}$ \qquad b) $x=\frac{1}{4}$ ja $y= -\frac{3}{8}$.
\begin{vastaus}
a) $1$ \qquad b) $-\frac{1}{7}$
\end{vastaus}
\end{tehtava}


 %   Yksi prosentti tarkoittaa yhtä sadasosaa: $1~\% = \frac{1}{100}$
 
 %TODO pitäisikö selittää leipätekstissä ja antaa joku esimerkki?
    
        \begin{tehtava} Laske:
            \begin{alakohdat}
        	\alakohta{$\frac{3}{5} + \frac{1}{5}$}
        	\alakohta{$\frac{5}{7} + \frac{4}{7}$}
        	\alakohta{$2 + \frac{2}{3}$}
        	\alakohta{$3 + \frac{3}{5} + \frac{2}{5}$}
            \end{alakohdat}
            \begin{vastaus}
        		\begin{alakohdat}
        			\alakohta{$\frac{4}{5}$}
        			\alakohta{$\frac{9}{7} = 1 \frac{2}{7}$}
        			\alakohta{$2 \frac{2}{3} = \frac{8}{3}$}
        			\alakohta{$4$}
        		\end{alakohdat}
            \end{vastaus}
        \end{tehtava}
        
        \begin{tehtava}
        
        \begin{alakohdat}
        	\alakohta{$\frac{6}{2} + \frac{3}{5}$}
        	\alakohta{$\frac{7}{8} - \frac{1}{4}$}
        	\alakohta{$2 \frac{1}{3} + \frac{4}{6}$}
        	\alakohta{$4 \frac{7}{2} - 6 \frac{5}{4}$}
        \end{alakohdat}
            \begin{vastaus}		
        		\begin{alakohdat}
        			\alakohta{$\frac{18}{5}$}
        			\alakohta{$\frac{5}{8}$}
        			\alakohta{$3$}
        			\alakohta{$-\frac{41}{6}$ }
        		\end{alakohdat}
            \end{vastaus}
        \end{tehtava}
        
        \begin{tehtava}
        
        \begin{alakohdat}
        	\alakohta{$2 \cdot \frac{2}{5}$}
        	\alakohta{$2 \cdot \frac{2}{3}$}
        	\alakohta{$\frac{5}{4} \cdot 2 \cdot 3$}
        	\alakohta{$\frac{\frac{3}{7}}{4}$ }
        \end{alakohdat}
            \begin{vastaus}
        		\begin{alakohdat}
        			\alakohta{$\frac{4}{5}$}
        			\alakohta{$\frac{4}{3} = 1 \frac{1}{3}$}
        			\alakohta{$\frac{15}{2} = 7 \frac{1}{2}$}
        			\alakohta{$\frac{3}{28}$}
        		\end{alakohdat}
            \end{vastaus}
        \end{tehtava}
        
        \begin{tehtava}
        
        \begin{alakohdat}
        	\alakohta{$\frac{1}{3} \cdot \frac{6}{5}$}
        	\alakohta{$\frac{5}{4} \cdot (-\frac{2}{3})$ }
        	\alakohta{$\frac{2}{5} (2 - \frac{3}{4})$}
        	\alakohta{$(\frac{5}{6} - \frac{1}{3})(\frac{7}{4} - \frac{3}{2})$}
        \end{alakohdat}
            \begin{vastaus}		
        		\begin{alakohdat}
        			\alakohta{$\frac{2}{5}$}
        			\alakohta{$-\frac{5}{6}$}
        			\alakohta{$\frac{1}{2}$}
        			\alakohta{$\frac{1}{8}$ }
        		\end{alakohdat}
            \end{vastaus}
        \end{tehtava}
        
        \begin{tehtava}
        
        \begin{alakohdat}
        	\alakohta{$\displaystyle \frac{\frac{3}{7} + \frac{5}{4}}{3}$}
        	\alakohta{$\displaystyle \frac{\frac{10}{8}}{\frac{5}{2}}$}
        	\alakohta{$\displaystyle \frac{\frac{1}{3} - \frac{5}{10}}{\frac{3}{4} + \frac{1}{2}}$}
        	\alakohta{$\displaystyle 3\frac{\frac{4}{2} + \frac{10}{4}}{\frac{3}{2} - \frac{2}{3}}$}
        \end{alakohdat}
            \begin{vastaus}		
        		\begin{alakohdat}
        			\alakohta{$\frac{47}{28}$}
        			\alakohta{$\frac{1}{2}$}
        			\alakohta{$-\frac{1}{3}$}
        			\alakohta{$\frac{54}{5}$}
        		\end{alakohdat}
            \end{vastaus}
        \end{tehtava}
        
        \begin{tehtava} Laske:
	\begin{alakohdatrivi}
		\alakohta{$\frac{4}{9} : \frac{1}{5}$}
		\alakohta{$\frac{2}{7} : \frac{5}{9}$}
		\alakohta{$\frac{2}{3}:\frac{4}{3}$}
	\end{alakohdatrivi}
	\begin{vastaus}
		\begin{alakohdatrivi}
			\alakohta{$\frac{20}{9}$}
			\alakohta{$\frac{18}{35}$}
			\alakohta{$\frac{1}{2}$}
		\end{alakohdatrivi}
	\end{vastaus}
\end{tehtava}

\begin{tehtava}
    Laske 
    \[ \frac{10}{9}\cdot \frac{9}{8}\cdot \frac{8}{7}\cdot \frac{7}{6}\cdot \frac{6}{5}
    \cdot \frac{5}{4}\cdot \frac{4}{3}\cdot \frac{3}{2}. \]
    \begin{vastaus}
		$\frac{10}{2}=5$.
    \end{vastaus}        
\end{tehtava}
    
    \begin{tehtava} %syvteht
        Mira, Pontus, Viljami, Jarkko-Kaaleppi ja Milla leipoivat lanttuvompattipiirakkaa.
        Pontus kuitenkin söi piirakasta kolmanneksen ennen muita, ja loput piirakasta
        jaettiin muiden kanssa tasan. Kuinka suuren osan muut saivat?
        
        \begin{vastaus}
            Muut saivat piirakasta kuudesosan.
        \end{vastaus}
    \end{tehtava}
    
\begin{tehtava} %perusteht
    Huvipuiston sisäänpääsylippu maksaa 20 euroa, ja lapset pääsevät sisään puoleen hintaan.
	\begin{alakohdat}
		\alakohta{Kuinka paljon kolmen lapsen yksinhuoltajaperheelle maksaa päästä sisään?}
		\alakohta{Kuinka paljon sisäänpääsy maksaa perheelle avajaispäivänä, kun silloin sisään pääsee neljänneksen (25~\%) halvemmalla?}
    \end{alakohdat}
    \begin{vastaus}
		\begin{alakohdat}
			\alakohta{50 euroa }
			\alakohta{50 euroa }
			\alakohta{37,50 euroa}
		\end{alakohdat} 
    \end{vastaus}
\end{tehtava}  
  

    
\begin{tehtava}
	Eräässä kaupassa on käynnissä loppuunmyynti, ja kaikki tuotteet
    myydään puoleen hintaan. Lisäksi kanta-asiakkaat saavat aina
    viidenneksen alennusta ostoksistaan.
	Paljonko kanta-asiakas maksaa nyt tuotteesta, joka normaalisti
    maksaisi 40 euroa?
    \begin{vastaus}
		$40\cdot \frac{1}{2} \cdot \frac{4}{5}=40\cdot \frac{4}{10}= 16$. 
	\end{vastaus}
\end{tehtava}
    
\begin{tehtava}
	Kokonaisesta kakusta syödään maanantaina iltapäivällä puolet, ja jäljelle
	jääneestä palasta syödään tiistaina iltapäivällä taas puolet.
	Jos kakun jakamista ja syömistä jatketaan samalla tavalla koko viikko,
	kuinka suuri osa alkuperäisestä kakusta on
	jäljellä seuraavana maanantaiaamuna?
	\begin{vastaus}
		Toisena päivänä aamulla kakkua on jäljellä puolet, kolmantena
		päivänä aamulla
		$1-\left(\frac{1}{2} + \frac{1}{4}\right) = \frac{1}{4}$, 
		neljäntenä päivänä
		$1-\left(\frac{1}{2} + \frac{1}{4} + \frac{1}{8}\right)
		= \frac{1}{8}$, jne.
		Siis seitsemän päivän jälkeen kakkua on jäljellä
		$1-\left(\frac{1}{2} + \frac{1}{4} + \frac{1}{8} +
		\frac{1}{16} + \frac{1}{32} + \frac{1}{64} + \frac{1}{128}\right)
		= \frac{1}{128}$.  
	\end{vastaus}
\end{tehtava}


\begin{tehtava}
% Muokannut Henri Ruoho 9.11.2013
% Ratkaisun tarkistanut Sampo Tiensuu 2013-11-10
% Vastaus ja tehtävä ovat oikein, vaikka siltä ei ehkä ensin vaikuta.
% Jos ihmetyttää, kannattaa lukea tarkkaan, mitä tehtävässä oikein sanotaan.
%Tähtitehtäväksi, jos vaikea?
	Vanhalla matemaatikolla on kolme lasta. Eräänä päivänä hän antaa lapsilleen laatikollisen vuosien varrella 						
	ongelmanratkaisukilpailuista voitettuja palkintoja. Hän kertoo antavansa vanhimmalle lapselleen puolet
	saamistaan arvoesineistä, keskimmäiselle neljäsosan ja nuorimmalle kuudesosan. Laatikossa on kuitenkin vain 11 palkintoa. Miten 	
	palkinnot jaetaan ja kuinka monta arvoesinettä matemaatikko pitää itsellään?
	\begin{vastaus}
		Vanhin sai 6 esinettä, keskimmäinen 3 esinettä ja nuorin 2 esinettä. Vanha
		matemaatikko pitää yhden palkinnon itsellään, sillä $\frac{1}{2} + \frac{1}{4}
		+ \frac{1}{6} = \frac{11}{12}$. Tämä vastaus on oikein. Jos ihmetyttää, kannattaa lukea tarkkaan, mitä tehtävässä oikein sanotaan.
	\end{vastaus}
\end{tehtava}

%	\begin{vastaus}
%		Vanhin sai 6 esinettä, keskimmäinen 3 esinettä ja nuorin 2 esinettä. Vanha
%		matemaatikko oli pitänyt yhden esineen itsellään, sillä $\frac{1}{2} + \frac{1}{4}
%		+ \frac{1}{6} = \frac{11}{12}$.
%	\end{vastaus}
%\end{tehtava}
%\begin{tehtava}
%	Vanhalla matemaatikolla oli kolme lasta. Eräänä päivänä hän antoi lapsilleen laatikon ja kertoi, että sen sisällä oli erilaisia palkintoja, joita hän oli saanut ratkottuaan pulmatilanteita ympäri maailmaa. Hän kertoi antavansa vanhimmalle lapsilleen puolet
%	saamistaan arvoesineistä, keskimmäiselle neljäsosan ja nuorimmalle kuudesosan. Avattuaan 
%	laatikon lapset näkivät 11 erilaista esinettä, ja matemaatikon suureksi iloksi osasivat 
%	jakaa esineet oikein. Kuinka monta esinettä kukin sai?
%	\begin{vastaus}
%		Vanhin sai 6 esinettä, keskimmäinen 3 esinettä ja nuorin 2 esinettä. Vanha
%		matemaatikko oli pitänyt yhden esineen itsellään, sillä $\frac{1}{2} + \frac{1}{4}
%		+ \frac{1}{6} = \frac{11}{12}$.
%	\end{vastaus}
%\end{tehtava}

\begin{tehtava}
	Laske lausekkeen $\frac{1}{n}-\frac{1}{m}$ arvo, kun tiedetään, että $n = \frac{1}{9}$ ja $m=n+1$.
	\begin{vastaus}
		$\frac{81}{10}$
	\end{vastaus}
\end{tehtava}

\begin{tehtava}
	Laske lausekkeen $\frac{1}{n}-\frac{1}{2n}+\frac{1}{3n}$ arvo, kun tiedetään, että $n = 10$.
	\begin{vastaus}
		$\frac{1}{12}$
	\end{vastaus}
\end{tehtava}
\begin{tehtava}
	Määritä nollasta poikkeavien rationaalilukujen \(r\) ja \(s\) käänteislukujen summan käänteisluku. Minkä arvon saat, jos \(r=\frac{2}{3}\) ja \(s=3\)?
	\begin{vastaus}
		Summa on $\frac{rs}{r+s}$. Jos \(r=\frac{2}{3}\) ja \(s=3\), niin lausekkeen arvo on \(\frac{6}{11}\).
	\end{vastaus}
\end{tehtava}
%Laatinut Henri Ruoho 9.11.2013

%\begin{tehtava}
%	Tarkastellaan jaksollista desimaalilukua \(a=0,1212\ldots\) Jakson pituus on 2 ja \(100a=12,1212\ldots\) Nyt \(100a-a=12\), joten \[a=\frac{12}{99}.\] Vastaavasti jos \(b=0,314314\ldots\), niin jakson pituus on 3 ja \(1000b=314,314314\ldots\) Siis \(999b=314\) ja niinpä \[b=\frac{314}{999}.\] Määritä samalla tekniikalla jaksollisten desimaalilukujen
%	\begin{alakohdat}
%		\alakohta{$0,2020\ldots$}
%		\alakohta{$0,118118\ldots$}
%		\alakohta{$0,333\ldots$}
%		\alakohta{$3,1414\ldots$}
%	\end{alakohdat}
%murtolukuesitykset.	
%	\begin{vastaus}
%		\begin{alakohdat}
%			\alakohta{$20/99$}
%			\alakohta{$118/999$}
%			\alakohta{$1/3$}
%			\alakohta{$3+14/99$}
%		\end{alakohdat}
%	\end{vastaus}
%\end{tehtava}
%%Laatinut Henri Ruoho 9.11.2013


\begin{tehtava}
	\begin{alakohdat}
		\alakohta{Jos $n$ on positiivinen kokonaisluku, laske lukujen $n$ ja $(n+1)$ käänteislukujen erotus.}
		\alakohta{Laske summa \[ \frac{1}{1\cdot 2}+\frac{1}{2 \cdot 3}+ \ldots + \frac{1}{(n-1)n} \]}
	\end{alakohdat}
	\begin{vastaus}
		\begin{alakohdat}
			\alakohta{$\frac{1}{n(n+1)}$}
			\alakohta{$1-\frac{1}{n}$}
		\end{alakohdat}
	\end{vastaus}
\end{tehtava}

\begin{tehtava}
%Laatinut Jaakko Viertiö 2013-11-9
%Vaativa, periaatteessa epäyhtälön ratkaisua, mutta myös ajateltavissa ilman epäyhtälöitä.
	Määritä ne positiiviset reaaliluvut \(x\), jotka ovat käänteislukuaan $\frac{1}{x}$ suurempia.
	\begin{vastaus}
	 Kun $x>1$.
	\end{vastaus}
\end{tehtava}

\begin{tehtava}
%Laatinut Jaakko Viertiö 2013-11-9
Sievennä:
 \begin{alakohdat}
  \alakohta{$a+{b}\cdot{\frac{a}{b}}\cdot{\frac{\frac{b}{c}}{\frac{ab}{c}}}$}
  \alakohta{$\dfrac{208ab+52b+26b({\frac{a}{26}}-{\frac{6a}{3a}})-ab}{-42ab+52b+42ba}$}
 \end{alakohdat}
 \begin{vastaus}
	\begin{alakohdat}
		\alakohta{$a+1$}
		\alakohta{$4a$}
	\end{alakohdat}
 \end{vastaus}


\end{tehtava}

\begin{tehtava}
	$\star$ Fibonaccin luvut 0, 1, 1, 2, 3, 5, 8, 13, 21, $\ldots$ määritellään seuraavasti: Kaksi ensimmäistä
	Fibonaccin lukua ovat 0 ja 1, ja siitä seuraavat saadaan kahden
	edellisen summana: \[ 0+1=1, \quad 1+1=2, \quad 1+2 = 3, \quad 2+3=5 \] 
	ja niin edelleen. 
	Tutki, miten Fibonaccin luvut liittyvät lukuihin
	\[ \frac{1}{1+1}, \quad \frac{1}{1+\frac{1}{1+1}}, \quad
	\frac{1}{1+\frac{1}{1+\frac{1}{1+1}}}, \quad 
	\frac{1}{1+\frac{1}{1+\frac{1}{1+\frac{1}{1+1}}}}, \quad \ldots\]
	\begin{vastaus}
		Luvut ovat sievennettynä peräkkäisten Fibonaccin
		lukujen osamääriä:
		\[\frac{1}{2}, \ \frac{2}{3}, \ \frac{3}{5}, \frac{5}{8} \ldots  \]
	\end{vastaus}
\end{tehtava}


\end{tehtavasivu}
