\begin{tehtavasivu}

\begin{tehtava}
Laske ja totea murtolukujen 
 $ \frac{1}{10} = 0,1$ , 
$ \frac{1}{100} = 0,01$ , 
 $ \frac{1}{2} = 0,5$ , 
$ \frac{1}{4} = 0,25$ , 
$ \frac{3}{4} = 0,75$
desimaaliesitysten paikkansapitävyys.
\end{tehtava}

\begin{tehtava}
Muuta murtoluvuksi.
%selkeys voitti täsmällisyyden ei siis murtolukumuotoon
	\begin{alakohdat}
		\alakohta{$43{,}532$}
		\alakohta{$5{,}031$}
		\alakohta{$0{,}23$}
		\alakohta{$0{,}3002$}
		\alakohta{$0{,}101$}
	\end{alakohdat}
\begin{vastaus}
	\begin{alakohdat}
		\alakohta{$ \frac{43532}{1000}$}
		\alakohta{$ \frac{5031}{1000}$}
		\alakohta{$ \frac{23}{100}$}
		\alakohta{$ \frac{3002}{1000}$}
		\alakohta{$ \frac{101}{1000}$}
	\end{alakohdat}
\end{vastaus}
\end{tehtava}

\begin{tehtava}%sovteht tai vaikea tehtävä? sivevennyksen takia
Muuta murtoluvuksi ja sievennä.
%selkeys voitti täsmällisyyden ei siis murtolukumuotoon
	\begin{alakohdat}
		\alakohta{$0{,}01$}
		\alakohta{$0{,}0245$}
		\alakohta{$0{,}004$}
		\alakohta{$0{,}001004$}
	\end{alakohdat}
\begin{vastaus}
	\begin{alakohdat}
		\alakohta{$ \frac{1}{100}$}
		\alakohta{$ \frac{49}{200}$}
		\alakohta{$ \frac{1}{250}$}
		\alakohta{$ \frac{251}{250~000}$}
	\end{alakohdat}
\end{vastaus}
\end{tehtava}

\begin{tehtava}
Muuta murtoluvuksi.
	\begin{alakohdat}
		\alakohta{$0,77777\ldots$}
		\alakohta{$0,151515 \ldots$}
		\alakohta{$2,05\overline{631}$}
		\alakohta{$0,99999\ldots$}
	\end{alakohdat}
\begin{vastaus}
	\begin{alakohdat}
		\alakohta{$\frac{7}{9}$ }
		\alakohta{$\frac{15}{99}=\frac{5}{33}$}
		\alakohta{$\frac{205\ 426}{99\ 900} = \frac{102\ 713}{49\ 950}$}
		\alakohta{$\frac{9}{9} = 1$}
	\end{alakohdat}
\end{vastaus}
\end{tehtava}

\begin{tehtava}
Muuta desimaaliluvuksi.
	\begin{alakohdat}
		\alakohta{$\frac{151}{250}$}
		\alakohta{$\frac{251}{625}$}
		\alakohta{$\frac{386}{1\ 250}$}
		\alakohta{$\frac{493}{500}$}
	\end{alakohdat}
\begin{vastaus}
	\begin{alakohdat}
		\alakohta{$0,604$}
		\alakohta{$0,4016$}
		\alakohta{$0,3088$}
		\alakohta{$0,986$}
	\end{alakohdat}
\end{vastaus}
\end{tehtava}

\begin{tehtava}
Muuta desimaaliluvuksi.
	\begin{alakohdat}
		\alakohta{$\frac{42}{11}$}
		\alakohta{$\frac{37}{13}$}
		\alakohta{$\frac{38}{99}$}
		\alakohta{$\frac{14}{15}$}
	\end{alakohdat}
\begin{vastaus}
	\begin{alakohdat}
		\alakohta{$3,\overline{81}$}
		\alakohta{$2,\overline{846153}$}
		\alakohta{$0,\overline{38}$}
		\alakohta{$0,9\overline{3}$}
	\end{alakohdat}
\end{vastaus}
\end{tehtava}

\begin{tehtava}
	Tunnissa on 60 minuuttia ja minuutissa on 60 sekuntia. Muuta seuraavat ajat tunneiksi.
	\begin{alakohdat}
		\alakohta{73 minuuttia}
		\alakohta{649 sekuntia}
		\alakohta{15 minuuttia ja 50 sekuntia}
		\alakohta{42 minuuttia ja 54 sekuntia}
	\end{alakohdat}
	\begin{vastaus}
		\begin{alakohdat}
			\alakohta{1,21666... tuntia}
			\alakohta{0,1802777... tuntia}
			\alakohta{0,263888... tuntia}
			\alakohta{0,715 tuntia}
		\end{alakohdat}
	\end{vastaus}
\end{tehtava}

\end{tehtavasivu}

\laatikko{
Joissain yhteyksissä käytetään muitakin \emph{lukujärjestelmiä} kuin kymmenjärjestelmää. Esimerkiksi tietokoneet käyttävät kaksikantajärjestelmää eli \emph{binäärijärjestelmää}. Binäärijärjestelmässä jokainen luvun numero kertoo sen paikkaa vastaavan kakkosen potenssien määrän kymmenen potenssien sijasta.

Binäärijärjestelmässä esitetty luku merkitään yleensä kirjoittamalla pieni kakkonen luvun jälkeen, esim. $10,01_2$.
}

\begin{esimerkki}
$10,01_2 = 1 \cdot 2^1 + 0 \cdot 2^0 + 0 \cdot 2^{-1} + 1 \cdot 2^{-2} = 2,25_{10}$
\end{esimerkki}

\begin{tehtavasivu}

\begin{tehtava}
Muunna seuraavat binääriluvut kymmenjärjestelmään.
	\begin{alakohdat}
		\alakohta{$101,0_2$}
		\alakohta{$1,00101_2$}
		\alakohta{$100101,1101_2$}
	\end{alakohdat}
\begin{vastaus}
	\begin{alakohdat}
		\alakohta{$5,0_{10}$}
		\alakohta{$1,15625_{10}$}
		\alakohta{$37.8125_{10}$}
	\end{alakohdat}
\end{vastaus}
\end{tehtava}

\begin{tehtava}
Muunna seuraavat luvut binäärijärjestelmään.
	\begin{alakohdat}
		\alakohta{$7,0_{10}$}
		\alakohta{$2,5_{10}$}
		\alakohta{$11,1875_{10}$}
	\end{alakohdat}
\begin{vastaus}
	\begin{alakohdat}
		\alakohta{$111,0_2$}
		\alakohta{$10,1_2$}
		\alakohta{$1011,0011_2$}
	\end{alakohdat}
\end{vastaus}
\end{tehtava}

\begin{tehtava}
	Laske
	\begin{alakohdat}
		\alakohta{$1101_2 \cdot 100_2$}
		\alakohta{$1101001_2 \cdot 100000_2$}
		\alakohta{$10110_2 \cdot 0,01_2$.}
	\end{alakohdat}
	\begin{vastaus}
		\begin{alakohdat}
			\alakohta{$110100_2$}
			\alakohta{$110100100000_2$}
			\alakohta{$101,1_2$}
		\end{alakohdat}
	\end{vastaus}
\end{tehtava}

\end{tehtavasivu}
