\subsection*{Harjoituskoe 1}

%Tässä vielä työstettävää... >_>

\begin{description}
	\item[] Ei laskinta.
	\item[1.] Mainitse jokin \\
	(a) aina tosi yhtälö \\
	(b) joskus tosi yhtälö (ja milloin se on tosi) \\
	(c) aina epätosi yhtälö.
	\item[2.] Sievennä \\
	(a) $\sqrt{144}$ \qquad
	(b) $84^\frac{3}{2}$ \qquad
	(c) $ 6^\frac{3}{2}\cdot 8^\frac{2}{3} $ \qquad
	(d) $\frac{\sqrt{a}}{\sqrt{2}}(2a)^{\frac{-5}{2}}$.
	\item[3.] Muuta murtoluvuksi \\
	(a) $0,45$ \qquad
	(b) $0,33\ldots$ \qquad
	(c) $0,2857142\ldots$ \qquad
	(d) $0,388\ldots$.
	\item[4.] Sievennä \\
	(a) Eräs bakteerikanta kasvaa päivässä $75$:llä prosentilla. Montako prosenttia bakteerikannan alkuarvo $1 g$ on bakteerikannan määrästä kolmen päivän kuluttua? \\
	(b) Mursupuvun hinnasta puolet tulee valmistuskuluista. Valmistuskuluista $75\%$ on materiaalikuluja. Montako prosenttia materaalikulujen pitää laskea, että mursupuvun hinta $10\%$? \qquad
	\item[5.] Jaa luvut tekijöihin. Mitkä luvuista ovat alkulukuja? \\
	(a) $111$ \qquad
	(b) $75$ \qquad
	(c) $97$ \qquad
	(d) $360$.
	\item[6.] Ratkaise \\
	(a) $x^3 - \frac{14}{27} = \frac{20}{6}+\frac{14}{18}$ \qquad
	(b) $2(x-1)=\frac{3x}{4}$
	\item[7.] Tee näitä.  \\
	(a) Laske $f(2)$, kun $f(x)=5x^{3}-12x^{-2}-2$ \\
	(b) Tutki kuvaajasta, mitkä ovat funktion $f(x)=2x^2-3x-2$ nollakohdat ja lisäksi arvo, kun $x=1$. Millä muuttujan arvoilla $f$ saa arvon $3$? Laske $f(7)$.
	
	\begin{center}
		\begin{kuvaajapohja}{1.0}{-3}{5}{-4}{4}
			\kuvaaja{2*x**2-3*x-2}{$f(x)=2x^2-3x-2$}{black}
		\end{kuvaajapohja}
	\end{center}

	\item[8.] Kalle-Petterillä on suorakulmainen pala kartonkia, jonka pinta-ala on $36cm^2$, lyhyen sivu pituus on $4cm$ ja pitemmän sivun pituus on $x$ $cm$. 
	Hän haluaa taivuttaa kartongista putken niin että lyhyet sivut ovat yhdessä. 
	
	Laske putken tilavuus, kun tilavuuden kaava on $V=\pi\cdot r^2\cdot h$, jossa $V$ on tilavuus, $h$ on kartonkipalan lyhyen sivun pituus, ja $x=2\pi\cdot r$.    
\end{description}

\subsection*{Harjoituskoe 2}

\begin{description}
	\item[] Ei laskinta.
	\item[1.] Mainitse jokin \\
	(a) kokonaisluku, joka ei ole luonnollinen luku \\
	(b) rationaaliluku, joka ei ole kokonaisluku \\
	(c) reaaliluku, joka ei ole rationaaliluku.
	\item[2.] Ratkaise \\
	(a) $11x=77$ \qquad
	(b) $8x+174=50x$ \\
	(c) $\frac{5}{4}x-1=\frac{4}{5}x$ \qquad
	(d) $\sqrt{2}x+\sqrt{2}=2x$.
	\item[3.] Muuta a ja b desimaaliluvuksi sekä c ja d murtopotensseiksi sieventäen. \\
	(a) $\frac{3}{10}$ \qquad
	(b) $\frac{3}{16}$ \qquad
	(c) $a\cdot \sqrt[3]{a^4}$\qquad
	(d) $\frac{\sqrt[3]{a}}{\sqrt[3]{a}}-\frac{1}{\sqrt[3]{3a}}$ .
	\item[4.] Tuoreessa ananaksessa veden osuus on 80\% ananaksen massasta ja A-, B- ja C-vitamiinien yhteenlaskettu osuus $0,05\%$ massasta. Ananas kuivatetaan niin, että veden osuus laskee 8 prosenttiin ananaksen massasta. Kuinka suuri on A-, B- ja C-vitamiinien osuus kuivatun ananaksen massasta? (Luvut eivät ole faktuaalisia.)
	
	\item[5.] Pertsa ajaa kotoansa mummolaan tunnissa, jos ajovauhti on lupsakka $60 km/h$. \\
	(a) Nyt Pertsalla on kuitenkin kiire ja hän yrittää keretä mummolaan kahdessa kolmasosa tunnissa. Kuinka nopeasti Pertsan pitää ajaa? \\
	(b) Pertsan ajaessa tätä vauhtia, tielle ryntää yhtäkkiä kirahvi. Jos jarrutusmatka on suoraan verranollinen ajonopeuden neliöön, kuinka kaukana kirahvista täytyy Pertsan ruveta jarruttamaan, että ei käy huonosti?
	\item[6.] Sievennä \\
	(a) $\frac{a^2 b^2}{a}$, $a \neq 0$ \qquad
	(b) $3(a^2+1)-2(a^2-1)$ \\
	(c) $ab(a+2a)$ \qquad
	(d) $(a^3 b^2 c)^2$.
	\item[7.] Olkoon $f(t) = 35 \cdot 2^t$ bakteerien lukumäärä soluviljelmässä ajanhetkellä $t$ (sekuntia). Monenko sekunnin kuluttua bakteereita on yli 1000? Yhden sekunnin tarkkuus ylöspäin pyöristettynä riittää.
	\item[8.] Määritellään kahdelle järjestetylle lukunelikolle $(a, b, c, d)$ laskutoimitus $\odot$ seuraavasti: $(a_1, b_1, c_1, d_1) \odot (a_2, b_2, c_2, d_2) = (a_1 a_2 + b_1 c_2, a_1 b_2 + b_1 d_2, c_1 a_2 + d_1 c_2, c_1 b_2 + d_1 d_2)$. Laske $(1, 1, 1, 0) \odot (1, 1, 1, 0)$.

\end{description}
