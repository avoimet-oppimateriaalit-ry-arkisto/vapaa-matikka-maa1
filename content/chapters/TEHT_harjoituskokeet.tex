\subsection*{Harjoituskoe 1}

%Tässä vielä työstettävää... >_>

\begin{description}
	\item[] Ei laskinta.
	\item[1.] Mainitse jokin \\
	(a) aina tosi yhtälö \\
	(b) joskus tosi yhtälö (ja milloin se on tosi) \\
	(c) aina epätosi yhtälö.
	\item[2.] Sievennä \\
	(a) $\sqrt{144}$ \qquad
	(b) $84^\frac{3}{2}$ \qquad
	(c) $ 6^\frac{3}{2}\cdot 8^\frac{2}{3} $ \qquad
	(d) $\frac{\sqrt{a}}{\sqrt{2}}(2a)^{\frac{-5}{2}}$.
	\item[3.] Muuta murtoluvuksi \\
	(a) $0,45$ \qquad
	(b) $0,33\ldots$ \qquad
	(c) $0,2857142\ldots$ \qquad
	(d) $0,388\ldots$.
	\item[4.] Sievennä \\
	(a) Eräs bakteerikanta kasvaa päivässä $75$:llä prosentilla. Montako prosenttia bakteerikannan alkuarvo $1 g$ on bakteerikannan määrästä kolmen päivän kuluttua? \\
	(b) Mursupuvun hinnasta puolet tulee valmistuskuluista. Valmistuskuluista $75\%$ on materiaalikuluja. Montako prosenttia materaalikulujen pitää laskea, että mursupuvun hinta $10\%$? \qquad
	\item[5.] Jaa luvut tekijöihin. Mitkä luvuista ovat alkulukuja? \\
	(a) $111$ \qquad
	(b) $75$ \qquad
	(c) $97$ \qquad
	(d) $360$.
	\item[6.] Ratkaise \\
	(a) $x^3 - \frac{\sqrt[3]{81}}{\sqrt[3]{3}} = 61$ \qquad
	(b) $x=0$
	\item[7.] ölaglrypk aopmf anseof  \\
	(a) Laske $f(2)$, kun $f(x)=5x^{3}-12x^{-2}-2$ \\
	(b) Tutki kuvaajasta mikä on funktion $f(x)=x^3+x^2+\frac{1}{x}$, kun $f(4)$. Laske $f(-1)$.
	\item[8.] Kalle-Petterillä on suorakulmainen pala kartonkia, jonka pinta-ala on $36cm^2$, lyhyen sivu pituus on $4cm$ ja pitemmän sivun pituus on $x$ $cm$. 
	Hän haluaa taivuttaa kartongista putken niin että lyhyet sivut ovat yhdessä. 
	
	Laske putken tilavuus, kun tilavuuden kaava on $V=\pi\cdot r^2\cdot h$, jossa $V$ on tilavuus, $h$ on kartonkipalan lyhyen sivun pituus, ja $x=2\pi\cdot r$.    
\end{description}

\subsection*{Harjoituskoe 2}

\begin{description}
	\item[] Ei laskinta.
	\item[1.] Mainitse jokin \\
	(a) kokonaisluku, joka ei ole luonnollinen luku \\
	(b) rationaaliluku, joka ei ole kokonaisluku \\
	(c) reaaliluku, joka ei ole rationaaliluku.
	\item[2.] Ratkaise \\
	(a) $11x=77$ \qquad
	(b) $8x+174=50$ \\
	(c) $7x+7=5x$ \qquad
	(d) $6x+7=8x+1$.
	\item[3.] Muuta desimaaliluvuksi \\
	(a) $\frac{3}{10}$ \qquad
	(b) $\frac{3}{16}$ \qquad
	(c) $\frac{5}{9}$ \qquad
	(d) $\frac{5}{11}$.
	\item[4.] Tuoreessa ananaksessa veden osuus on 80\% ananaksen massasta ja A-, B- ja C-vitamiinien yhteenlaskettu osuus 0,05\% massasta. Ananas kuivatetaan niin, että veden osuus laskee 8 prosenttiin ananaksen massasta. Kuinka suuri on A-, B- ja C-vitamiinien osuus kuivatun ananaksen massasta? (Luvut eivät ole faktuaalisia.)
	\item[5.] Muuta seuraavat kymmenjärjestelmän luvut binääriluvuiksi. \\
	(a) $100$ \qquad
	(b) $128$.
	\item[6.] Sievennä \\
	(a) $\frac{a^2 b^2}{a}$, $a \neq 0$ \qquad
	(b) $3(a^2+1)-2(a^2-1)$ \\
	(c) $ab(a+2a)$ \qquad
	(d) $(a^3 b^2 c)^2$.
	\item[7.] Olkoon $f(t) = 35 \cdot 2^t$ bakteerien lukumäärä soluviljelmässä ajanhetkellä $t$ (sekuntia). Monenko sekunnin kuluttua bakteereita on yli 1000? Yhden sekunnin tarkkuus ylöspäin pyöristettynä riittää.
	\item[8.] Määritellään kahdelle järjestetylle lukunelikolle $(a, b, c, d)$ laskutoimitus $\odot$ seuraavasti: $(a_1, b_1, c_1, d_1) \odot (a_2, b_2, c_2, d_2) = (a_1 a_2 + b_1 c_2, a_1 b_2 + b_1 d_2, c_1 a_2 + d_1 c_2, c_1 b_2 + d_1 d_2)$. Laske $(1, 1, 1, 0) \odot (1, 1, 1, 0)$.

\end{description}
