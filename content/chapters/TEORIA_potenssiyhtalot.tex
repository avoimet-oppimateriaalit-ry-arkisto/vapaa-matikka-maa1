Ensimmäisen asteen yhtälöä $ax+b = 0$ monimutkaisempia yhtälöitä saadaan,
kun tuntematon $x$ korotetaan potenssiin.

\laatikko{\termi{potenssiyhtälö}{Potenssiyhtälö} on muotoa 
$
x^n=a
$,
oleva yhtälö, jossa $n$ on positiivinen kokonaisluku. Eksponentin $n$ arvoa kutsutaan potenssiyhtälön \termi{aste (potenssiyhtälö)}{asteeksi}.}

Potenssiyhtälöitä tarvitaan esimerkiksi tilanteissa, joissa lasketaan
korolle korkoa. Myös pinta-ala- ja tilavuuslaskuissa esiintyy potenssiyhtälöitä.

\begin{esimerkki}
\begin{alakohdat}
\alakohta{Yhtälö $27x^3=7$ on potenssiyhtälö, sillä jakamalla se puolittain luvulla $27$ saadaan $x^3 = \frac{7}{27}$.}
\alakohta{Yhtälö $2x^{4}-7=3$ on potenssiyhtälö, sillä se voidaan muokata muotoon $x^n = a$,
\begin{eqnarray*}
2x^{4} -7 &=& 3 \\
2x^{4} &=& 3+7 \\
x^{4} &=& \frac{10}{2} \\
x^{4} &=& 5.
\end{eqnarray*}}
\alakohta{Yhtälö $x^{\frac{3}{2}}=42$ ei ole potenssiyhtälö, sillä eksponentti $\frac{3}{2}$ ei ole kokonaisluku. Yhtälö voidaan kuitenkin kirjoittaa uuden tuntemattoman $z=x^{\frac{1}{2}}=\sqrt{x}$ avulla: tällöin saadaan potenssiyhtälö $z^3 = 42$.)}
\alakohta{Yhtälö $x^{-2}=44$ ei ole potenssiyhtälö, sillä $-2$ ei ole positiivinen kokonaisluku. Sekin voidaan kirjoittaa potenssiyhtälönä, kun merkitään $z=x^{-1}=\frac{1}{x}$. Tällöin saadaan yhtälö $z^2 = 44$.)}
\end{alakohdat}
\end{esimerkki}

\laatikko{Potenssiyhtälön ratkaiseminen:
\begin{alakohdat}
\alakohta{Jos potenssiyhtälön aste $n$ on parillinen ja $a \ge 0$, yhtälöllä on kaksi ratkaisua, $$ x = \pm \sqrt[n]{a} \textrm{.} $$ ($\pm$ tarkoittaa, että sekä positiivinen että negatiivinen arvo käyvät: $\pm 5$ tarkoittaa sekä lukuja $5$ että $-5$.)}
\alakohta{Jos aste on pariton, yhtälöllä on täsmälleen yksi ratkaisu,$ x = \sqrt[n]{a}.$}
\alakohta{Jos aste on parillinen ja $a < 0$, potenssiyhtälöllä ei ole yhtään ratkaisua.}
\end{alakohdat}
}

\begin{esimerkki}
\begin{alakohdat}
\alakohta{Potenssiyhtälön $x^3 = 100$ ratkaisu on $x=100^{\frac{1}{3}}=4{,}6416...$.}
\alakohta{ Potenssiyhtälöllä $x^4=50$ on kaksi ratkaisua $x=50^{\frac{1}{4}}=2{,}6591...$ ja $x=-50^{\frac{1}{4}}=-2{,}6591...$.}
\alakohta{Potenssiyhtälöllä $x^6 = -1$ ei ole ratkaisua, sillä $x^4 = (x^3)^2 \ge 0$ kaikille $x$.}
\end{alakohdat}

\begin{center}
\includegraphics[width=12cm]{pictures/xpot346.pdf}
\end{center}
\end{esimerkki}


\begin{esimerkki}
Suursijoittaja Nalle Mursulla on $5~000$ euroa ylimääräistä rahaa, jonka hän aikoo sijoittaa $30$ vuodeksi.  Nalle Mursu haluaa sijoittamansa pääoman kasvavan $100~000$ euroksi $30$ vuodessa.  Kuinka suuren vuotuisen korkokannan Nalle Mursu tarvitsee sijoitukselleen? 

	\begin{esimratk}
	Olkoon vuotuinen korkokanta $r$. Korkoa korolle -periaatteen nojalla $5~000$ euron sijoitus kasvaa $30$ vuodessa summaksi $5~000\cdot(1+r)^{30}$.  Merkitsemällä $x=1+r$ saamme yhtälön $5~000\cdot x^{30} = 100~000$.  Jakamalla yhtälö puolittain luvulla $5~000$ päädymme 
	potenssiyhtälöön
	$$
	x^{30} = 20~000,
	$$ 
	jonka ratkaisuksi saadaan $x=20~000^{\frac{1}{30}} = 1{,}39\ldots$. Näin
	ollen suursijoittaja Nalle Mursun vaatima korkokanta sijoitukselleen on noin $r=1-x=1-1,39=0,39=39~\%$.
	\end{esimratk}
\end{esimerkki}

%Potenssifunktiot ovat tapa katsoa potenssiyhtälöitä.

%\laatikko{\termi{potenssifunktio}{Potenssifunktio} on funktio, jossa muuttuja $x$ korotetaan %potenssiin $n$. Potenssifunktion lauseke siis on
%$$
%f(x) = x^n.
%$$
%}

%Potenssiyhtälöä $x^n=a$ voidaan nyt tarkastella tarkastelemalla funktiota %$f(x)=x^n$ ja sen kuvaajaa $y=f(x)$. Tällöin siis $y=a$, eli $y$-akseli vastaa %$a$:n arvoja.

%\missingfigure{Funktioiden $f(x)={x}^3$ ja $g(x)=x^4$ kuvaajat.}

%\laatikko{
%\begin{itemize}
%\item
%Olkoon $n$ pariton. Tällöin potenssifunktio $f(x)=x^n$ on kaikkialla %aidosti kasvava ja jatkuva. Tästä seuraa, että yhtälöllä $y=f(x)$ on aina tasan %yksi ratkaisu kaikilla $y$.    
%\item
%Olkoon $n$ parillinen. Tällöin potenssifunktio $f(x)=x^n$ on positiivinen, %symmetrinen, jatkuva ja aidosti kasvava, kun $x$ on positiivinen.  %Positiivisuudesta seuraa, että yhtälöllä $y=f(x)$ ei ole ratkaisuja, jos $y$ on %negatiivinen. Symmetriasta $f(x)=f(-x)$ seuraa, että jos $x$ on ratkaisu, niin %myös $-x$ on ratkaisu.  
%\end{itemize}
%}

Yhtälöt, jotka ovat muotoa $a\cdot x^n = b$ (joskus myös muotoa ($a\cdot x^n - b = 0$), ratkaistavuus riippuu useammasta seikasta. Mikäli $n$ on pariton, yhtälö on aina ratkaistavissa (eli sillä on reaalinen juuri):
\begin{align*}
a\cdot x^n &= b \\
x^n &= \frac{b}{a} \\
x &= \sqrt[n]{\frac{b}{a}}
\end{align*}

\begin{esimerkki}
$2x^3 + 16 = 0 \Leftrightarrow 2x^3 = -16 \Leftrightarrow x^3 = -8  \Leftrightarrow x = \sqrt[3]{-8} = -2 $
\end{esimerkki}

Mikäli $n$ on parillinen, yhtälö on ratkaistavissa jos ja vain jos $\frac{b}{a} \geq 0 $. Parillisella potenssifunktiolla voi olla yksi, kaksi tai ei yhtään ratkaisua.

\begin{esimerkki}
$x^2 = 0 \Leftrightarrow x = 0 \\
\Rightarrow$ Yhtälöllä on yksi ratkaisu.
\end{esimerkki}

\begin{esimerkki}
$x^2 - 9 = 0 \Leftrightarrow x^2 = 9 \Leftrightarrow x = \pm 3 \\
\Rightarrow$ Yhtälöllä on kaksi ratkaisua, sillä $3^2 = 9$ ja $(-3)^2 = 9$.
\end{esimerkki}

\begin{esimerkki}
$x^2 + 9 = 0 \Leftrightarrow x^2 = -9 \Leftrightarrow x = \sqrt{-9} \\
\Rightarrow$ Yhtälöllä ei ole reaalista ratkaisua.
\end{esimerkki}
