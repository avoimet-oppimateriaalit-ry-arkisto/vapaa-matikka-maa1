\laatikko{
Ensimmäisen asteen yhtälöksi kutsutaan yhtälöä, joka on esitettävissä muodossa $ax+b=0$, jossa $a \neq 0$.
}

Yhtälötyypin nimi tulee siitä, että korkein potenssi, johon tuntematon $x$
yhtälössä korotetaan, on $1$.

\begin{esimerkki}
Muun muassa seuraavat yhtälöt ovat ensimmäisen asteen yhtälöitä:
\begin{alakohdat}
\alakohta{$2x = 4$}
\alakohta{$5x+3 = 0$}
\alakohta{$x+2 = 3x-4$}
\end{alakohdat}
\end{esimerkki}

\laatikko{
Ensimmäisen asteen yhtälö ratkaistaan siirtämällä ensin tuntemattoman $x$ sisältävät termit yhtälön vasemmalle puolelle ja vakiotermit oikealle puolelle. Tämän jälkeen yhtälö jaetaan puolittain tuntemattoman $x$ kertoimella.
}
% FIXME ruma todo-looda taas pois\todo{onko 1. asteen yhtälönratkaisun laatikossa käytettävä ''siirtää toiselle puolelle'' ehkä kuitenkin voi olla noin}

\begin{esimerkki}
Yhtälön $7x+4=4x+7$ ratkaisu saadaan seuraavasti:
\begin{align*}
7x+4 &= 4x+7 & &| \, \text{Vähennetään molemmilta puolilta $4x$.} \\
3x+4 &= 7 & &| \, \text{Vähennetään molemmilta puolilta 4.} \\
3x &= 3 & &| \, \text{Jaetaan molemmat puolet luvulla 3.} \\
x &= 1 & & \\
\end{align*}

\textbf{Vastaus.} $x=1$
\end{esimerkki}

Ensimmäisen asteen yhtälöllä on aina täsmälleen yksi ratkaisu.

Kaikki muotoa $ax+b=cx+d$ olevat yhtälöt, joissa $a \neq c$, ovat ensimmäisen asteen yhtälöitä. Tämä voidaan todistaa seuraavasti:

\begin{align*}
ax+b &= cx+d & &| \, \text{Vähennetään molemmilta puolilta $cx+d$}. \\
ax+b - (cx+d) &= 0 & &| \, \text{Järjestellään termejä uudelleen.} \\
ax - cx + b - d &= 0 & &| \, \text{Otetaan yhteinen tekijä.} \\
(a-c)x + (b-d) &= 0 & &
\end{align*}

Tämä on määritelmän mukainen ensimmäisen asteen yhtälö, koska $a \neq c$.

\begin{esimerkki}
Yleinen lähestymistapa muotoa $ax+b = cx+d$ olevien yhtälöiden ratkaisuun: \\
(1) Vähennä molemmilta puolilta $cx$. Saat yhtälön $(a-c)x + b = d$. \\
(2) Vähennä molemmilta puolilta $b$. Saat yhtälön $(a-c)x = d-b$. \\
(3) Jaa molemmat puolet lausekkeella $(a-c)$. Saat yhtälön ratkaistuun muotoon $x = \frac{d-b}{a-c}$.
\end{esimerkki}
