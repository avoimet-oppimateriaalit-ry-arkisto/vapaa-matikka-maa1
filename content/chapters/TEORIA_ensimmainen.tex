\laatikko{
Ensimmäisen asteen yhtälöksi kutsutaan yhtälöä, joka on esitettävissä muodossa $ax+b=0$, jossa $a \neq 0$.
}

Yhtälötyypin nimi tulee siitä, että korkein potenssi, johon tuntematon $x$
yhtälössä korotetaan, on $1$. (Sillä $x^{1}=x$.)

\begin{esimerkki}
Muun muassa seuraavat yhtälöt ovat ensimmäisen asteen yhtälöitä:
\begin{alakohdat}
\alakohta{$2x = 4$}
\alakohta{$5x+3 = 0$}
\alakohta{$x+2 = 3x-4$}
\end{alakohdat}
\end{esimerkki}

%pitää työstää tätä selitystä...
%\laatikko[Yhtälöiden muokkaaminen]{
%	\begin{description}
%		\item[Laskutoimitukset], joita ovat
%		
%			\begin{itemize}
%				\item{Puolittain lisääminen ja vähentäminen: yhtälön molemmille puolille voidaan lisätä tai molemmilta puolilta voidaan vähentää luku tai muuttuja. 
%				Esimerkiksi yhtälö $3x+5 = 3$ saadaan näin muotoon $3x = -2$.}
%				\item{Puolittain kertominen tai jakaminen: yhtälön molemmat puolet voidaan kertoa tai jakaa nollasta poikkeavalla luvulla, muuttujalla tai lausekkeella. Muuttujalla tai lausekkeella jakaessa täytyy ottaa huomioon ja merkitä, että yhtälö ei ole määritelty, kun jakaja on nolla.
%				Esimerkiksi kertomalla yhtälön $2x = 4$ molemmat puolet luvulla $\frac{1}{2}$ saadaan yhtälö $x = 2$.}
%				\item{Huomaa, että itseasiassa vähennyslasku on negatiivisen luvun summaamista [$a-b=a+(-b)$], 
%				ja jakaminen on jakajan käänteisluvulla kertomista [$a:b=a\cdot{\frac{1}{b}}=\frac{a}{b}$]}				
%			\end{itemize}
%		\item[Sieventäminen laskusääntöjä käyttäen] Yhtälön lausekkeista voi esimerkiksi hakea yhteistä tekijää, avata sulkuja tai supistaa, 
%		ja pyrkiä täten saamaan lauseke yksinkertaisempaan muotoon. Esimerkiksi yhtälöä $3(x-2)=5(x+6)$ kannattaa lähteä ratkaisemaan avaamalla ensin 
%		sulut soveltamalla tuttuja laskusääntöjä. Näin yhtälö saadaan muotoon $3x-6=5x+30$, jonka jälkeen on helppoa jatkaa käyttämällä laskutoimituksia 
%		tuntemattoman $x$ ratkaisemiseksi. Yhtälön ratkaisu on $x=-18$.
%	\end{description}

}

% \laatikko{
% Ensimmäisen asteen yhtälö ratkaistaan siirtämällä ensin tuntemattoman $x$ sisältävät termit yhtälön vasemmalle puolelle ja vakiotermit oikealle puolelle. Tämän jälkeen yhtälö jaetaan puolittain tuntemattoman $x$ kertoimella $a$.
% }
% % FIXME ruma todo-looda taas pois\todo{onko 1. asteen yhtälönratkaisun laatikossa käytettävä ''siirtää toiselle puolelle'' ehkä kuitenkin voi olla noin}

\begin{esimerkki}
Yhtälön $7x+4=4x+7$ ratkaisu saadaan seuraavasti:
\begin{align*}
7x+4 &= 4x+7 & &| \, \text{Vähennetään molemmilta puolilta $4x$.} \\
3x+4 &= 7 & &| \, \text{Vähennetään molemmilta puolilta 4.} \\
3x &= 3 & &| \, \text{Jaetaan molemmat puolet luvulla 3.} \\
x &= 1 & & \\
\end{align*}

\textbf{Vastaus.} $x=1$
\end{esimerkki}

Ensimmäisen asteen yhtälöllä on aina täsmälleen yksi ratkaisu.

Kaikki muotoa $ax+b=cx+d$ olevat yhtälöt, joissa $a \neq c$, ovat ensimmäisen asteen yhtälöitä. Tämä voidaan todistaa seuraavasti:

\begin{align*}
ax+b &= cx+d & &| \, \text{Vähennetään molemmilta puolilta $cx+d$}. \\
ax+b - (cx+d) &= 0 & &| \, \text{Järjestellään termejä uudelleen.} \\
ax - cx + b - d &= 0 & &| \, \text{Otetaan yhteinen tekijä.} \\
(a-c)x + (b-d) &= 0 & &
\end{align*}

Tämä on määritelmän mukainen ensimmäisen asteen yhtälö, koska $a \neq c$.

\begin{esimerkki}
Yleinen lähestymistapa muotoa $ax+b = cx+d$ olevien yhtälöiden ratkaisuun: \\
(1) Vähennä molemmilta puolilta $cx$. Saat yhtälön $(a-c)x + b = d$. \\
(2) Vähennä molemmilta puolilta $b$. Saat yhtälön $(a-c)x = d-b$. \\
(3) Jaa molemmat puolet lausekkeella $(a-c)$. Saat yhtälön ratkaistuun muotoon $x = \frac{d-b}{a-c}$. 
\end{esimerkki}
