\subsection*{Jaollisuus ja tekijöihinjako}

\laatikko{
	\textbf{Määritelmä 1}
	
	Kokonaisluku $a$ on jaollinen kokonaisluvulla $b \neq 0$, jos on olemassa kokonaisluku $c$ niin, että $a = b \cdot c$.
	Tällöin sanotaan myös, että $b$ on $a$:n tekijä.
	
	\vspace*{24pt}
	
	\textbf{Määritelmä 2}
	Luku $a$ on jaollinen luvulla $b \neq 0$, mikäli $\frac{a}{b}$ on kokonaisluku.
	
	Määritelmät ovat yhtäpitäviä.
}
    
\begin{esimerkki}
	\begin{alakohdat}
		\alakohta{Luku $-12$ on jaollinen luvulla $3$, sillä $-12 = 3 \cdot (-4)$.}
		\alakohta{$-12$ ei ole jaollinen luvulla $5$, sillä ei ole kokonaislukua, joka kerrottuna viidellä olisi $12$.}
	\end{alakohdat}
\end{esimerkki}
   
\begin{center}
\includegraphics[scale=0.85]{pictures/Kuva2-4-3x4.pdf}
\end{center}
    
Kaikki luvut ovat jaollisia itsellään ja luvulla $1$. Esimerkiksi $7=7 \cdot 1=1 \cdot 7$, joten $7$ on jaollinen luvuilla $1$ ja $7$.
    
\laatikko{
	\termi{alkuluku}{Alkuluku} on ykköstä suurempi kokonaisluku, joka ei ole jaollinen muilla positiivisilla kokonaisluvuilla kuin luvulla $1$ ja itsellään.
}
    
Esimerkiksi luvut 2, 3, 5, 7, 11, 13, 17 ja 19 ovat alkulukuja. Jos luvun tekijä on alkuluku, sitä kutsutaan alkutekijäksi.
    
\laatikko{
    \textbf{Aritmetiikan peruslause}
	
    Jokainen ykköstä suurempi kokonaisluku voidaan esittää (termien järjestystä lukuunottamatta) yksikäsitteisesti alkulukujen tulona.
	(Jokainen ykköstä suurempi kokonaisluku voidaan jakaa alkutekijöihin (termien järjestystä lukuunottamatta) yksikäsitteisellä tavalla.)
}
    
Aritmetiikan peruslause todistetaan kurssilla Logiikka ja lukuteoria.
    
Esimerkiksi luku $84$ voidaan kirjoittaa muodossa $2\cdot 2\cdot 3\cdot 7$. Havaitaan, että $2$, $3$, ja $7$ ovat kaikki alkulukuja. Aritmetiikan peruslauseen nojalla tiedetään, että tämä on ainoa tapa kirjoittaa $84$ alkulukujen tulona -- mahdollista kerrottavien termien järjestyksen vaihtoa lukuunottamatta.
    
Luvun alkutekijät voi löytää etsimällä luvulle ensin jonkin esityksen kahden luvun tulona. Näiden kahden luvun ei tarvitse olla alkulukuja. Sen jälkeen sama toistetaan näille kahdelle luvulle ja edelleen aina uusille luvuille, kunnes jäljellä on vain alkulukuja.

\begin{tehtavasivu}

\begin{tehtava}
	Mitkä seuraavista luvuista ovat jaollisia luvulla $4$?
	Jos luku $a$ on jaollinen luvulla $4$, kerro, millä kokonaisluvulla $b$ pätee $a = 4 \cdot b$. \\
	\begin{alakohdatrivi}
		\alakohta{$1$}
		\alakohta{$12$}
		\alakohta{$13$}
		\alakohta{$2$}
		\alakohta{$-20$}
		\alakohta{$0$}
	\end{alakohdatrivi}
	\begin{vastaus}
		\begin{alakohdat}
			\alakohta{Ei ole jaollinen luvulla $4$}
			\alakohta{On jaollinen luvulla $4$, $12 = 4 \cdot 3$}
			\alakohta{Ei ole jaollinen luvulla $4$}
			\alakohta{Ei ole jaollinen luvulla $4$}
			\alakohta{On jaollinen luvulla $4$, $-20 = 4 \cdot (-5)$}
			\alakohta{On jaollinen luvulla $4$, $0 = 4 \cdot 0$} 
		\end{alakohdat}
    \end{vastaus}
\end{tehtava}

\begin{tehtava}
    Jaa seuraavat luvut alkutekijöihin. \\
	\begin{alakohdatrivi}
		\alakohta{$12$}
		\alakohta{$15$}
		\alakohta{$28$}
		\alakohta{$30$}
		\alakohta{$64$}
		\alakohta{$90$}
		\alakohta{$100$}
	\end{alakohdatrivi}
    \begin{vastaus}
		\begin{alakohdat}
			\alakohta{$12 = 2^2 \cdot 3$}
			\alakohta{$15 = 3 \cdot 5$}
			\alakohta{$28 = 2^2 \cdot 7$}
			\alakohta{$30 = 2 \cdot 3 \cdot 5$}
			\alakohta{$64 = 2^6$}
			\alakohta{$90 = 2 \cdot 3^2 \cdot 5$}
			\alakohta{$100 = 2^2 \cdot 5^2$}
		\end{alakohdat}
    \end{vastaus}
\end{tehtava}

\end{tehtavasivu}
