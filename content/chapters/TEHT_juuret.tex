\begin{tehtavasivu}

\paragraph*{Opi perusteet}


\begin{tehtava}
Laske.
a) $\sqrt{64}$ \quad b) $\sqrt{-64}$ \quad c) $\sqrt[3]{64}$ \quad d) $\sqrt[3]{-64}$

\begin{vastaus}
a) 8 b) Ei määritelty c) 4 d) -4
\end{vastaus}
\end{tehtava}

\begin{tehtava}
Laske.
a) $\sqrt[4]{81}$ \quad b) $\sqrt[4]{-81}$ \quad c) $\sqrt[5]{32}$ \quad d) $\sqrt[5]{-32}$

\begin{vastaus}
a) 3 b) Ei määritelty c) 2 d) -2
\end{vastaus}
\end{tehtava}

%Tehtävät laatinut Aleksi Sipola 10.11.2013
%Ratkaisut tehnyt Aleksi Sipola 10.11.2013
\begin{tehtava}
Laske.
a) $\sqrt{81}$ \quad b) $\sqrt[3]{27}$ \quad c) $\sqrt[4]{-81}$ \quad d) $\sqrt[8]{256}$

\begin{vastaus}
a) 9 b) 3 c) Ei määritelty d) 2
\end{vastaus}
\end{tehtava}

%Tehtävät laatinut Aleksi Sipola 09.11.2013
%Ratkaisut tehnyt Aleksi Sipola 09.11.2013
\begin{tehtava} Laske. 
a) $\sqrt{2}\sqrt{8}$  \quad b)  $\sqrt{3}\sqrt{5}$ \quad c)  $\sqrt{3}\sqrt{27}$ \quad  d) $\sqrt[3]{3}\sqrt[3]{25} $
\begin{vastaus}
a) $4$ \quad b) $\sqrt{15}$  \quad Ei sievene enempää.  \quad c) $9$ \quad d) $5$
\end{vastaus}
\end{tehtava}


%Tehtävät laatinut Aleksi Sipola 09.11.2013
%Ratkaisut tehnyt Aleksi Sipola 09.11.2013
\begin{tehtava} Laske. 
a) $ \frac{\sqrt{8}}{\sqrt{2}}$  \quad b)   $ \frac{\sqrt[3]{27}}{\sqrt[3]{8}}$   \quad c)  $ \frac{\sqrt{3} \cdot \sqrt{6}}{\sqrt{2}}$ \quad d) $ \frac {\sqrt[3]{2}}{\sqrt[3]{16}} \cdot \sqrt[3]{4}$ 
\begin{vastaus}
a) $2$ \quad b) $\frac{3}{2}$  \quad c) $3$ \quad d) $\frac{1}{\sqrt{2}}$
\end{vastaus}
\end{tehtava}


\begin{tehtava}
Laske $\sqrt{8100}$ päässä ajattelemalla juurrettava luku sopivana tulona.
\begin{vastaus}
$\sqrt{8100}=\sqrt{81}\sqrt{100}=9\cdot 10$.
\end{vastaus}
\end{tehtava}

%Antti, lisää tämä
\begin{tehtava}
Laske seuraavat neliöjuuret laskimella.\\
a) $\sqrt{320}$,\ b) $\sqrt{15}$,\ c) $\sqrt{71}$
\begin{vastaus}
a) $17{,}89$ b) $3{,}87$ c) $8{,}43$
\end{vastaus}
\end{tehtava}


\begin{tehtava}
Oletetaan että suorakaiteen leveyden suhde korkeuteen on $2$ ja suorakaiteen pinta-ala on $10$. Mikä on suorakaiteen 
leveys ja korkeus?
\begin{vastaus}
Suorakaide muodostuu kahdesta vierekkäisestä neliöstä, joiden pinta-ala on $5$. Tämän neliön sivun pituus on $\sqrt{5}$.
Siis suorakaiteen korkeus on $\sqrt{5}$ ja leveys $2\sqrt{5}$.
\end{vastaus}
\end{tehtava}


%Tehtävät laatinut Aleksi Sipola 09.11.2013
%Ratkaisut tehnyt Aleksi Sipola 09.11.20133
\begin{tehtava}
Pienen kuution sivu on puolet ison kuution sivusta. Kuinka pitkä pienen kuution sivu on, jos ison kuution tilavuus on 64 yksikköä? \\
\begin{vastaus}
Pienen kuution sivu on $\sqrt[3]{64}/2=2$ 
\end{vastaus}
\end{tehtava}



\begin{tehtava}
%Laatinut Jaakko Viertiö 2013-11-10
 Sievennä.
	\begin{alakohdat}
		\alakohta{$\sqrt{\sqrt[3]{27}}$}
		\alakohta{$\sqrt[3]{{\sqrt[16]{{\sqrt{2}}^{16}}}^2}$}
		\alakohta{${\sqrt{\sqrt[3]{{\sqrt[4]{\sqrt[5]{{\sqrt[6]{42}}^{22}}}}^2}}}^3$}
	\end{alakohdat}
	  \begin{vastaus}
		\begin{alakohdat}
		 \alakohta{$\sqrt{3}$}
		 \alakohta{$\sqrt[3]{2}$}
		 \alakohta{${\sqrt[60]{42}}^{11}$}
		\end{alakohdat}
	\end{vastaus}
\end{tehtava}


%Tehtävät laatinut Aleksi Sipola 09.11.2013
%Ratkaisut tehnyt Aleksi Sipola 09.11.2013
\begin{tehtava} Kumpi juurista on suurempi? Arvaa ja laske.
a) $\sqrt[3]{8}$ vai $\sqrt{16}$ \quad b)  $\sqrt{8}$ vai $2\sqrt{2}$  \quad c) $\sqrt[3]{64}$ vai $\sqrt[5]{32}$ \quad d) $\sqrt[2]{121}$ vai $\sqrt[5]{243}$ 
\begin{vastaus}
a) $\sqrt[3]{8}=2$   <   $\sqrt{16}=4$ \quad b) $\sqrt{8}=\sqrt{4}\sqrt{2}=2\sqrt{2}$ \quad = \quad $2\sqrt{2}$ \quad c) $\sqrt[3]{64}=4$   >   $\sqrt[5]{32}=2$ \quad d) $\sqrt[2]{121}=11$ \quad  > \quad $\sqrt[5]{243}=3$ 
\end{vastaus}
\end{tehtava}

\paragraph*{Hallitse kokonaisuus}



\begin{tehtava}
%Tehtävän laatinut Johanna Rämö 9.11.2013.
%Ratkaisun tehnyt Johanna Rämö 9.11.2013.
        Selvitä ilman laskinta, kumpi luvuista $3\sqrt{2}$ ja $2\sqrt{3}$ on suurempi. 
       
        \begin{vastaus}
        Korotetaan luvut toiseen potenssiin: $(3\sqrt{2})^2=3^2\cdot\sqrt{2}^2=9 \cdot 2=18$ ja $(2\sqrt{3})^2=2^2\cdot\sqrt{3}^2=4 \cdot 3=12$. Koska $3\sqrt{2}$ ja $2\sqrt{3}$ ovat molemmat lukua 1 suurempia, voidaan niiden keskinäinen suuruusjärjestys lukea neliöiden suuruusjärjestyksestä. Edellisten laskujen perusteella $(3\sqrt{2})^2 > (2\sqrt{3})^2$, joten $3\sqrt{2} > 2\sqrt{3}$.
        \end{vastaus}
\end{tehtava}

\begin{tehtava}
%Tehtävän laatinut Johanna Rämö 9.11.2013.
%Ratkaisun tehnyt Johanna Rämö 9.11.2013.
        Mitä tapahtuu neliöjuuren arvolle, kun juurrettavaa kerrotaan luvulla 100? Miten tulos yleistyy?
       
        \begin{vastaus}
        Neliöjuuri kasvaa 10-kertaiseksi. Reaalilukujen $a$ ja $100a$ neliöjuuret ovat nimitäin $\sqrt{a}$ ja $\sqrt{100a}=\sqrt{100}\sqrt{a}=10\sqrt{a}$.
        
        Yleisesti jos juurrettavaa kerrotaan reaaliluvulla $k$, neliöjuuri kasvaa $\sqrt{k}$-kertaiseksi.
        \end{vastaus}
\end{tehtava}

\begin{tehtava}
%Tehtävän laatinut Johanna Rämö 9.11.2013.
%Ratkaisun tehnyt Johanna Rämö 9.11.2013.
        Suorakulmion muotoisen levyn mitat ovat $\text{230 cm} \times \text{250 cm}$. Mahtuuko se sisään oviaukosta, joka on 90 cm leveä ja 205 cm korkea?
       
        \begin{vastaus}
        Ei mahdu. Suurin mahdollinen levy, joka mahtuu ovesta sisään on Pythagoraan lauseen perusteella leveydeltään $\sqrt{90^2+205^2}\approx 224$ cm.
        \end{vastaus}
\end{tehtava}


\begin{tehtava}
Etsi luku $a>0$ jolle $a^4=83521$.
\begin{vastaus}
$a=\sqrt{\sqrt{83521}}$.
\end{vastaus}
\end{tehtava}



\begin{tehtava}
Laske luvun $10$ potensseja: $10^1, 10^2, 10^3, 10^4, \ldots$ Kuinka monta nollaa on luvussa $10^n$? Laske sitten $\sqrt[6]{1~000~000}$ ja $\sqrt[10]{10~000~000~000}$.

\begin{vastaus}
$10^1 = 10, 10^2 = 100, 10^3 = 1~000, 10^4 = 10~000$. Luvussa $10^n$ on $n$ kappaletta nollia. Niinpä $\sqrt[6]{1~000~000} = 10$ ja $\sqrt[10]{10~000~000~000} = 10$.
\end{vastaus}
\end{tehtava}

\begin{tehtava}
Onko annettu juuri määritelty kaikilla luvuilla $a$? Millaisia arvoja juuri voi saada luvusta $a$ riippuen?\\
a) $\sqrt[4]{a^2}$ \quad b) $\sqrt[4]{-a^2}$ \quad c) $\sqrt[4]{(-a)^2}$ \quad d) $- \sqrt[4]{a^2}$

\begin{vastaus}

\begin{alakohdat}
	\alakohta{Juuri on määritelty kaikilla luvuilla $a$, koska kaikkien lukujen neliöt ovat vähintään nolla. Vastaus on aina epänegatiivinen.}
	\alakohta{Juuri on määritelty vain luvulla $a = 0$. Muilla $a$:n arvoilla $-a^2$ on negatiivinen, jolloin parillinen juuri ei ole määritelty. Ainoa vastaus, joka voidaan saada, on siis $\sqrt[4]{0} = 0$.}
	\alakohta{Juuri on määritelty kaikilla luvuilla $a$, koska $(-a)^2$ on aina vähintään nolla. Vastaus on aina epänegatiivinen.}
	\alakohta{Juuri on määritelty kaikilla luvuilla $a$, koska kaikkien lukujen neliöt ovat vähintään nolla. Vastaus on aina ei-positiivinen, koska $\sqrt[4]{a^2}$ on aina epänegatiivinen.}
\end{alakohdat}
\end{vastaus}
\end{tehtava}

\begin{tehtava}
Onko annettu juuri määritelty kaikilla luvuilla $a$? Millaisia arvoja juuri voi saada luvusta $a$ riippuen?\\
a) $\sqrt[5]{a^2}$ \quad b) $\sqrt[5]{a^3}$ \quad c) $\sqrt[5]{-a^2}$ \quad d) $- \sqrt[5]{a^2}$

\begin{vastaus}

\begin{alakohdat}
	\alakohta{Juuri on määritelty kaikilla luvuilla $a$. Vastaus on aina epänegatiivinen.}
	\alakohta{Juuri on määritelty kaikilla luvuilla $a$. Vastaus voi olla mikä tahansa luku.}
	\alakohta{Juuri on määritelty kaikilla luvuilla $a$. Vastaus on aina ei-positiivinen.}
	\alakohta{Juuri on määritelty kaikilla luvuilla $a$. Vastaus on aina ei-positiivinen.}
\end{alakohdat}
\end{vastaus}
\end{tehtava}

\begin{tehtava}
Sievennä $(3+\sqrt{3}x)^4:(\sqrt{3}+x)^3$.
\begin{vastaus}
$3 + \sqrt{3}x$
\end{vastaus}
\end{tehtava}

\begin{tehtava}
Ajatellaan suorakulmaista hiekkakenttää, jonka pinta-ala on 1 aari ($100~\mathrm{m}^2$). Lyhyemmän ja pidemmän sivujen 
pituuksien suhde on 4:3. Laske Pythagoraan lauseen avulla matka hiekkakentän kulmasta kauimmaisena olevaan kulmaan.
\begin{vastaus}
$\frac{4}{3}x^2=100$ joten $x = \sqrt{\frac{300}{4}}$. 
Hypotenuusa: $\sqrt{x^2 + (\frac{4}{3}x)^2}=\sqrt{\frac{300}{4}+\frac{16}{9}\cdot \frac{300}{4}}
=\frac{5}{3}\sqrt{\frac{300}{4}}\approx 14{,}4$.
\end{vastaus}
\end{tehtava}

%Piilotin tämän tehtävän kommenteilla, koska se vaatii vielä tässä vaiheessa läpikäymättömien kaavojen osaamista
%\begin{tehtava}
%(YO 1888 1) Mikä on a/b:n arvo, jos \\
%$ (\sqrt{a}+\sqrt{b}):(\sqrt{a}-\sqrt{b})=\sqrt{2}$ ?
%\begin{vastaus}
%$(3-2\sqrt{2}):(3+2\sqrt{2})$
%\end{vastaus}
%\end{tehtava}






%Tehtävät laatinut Aleksi Sipola 09.11.2013
%Ratkaisut tehnyt Aleksi Sipola 09.11.2013
\begin{tehtava} Laske. 
a) $ \frac{\sqrt{15}}{\sqrt{7}} \cdot  \frac{\sqrt{27}}{\sqrt{35}}$  \quad b)  $ \frac{\sqrt{64}-\sqrt{9}}{\sqrt{5}}$   \quad c)  $ \frac{2 \cdot \sqrt[3]{32}}{\sqrt[3]{4}}$ \quad 
\begin{vastaus}
a) $9/7$ \quad b) $\sqrt{5}$ \quad c) $8$ \quad
\end{vastaus}
\end{tehtava}

%Tehtävät laatinut Aleksi Sipola 09.11.2013
%Ratkaisut tehnyt Aleksi Sipola 09.11.2013
\begin{tehtava} Etsi positiivinen rationaaliluku a, jolla
a) $\sqrt{a} \sqrt{2} = 4$  \quad b)   $ \sqrt{a}\cdot{\sqrt{3}} =9 $   \quad 
\begin{vastaus}
a) $a=8$ \quad b) $a=27$ \quad 
\end{vastaus}
\end{tehtava}


%Tehtävät laatinut Aleksi Sipola 09.11.2013
%Ratkaisut tehnyt Aleksi Sipola 09.11.2013
\begin{tehtava} Etsi jotkin erisuuret ykköstä suuremmat luonnolliset luvut a ja b, jolla 
a) $\frac{\sqrt{a}}{\sqrt{b}}$ on jokin luonnollinen luku  \quad b) $\sqrt[b]{a}=2$   
\begin{vastaus}
a) Esimerkiksi $a=27$ \quad ja \quad $b=3$  \quad b) Esimerkiksi $a=16$ \quad ja \quad $b=4$ 
\end{vastaus}
\end{tehtava}



\paragraph*{Sekalaisia tehtäviä}

%Tehtävät laatinut Aleksi Sipola 09.11.2013
%Ratkaisut tehnyt Aleksi Sipola 09.11.2013
\begin{tehtava} Laske laskimella likiarvot viiden merkitsevän luvun tarkkuudella. Mikä yhteys luvuilla on?
a) $ \sqrt{1^2+1^2}$ \quad b)  $ \frac {\sqrt{3^2+3^2}}{3}$    \quad c)  $ \frac {\sqrt{9^2+9^2}}{9}$  \quad 
\begin{vastaus}
Luku a) $1.41421$ \quad b) $1.41421$ \quad c) $1.41421$ \quad on tietysti $\sqrt{2}$ ja annetut laskut mukailevat neliön halkaisijan suhdetta sivuun.
\end{vastaus}
\end{tehtava}


%\begin{tehtava}
%Nämä eivät taida aivan sopia vielä tähän vaiheeseen ja aihealueeseen
%Tehtävän laatinut Johanna Rämö 9.11.2013.
%Ratkaisun tehnyt Johanna Rämö 9.11.2013.
%Osoita, että seuraavat laskukaavat eivät päde. Tee se etsimällä konkereettiset reaaliluvut $a$ ja $b$, joilla yhtälö ei päde.
%        \begin{alakohdat}
%        \alakohta{$(a+b)^2=a^2+b^2$}
%        \alakohta{$\sqrt{a+b}=\sqrt{a}+\sqrt{b}$}
%        \alakohta{$\dfrac{3a}{3b+c}=\dfrac{a}{b+c}$}
%        \end{alakohdat}
%        
%        \begin{vastaus}
%        \begin{alakohdatrivi}
%            \alakohta{Kaava ei päde, sillä esimerkiksi $(1+2)^2=3^2=9$, mutta $1^2+2^2=1+4=5$.}
%            \alakohta{Kaava ei päde, sillä esimerkiksi $\sqrt{1+1}=\sqrt{2}$, mutta $\sqrt{1}+\sqrt{1}=1+1=2$.}
%            \alakohta{Kaava ei päde, sillä esimerkiksi
%            $$\frac{3 \cdot 2}{3 \cdot 1+1}=\frac{6}{5}$$,
%            mutta $$\frac{2}{1+1}=1$$.}
%        \end{alakohdatrivi}
%        \end{vastaus}
%\end{tehtava}


%Tehtävät laatinut Aleksi Sipola 09.11.2013
%Ratkaisut tehnyt Aleksi Sipola 09.11.2013
\begin{tehtava} Etsi jotkin erisuuret luonnolliset luvut a ja b, jolla 
$\frac{\sqrt{\sqrt[a]{b}}}{\sqrt{a}}=1$ 
\begin{vastaus}
Esimerkiksi $a=3$, \quad $b=27$ 
\end{vastaus}
\end{tehtava}



\end{tehtavasivu}
