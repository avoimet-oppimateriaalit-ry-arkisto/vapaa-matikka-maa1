% vastaukset voisi olla jossain takana eikä tässä ettei kovin moni lukisi niitä ennen testiä

% FIXME: On mietittävä, mikä on lähtötasotestin tarkoitus. Tarvitaanko sitä tässä oppikirjassa vai ei. Luultavasti voisi poistaa. Onko sen teettäminen järkevää ajankäyttöä? Mikä on sen rooli? Kannustava alkuverryttely, opettajalle koko vuosiluokan tarkasteluun työkalu vai jotain muuta?
% no ei se negatiivista voi olla, että tämä on. Käyttäkööt, jos käyttävät. Hommas selvä. –Joonas%

% FIXME: Muokataan tehtäväympäristöä niin, etteivät nämä jatka vanhaa tehtävänumerointia.

\begin{tehtava}
	Laske. Merkitse välivaiheet näkyviin. 
	\begin{alakohdat}
		\alakohta{$2+3\cdot 2-1\cdot5$}
		\alakohta{$(2+1)^2+\frac{4-2}{2}$}
        \alakohta{$1 + (2-(-5)) \cdot (5 - 1)$}
        \alakohta{$-8 -2 \cdot ( 3 - 2 \cdot (3\cdot2 - 1))$}
		\alakohta{$4\cdot \frac{2}{5} + \frac{2}{3}\cdot \frac{3}{5}$}
		\alakohta{$\frac{2}{5} : \frac{3}{2}$}
	\end{alakohdat}
	\begin{vastaus}
		\begin{alakohdat}
			\alakohta{$2+6-5=3$}
			\alakohta{$3^2+\frac{2}{2}=9+1=10$}
			\alakohta{$1+7\cdot4=1+28=29$}
			\alakohta{$-8 -2\cdot ( 3 - 2 \cdot (6-1)) = -8 -2\cdot (3 - 2 \cdot 5) = -8 -2\cdot (3-10) = -8 -2\cdot (-7) = -8 -(-14) = 6$}
			\alakohta{$\frac{8}{5} + \frac{2}{5}=\frac{10}{5} = 2$}
			\alakohta{$\frac{2}{5} \cdot \frac{2}{3}=\frac{4}{15}$}
		\end{alakohdat}
	\end{vastaus}
\end{tehtava}

\begin{tehtava}
	\begin{alakohdat}
		\alakohta{Kuinka paljon on 5~\% luvusta 40?}
		\alakohta{Kuinka monta prosenttia 2 on luvusta 4?}
	\end{alakohdat}
	\begin{vastaus}
		\begin{alakohdat}
			\alakohta{2}
			\alakohta{50~\%}
		\end{alakohdat}
	\end{vastaus}
\end{tehtava}

\begin{tehtava}
	Sievennä.
	\begin{alakohdat}
		\alakohta{$x + 2x+x^2$}
		\alakohta{$2(4x+1)$}
		\alakohta{$-3x+4x^2-2x+x^2$}
	\end{alakohdat}
	\begin{vastaus}
		\begin{alakohdat}
			\alakohta{$2x +x^2$}
			\alakohta{$8x+2$}
			\alakohta{$5x^2-5x$}
		\end{alakohdat}
	\end{vastaus}
\end{tehtava}

\begin{tehtava}
	Matka ja aika ovat suoraan verrannollisia. Jos matka kaksinkertaistuu, niin miten käy ajalle?
	\begin{vastaus}
		Aika kaksinkertaistuu.
	\end{vastaus}
\end{tehtava}

\begin{tehtava}
	Ratkaise yhtälöt.
	\begin{alakohdat}
		\alakohta{$2x+5 = -1$}
		\alakohta{$x^2 = 9$}
	\end{alakohdat}
	\begin{vastaus}
		\begin{alakohdat}
			\alakohta{$x=-3$}
			\alakohta{$x=-3$ tai $x=3$}
		\end{alakohdat}
	\end{vastaus}
\end{tehtava}

\begin{tehtava}
	Funktiot on määritelty seuraavasti: $f(x)= x^2+3x$ ja $g(x)=2x-8$.
	\begin{alakohdat}
		\alakohta{Laske $f(-2)$.}
		\alakohta{Millä $x$:n arvolla $g(x)=0$?}
	\end{alakohdat}
	\begin{vastaus}
		\begin{alakohdat}
			\alakohta{$(-2)^2+3\cdot(-2)=-2$}
			\alakohta{$x=4$}
		\end{alakohdat}
	\end{vastaus}
\end{tehtava}
