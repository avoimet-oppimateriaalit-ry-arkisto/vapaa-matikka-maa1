\begin{tehtavasivu}

\paragraph*{Opi perusteet}

\begin{tehtava}
Mikä on seuraavien potenssifunktioiden aste?
\begin{alakohdat}
\alakohta{$f(x) = x$}
\alakohta{$f(x) = 5x^5$}
\alakohta{$f(x) = \frac{1}{2x}$}
\alakohta{$f(x) = x^{-2}$}
\end{alakohdat}
\begin{vastaus}
\begin{alakohdat}
\alakohta{$1$}
\alakohta{$5$}
\alakohta{$-1$}
\alakohta{$-2$}
\end{alakohdat}
\end{vastaus}
\end{tehtava}

\begin{tehtava}
Olkoon $f(x)=x^{-3}$. Laske
\begin{alakohdat}
\alakohta{$f(1)$}
\alakohta{$f(2)$}
\alakohta{$f(-\frac{1}{3})$}
\end{alakohdat}
\begin{vastaus}
\begin{alakohdat}
\alakohta{$1$}
\alakohta{$\frac{1}{8}$}
\alakohta{$-27$}
\end{alakohdat}
\end{vastaus}
\end{tehtava}

\begin{tehtava}
Saako potenssifunktio $f$ negatiivisia arvoja, jos
\begin{alakohdat}
\alakohta{$f(x) = x^2$}
\alakohta{$f(x) = -2x^7$}
\alakohta{$f(x) = 3x^{-3}$}
\alakohta{$f(x) = -6x^{-4}$}
\alakohta{$f(x) = x^{-6}$?}
\end{alakohdat}
\begin{vastaus}
\begin{alakohdat}
\alakohta{Ei saa.}
\alakohta{Saa.}
\alakohta{Saa.}
\alakohta{Saa.}
\alakohta{Ei saa.}
\end{alakohdat}
\end{vastaus}
\end{tehtava}

\paragraph*{Hallitse kokonaisuus}

\begin{tehtava}
Olkoon $f(x)=x^2$. Millä $x$:n arvoilla
\begin{alakohdat}
\alakohta{$f(x)=4$}
\alakohta{$f(x)=25$}
\alakohta{$f(x)=121$}
\end{alakohdat}
\begin{vastaus}
\begin{alakohdat}
\alakohta{$2$}
\alakohta{$5$}
\alakohta{$11$}
\end{alakohdat}
\end{vastaus}

\end{tehtava}
\begin{tehtava}
Minkä kahden peräkkäisen kokonaisluvun välissä yhtälön $x^2 = 12$ positiivinen ratkaisu on?
\begin{vastaus}
Ratkaisu on lukujen $3$ ja $4$ välissä.
\end{vastaus}
\end{tehtava}

\paragraph*{Sekalaisia tehtäviä}

\begin{tehtava}
Kuution tilavuus särmän pituuden funktiona on $f(x) = x^3$. Jos kuution tilavuus on $64$, mikä on särmän pituus?
\begin{vastaus}
Särmän pituus on $4$.
\end{vastaus}
\end{tehtava}

\begin{tehtava}
Suorakulmaisen kolmion molempien kateettien pituus on $x$.
\begin{alakohdat}
\alakohta{Muodosta kolmion pinta-alan lauseke.}
\alakohta{Mikä on kolmion pinta-ala, jos $x=10$?}
\alakohta{Kolmion pinta-ala on $72$. Mikä on kateettien pituus?}
\end{alakohdat}
\begin{vastaus}
\begin{alakohdat}
\alakohta{$\frac{1}{2}x^2$}
\alakohta{$50$}
\alakohta{$12$}
\end{alakohdat}
\end{vastaus}
\end{tehtava}

\end{tehtavasivu}
