\begin{tehtavasivu}

\paragraph*{Opi perusteet}

%Tarkistanut V-P Kilpi 2013-11-10
\begin{tehtava}
Mikä on seuraavien potenssifunktioiden aste?
\begin{alakohdat}
\alakohta{$f(x) = x$}
\alakohta{$f(x) = 5x^5$}
\alakohta{$f(x) = \frac{1}{2x}$}
\alakohta{$f(x) = x^{-2}$}
\end{alakohdat}
\begin{vastaus}
\begin{alakohdat}
\alakohta{$1$}
\alakohta{$5$}
\alakohta{$-1$}
\alakohta{$-2$}
\end{alakohdat}
\end{vastaus}
\end{tehtava}

%Laatinut V-P Kilpi 2013-11-10
\begin{tehtava}
Olkoon $g(x)=-2x^{-2}$. Laske
\begin{alakohdat}
\alakohta{$g(4)$}
\alakohta{$g(\frac{1}{4})$}
\alakohta{$g(-\frac{1}{4})$}
\alakohta{$g(-4)$}
\end{alakohdat}
\begin{vastaus}
\begin{alakohdat}
\alakohta{$-\frac{1}{8}$}
\alakohta{$-32 $}
\alakohta{$-32 $}
\alakohta{$-\frac{1}{8}$}
\end{alakohdat}
\end{vastaus}
\end{tehtava}

%Tarkistanut V-P Kilpi 2013-11-10
\begin{tehtava}
Olkoon $f(x)=x^{-3}$. Laske
\begin{alakohdat}
\alakohta{$f(1)$}
\alakohta{$f(2)$}
\alakohta{$f(-\frac{1}{3})$}
\end{alakohdat}
\begin{vastaus}
\begin{alakohdat}
\alakohta{$1$}
\alakohta{$\frac{1}{8}$}
\alakohta{$-27$}
\end{alakohdat}
\end{vastaus}
\end{tehtava}

%Tarkistanut V-P Kilpi 2013-11-10
\begin{tehtava}
Saako potenssifunktio $f$ negatiivisia arvoja, jos
\begin{alakohdat}
\alakohta{$f(x) = x^2$}
\alakohta{$f(x) = -2x^7$}
\alakohta{$f(x) = 3x^{-3}$}
\alakohta{$f(x) = -6x^{-4}$}
\alakohta{$f(x) = x^{-6}$?}
\end{alakohdat}
\begin{vastaus}
\begin{alakohdat}
\alakohta{Ei saa.}
\alakohta{Saa.}
\alakohta{Saa.}
\alakohta{Saa.}
\alakohta{Ei saa.}
\end{alakohdat}
\end{vastaus}
\end{tehtava}

%Laatinut V-P Kilpi 2013-11-10
\begin{tehtava}
Saavatko funktiot $ f(x)=x^{3}-1$ ja $ g(x)=x^{3}+1$ missään kohdassa samaa arvoa?
\begin{vastaus}
Eivät saa.
\end{vastaus}
\end{tehtava}

\begin{tehtava}
Yhdistä funktioiden kuvaajat oikeaan funktioon.

\begin{alakohdat}
	\alakohta{$f(x)=x^{\frac{3}{2}}$}
	\alakohta{$g(x)=x$}
	\alakohta{$h(x)=x^{-8}$}
	\alakohta{$a(x)=\sqrt[3]{x}$}
	\alakohta{$b(x)=x^{\frac{1}{10}}$}
	\alakohta{$c(x)=x^{10}$}
	\alakohta{$d(x)=x^{-1}$}
\end{alakohdat}

\begin{center}
	\begin{kuvaajapohja}{0.9}{-3.5}{3.5}{-3.5}{3.5}
	
	\newcommand{\kuvaajaneg}[3]{
\draw[smooth,color=#3,thick,domain=\kuvaajaminx:-0.01,scale=\kuvaajascale,samples=300] plot function{(#1) < \kuvaajamaxy ? ((#1) > \kuvaajaminy ? (#1) : NaN) : NaN} node[right] {#2};
}
\newcommand{\kuvaajapos}[3]{
\draw[smooth,color=#3,thick,domain=0.01:\kuvaajamaxx,scale=\kuvaajascale,samples=300] plot function{(#1) < \kuvaajamaxy ? ((#1) > \kuvaajaminy ? (#1) : NaN) : NaN} node[right] {#2};
}
	 \kuvaaja{x**10}{2}{red}
	 \kuvaaja{x**1.5}{1}{green}
	 \kuvaaja{x}{3}{black}
	 \kuvaaja{x**0.1}{5}{purple}
	 \kuvaajapos{x**0.33333333333}{7}{blue}
	 \kuvaajaneg{-(-x)**0.33333333333}{}{blue}
	 \kuvaajaneg{x**(-1)}{}{violet}
	 \kuvaajaneg{x**(-8)}{6}{orange}
	 \kuvaajapos{x**(-1)}{4}{violet}
	 \kuvaajapos{x**(-8)}{}{orange}
	\end{kuvaajapohja}
\end{center}

\begin{vastaus}
	\begin{alakohdat}
		\alakohta{1}
		\alakohta{3}
		\alakohta{6}
		\alakohta{7}
		\alakohta{5}
		\alakohta{2}
		\alakohta{4}
	\end{alakohdat}
\end{vastaus}

\end{tehtava}


\paragraph*{Hallitse kokonaisuus}

\begin{tehtava}
Olkoon $f(x)=x^2$. Millä $x$:n arvoilla
\begin{alakohdat}
\alakohta{$f(x)=4$}
\alakohta{$f(x)=25$}
\alakohta{$f(x)=121$}
\end{alakohdat}
\begin{vastaus}
\begin{alakohdat}
\alakohta{$2$}
\alakohta{$5$}
\alakohta{$11$}
\end{alakohdat}
\end{vastaus}
\end{tehtava}

\begin{tehtava}
Minkä kahden peräkkäisen kokonaisluvun välissä yhtälön $x^2 = 12$ positiivinen ratkaisu on?
\begin{vastaus}
Ratkaisu on lukujen $3$ ja $4$ välissä.
\end{vastaus}
\end{tehtava}

%Laatinut V-P Kilpi 2013-11-10
\begin{tehtava}
Onko olemassa reaalilukua, jonka neliö on yhtä suuri kuin sen summa itsensä kanssa?
\begin{vastaus}
On, luku $ 2 $.
\end{vastaus}
\end{tehtava}


%Laatinut V-P Kilpi 2013-11-10
\begin{tehtava}
Vapaassa pudotuksessa olevan kappaleen etäisyydelle metreinä pudotuskorkeudesta antaa hyvän arvion funktio $ s(t)=4,9t^{2}$, missä $ t $ on putoamisaika sekunteina. Kuinka monen sekuntin päästä pudotettu kappale on pudonnut 100 metriä?
\begin{vastaus}
Noin $4,5$ sekuntin päästä.
\end{vastaus}
\end{tehtava}

%Laatinut V-P Kilpi 2013-11-10
\begin{tehtava}
Millä muutujan $x$ arvoilla funktio $ f(x)=x^{5}$ saa saman arvon kuin funktio $ g(x)=x^{4}$?
\begin{vastaus}
Muuttujan arvoilla $0$ ja $1$.
\end{vastaus}
\end{tehtava}

\paragraph*{Sekalaisia tehtäviä}

%Laatinut V-P Kilpi 2013-11-10
\begin{tehtava}
Tasasivuisen kolmion pinta-ala sivun pituuden funktiona on $a(s) = \frac{\sqrt{3}}{4}s^{2}$. Jos tasasivuisen kolmion sivun pituus on $10$, mikä on sen pinta-ala?
\begin{vastaus}
Kolmion pinta-ala on $25\sqrt{3}\approx43,3$.
\end{vastaus}
\end{tehtava}

\begin{tehtava}
Kuution tilavuus särmän pituuden funktiona on $f(x) = x^3$. Jos kuution tilavuus on $64$, mikä on särmän pituus?
\begin{vastaus}
Särmän pituus on $4$.
\end{vastaus}
\end{tehtava}

%Laatinut V-P Kilpi 2013-11-10
\begin{tehtava}
Veijo kehitti funktion $t(p)=\frac{1}{10}p^{\frac{6}{5}}$, joka ennustaa palapelin kokoamiseen kuluvaa aikaa minuutteina palojen määrän $ p $ funktiona. 
\begin{alakohdat}
\alakohta{Kuinka kauan mallin mukaan kuluu aikaa $100$ palan palapelin kokoamiseen?}
\alakohta{$1000$ palan palapelin kokoamiseen?}
\alakohta{Kahden palan palapelin kokoamiseen?}
\end{alakohdat}
\begin{vastaus}
\begin{alakohdat}
\alakohta{Noin $ 25 $ minuuttia.}
\alakohta{Noin $ 6 $ tuntia ja $ 40 $ minuuttia}
\alakohta{Noin $ 14 $ sekuntia}
\end{alakohdat}
\end{vastaus}
\end{tehtava}

\begin{tehtava}
Suorakulmaisen kolmion molempien kateettien pituus on $x$.
\begin{alakohdat}
\alakohta{Muodosta kolmion pinta-alan lauseke.}
\alakohta{Mikä on kolmion pinta-ala, jos $x=10$?}
\alakohta{Kolmion pinta-ala on $72$. Mikä on kateettien pituus?}
\end{alakohdat}
\begin{vastaus}
\begin{alakohdat}
\alakohta{$\frac{1}{2}x^2$}
\alakohta{$50$}
\alakohta{$12$}
\end{alakohdat}
\end{vastaus}
\end{tehtava}

%Laatinut V-P Kilpi 2013-11-10
\begin{tehtava}
Säteilyn intensiteetti säteilylähteen etäisyyden $ r $ funktiona on $ I(r)=\frac{I_{0}}{r^{2}}$, missä $I_{0}$ on vakio, joka kuvaa säteilyn intensiteettiä etäisyydellä $ 0 $. Arvioi funktion avulla kuinka monta prosenttia Auringon säteilyn intensiteetti Maan pinnalla vähenee, jos Maan ja Auringon välinen etäisyys kasvaa $ 20 $ prosenttia?
\begin{vastaus}
Säteilyn intensiteetti Maan pinnalla vähenee noin $ 30 $ prosenttia.
\end{vastaus}
\end{tehtava}

%Laatinut V-P Kilpi 2013-11-10
\begin{tehtava}
Stefan Boltzmannin lain mukaan kappaleen säteilyteho tiettyä pinta-alaa kohden on suoraan verrannollinen sen absoluuttisen lämpötilan neljänteen potenssiin. Lakia kuvaa funktio $ P(T)=A\varepsilon\sigma T^{4} $, missä $ A$ , $\varepsilon$ ja $\sigma $ ovat vakioita.  Jos absoluuttinen lämpötila kaksinkertaistuu, kuinka moninkertaiseksi säteilyteho kasvaa?
\begin{vastaus}
16-kertaiseksi.
\end{vastaus}
\end{tehtava}

\end{tehtavasivu}
