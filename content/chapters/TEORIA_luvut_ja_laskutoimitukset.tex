% engl. natural numbers, counting numbers
% ruots. naturliga tal

% Luonnollisia lukuja käytetään kolmeen eri tarkoitukseen:
% Lukumäärien ilmoittamiseen (kardinaaliluvut)
% Järjestyksen ilmoittamiseen (ordinaaliluvut)
% Indeksointiin ja asioiden nimeämiseen

% FIXME: tähän voisi lisätä lyhyen, selkeän johdannon muuttujien käytön etuihin

\subsection*{Luvun käsite}

\termi{luku}{Luku} on käsite, jolla voidaan ilmaista esimerkiksi suuruutta, lukumäärää tai järjestystä. Erityisesti on huomattava, että
luku ja \termi{numero}{numero} eivät ole synonyymejä; numero on yksittäinen lukujärjestelmän merkki.

\begin{esimerkki}
	Luku $715531$ koostuu numeroista 7, 1, 5, 5, 3 ja 1.
	
	Luku $9$ koostuu ainoastaan vastaavasta numeromerkistä 9.
\end{esimerkki}

Englannin kielen sana \textit{number} voi viitata sekä numeroon että lukuun.
Sana \textit{digit} tarkoittaa pelkästään yhtä numeromerkkiä.
Ruotsiksi luku on \textit{tal}, lukumäärä \textit{antal} ja numeroa tai lukumäärää tarkoittamatonta
numeroyhdistelmää kuvaa suomen kielen tapaan sana \textit{nummer}.

Luvulla on aina suuruus mutta numerolla ei välttämättä ole. Esimerkiksi arkipäivän käsitteet
postinumero ja puhelinnumero eivät ole lukuja, vaikka niissä numeroita yhdistelläänkin. 
Emme voi esimerkiksi sanoa, onko postinumero 00950 jollakin tapaa suurempi kuin postinumero 00900.

\termi{tuhaterotin}{Tuhaterottimena} käytetään suomenkielisessä tekstissä välilyöntiä, ei pilkkua tai pistettä.
\termi{desimaalierotin}{Desimaalierotin} puolestaan on suomenkielisessä tekstissä pilkku, ei piste, kuten yhdysvaltalaisissa laskimissa.

Matematiikassa lukuja merkitään usein kirjaimilla. Tästä on se etu, että kirjain voi tarkoittaa mitä lukua
tahansa, joten kirjainten avulla pystytään ilmaisemaan luvuille päteviä yleisiä sääntöjä ja laskutoimituksia, joissa ei
tiedetä, mikä luku johonkin paikkaan kuuluu. Esimerkiksi voimme sanoa, että kaikille luvuille
(ainakin sellaisille, joita lukion matematiikassa käsitellään) pätee $x+y=y+x$. Tämä on paljon helpompaa
kuin luetella $2+3=3+2$, $2+4=4+2$, $3+4=4+3$, $2+5=5+2$, jne.

Yleensä samalla kirjaimella tarkoitetaan samassa asiayhteydessä aina samaa lukua.
Esmerkiksi edellisessä esimerkissä $x$ tarkoittaa samaa lukua yhtälön kummallakin puolella. $y$ voi tarkoittaa eri lukua kuin $x$, mutta
ei välttämättä. On myös mahdollista, että $x$ ja $y$ voivat tarkoittaa samaa lukua, eli yllä mainittu sääntö kertoo myös, että $2+2=2+2$.

%Laskea-verbin kaksi merkitystä

Kirjaimia, joiden lukuarvo voi vaihdella, on tapana kutsua \termi{muttuja}{muuttujiksi} ja sellaisia,
joiden lukuarvon ajatellaan pysyvän muuttumattomana, on tapana kutsua \termi{vakio}{vakioiksi}.

\subsection*{Lukualueet}

\termi{luonnolliset luvut}{Luonnolliset luvut} ovat lukuja, joita voidaan käyttää lukumäärän ilmaisemiseen.
Luonnollisten lukujen joukkoa merkitään kirjaimella $\nn$ ja niihin katsotaan kuuluvan luvut 0, 1, 2, 3, jne.
Matematiikan kielellä tämä on usein tapana ilmaista seuraavanlaisella merkinnällä: \[\nn = \{0, 1, 2, 3, \ldots\}\]

Tässä kirjasarjassa katsotaan, että nolla on myös luonnollinen luku. Joissain muissa lähteissä saatetaan kuitenkin
tarkoittaa luonnollisilla luvuilla joukkoa $\nn_+ = \{1, 2, 3, \ldots\}$ jota kutsutaan myös positiivisten kokonaislukujen joukoksi.

Luonnollisille luvuille $m$ ja $n$ on määritelty yhteenlasku $m + n$, esimerkiksi $5 + 3 = 8$.

Luonnollisten lukujen kertolasku määritellään peräkkäisinä yhteenlaskuina

\[5 \cdot 3 = 5 + 5 + 5 = 3 + 3 + 3 + 3 + 3.\]

tai yleisesti kirjaimia käyttäen

\laatikko{ \[m \cdot n = \underbrace{m + m + \ldots + m}_{n\text{ kpl}} = \underbrace{n + n + \ldots + n}_{m\text{ kpl}}.\] }

Nollalla kertomisen ajatellaan olevan ''tyhjä yhteenlasku'' eli nolla:
\laatikko{ \[0 \cdot m = 0\] }

%Luonnollisten lukujen $m$ ja $n$ erotus määritellään yhteenlaskun avulla:
%$m-n$ on luku $k$, jolle $k + n = m$. Kahden luonnollisen luvun erotus
%ei kuitenkaan aina ole luonnollinen luku, esimerkkinä $3 - 5$.
%Ratkaisemme ongelman määrittelemällä kullekin luonnolliselle
%luvulle vastaluvun.

Peruskoulun matematiikasta tutuista laskutoimituksista yhteenlasku ja kertolasku ovat ainoat,
joilla voidaan laskea niin, että pysytään luonnollisten lukujen sisällä. Esimerkiksi vähennyslaskua
$3-5$ ei voida laskea luonnollisten lukujen joukossa. Jotta vähennyslasku olisi kaikille luvuille
mahdollinen, on keksitty niin sanottu vastaluvun käsite. Vastaluku määritellään seuraavasti:

\laatikko{ Jokaisella luvulla $n$ on vastaluku $-n$, jolle pätee $n+(-n)=0$. }

Esimerkiksi luvun $2$ vastaluku on $-2$. Yllä olevan määritelmän mukaan, kun luku lasketaan yhteen vastalukunsa kanssa, saadaan tulokseksi $0$.
Esierkiksi $2+(-2)=0$. Vastaavasti luvun $-2$ vastaluku on sellainen luku, joka laskettuna yhteen luvun $-2$ kanssa antaa luvun $0$.
Tämä on tietysti $2$, koska $-2+2=0$. Näin voidaan huomata, että $-(-2)=2$. Yleisesti pätee

\laatikko{$$-(-a)=a\text{.}$$}

Luonnolliset luvut ja niiden vastaluvut muodostavat yhdessä \termi{kokonaisluvut}{kokonaislukujen} joukon

\[\zz = \{\ldots, -2, -1, 0, 1, 2, \ldots\},\] jota voidaan havainnollistaa \termi{lukusuora}{lukusuoran} avulla

\begin{kuva}
	lukusuora.pohja(-4,4,12)
	lukusuora.piste(-3, "$-3$")
	lukusuora.piste(-2,"$-2$")
	lukusuora.piste(-1,"$-1$")
	lukusuora.piste(0,"$0$")
	lukusuora.piste(1,"$1$")
	lukusuora.piste(2,"$2$")
	lukusuora.piste(3,"$3$")
\end{kuva}

%\begin{center}
%\begin{lukusuora}{-4}{4}{10}
%\lukusuorapiste{-3}{}
%\lukusuorapiste{-2}{}
%\lukusuorapiste{-1}{}
%\lukusuorapiste{0}{}
%\lukusuorapiste{1}{}
%\lukusuorapiste{2}{}
%\lukusuorapiste{3}{}
%\lukusuoraalanimi{-3}{$-3$}
%\lukusuoraalanimi{-2}{$-2$}
%\lukusuoraalanimi{-1}{$-1$}
%\lukusuoraalanimi{0}{$0$}
%\lukusuoraalanimi{1}{$1$}
%\lukusuoraalanimi{2}{$2$}
%\lukusuoraalanimi{3}{$3$}
%
%\end{lukusuora}
%\end{center}

Kun käytämme kokonaislukuja, voidaan kahden luvun erotus määritellä
yhteenlaskun ja vastaluvun avulla yksinkertaisesti:

\laatikko{
\[m-n = m+(-n)\]
}

Tämä perusteella voidaan yleistää seuraavat merkkisäännöt yhteen- ja vähennyslaskulle:

\laatikko{
\begin{itemize}
\item{$a+(+b)=a+b$}
\item{$a+(-b)=a-b$}
\item{$a-(+b)=a-b$}
\item{$a-(-b)=a+b$}
\end{itemize}
}


% === SIISTITTÄVÄ PÄTKÄ ALKAA ===

\subsection*{Yhteen- ja vähennyslasku}

    Kysymys: Mitä saadaan, kun luvusta $5$ vähennetään luku $-8$?
    
    Negatiivisten ja positiivisten lukujen yhteen- ja vähennyslaskut voidaan helposti tulkita lukusuoran avulla.
    
    % tässä on vähän kyseenalaista käyttää sekaisin sanallista ja numeerista esitystä
    
    $5+8$ ''viiteen lisätään kahdeksan''
\begin{center}
\begin{kuva}
	lukusuora.pohja(-4,14,12,n=2)
	with vari("red"):
		lukusuora.vali(0,5,i=1)
		lukusuora.vali(8,13,i=2)
	with vari("blue"): lukusuora.vali(0,8,i=2)
	lukusuora.piste(0,"$0$")
	lukusuora.piste(5,"$5$",1)
	lukusuora.piste(8,"$8$",2)
	lukusuora.piste(13,"$13$",2)
\end{kuva}
%      \begin{lukusuora}{-1}{14}{14}
%        %\lukusuoranuolialas{5}{13}
%        %\lukusuoranuolialas{0}{8}
%        {\color{red} \lukusuoravaliss{0}{5}{$0$}{$5$}}
%        \lukusuorauusi
%        {\color{red} \lukusuoravaliss{8}{13}{$8$}{${\color{black}13}$}}
%        {\color{blue} \lukusuoravaliss{0}{8}{$0$}{$8$}}
%       \end{lukusuora}
       ${\color{red}5}+{\color{blue}8}=13$
\end{center}
    
    $5+(+8)$ ''viiteen lisätään plus kahdeksan''
    
    $+8$ tarkoittaa samaa kuin $8$. '$+$'-merkkiä käytetään luvun edessä silloin, kun halutaan korostaa, että kyseessä on nimenomaan positiivinen luku.
    
%säädetään kuva vähän irti tuosta edeltävästä tektistä
%\vspace{0.3cm}     
    
\begin{center}
\begin{kuva}
	lukusuora.pohja(-4,14,12,n=2)
	with vari("red"):
		lukusuora.vali(0,5,i=1)
		lukusuora.vali(8,13,i=2)
	with vari("blue"): lukusuora.vali(0,8,i=2)
	lukusuora.piste(0,"$0$")
	lukusuora.piste(5,"$5$",1)
	lukusuora.piste(8,"$8$",2)
	lukusuora.piste(13,"$13$",2)
\end{kuva}
%          \begin{lukusuora}{-1}{14}{14}
%        {\color{red} \lukusuoravaliss{0}{5}{$0$}{$5$}}
%        \lukusuorauusi
%        {\color{red} \lukusuoravaliss{8}{13}{$8$}{${\color{black}13}$}}
%        {\color{blue} \lukusuoravaliss{0}{8}{$0$}{$8$}}
%       \end{lukusuora}
       ${\color{red}5}+({\color{blue}+8})=13$
\end{center}
    
    $5-(+8)$ ''viidestä vähennetään $+8$''
    
    Tämä tarkoittaa samaa kuin $5-8$. Lukusuoralla siis liikutaan 8 pykälää taaksepäin.

%säädetään kuva vähän irti tuosta edeltävästä tektistä
\vspace{0.3cm}     
    
\begin{center}
\begin{kuva}
	lukusuora.pohja(-4,14,12,n=2)
	with vari("red"):
		lukusuora.vali(0,5,i=2)
	with vari("blue"): lukusuora.vali(-3,5,i=1)
	lukusuora.piste(0,"$0$",2)
	lukusuora.piste(5,"$5$")
	lukusuora.piste(-3,"$-3$",1)
\end{kuva}
%              \begin{lukusuora}{-4}{8}{14}
%        {\color{blue} \lukusuoravaliss{-3}{5}{\color{black}$-3$}{$5$}}
%        \lukusuorauusi
%%        {\color{red} \lukusuoravaliss{8}{13}{$8$}{${\color{black}13}$}}
%        {\color{red} \lukusuoravaliss{0}{5}{$0$}{$5$}}
%       \end{lukusuora}
       ${\color{red}5}-({\color{blue}+8})=-3$
\end{center}


%lukiolaiset pitivät näitä epäselvinä kuvina
%[Joonas]: Tarvittaisiin vierekkäin olevat nuolet
    
Mitä tapahtuu, kun lisätään negatiivinen luku? Kun lukuun lisätään $1$, se kasvaa yhdellä. Kun lukuun lisätään $0$, se ei kasva lainkaan. Kun lukuun lisätään negatiivinen luku, esimerkiksi $-1$, on luonnollista ajatella, että se pienenee. Tällä logiikalla negatiivisen luvun lisäämisen pitäisi siis pienentää alkuperäistä lukua. Juuri näin vähennyslasku määritellään: $5+(-8)$ on yhtä suuri kuin $5-8$.
    
\vspace{0.3cm}     
\begin{center}
\begin{kuva}
	lukusuora.pohja(-4,14,12,n=2)
	with vari("red"):
		lukusuora.vali(0,5,i=2)
	with vari("blue"): lukusuora.vali(-3,5,i=1)
	lukusuora.piste(0,"$0$",2)
	lukusuora.piste(5,"$5$")
	lukusuora.piste(-3,"$-3$",1)
\end{kuva}
%                 \begin{lukusuora}{-4}{8}{14}
%        {\color{blue} \lukusuoravaliss{-3}{5}{\color{black}$-3$}{$5$}}
%        \lukusuorauusi
%%        {\color{red} \lukusuoravaliss{8}{13}{$8$}{${\color{black}13}$}}
%        {\color{red} \lukusuoravaliss{0}{5}{$0$}{$5$}}
%       \end{lukusuora}
       ${\color{red}5}+({\color{blue}-8})=-3$
\end{center}
    
    
    $5-(-8)$ ''viidestä vähennetään miinus kahdeksan''
    
    Negatiivisen luvun lisääminen on vastakohta positiivisen luvun lisäämiselle. Tällöin on luonnollista, että negatiivisen luvun vähentäminen on vastakohta positiivisen luvun vähentämiselle. Koska positiivisen luvun vähentäminen pienentää lukua, pitäisi negatiivisen luvun vähentämisen kasvattaa lukua. Lasku $5-(-8)$ tarkoittaa siis samaa kuin $5+8$.
\vspace{0.3cm}     
        
\begin{center}
\begin{kuva}
	lukusuora.pohja(-4,14,12,n=2)
	with vari("red"):
		lukusuora.vali(0,5,i=1)
		lukusuora.vali(8,13,i=2)
	with vari("blue"): lukusuora.vali(0,8,i=2)
	lukusuora.piste(0,"$0$")
	lukusuora.piste(5,"$5$",1)
	lukusuora.piste(8,"$8$",2)
	lukusuora.piste(13,"$13$",2)
\end{kuva}
%    \begin{lukusuora}{-1}{14}{14}
%        {\color{red} \lukusuoravaliss{0}{5}{$0$}{$5$}}
%        \lukusuorauusi
%        {\color{red} \lukusuoravaliss{8}{13}{$8$}{${\color{black}13}$}}
%        {\color{blue} \lukusuoravaliss{0}{8}{$0$}{$8$}}
%       \end{lukusuora}
       ${\color{red}5}-({\color{blue}-8})=13$
\end{center}


%Koska $+b-b=0$, yhteen- ja vähennyslasku kumoavat toisensa, eli
%
%\begin{itemize}
%\alakohta{$a+b-b=a$
%\alakohta{$a-b+b=a$
%\end{itemize}



\subsection*{Kertolasku}

    Samaan logiikkaan perustuen on sovittu myös merkkisäännöt positiivisten ja negatiivisten lukujen kertolaskuissa. Kun negatiivinen ja positiivinen luku kerrotaan keskenään, saadaan negatiivinen luku, mutta kun kaksi negatiivista lukua kerrotaan keskenään, saadaan positiivinen luku.

    $3 \cdot 4$ ''kolme kappaletta nelosia''
    
\begin{center}
\begin{kuva}
	lukusuora.pohja(-14,14,12)
	vari("red")
	lukusuora.vali(0,4)
	lukusuora.vali(4,8)
	lukusuora.vali(8,12)
	vari("black")
	lukusuora.piste(0,"$0$")
	lukusuora.piste(4,"$4$")
	lukusuora.piste(8,"$8$")
	lukusuora.piste(12,"$12$")
\end{kuva}
%    \begin{lukusuora}{-1}{14}{14}
%	\color{red} \lukusuoravaliss{0}{4}{$0$}{$4$}
%	\color{red} \lukusuoravaliss{4}{8}{$4$}{$8$}
%	\color{red} \lukusuoravaliss{8}{12}{$8$}{$12$}
%
%      \end{lukusuora}
      $3\cdot {\color{red}4}=12$
\end{center}
    
    $3 \cdot (-4)$ ''kolme kappaletta miinus-nelosia''
    
    
\begin{center}
\begin{kuva}
	lukusuora.pohja(-14,14,12)
	vari("red")
	lukusuora.vali(-4,0)
	lukusuora.vali(-8,-4)
	lukusuora.vali(-12,-8)
	vari("black")
	lukusuora.piste(0,"$0$")
	lukusuora.piste(-4,"$-4$")
	lukusuora.piste(-8,"$-8$")
	lukusuora.piste(-12,"$-12$")
\end{kuva}
%    \begin{lukusuora}{-13}{2}{14}
%	\color{red} \lukusuoravaliss{-12}{-8}{$-12$}{$-8$}
%	\color{red} \lukusuoravaliss{-8}{-4}{$-8$}{$-4$}
%	\color{red} \lukusuoravaliss{-4}{0}{$-4$}{$0$}
%
%      \end{lukusuora}
      $3\cdot ({\color{red}-4})=-12$
\end{center}
    
    $-3 \cdot 4$ ''miinus-kolme nelinkertaistetaan''
    
\begin{center}
\begin{kuva}
	lukusuora.pohja(-14,14,12)
	vari("blue")
	lukusuora.vali(-3,0)
	lukusuora.vali(-6,-3)
	lukusuora.vali(-9,-6)
	lukusuora.vali(-12,-9)
	vari("black")
	lukusuora.piste(0,"$0$")
	lukusuora.piste(-3,"$-3$")
	lukusuora.piste(-6,"$-6$")
	lukusuora.piste(-9,"$-9$")
	lukusuora.piste(-12,"$-12$")
\end{kuva}
%    \begin{lukusuora}{-13}{2}{14}
%	\color{red} \lukusuoravaliss{-12}{-9}{$-12$}{$-9$}
%	\color{red} \lukusuoravaliss{-9}{-6}{$-9$}{$-6$}
%	\color{red} \lukusuoravaliss{-6}{-3}{$-6$}{$-3$}
%	\color{red} \lukusuoravaliss{-3}{0}{$-3$}{$0$}
%
%      \end{lukusuora}
      ${\color{blue}-3}\cdot 4=-12$
\end{center}
    
    $-3 \cdot (-4)$ ''miinus-kolme miinus-nelinkertaistetaan''
    
\begin{center}
\begin{kuva}
	lukusuora.pohja(-14,14,12)
	vari("blue")
	lukusuora.vali(0,3)
	lukusuora.vali(3,6)
	lukusuora.vali(6,9)
	lukusuora.vali(9,12)
	vari("black")
	lukusuora.piste(0,"$0$")
	lukusuora.piste(3,"$3$")
	lukusuora.piste(6,"$6$")
	lukusuora.piste(9,"$9$")
	lukusuora.piste(12,"$12$")
\end{kuva}
%    \begin{lukusuora}{-1}{14}{14}
%	\color{red} \lukusuoravaliss{12}{9}{$12$}{$9$}
%	\color{red} \lukusuoravaliss{9}{6}{$9$}{$6$}
%	\color{red} \lukusuoravaliss{6}{3}{$6$}{$3$}
%	\color{red} \lukusuoravaliss{3}{0}{$3$}{$0$}
%
%      \end{lukusuora}
      ${\color{blue}-3}\cdot (-4)=12$
\end{center}

\subsection*{Jakolasku}

        Kun ensin kerrotaan jollain ja sitten jaetaan samalla luvulla, päädytään takaisin samaan, mistä lähdettiin.  Jakolaskujen merkkisäännöt on sovittu niin, että tämä ominaisuus säilyy. Ne ovat siis samat kuin kertolaskujen merkkisäännöt.
    
    Esimerkiksi haluamme, että $(-12):(-3)\cdot (-3)=-12$. Nyt voimme kysyä, mitä laskun $(-12):(-3)$ tulokseksi pitäisi tulla, jotta jakolasku ja kertolasku säilyvät toisilleen käänteisinä, eli mikä luku kerrottuna luvulla $-3$ on $-12$. Kertolaskun merkkisäännöistä nähdään helposti, että tämän luvun täytyy olla $+4$ eli $4$. Niinpä on sovittu, että $(-12):(-3)=12:3=4$.

\laatikko{
Kerto- ja jakolaskun merkkisäännöt
\begin{alakohdat}
\alakohta{$a\cdot (-b)=(-a)\cdot b=-(ab)$}
\alakohta{$(-a)\cdot (-b)=a\cdot b=ab$}
\alakohta{$(-a):b=a:(-b)=-\dfrac{a}{b}$}
\alakohta{$(-a):(-b)=a: b=\dfrac{a}{b}$}
\end{alakohdat}
}

\laatikko{
Kerto- ja jakolasku kumoavat toisensa
\begin{alakohdat}
\alakohta{$a\cdot b:b=a$}
\alakohta{$a:b\cdot b=a$}
\end{alakohdat}
}

\subsection*{Laskujärjestys}

Yleisen käytännön mukaan laskulausekkeessa, jossa on useita laskutoimituksia, tehdään laskutoimitukset seuraavan järjestyksen mukaan:

\laatikko{
\begin{enumerate}
\item{Suluissa olevat lausekkeet}
\item{Potenssilaskut (potenssia käsitellään tarkemmin myöhemmin tässä kirjassa)}
\item{Kerto- ja jakolaskut vasemmalta oikealle}
\item{Yhteen- ja vähennyslaskut vasemmalta oikealle}
\end{enumerate}
}