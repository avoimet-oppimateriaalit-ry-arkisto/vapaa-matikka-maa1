\begin{tehtavasivu}

\paragraph*{Opi perusteet}

\begin{tehtava}
Olkoon $f(x) = 4^x$. Laske
\begin{alakohdat}
\alakohta{$f(0)$}
\alakohta{$f(3)$}
\alakohta{$f(\frac{1}{2})$}
\end{alakohdat}
\begin{vastaus}
\begin{alakohdat}
\alakohta{$1$}
\alakohta{$64$}
\alakohta{$2$}
\end{alakohdat}
\end{vastaus}
\end{tehtava}

\begin{tehtava}
Olkoon $f(x) = 10^x$. Millä $x$:n arvoilla
\begin{alakohdat}
\alakohta{$f(x) = 1000$}
\alakohta{$f(x) = \frac{1}{100}$}
\alakohta{$f(x) = -1$?}
\end{alakohdat}
\begin{vastaus}
\begin{alakohdat}
\alakohta{$3$}
\alakohta{$6$}
\alakohta{Ei ratkaisua.}
\end{alakohdat}
\end{vastaus}
\end{tehtava}

\paragraph*{Hallitse kokonaisuus}
\begin{tehtava}
Minkä kahden peräkkäisen kokonaisluvun välissä yhtälön
$10^x = 500$ ratkaisu on?
\begin{vastaus}
Ratkaisu on lukujen $2$ ja $3$ välissä.
\end{vastaus}
\end{tehtava}


\begin{tehtava}
Olkoon $f(t) = 20 \cdot 2^t$ bakteerien lukumäärä soluviljelmässä
ajanhetkellä $t$. Millä ajanhetkellä bakteerien lukumäärä on tasan 160?
\begin{vastaus}
Ajanhetkellä $t = 3$.
\end{vastaus}
\end{tehtava}

\begin{tehtava}
Miten muokkaisit edellisen tehtävän funktiota, jos bakteerien lukumääräksi
halutaan 5 ajanhetkellä $t = 0$?
\begin{vastaus}
$f(t) = 5 \cdot 2^t$
\end{vastaus}
\end{tehtava}

\begin{tehtava}
Millä ajanhetkellä atomiydinten määrä on alle $1/200$ alkuperäisestä?
\begin{vastaus}
Ajanhetkellä $t = 8$.
\end{vastaus}
\end{tehtava}

\paragraph*{Sekalaisia tehtäviä}

\begin{tehtava}
(YO 1877 4) Vuosikymmenen 1860--70 kuluessa lisääntyi Helsingin väkiluku puolella vuoden 1860 väkiluvulla. Jos väkiluvun lisäys tapahtuisi seuraavinakin vuosikymmeninä samassa suhteessa, paljonko väkeä Helsingissä olisi 1890, kun siellä 1860 oli 21~700 asukasta? 
	\begin{vastaus}
	73~200 (pyöristämättä 73~237,5)
	\end{vastaus}
\end{tehtava}

% % tähän parempi tehtävä atomiytimistä
%\begin{tehtava}
%Millä ajanhetkellä atomiydinten määrä on alle $1/200$ alkuperäisestä?
%\begin{vastaus}
%Ajanhetkellä $t = 8$.
%\end{vastaus}
%\end{tehtava}

\end{tehtavasivu}
