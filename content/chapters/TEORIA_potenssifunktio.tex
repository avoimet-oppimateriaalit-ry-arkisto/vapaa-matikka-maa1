Kahden muuttujan välinen riippuvuus ei monesti ole suoraan tai kääntäen verrannollista.
Potenssifunktion käyttäminen on eräs keino kuvata tällaisia riippuvuuksia.

\laatikko{Potenssifunktioita ovat muotoa
$ f(x) = a x^n $ olevat funktiot, jossa $a \neq 0$. }

Eksponenttia $n$ kutsutaan potenssifunktion \termi{asteluku}{asteluvuksi}
tai \termi{aste}{asteeksi}.
Potenssifunktion eksponentti voi olla mikä tahansa reaaliluku, mutta
rajoitumme ensin käsittelemään tapauksia, jossa $n = 1, 2, 3\ldots $.

\begin{esimerkki}
Jos neliön sivun pituus on $x$, neliön pinta-ala voidaan laskea
funktiolla $A(x)=x^2$.
Esimerkiksi jos $x = 3$ cm, saadaan neliön pinta-alaksi $A(x) = 9$ cm$^2$.
Vastaavasti kuutiolle: jos $x$ kuvaa kuution särmän pituutta, funktiolla
$V(x)=x^3$ voidaan laskea kuution tilavuus. Sekä $A(x)$ että $V(x)$ ovat
esimerkkejä potenssifunktioista.
\end{esimerkki}

Potenssifunktion aste vaikuttaa funktion kuvaajan muotoon:
\begin{alakohdat}
  \alakohta{%
Jos aste on parillinen, kuvaaja on U-kirjaimen muotoinen ja funktio
saa $a$:n merkistä riippuen joko pelkästään positiivisia tai pelkästään
negatiivisia arvoja.}
  \alakohta{%
Jos aste on pariton, kuvaaja muodostaa ''kaksoismutkan'' ja potenssifunktio
saa sekä positiivisia että negatiivisia arvoja.}
\end{alakohdat}

%\missingfigure{Potenssifunktioiden kuvaajat -- yksi parillisilla potensseilla ja toinen parittomilla (kerroin a positiivinen).}
\begin{center}
\begin{kuvaajapohja}{0.5}{-6}{6}{-6}{6}
\kuvaaja{0.2*x**2}{$y=\frac{1}{5}x^2$}{red}
\kuvaaja{0.2*x**3}{$y=\frac{1}{5}x^3$}{blue}
\end{kuvaajapohja}
\end{center}
Potenssifunktion tärkeitä erikoistapauksia ovat asteluvut $n = 1$ ja $n = -1$.
Kun $n = 1$, saadaan edellisessä luvussa esitelty suoraan verrannollinen
riippuvuus $x$:n ja $f(x)$:n välillä, ja kun $n = -1$, muuttuja $x$ ja
funktion arvo $f(x)$ ovat kääntäen verrannolliset.

Potenssifunktiota voidaan laajentaa sallimalla eksponentille $n$
myös negatiiviset arvot.
Tällöin funktion muoto muuttuu merkittävästi. Funktio ei myöskään ole
enää määritelty kohdassa $x = 0$, vaan funktion arvot
''räjähtävät äärettömyyteen'', kun y-akselia lähestytään:

%\missingfigure{Potenssifunktioiden kuvaajat - yksi potenssilla -1 ja toinen potenssilla -2.}

\begin{center}
\begin{kuvaajapohja}{0.5}{-6}{6}{-6}{6}
%\kuvaaja{x**(-1)}{$\frac{1}{5}x^2$}{red}
%\kuvaaja{x**(-2)}{$\frac{1}{5}x^3$}{blue}
% FIXME: alla oleva puukotus, että saa äärettömyyteen menevät käppyrät piirtymään oikein
\newcommand{\kuvaajaneg}[3]{
\draw[smooth,color=#3,thick,domain=\kuvaajaminx:-0.01,scale=\kuvaajascale,samples=300] plot function{(#1) < \kuvaajamaxy ? ((#1) > \kuvaajaminy ? (#1) : NaN) : NaN} node[right] {#2};
}
\newcommand{\kuvaajapos}[3]{
\draw[smooth,color=#3,thick,domain=0.01:\kuvaajamaxx,scale=\kuvaajascale,samples=300] plot function{(#1) < \kuvaajamaxy ? ((#1) > \kuvaajaminy ? (#1) : NaN) : NaN} node[right] {#2};
}
\kuvaajaneg{x**(-1)}{$y=x^{-1}$}{red}
\kuvaajaneg{x**(-2)}{}{blue}

\kuvaajapos{x**(-1)}{}{red}
\kuvaajapos{x**(-2)}{$y=x^{-2}$}{blue}

\end{kuvaajapohja}
\end{center}

Aiemmin määriteltiin murtopotenssi $x^{\frac{m}{n}}$, missä kantaluku $x\geq0$. Murtopotenssifunktiota määritellessä pitää myös muistaa kantaluvun positiivisuus, 
joten murtopotenssifunktion $f(x)=x^{\frac{m}{n}}$ määrittelyjoukko on positiivisten reaalilukujen joukko, eli $x\geq0$.

Huomaa, että koska murtopotensseilla pätee $x^{\frac{m}{n}}={\sqrt[n]{x}}^m$, funktio $f(x)=x^{\frac{1}{2}}$ on itse asiassa sama kuin funktio $g(x)=\sqrt{x}$.
Kuitenkin, koska murtopotensseja ei määritellä negatiivisilla kantaluvuilla, funktio $h(x)=x^{\frac{1}{3}}$ on määritelty ainoastaan kun $x\geq0$, mutta funktion
$w(x)=\sqrt[3]{x}$ määrittelyjoukko on koko reaalilukujen joukko. Alla jotain murtopotenssifunktioita ja juurifunktioita esitettynä koordinaatistossa.

\begin{center}
	\begin{kuvaajapohja}{1.0}{-3}{3}{-3}{3}
		\kuvaaja{x**1.6666666667}{$y=x^\frac{5}{3}$}{red}
		\kuvaaja{x**0.5}{$y=x^\frac{1}{2}$}{blue}
	\end{kuvaajapohja}
\end{center}

Juurifunktioiden kuvaajissa huomaa määrittelyjoukko ja yhteys murtopotenssifunktioihin.

\begin{center}
	\begin{kuvaajapohja}{1.0}{-3}{3}{-3}{3}
	
\newcommand{\kuvaajaneg}[3]{
\draw[smooth,color=#3,thick,domain=\kuvaajaminx:-0.01,scale=\kuvaajascale,samples=300] plot function{(#1) < \kuvaajamaxy ? ((#1) > \kuvaajaminy ? (#1) : NaN) : NaN} node[right] {#2};
}
\newcommand{\kuvaajapos}[3]{
\draw[smooth,color=#3,thick,domain=0.01:\kuvaajamaxx,scale=\kuvaajascale,samples=300] plot function{(#1) < \kuvaajamaxy ? ((#1) > \kuvaajaminy ? (#1) : NaN) : NaN} node[right] {#2};
}
		\kuvaajapos{x**0.333333333333333}{}{red}
		\kuvaajapos{x**0.5}{$y=\sqrt{x}$}{blue}
		
		\kuvaajaneg{-(-x)**0.333333333333333}{$y=\sqrt[3]{x}$}{red}
	\end{kuvaajapohja}
\end{center}


Kuvaajista voidaan nähdä yhteys potenssin ja kuvaajan tyypin välillä. 

\laatikko{
Tarkastellaan potenssifunktiota $f(x)=x^a$.

	\begin{itemize}
		\item Jos potenssi $a>1$, kuvaajan kasvu kiihtyy kun $x$ kasvaa.
		\item Jos potenssi $a<1$, kuvaaja kasvaa aluksi nopeasti, mutta funktioiden arvojen kasvu hidastuu todella vähäiseksi $x$:n kasvaessa.
		\item Jos potenssi $a=1$, funktio $f$ saa muodon $f(x)=x^1=x$, ja funktion kuvaaja on suora $y=x$.
		\item Jos potenssi $a<0$, funktiota $f(x)=x^(-a)=\frac{1}{x^a}$ ei ole määritelty kun $x=0$, ja funktio käyttäytyy origon ympäristössä erikoisesti.
	\end{itemize}
}

%Tämä laitetaankin harjoitustehtäväksi t. Jaakko Viertiö
%\begin{center}
%	\begin{kuvaajapohja}{1.0}{-5}{5}{-5}{5}
%	
%	\newcommand{\kuvaajaneg}[3]{
%\draw[smooth,color=#3,thick,domain=\kuvaajaminx:-0.01,scale=\kuvaajascale,samples=300] plot function{(#1) < \kuvaajamaxy ? ((#1) > \kuvaajaminy ? (#1) : NaN) : NaN} node[right] {#2};
%}
%\newcommand{\kuvaajapos}[3]{
%\draw[smooth,color=#3,thick,domain=0.01:\kuvaajamaxx,scale=\kuvaajascale,samples=300] plot function{(#1) < \kuvaajamaxy ? ((#1) > \kuvaajaminy ? (#1) : NaN) : NaN} node[right] {#2};
%}
%	 \kuvaaja{x**10}{$x^{10}$}{blue}
%	 \kuvaaja{x**1.5}{$x^{\frac{3}{2}}$}{blue}
%	 \kuvaaja{x}{$x$}{black}
%	 \kuvaaja{x**0.1}{$\sqrt[10]{x}=x^{\frac{1}{10}}$}{red}
%	 \kuvaajapos{x**0.33333333333}{$\sqrt[3]{x}$}{red}
%	 \kuvaajaneg{-(-x)**0.33333333333}{}{red}
%	 \kuvaajaneg{x**(-1)}{}{violet}
%	 \kuvaajaneg{x**(-8)}{$x^{-8}=\frac{1}{x^8}$}{violet}
%	 \kuvaajapos{x**(-1)}{$x^{-1}=\frac{1}{x}$}{violet}
%	 \kuvaajapos{x**(-8)}{}{violet}
%	\end{kuvaajapohja}
%\end{center}


Tässä kurssissa on tarkoitus vain sivuta erilaisia funktioita ja niiden kuvaajia, ja tutustua niiden käyttäytymiseen alustavasti. 
Asiaan palataan useaan otteeseen myöhemmillä kursseilla. Esimerkiksi funktioita kokonaislukupotenssilla tarkastellaan lähemmin kurssilla MAA2 ja
juuri- ja murtopotenssifunktioita kurssilla MAA8.



%Potenssifunktiota käsitellään samalla lailla riippumatta eksponentin
%etumerkistä. Huomaa kuitenkin, että $\frac{1}{x^n} \neq 0 $ kaikilla $x$:n %arvoilla.

%\laatikko{
%\begin{itemize}
%\item
%Olkoon $n$ pariton. Tällöin potenssifunktio $f(x)=x^n$ on kaikkialla %aidosti kasvava ja jatkuva. Tästä seuraa, että yhtälöllä $y=f(x)$ on aina tasan %yksi ratkaisu kaikilla $y$.    
%\item
%Olkoon $n$ parillinen. Tällöin potenssifunktio $f(x)=x^n$ on positiivinen, %symmetrinen, jatkuva ja aidosti kasvava, kun $x$ on positiivinen.  %Positiivisuudesta seuraa, että yhtälöllä $y=f(x)$ ei ole ratkaisuja, jos $y$ on %negatiivinen. Symmetriasta $f(x)=f(-x)$ seuraa, että jos $x$ on ratkaisu, niin %myös $-x$ on ratkaisu.  
%\end{itemize}
%}

% potenssiyhtälöiden tarkastelu graafisesti
