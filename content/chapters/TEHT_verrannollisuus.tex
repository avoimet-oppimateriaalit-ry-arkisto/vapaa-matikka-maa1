\begin{tehtavasivu}

\begin{tehtava}
    Tutkitaan seuraavia muuttujapareja:
    \begin{alakohdat}
        \alakohta{Neliön sivun pituus ja neliön pinta-ala.}
        \alakohta{Kuljettu matka ja kulunut aika, kun nopeus on vakio.}
        \alakohta{Dieselin hinta ja 50 eurolla saatavan dieselin määrä.}
        \alakohta{Irtomyynnistä ostetun suolakurkkuerän paino ja hinta.}
    \end{alakohdat}
    Miten toinen muuttuja muuttuu ensimmäisen kaksinkertaistuessa tai puolittuessa? Ovatko muuttujat suoraan tai kääntäen verrannolliset vai eivät kumpaakaan?
    \begin{vastaus}
        Vastaus:
        \begin{alakohdat}
            \alakohta{Eivät ole.}
            \alakohta{Suoraan verrannolliset.}
            \alakohta{Kääntäen verrannolliset.}
            \alakohta{Suoraan verrannolliset.}
        \end{alakohdat}
    \end{vastaus}
\end{tehtava}

\begin{tehtava}
Ratkaise
\begin{alakohdat}
\alakohta{$ \frac{x}{3} = 1$}
\alakohta{$ \frac{8}{x} = 2$}
\alakohta{$ \frac{7}{x} = \frac{16}{8}$}
\alakohta{$ \frac{x}{3} = \frac{1}{7}$}
\end{alakohdat}
\begin{vastaus}
\begin{alakohdat}
\alakohta{$x= \frac{1}{3}$}
\alakohta{$y= \frac{1}{4}$}
\alakohta{$x= \frac{7}{2}$}
\alakohta{$x= \frac{3}{7}$}
\end{alakohdat}
\end{vastaus}
\end{tehtava}

\begin{tehtava}
Muodosta seuraavia tilanteita kuvaavat yhtälöt. 
%Voit käyttää vakion merkkinä esimerkiksi kirjainta $c$. %%%miksei vakiintunutta k:ta?
\begin{alakohdat}
\alakohta{Kultakimpaleen arvo ($x$) on suoraan verrannollinen sen massaan ($m$), eli mitä painavampi kimpale on, sitä enemmän siitä saa rahaa.}
\alakohta{Aidan maalaamiseen osallistuvien ihmisten määrä {$x$} on kääntäen verrannollinen maalaamiseen kuluvaan aikaan ($t$). Toisin sanoen, mitä}
enemmän maalaajia, sitä nopeammin homma on valmis.
\alakohta{Planeettojen toisiinsa aiheuttama vetovoima ($F$) on suoraan verrannollinen planeettojen massoihin ($m_1$ ja $m_2$) ja kääntäen verrannollinen niiden välisen etäisyyden ($r$) neliöön.}
\end{alakohdat}
\begin{vastaus}
\begin{alakohdat}
\alakohta{$ \frac{x}{m}=c$}
\alakohta{$ xt=c $}
\alakohta{$ \frac{Fr^2}{m_1+m_2}=c$}
\end{alakohdat}
\end{vastaus}
\end{tehtava}

\begin{tehtava}
Rento pyöräilyvauhti kaupunkiolosuhteissa on noin $20$~km/h. Lukiolta urheiluhallille on matkaa $7$~km. Kuinka monta minuuttia kestää arviolta pyöräillä lukiolta urheiluhallille?
\begin{vastaus}
Viiden minuutin tarkkuudella $20$ min.
\end{vastaus}
\end{tehtava}

\begin{tehtava}
    Isä ja lapset ovat ajamassa mökille Sotkamoon. On ajettu jo neljä
    viidesosaa matkasta, ja aikaa on kulunut kaksi tuntia. ''Joko ollaan perillä?''
    lapset kysyvät takapenkiltä. Kuinka pitkään vielä arviolta kuluu, ennen
    kuin ollaan mökillä?
    
    \begin{vastaus}
        Vastaus: $30$~min
    \end{vastaus}
\end{tehtava}

\begin{tehtava}
    Äidinkielen kurssilla annettiin tehtäväksi lukea 300-sivuinen romaani.
    Eräs opiskelija otti aikaa ja selvitti lukevansa vartissa seitsemän sivua.
    Kuinka monta tuntia häneltä kuluu koko romaanin lukemiseen, jos
    taukoja ei lasketa?
    
    \begin{vastaus}
        Vastaus: $642$ minuuttia eli $10$~h $42$~min.
    \end{vastaus}
\end{tehtava}

\begin{tehtava}
Alkuperäiskansojen perinteitä tutkiva etnologi kerää saamelaisten kansantarinoita. Kierrettyään viikkojen ajan saamelaisperheiden vieraana palaa etnologi yliopistolle mukanaan äänitiedostoiksi tallennettuja haastatteluja. Jotta etnologi voi ryhtyä kirjoittamaan tutkimusraporttia, pitää haastattelutiedostot ensin purkaa tekstiksi eli \emph{litteroida.} Tähän etnologi pyytää apua laitoksen opiskelijoilta. Neljä opiskelijaa litteroi tiedostot $25$~tunnissa. Kuinka kauan aikaa kuluu, jos työhön saadaan vielä kolme opiskelijaa lisää? Kuinka monta opiskelijaa tarvitaan, jotta työ saataisiin tehdyksi $9$~tunnissa? Oletetaan, että kaikkien opiskelijoiden työteho on sama.

\begin{vastaus}
Seitsemällä opiskelijalla työ valmistuu noin $14$~tunnissa. Työ valmistuu $8$~tunnissa, jos opiskelijoita on $12$.
\end{vastaus}
\end{tehtava}

\begin{tehtava}
	Kappaleen liike-energia on suoraan verrannollinen nopeuden neliöön.
	Kun kappale liikkui nopeudella $9,0$~m/s, sillä oli $150$~joulea liike-energiaa.
	Laske kappaleen liike-energia, kun sen nopeus on $5,0$~m/s. 
	\begin{vastaus}
		$46$~J
	\end{vastaus}
\end{tehtava}

\begin{tehtava}
	Kymmenen miestä kaivaa viisitoista kuoppaa kymmenessä tunnissa.
	Kuinka kauan viidellä miehellä kestää kaivaa $30$ kuoppaa?	
	\begin{vastaus}
		$40$ tuntia
	\end{vastaus}
\end{tehtava}

\end{tehtavasivu}
