\begin{tehtavasivu}

\subsubsection*{Opi perusteet}

%Tarkistanut V-P Kilpi 2013-11-09
\begin{tehtava}
    Tutkitaan seuraavia muuttujapareja:
    \begin{alakohdat}
        \alakohta{Neliön sivun pituus ja neliön pinta-ala.}
        \alakohta{Neliön, jonka pinta-ala on 5, sivunpituudet.}
        \alakohta{Kuljettu matka ja kulunut aika, kun nopeus on vakio.}
        \alakohta{Dieselin hinta ja 50 eurolla saatavan dieselin määrä.}
        \alakohta{Irtomyynnistä ostetun suolakurkkuerän paino ja hinta.}
    \end{alakohdat}
    Tutki miten toinen muuttuja muuttuu ensimmäisen kaksinkertaistuessa tai puolittuessa. Ovatko muuttujat suoraan tai kääntäen verrannolliset vai eivät kumpaakaan?
    \begin{vastaus}
        Vastaus:
        \begin{alakohdat}
            \alakohta{Eivät ole kumpaakaan.}
            \alakohta{Kääntäen verrannolliset.}
            \alakohta{Suoraan verrannolliset.}
            \alakohta{Kääntäen verrannolliset.}
            \alakohta{Suoraan verrannolliset.}
        \end{alakohdat}
    \end{vastaus}
\end{tehtava}

%Tarkistanut V-P Kilpi 2013-11-09
\begin{tehtava}
Ratkaise
\begin{alakohdat}
\alakohta{$ \frac{x}{3} = 1$}
\alakohta{$ \frac{8}{x} = 2$}
\alakohta{$ \frac{7}{x} = \frac{16}{8}$}
\alakohta{$ \frac{x}{3} = \frac{1}{7}$}
\end{alakohdat}
\begin{vastaus}
\begin{alakohdat}
\alakohta{$x= \frac{1}{3}$}
\alakohta{$y= \frac{1}{4}$}
\alakohta{$x= \frac{7}{2}$}
\alakohta{$x= \frac{3}{7}$}
\end{alakohdat}
\end{vastaus}
\end{tehtava}

%Tarkistanut V-P Kilpi 2013-11-09
\begin{tehtava}
Muodosta seuraavia tilanteita kuvaavat yhtälöt. Voit käyttää vakion merkkinä esimerkiksi kirjainta $c$. %%%miksei vakiintunutta k:ta?
\begin{alakohdat}
\alakohta{Kultakimpaleen arvo $x$ on suoraan verrannollinen sen massaan $m$, eli mitä painavampi kimpale, sitä arvokkaampi.}
\alakohta{Aidan maalaamiseen osallistuvien ihmisten määrä {$x$} on kääntäen verrannollinen maalaamiseen kuluvaan aikaan $t$. Toisin sanoen, mitä
enemmän maalaajia, sitä nopeammin työ on valmis.}
\alakohta{Planeettojen toisiinsa aiheuttama vetovoima $F$ on suoraan verrannollinen planeettojen massojen summaan $m_1 +m_2$ ja kääntäen verrannollinen niiden välisen etäisyyden $r$ neliöön.}
\end{alakohdat}
\begin{vastaus}
\begin{alakohdat}
\alakohta{$ \frac{x}{m}=c$ tai $ x=cm$. }
\alakohta{$ xt=c $ tai $ x=\frac{c}{t}$.}
\alakohta{$ \frac{Fr^2}{m_1+m_2}=c$ tai $ F=c\frac{m_1 +m_2}{r²} $}
\end{alakohdat}
\end{vastaus}
\end{tehtava}

%Laatinut V-P Kilpi 2013-11-09
\begin{tehtava}
Hiusten pituuskasvu on suoraan verrannollinen kuluneeseen aikaan. Iinan hiukset kasvavat noin 1,5 senttimetriä kuukaudessa. Iina päättää kasvattaa hiuksia vuoden ajan. Kuinka paljon Iinan hiukset tänä aikana kasvavat?
\begin{vastaus}
Hiukset kasvavat vuodessa noin 18cm.
\end{vastaus}
\end{tehtava}

%Tarkistanut V-P Kilpi 2013-11-09
\begin{tehtava}
Rento pyöräilyvauhti kaupunkiolosuhteissa on noin $20$~km/h. Lukiolta urheiluhallille on matkaa $7$~km. Kuinka monta minuuttia kestää arviolta pyöräillä lukiolta urheiluhallille?
\begin{vastaus}
Viiden minuutin tarkkuudella $20$ min.
\end{vastaus}
\end{tehtava}

%Laatinut V-P Kilpi 2013-11-09
\begin{tehtava}
Kun muuttujan $ x $ arvo on $ 6 $, muuttuja $ y $ saa arvon $ 12 $. Minkä arvon muuttuja $ y $ saa silloin, kun $ x=18 $ ja
	\begin{alakohdat}
\alakohta{muuttujat $ x $ ja $ y $ ovat suoraan verrannollisia?}
\alakohta{muuttujat $ x $ ja $ y $ ovat kääntäen verrannollisia?}
\end{alakohdat} 
	\begin{vastaus}
		\begin{alakohdat}
\alakohta{36}
\alakohta{4}
\end{alakohdat} 
	\end{vastaus}
\end{tehtava}

%Laatinut V-P Kilpi 2013-11-09
\begin{tehtava}
Yhteen sokerikakkuun tarvitaan neljä kananmunaa, 1,5 desilitraa sokeria ja kaksi desilitraa vehnäjauhoja. Timolla on keittiössään 20 kananmunaa, litra sokeria ja 1,5 litraa jauhoja. Kuinka monta sokerikakkua Timo voi leipoa?  
\begin{vastaus}
Munia riittäisi viiteen, sokeria kuuteen ja jauhoja seitsemään kokonaiseen kakkuun. Kaikki aineksia tarvitaan, joten Timo voi leipoa vain viisi kakkua.
\end{vastaus}
\end{tehtava}

\subsubsection*{Hallitse kokonaisuus}


%Laatinut V-P Kilpi 2013-11-09
\begin{tehtava}
Ovatko muuttujat $ x $ ja $ y $ suoraan tai kääntäen verrannolliset vai eivät kumpaakaan,
\begin{alakohdat}
\alakohta{jos niiden osamäärä on aina 2?}
\alakohta{jos niiden summa on aina 5?}
\alakohta{jos niiden tulo on aina 4?}
\alakohta{jos niiden osamäärä on aina 1?}
\end{alakohdat}
\begin{vastaus}
\begin{alakohdat}
\alakohta{Suoraan verrannollisia}
\alakohta{Eivät kumpaakaan, sillä niiden tulo eikä osamäärä ei ole vakio. }
\alakohta{Kääntäen verrannollisia}
\alakohta{Kääntäen ja suoraan verrannollisia, sillä niiden tulo ja osamäärä ovat vakioita.}
\end{alakohdat}
\end{vastaus}
\end{tehtava}

%Tarkistanut V-P Kilpi 2013-11-09
\begin{tehtava}
    Isä ja lapset ovat ajamassa mökille Sotkamoon. On ajettu jo neljä
    viidesosaa matkasta, ja aikaa on kulunut kaksi tuntia. ''Joko ollaan perillä?''
    lapset kysyvät takapenkiltä. Kuinka pitkään vielä arviolta kuluu, ennen
    kuin ollaan mökillä?
    
    \begin{vastaus}
        $30$~min
    \end{vastaus}
\end{tehtava}

%Laatinut V-P Kilpi 2013-11-09
\begin{tehtava}
Sotkamossa isä lämmittää savusaunaa, jonka käyttöaika on suoraan verrannollinen lämmitysaikaan. Jos saunaa lämmittää kolme tuntia, pysyy se kuumana kaksi tuntia. Perhe haluaa saunoa kolme tuntia kello kahdeksasta eteenpäin. Milloin saunaa pitää alkaa lämmittää?
\begin{vastaus}
Puoli neljältä.
\end{vastaus}
\end{tehtava}

%Tarkistanut V-P Kilpi
\begin{tehtava}
    Äidinkielen kurssilla annettiin tehtäväksi lukea 300-sivuinen romaani.
    Eräs opiskelija otti aikaa ja selvitti lukevansa vartissa seitsemän sivua.
    Kuinka monta tuntia häneltä kuluu koko romaanin lukemiseen, jos
    taukoja ei lasketa?
    
    \begin{vastaus}
        Noin $642$ minuuttia eli $10$~h $42$~min.
    \end{vastaus}
\end{tehtava}

%Tarkistanut V-P Kilpi
\begin{tehtava}
Alkuperäiskansojen perinteitä tutkiva etnologi kerää saamelaisten kansantarinoita. Kierrettyään viikkojen ajan saamelaisperheiden vieraana palaa etnologi yliopistolle mukanaan äänitiedostoiksi tallennettuja haastatteluja. Jotta etnologi voi ryhtyä kirjoittamaan tutkimusraporttia, pitää haastattelutiedostot ensin purkaa tekstiksi eli \emph{litteroida.} Tähän etnologi pyytää apua laitoksen opiskelijoilta. Neljä opiskelijaa litteroi tiedostot $25$~tunnissa. Kuinka kauan aikaa kuluu, jos työhön saadaan vielä kolme opiskelijaa lisää? Kuinka monta opiskelijaa tarvitaan, jotta työ saataisiin tehdyksi $9$~tunnissa? Oletetaan, että kaikkien opiskelijoiden työteho on sama.

\begin{vastaus}
Seitsemällä opiskelijalla työ valmistuu noin $14$~tunnissa. Opiskelijoita tarvitaan $12$. Silloin työ valmistuu $8$~tunnissa ja $20$ minuutissa. 
\end{vastaus}
\end{tehtava}

\begin{tehtava}
	Kappaleen liike-energia on suoraan verrannollinen nopeuden neliöön.
	Kun kappale liikkui nopeudella $9,0$~m/s, sillä oli $150$~joulea liike-energiaa.
	Laske kappaleen liike-energia, kun sen nopeus on $5,0$~m/s. 
	\begin{vastaus}
		Noin $46$ joulea.
	\end{vastaus}
\end{tehtava}

\subsubsection*{Lisätehtäviä}

%Laatinut V-P Kilpi 2013-11-09
\begin{tehtava}
	Metsäpalon pinta-ala on suoraan verrannollinen palamisajan neliöön. Kahdessa päivässä metsää on palanut neliökilometrin verran. 
	\begin{alakohdat}
\alakohta{Kuinka monta neliökilometriä metsää on palanut, kun metsäpalo on kestänyt kuusi päivää?}
\alakohta{Kuinka monen päivän päästä metsää on palanut yli 30 neliökilometriä?}
\end{alakohdat} 
	\begin{vastaus}
		\begin{alakohdat}
\alakohta{Yhdeksän neliökilometriä.}
\alakohta{Viiden päivän päästä.}
\end{alakohdat} 
	\end{vastaus}
\end{tehtava}

%Laatinut V-P Kilpi 2013-11-09
\begin{tehtava}
Pallon muotoisen lankakerän halkaisija on suoraan verrannollinen langan pituuden kuutioon. Kun lankakerässä on 500 metriä lankaa on sen halkaisija 4 cm. Jättimäisen lankakerän halkaisija on kaksi metriä. Riitäisikö siinä oleva lanka levitettäväksi maapallon ympäri? Maapallon ympärysmitta on noin 40 000km.
\begin{vastaus}
Riittäisi, sillä lankakerä sisältää noin 60 000km lankaa.
\end{vastaus}
\end{tehtava}

%Tarkistanut V-P Kilpi 2013-11-09
\begin{tehtava}
	Kymmenen miestä kaivaa viisitoista kuoppaa kymmenessä tunnissa.
	Kuinka kauan viidellä miehellä kestää kaivaa $30$ kuoppaa?	
	\begin{vastaus}
		$40$ tuntia
	\end{vastaus}
\end{tehtava}

\begin{tehtava}
	Muuttujat $x$ ja $y$ ovat
	\begin{alakohdatrivi}
		\alakohta{suoraan}
		\alakohta{kääntäen}
	\end{alakohdatrivi}
	verrannollisia. Kun $x = 5$, $y = 8$. Mikä on $y$:n arvo kun $x = 4$?
	\begin{vastaus}
		\begin{alakohdat}
			\alakohta{$\frac{32}{5}$}
			\alakohta{$10$}
		\end{alakohdat}
	\end{vastaus}
\end{tehtava}

%Laatinut V-P Kilpi 2013-11-09
\begin{tehtava}
	Karkkikulhon tyhjentymisaika on kääntäen verrannollinen syöjien määrään. Kun syöjiä on 20, kulho syödään tyhjäksi kahdessa tunnissa. Juhliin kutsuttiin sata ihmistä, mutta kulhon tyhjäksi syömiseen kului puoli tuntia. Kuinka moni vieraista ei syönyt karkkeja?
	\begin{vastaus}
		Sadasta vieraasta 20 ei syönyt karkkeja.
	\end{vastaus}
\end{tehtava}

%Laatinut V-P Kilpi 2013-11-09
\begin{tehtava}
Kenkätehtaalla sata työntekijää valmistaa yhdeksässä tunnissa 500 paria kenkiä. Tehtaan omistaja päättää siirtää tuotannon halvemman työvoiman maahan, jossa hänellä on varaa palkata 300 työntekijää. 

\begin{alakohdat}
\alakohta{Jos uudet työntekijät ovat yhtä tehokkaita kuin vanhat, kuinka kauan heillä kestää tehdä 500 paria kenkiä?}
\alakohta{Uudet työntekijät suostuvat tekemään 15 tunnin työpäivää. Kuinka monta paria kenkiä yhden työpäivän aikana valmistuu?}
\end{alakohdat} 
\begin{vastaus}
\begin{alakohdat}
\alakohta{Kolme tuntia.}
\alakohta{2500 paria kenkiä}
\end{alakohdat} 
\end{vastaus}
\end{tehtava}

%Laatinut V-P Kilpi 2013-11-09
\begin{tehtava}
	Kuution tilavuus on suoraan verrannollinen sen pinta-alan neliöjuuren kolmanteen potenssiin. Kun kuution tilavuus on 64m³ sen pinta-ala on 96m². Jos kuution pinta-ala on 100m², mikä on sen tilavuus?
	\begin{vastaus}
		${\sqrt{\frac{100}{6}}}³\approx 68$
	\end{vastaus}
\end{tehtava}

\end{tehtavasivu}
