Mitä tahansa lukua tai kirjoitettua laskutoimitusta (kuten kuten $9$ tai $\frac{x}{3}-9$) kutsutaan \termi{Lauseke}{lausekkeeksi}. \termi{Termi}{Termi} puolestaan on yksi yhteenlaskettava osa lausekkeesta. Termissä voi olla yhdistelmä luvuista, vakioista ja muuttujista tulona, kuten $-4\sqrt{2}ax^2$ tai se voi olla jokin näistä yksittäisenä kuten pelkkä $1$. 

\termi{Yhtälö}{Yhtälö} on väite kahden lausekkeen yhtäsuuruudesta, kuten $x+3=5/7+2$. Yhtälöitä käyttämällä voidaan mallintaa monia käytännön tilanteita.

\begin{esimerkki}
	Yksinkertainen esimerkki yhtälön käyttämisestä on matkaan kuluva ajan ratkaiseminen matkan pituuden ja nopeuden avulla. Kuljettu matka on nopeus kerrottuna kuljetulla ajalla. 
	
	Jos matka olisi 15 kilometriä ja nopeus on 5 km/h voimme merkitä $5\cdot t=15$ , jossa t on matkaan kuluvaa aikaa kuvaava \termi{muuttuja}{muuttuja}
\end{esimerkki}

Toisaalta yhtälöjä voidaan käyttää myös abstraktimpien pulmien ratkaisuun.

\begin{esimerkki}
	Halutaan tietää onko olemassa lukua, jonka kolmas potenssi jaettuna kolmella on yhtä suuri kuin sen toinen potenssi jaettuna kahdella. Tällöin merkitään $\frac{x^3}{3}=\frac{x^2}{2}$. 
\end{esimerkki}
	%%\begin{esimerkki}
%%	Merkitään lausekkeet $5x+\sqrt{x}$ ja $7x+7$ yhtäsuuriksi, jolloin saadaan yhtälö $5x+\sqrt{x} = 7x+7$.
%%\end{esimerkki}

\laatikko{
	Jos yhtälön kummankin puolen lausekkeen arvo on sama, sanotaan että \termi{yhtälön päteminen}{yhtälö pätee}.
}

\begin{esimerkki}
	\begin{alakohdat}
		\alakohta{Yhtälö $3x + 2 = 0$ pätee, kun $x = - \frac{2}{3}$.}
		\alakohta{Yhtälö $5 = 3$ ei päde.}
	\end{alakohdat}
\end{esimerkki}


\laatikko[Yhtälöihin liittyviä termejä]{
	\begin{description}
		\item[Tuntematon eli muuttuja] Luku, jonka arvoa ei tiedetä. Tuntemattomia merkitään vaihtelevilla symboleilla. Jos tuntemattomia on vain yksi, sitä merkitään 	yleensä kirjaimella $x$.
		\item[Yhtälön ratkaisu(t)] Muuttujan arvo(t), jolla yhtälö pätee.
		\item[Yhtälön ratkaiseminen] Kaikkien yhtälön ratkaisujen selvittäminen.
	\end{description}
%Muuttujista höpöttää Jaakko Viertiö 2013-11-10
	
}

Huomioi, että yhtälössä voi olla useitakin eri tuntemattomia. Kaikki kirjaimet eivät myöskään ole välttämättä tuntemattomia, vaan niillä voidaan tarkoittaa myös vakioita eli muuttumattomia lukuarvoja. Esimerkiksi fysiikassa käytetään usein eri kirjaimia merkitsemään eri luonnonvakioita.


\begin{esimerkki}
 Tarkastellaan legendaarista fysiikan yhtälöä $E=mc^2$. Tässä yhtälössä on vakio $c$ eli valonnopeus, joka on luonnonvakio joka ei muutu. Lisäksi yhtälössä on muuttujat $E$ eli energia ja $m$ eli massa.
\end{esimerkki}

%Lisännyt Jaakko Viertiö 2013-11-10
\begin{esimerkki}
 Tarkastellaan yhtälöä $x^3-2x^2-x+2=0$. 
 
 Yhtälössä on yksi muuttuja $x$ ja neljä termiä: $x^3$, $-2x^2$, $-x$ ja $+2$.  
 
 Yhtälön \textit{eräs} ratkaisu on $x=1$, sillä $1^3-2\cdot{1^2}-1+2=0$. 
 Myös $x=2$ on kyseisen yhtälön ratkaisu, sillä $2^3-2\cdot{2^2}-2+2=0$. 
 Tuntemattomalle $x$ voidaan siis joissain tapauksissa löytää useita arvoja, jotka toteuttavat yhtälön. 
\end{esimerkki}


% FIXME: jokin helposti ymmärrettävä esimerkki tuntemattomista

Tyypillinen tapa ratkaista yhtälöitä on kirjoittaa ne ilmaistuna toisella tavalla.
Käytännössä tämä tarkoittaa niiden muokkaamista siten, ettei alkuperäisen yhtälön paikkansapitävyys muutu. Tällä tavalla saadaan eri yhtälö, jolla on kuitenkin samat ratkaisut.
%Lisännyt Jaakko Viertiö 2013-11-10. Rautalangasta aina paras!
Yleensä tavoitteena on saada yhtälö muotoon, jossa yhtäpitävyyden toisella puolella on vain haluttu muuttuja ilman kertoimia, ja toisella puolella kaikki muu. 

\laatikko[Yhtälöiden muokkaaminen]{
	\begin{description}
		\item[Laskutoimitukset], joita ovat
		
			\begin{itemize}
				\item{Puolittain lisääminen ja vähentäminen: yhtälön molemmille puolille voidaan lisätä tai molemmilta puolilta voidaan vähentää luku tai muuttuja. 
				Esimerkiksi yhtälö $3x+5 = 3$ saadaan näin muotoon $3x = -2$.}
				\item{Puolittain kertominen tai jakaminen: yhtälön molemmat puolet voidaan kertoa tai jakaa nollasta poikkeavalla luvulla, muuttujalla tai lausekkeella. Muuttujalla tai lausekkeella jakaessa täytyy ottaa huomioon ja merkitä, että yhtälö ei ole määritelty, kun jakaja on nolla.
				Esimerkiksi kertomalla yhtälön $2x = 4$ molemmat puolet luvulla $\frac{1}{2}$ saadaan yhtälö $x = 2$.}
				\item{Huomaa, että itseasiassa vähennyslasku on negatiivisen luvun summaamista [$a-b=a+(-b)$], 
				ja jakaminen on jakajan käänteisluvulla kertomista [$a:b=a\cdot{\frac{1}{b}}=\frac{a}{b}$]}				
			\end{itemize}
		\item[Sieventäminen laskusääntöjä käyttäen] Yhtälön lausekkeista voi esimerkiksi hakea yhteistä tekijää, avata sulkuja tai supistaa, 
		ja pyrkiä täten saamaan lauseke yksinkertaisempaan muotoon. Esimerkiksi yhtälöä $3(x-2)=5(x+6)$ kannattaa lähteä ratkaisemaan avaamalla ensin 
		sulut soveltamalla tuttuja laskusääntöjä. Näin yhtälö saadaan muotoon $3x-6=5x+30$, jonka jälkeen on helppoa jatkaa käyttämällä laskutoimituksia 
		tuntemattoman $x$ ratkaisemiseksi. Yhtälön ratkaisu on $x=-18$.
	\end{description}
%Huomautuksen ja jakolaskumaininnan lisännyt Jaakko Viertiö 2013-11-10
%Muokannut laajemminkin Jaakko Viertiö 2013-11-10
}

Yhtälön ratkaisemiseksi näitä muunnoksia toistetaan, kunnes yhtälön ratkaisu on helposti luettavissa.

\begin{esimerkki}
Ratkaistaan esimerkin 2.36 matkan kesto. Yhtälö on siis $5t=15$, jossa t on aika, nopeus on 5 km/h ja matkan pituus 15 km.
Laskusääntöjen mukaan jakamalla molemmat puolet viidellä, saamme yhtälön helposti muotoon $t=3$, joka on siis kyseisen yhtälön ratkaisu.

\end{esimerkki}


\begin{esimerkki}
Ratkaistaan yhtälö $\frac{4\sqrt{x}a}{3}=8a$, jossa $a$ on vakio $\pi$ eli yhtälö voidaan kirjoittaa muotoon $\frac{4\pi\sqrt{x}}{3}=8\pi$

		\begin{align*}
			\frac{4 \cdot \pi \cdot \sqrt{x}}{3} &= 8 \cdot {\pi} && \text{| Jaetaan molemmat puolet $\frac{4}{3}$} \\
			{\pi}\sqrt{x} &= \frac{8\cdot 3\cdot {\pi}}{4} && \text{| Jaetaan molemmat puolet $\pi$} \\
			\sqrt{x} &= \frac{24}{4} && \text{| Sievennellään} \\
			\sqrt{x} &= 6 && \text{| Tässä huomataan, että $\sqrt{36}=6$} \\
\end{align*}

Joten yhtälöllä on ratkaisu $x=36$.

\end{esimerkki}

\begin{esimerkki}
	Kuvitellaan orsivaaka, joka on tasapainossa. Vasemmalla ja oikealla puolella on eripainoisia esineitä, mutta ne painavat yhteensä yhtä paljon. Jos molemmille puolille lisätään nyt saman verran painoa, vaaka on yhä tasapainossa. Samalla tavalla yhtälön molemmille puolille on sallittua lisätä sama luku.
\end{esimerkki}

\begin{esimerkki}
	Kuvassa oleva vaaka on tasapainossa. Toisessa vaakakupissa on kahden kilon siika ja toisessa puolen kilon ahven sekä tuntematon määrä lakritsia.
	Kuinka paljon vaakakupissa on lakritsia?
	\begin{center}
		\includegraphics[scale=0.6]{pictures/Kuva10-1-vaaka.pdf} % CC-BY Lilja Tamminen
		%\includegraphics{unused/kala-vaaka.png} % CC-BY Hannu Köngäs
		% FIXME: Hannun piirros on viimeistellympi (smoothit värjäykset jne.), Liljan piirros taas konsistentimpi
		% kirjan muun kuvituksen kanssa. Kumpi jätetään? Nyt päällä Liljan kuva.
	\end{center}
	\begin{esimratk}
		Merkitään lakritsin määrää tuntemattomalla $x$. Tilannetta kuvaa yhtälö $2 = 0,5 + x$.
		Ratkaistaan yhtälö.
		\begin{align*}
			2 &= 0,5 + x &&\text{| $-0,5$} \\
			2 - 0,5 &= x && \\
			1,5 &= x && \\
			x &= 1{,5} &&
		\end{align*}
	\end{esimratk}
	\begin{esimvast}
		Lakritsia on $1,5$ kg.
	\end{esimvast}
\end{esimerkki}

\begin{esimerkki}
	Juna Helsingistä Jyväskylään kulkee $342$ kilometrin matkan kolmessa tunnissa ja $23$ minuutissa tasaista vauhtia.
	$10$ minuuttia junan lähdön jälkeen auto lähtee Helsingistä Jyväskylään kulkien tasaista $100$ kilometrin tuntivauhtia.
	Jyväskylään on Helsingistä maanteitse $272$ kilometriä.
	Kuinka kauan auton lähdöstä on kulunut, kun juna ja auto ovat kulkeneet yhtä suuren osan omista matkoistaan?
	\begin{esimratk}
		Lasketaan aluksi junan nopeus.
		\[ 3 \; \text{h} \; 23 \; \text{min} \; = 12~180 \; \text{s} \]
		\[ \dfrac{342~000 \; \text{m}}{12~180 \; \text{s}} \approx  28,0788 \frac{\text{m}}{\text{s}} \]
		
		Muunnetaan lisäksi auton nopeus yksikköön $\frac{\text{m}}{\text{s}}$ (muuntokerroin $3,6$).
		\[ 100 \frac{\text{km}}{\text{h}} \approx 27,7778 \frac{\text{m}}{\text{s}} \]
		
		$10$ minuuttia on $600$ sekuntia.
		
		Voidaan nyt muotoilla yhtälö käyttäen yksiköitä metri ja sekunti. Tuntemattomana on $t$, aika sekunteina auton lähdöstä.
		\[ \dfrac{28,0788(t+600)}{342000} = \dfrac{27,7778t}{272000} \]
		
		Ratkaistaan yhtälö.
		%Välivaiheselityksiä lisäsi Jaakko Viertiö 2013-11-10
		\begin{align*}
			\dfrac{27,7778t}{272000} &= \dfrac{28,0788(t+600)}{342000} &&\text{| $\cdot{272000}$} &&\text{| $\cdot{342000}$}  \\
			342000 \cdot 27,7778t &= 272000 \cdot 28,0788(t+600) \\
			9500007,6t &= 7637433,6(t+600) \\
			9500007,6t &= 7637433,6t + 4582460160 &&\text{| $-7637433,6t$} \\ 
			1862574t &= 4582460160 &&\text{| $:{1862574}$} \\
			t &\approx 2460,28 \approx 2460
		\end{align*}
		
		Muutetaan yksiköksi minuutit.
		\[ 2460 \; \text{s} \; = 41 \; \text{min} \]
	\end{esimratk}
	\begin{esimvast}
		$41$ minuuttia.
	\end{esimvast}
\end{esimerkki}



\laatikko[Yhtälöiden luokittelu]{
	\begin{description}
		\item[Aina tosi yhtälö] Esimerkiksi yhtälöt $8=8$ ja $x=x$.
		\item[Joskus tosi yhtälö] Esimerkiksi yhtälö $x+4=7$ on tosi, kun $x=3$, ja epätosi muulloin.
		\item[Ei koskaan tosi yhtälö] Esimerkiksi yhtälö $0=1$.
	\end{description}
}

Tärkeimpiä näistä ovat joskus todet yhtälöt. Seuraavaksi on esimerkki, joka osoittaa, kuinka näihin vaihtoehtoisiin tilanteisiin saatetaan päätyä yhtälöä
ratkaistaessa.

\begin{esimerkki}
%Lisännyt Jaakko Viertiö 2013-11-10
	\begin{itemize}
		\item{Yhtälö $3x-12=6(\frac{1}{2}x-2)$ on yhtäpitävä yhtälön $0=0$ kanssa, sillä se saadaan tähän muotoon 
		tekemällä yhtäpitävyyden molemmille puolille aina samat laskutoimitukset. Yhtälön yksinkertaisemmasta esityksestä $0=0$ nähdään, että muuttujan $x$ 
		arvo ei vaikuta yhtälön ratkaisuihin. Toisin sanoen yhtälö on täysin sama, olipa muuttuja $x$ mikä tahansa luku. Yhtälö on siis aina tosi.}
		\item{Yhtälö $3x-11=6(\frac{1}{2}x-2)$ on yhtäpitävä yhtälön $0=1$ kanssa. Jälkimmäisestä muodosta nähdään, ettei yhtälö voi olla tosi millään
		muuttujan $x$ arvolla, sillä yhtälön muodossa $0=1$ ei esiinny muuttujaa $x$. Täten $x$ ei vaikuta yhtälön ratkaisuihin. 
		Yhtälö on siis aina epätosi.}
		\item{Yhtälö $3x-11=6(x-2)$ on yhtäpitävä yhtälön $x=\frac{1}{3}$ kanssa. Yhtälö on siis joskus tosi: täsmälleen silloin kun $x=\frac{1}{3}$.
		Tämä voidaan tarkistaa sijoittamalla yhtälöön $x$:n paikalle arvo $\frac{1}{3}$, jolloin yhtälön vasen ja oikea puoli ovat yhtä suuret.}
	\end{itemize}

\end{esimerkki}


Seuraavassa joitakin yleisiä toimenpiteitä yhtälöiden ratkaisuun, joita voi soveltaa päästäkseen eteenpäin. Kyseessä ei ole mikään aina toimiva
ratkaisuohje, vaan joitain vinkkejä.


\laatikko[Yleisiä yhtälönratkaisuperiaatteita]{
	\begin{description}
		\item[1)] Kerro tuntematonta sisältävät lausekkeet pois nimittäjistä.
		\item[2)] Yhdistä useat murtolausekkeet yhdeksi laventamalla ne samannimisiksi.
		\item[3)] Siirrä tuntemattomia sisältävät yhtälön osat samalle puolelle ja ota tuntematon yhteiseksi tekijäksi.
		\item[4)] Kumoa juuret korottamalla yhtälö puolittain sopivaan potenssiin.
		\item[5)] Sulkuja ei välttämättä aina kannata kertoa auki.
		\item[6)] Yhtälö on ratkaistu vasta, kun jäljellä on enää vain yksi kappale tuntematonta suuretta, ja se sijaitsee yksin omalla puolellaan yhtälöä.
	\end{description}
}
%Muokannut alustavasti Jaakko Viertiö 2013-11-10

Siirrymme nyt tarkastelemaan tärkeää yhtälöiden lajia, ensimmäisen asteen yhtälöitä.
