\begin{tehtavasivu}

\paragraph*{Opi perusteet}

\begin{tehtava}
Hahmottele funktion kuvaaja kynällä ja paperilla tai laskimen avulla. Voit myös käyttää tietokonetta.
\begin{alakohdat}
\alakohta{$f(x) = 2$}
\alakohta{$f(x) = 3x+2$}
\alakohta{$f(x) = x^2$}
\alakohta{$f(x) = \frac{2}{x}$}
\end{alakohdat}

%\begin{vastaus}
%\end{vastaus}
\end{tehtava}

\begin{tehtava}
  Laske taulukkoon funktion arvo pisteissä $x=-2$, $x=-1$, $x=0$, $x=1$ ja $x=2$. Hahmottele näiden tietojen avulla funktion kuvaaja.
  \begin{alakohdat}
    \alakohta{$f(x)= x^2+x+1$}
    \alakohta{$f(x)= 3x^2$}
    \alakohta{$f(x)= x^3 +2x+2$}
  \end{alakohdat}

  \begin{vastaus}
    \begin{alakohdat}
      \alakohta{$f(-2)=3,f(-1)=1, f(0)=1, f(1)=3$ ja $f(2)=7$}
      \alakohta{$f(-2)=12, f(-1)=3, f(0)=0, f(1)$ ja $f(2)=12$}
      \alakohta{$f(-2)=6, f(-1)=1, f(0)=2, f(1)=5$ ja $f(2)=14$}
    \end{alakohdat}
  \end{vastaus}
\end{tehtava}

\begin{tehtava}
%Laatinut Emilia Välttilä 9.11.2013
  Mikä on funktion $f(x)=\frac{x+1}{2x+5}$ määrittelyjoukko?

  \begin{vastaus}
    
  \end{vastaus}
\end{tehtava}

\paragraph*{Hallitse kokonaisuus}

\begin{tehtava}
%Laatinut Emilia Välttilä 9.11.2013
  Mikä on funktion määrittelyjoukko?
  \begin{alakohdat}
    \alakohta{$f(x)= \sqrt{x+6}$}
    \alakohta{$f(x)= \frac{2x}{\sqrt{x+1}}$}
    \alakohta{$f(x)= \sqrt{x-\frac{1}{2}}$}
  \end{alakohdat}

  \begin{vastaus}
  \end{vastaus}
\end{tehtava}

\begin{tehtava} % Emilia Välttilä 9.11.2013
	Olkoon $f(x) = 2x-1$ ja $g(x) = 2-x$. Millä x:n arvolla $f(x) = g(x)$? Piirrä funktioiden kuvaajat.
\end{tehtava}

\begin{tehtava} % Emilia Välttilä 9.11.2013
	Missä pisteessä funktio $P(t)=-2t+4$ saa arvon
	\begin{alakohdat}
	  \alakohta{4}
	  \alakohta{-2}
	  \alakohta{8}
	 \end{alakohdat}
	 Funktion nollakohdaksi kutsutaan kohtaa, jossa funktio saa arvon nolla. Mikä on funktion $P(t)$ nollakohta?
	 
\end{tehtava}

\paragraph*{Muita tehtäviä}

\begin{tehtava}
	Olkoon $f(x)=\frac{x^2+3x+1}{x^2-x}$. Laske
	\begin{alakohdat}
		\alakohta{$f(2)$}
		\alakohta{$f(1)$}
		\alakohta{$f(0)$}
		\alakohta{$f(-1)$}
	\end{alakohdat}
	\begin{vastaus}
		\begin{alakohdat}
			\alakohta{$\frac{11}{3}$}
			\alakohta{ei määritelty}
			\alakohta{ei määritelty}
			\alakohta{$\frac{-1}{2}$}
		\end{alakohdat}
	\end{vastaus}
\end{tehtava}

\begin{tehtava}
	Olkoon $f(x)=\frac{2^x+4}{x}$. Laske
	\begin{alakohdat}
		\alakohta{$f(2)$}
		\alakohta{$f(\frac{1}{3})$}
		\alakohta{$f(0)$}
		\alakohta{$f(-1)$}
	\end{alakohdat}
	\begin{vastaus}
		\begin{alakohdat}
			\alakohta{$4$}
			\alakohta{$3\sqrt[3]{2}+12$}
			\alakohta{ei määritelty}
			\alakohta{$\frac{-9}{2}$}
		\end{alakohdat}
	\end{vastaus}
\end{tehtava}

% kai vaikeahko
\begin{tehtava}
	Millä $x$:n arvoilla yhtälö $f(f(x)) = x$ pätee, kun
	\begin{alakohdat}
		\alakohta{$f(x) = 1$}
		\alakohta{$f(x) = x$}
		\alakohta{$f(x) = x+1$}
		\alakohta{$f(x) = 2x+1$?}
	\end{alakohdat}
	\begin{vastaus}
		\begin{alakohdat}
			\alakohta{$x = 1$}
			\alakohta{kaikilla $x\in\rr$}
			\alakohta{ei ratkaisuja}
			\alakohta{$x = -1$}
		\end{alakohdat}
	\end{vastaus}
\end{tehtava}

\begin{tehtava} % Emilia Välttilä 9.11.2013 (Sampo Tiensuun idea)
	Olkoon $f(x) =\sqrt{x} $ 
	ja 
	\begin{equation*}
	g(x)=
		\begin{cases}
		  0, & x \in \mathbb{R},\\
		  1, & x \notin \mathbb{R}.
		\end{cases}
	\end{equation*}
	laske
	\begin{alakohdat}
			\alakohta{$g(f(2))$}
			\alakohta{$g(f(0))$}
	\end{alakohdat}
	
\end{tehtava}

\begin{tehtava} % Emilia Välttilä 9.11.2013 (Sampo Tiensuun idea)
	Olkoon $f(x) \frac{x^2+3}{x}$. Laske
	\begin{alakohdat}
		\alakohta{$f(2)$}
		\alakohta{$f(a)$}
		\alakohta{$f(2a)$}
		\alakohta{$f(2x)$}
		\alakohta{$f(\sqrt{x+1}^2)$}
	\end{alakohdat}
	
\end{tehtava}

\end{tehtavasivu}
