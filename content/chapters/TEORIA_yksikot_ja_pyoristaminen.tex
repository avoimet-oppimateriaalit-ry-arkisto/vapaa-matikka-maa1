\subsection*{Suureet ja yksiköt}

Monissa sovelluksissa (ja erityisesti fysiikassa) luvut eivät esitä vain matemaattista lukuarvoa, vaan niitä käytetään yhdessä jonkin \termi{yksikkö}{yksikön} kanssa. Tällöin luku ja yksikkö yhdessä esittävät tietoa \termi{suure}{suureesta} eli jostakin mitattavasta ominaisuudesta.

%selitä vielä, mitä yksikkö tekee?

\begin{esimerkki}

Aika on suure, jota voidaan mitata käyttämällä yksikkönä esimerkiksi sekunteja tai vuosia.

Pituus on suure, ja sen arvo voidaan mitata ja esittää käyttäen yksikkönä esimerkiksi metrejä ja tuumia. (Myös matkaa ja etäisyyttä mitataan samoissa yksiköissä; näillä suureilla on sovellusriippuvainen, erityisesti geometriassa ja fysiikassa esille tuleva nyanssiero.)

Energia on suure, jota mitataan esimerkiksi yksiköissä joule tai kalori.
\end{esimerkki}

Jotkin suureet ovat \termi{skalaarisuure}{skalaarisuureita} ja toiset \termi{vektorisuure}{vektorisuureita}. Skalaarisuureet ovat sellaisia mitattavia ominaisuuksia, joiden kuvaamiseen riittää yksi luku. Vektorisuureet ovat ominaisuuksia, joilla on sekä suuruus että suunta, eikä yksi luku riitä kertomaan kaikkea. (Vektorilaskentaan tutustutaan tarkemmin MAA5-kurssilla sekä fysiikan mekaniikan kursseilla.)

\begin{esimerkki}

Lämpötilan esittämiseen riittää yksi luku, eli lämpötila on skalaarisuure. Käytetystä yksiköstä riippuen kyseinen luku saattaa olla joko negatiivinen, nolla tai positiivinen.

Nopeus on vektorisuure, koska nopeuteen liittyy sen suuruuden lisäksi myös suunta. Esimerkiksi puhuttaessa tuulen nopeudesta kerrotaan myös, mistä suunnasta tuuli puhaltaa.

%kuva nopeusvektorinuolesta (tuulennopeus)

\end{esimerkki}

Matemaattisesti suureita käsitellään aivan kuin luku ja yksikkö olisivat kerrotut keskenään:

\laatikko[Suure]{luku $\cdot$ yksikkö}

Tehtävissä yksiköistä käytetään standardilyhenteitä, eikä yksiköiden nimiä tarvitse kirjoittaa kokonaan. Käytäntö on jättää kertolaskun sijasta luvun ja yksikön väliin tyhjää. (Näin sanoo myös suomen kielioppi.) Tähän on muutama poikkeus: kulman suuruutta ilmaisevat asteet, minuutit ja sekunnit, joita merkataan symboleilla $^\circ$, ' ja '', sekä jotkin pituusyksiköt kuten tuuma, jota merkataan ''. Yksiköt kirjoitetaan konekirjoitetussa tekstissä pystyyn, ei \textit{kursiivilla}.

%Sivunlaitahuomio:: jos yksikkö on nimetty henkilön nimen mukaan, niin yksikkö kirjoitettuna lyhenteellään alkaa isolla kirjaimella, esim. N, J tai Hz. Auki kirjoitettuna kuitenkin "kolme newtonia", kymmenen kilojoulea" ja "5,7 hertziä".

%typografiahuomautus sivun laitaan :)

%onko tuuma varmasti merkattu oikein? (kaksoislainausmerkkejä on useita erilaisia)

Seuraava taulukko selventää oikeinkirjoitusta:

\begin{center}
\begin{tabular}{c|c}
Oikein & Väärin \\
\hline
3 m & 3m 	\\
100 $\frac{\text{km}}{\text{h}}$ & 100$\frac{km}{h}$	\\
2 \% & 2\% 	\\
3$^\circ$ & 3 $^\circ$\\
3,16$^\circ$ 5' 13'' & 3,16 $^\circ$ 5 ' 13 '' 	\\
6 $^\circ$C & 6$^\circ$C 	\\
\end{tabular}
\end{center}

%Hienosti ladottuna luvun ja yksikön väliin EI tule välilyöntiä, vaan ohuke (thin space), jonka latexissa saa komennolla \,

\subsection*{SI-järjestelmä}

SI-järjestelmä eli kansainvälinen yksikköjärjestelmä (ransk. Système international d'unités) on maailman käytetyin mittayksikköjärjestelmä. Siihen kuuluvat seuraavat \termi{perussuure}{perussuureet} ja vastaavat perusyksiköt:


\begin{center}
\begin{tabular}{c|c|c|c}
Suure & Suureen tunnus & Yksikkö & Yksikön tunnus\\
\hline
pituus & $l$ & metri &	m\\
massa & $m$ &kilogramma & kg 	\\
aika & $t$ & sekunti & s \\
sähkövirta & $I$ & ampeeri & A \\
(termodynaaminen) lämpötila & $T$ & kelvin & K \\
ainemäärä & $n$ & mooli & mol \\
valovoima & $I$ & kandela & cd
\end{tabular}
\end{center}

Toisin kuin yksiköt, suureiden kirjainlyhenteet kirjoitetaan \textit{kursiivilla}. %ja tämäkin lopulta sivun laitaan pois leipätekstistä... :)

Näitä suureita käsitellään tarkemmin fysiikassa, mutta matematiikan kursseilla tulee tuntea ainakin pituus, massa ja aika. (Huomaa, että arkikielessä käytetään usein sanaa paino, vaikka tarkoitetaan massaa.) Tämän lisäksi matematiikan tehtävissä esiintyy usein näistä yhdistämällä saatuja \termi{johdannaissuure}{johdannaissuureita} kuten nopeus, pinta-ala, tilavuus ja tiheys.

\laatikko[Tärkeitä johdannaissuureita]{

\begin{itemize}
\item Nopeus (tunnus $v$) on johdannaissuure, joka saadaan kuljetun matkan ja matkan kulkemiseen käytetyn ajan osamäärästä: $\text{nopeus}=\frac{\text{matka}}{\text{aika}}$. Vastaava nopeuden yksikkö saadaan jakamalla matkan yksikkö ajan yksiköllä, käytettiin mitä yksiköitä hyvänsä: $\frac{\text{m}}{\text{s}}$ ja $\frac{\text{km}}{\text{h}}$.

\item Pinta-ala (tunnus usein $A$) saadaan kertomalla pituus toisella pituudella. Vastaavasti pinta-alan yksiköt saadaan pituuden yksiköistä. Jos esimerkiksi pituutta on mitattu metreissä, niin vastaava pinta-alan yksikkö on $m\cdot m=m^2$ eli neliömetri.

\item Tilavuus (tunnus usein $V$) on edelleen johdettavissa kolmen pituuden tulona. Jos pituutta on mitattu metreissä, niin vastaava tilavuuden yksikkö on $m\cdot m \cdot m=m^3$ eli kuutiometri.

\item Tiheys (tunnus kreikkalainen pieni roo, $\rho$) on johdannaissuure, joka määritellään massan ja tilavuuden osamääränä: $\text{tiheys}=\frac{\text{massa}}{\text{tilavuus}}$. Vastaavasti tiheyden yksiköt saadaan samalla jakolaskulla massan ja tilavuuden yksiköistä: jos kappaleen tilavuus on mitattu kuutiometreissä ja massa kilogrammoina, esitetään kappaleen tiheys yksiköissä $\frac{\text{kg}}{\text{m}^3}$.
\end{itemize}
}

Kun johdannaisyksiköissä esiintyy jakolaskua, yksikön voi kirjoittaa useassa eri muodossa potenssisääntöjen avulla.

\begin{esimerkki}

Nopeuden yksikkö metriä sekunnissa voidaan kirjoittaa murtolausekkeena $\frac{\text{m}}{\text{s}}$, samalla rivillä m/s tai potenssien avulla ms$^{-1}$.

\end{esimerkki}

Joillekin johdannaisyksiköille on annettu oma nimensä ja lyhenteensä, eikä yksikössä esiinny alkuperäisiä perusyksiköitä. 

\begin{esimerkki}

Yksi litra vastaa yhtä kuutiodesimetriä, eli $1\,\text{l}=1\,\text{dm}^3$.

Energian yksikkö joule ilmaistuna vain SI-järjestelmän perusyksiköitä käyttämällä: $\text{J}=\frac{\text{m}^2\text{kg}}{\text{s}^2}$.

\end{esimerkki}

\subsection*{Etuliitteet}

Monesti suureita kuvataan \termi{kerrannaisyksikkö}{kerrannaisyksiköiden} avulla. Tällöin yksikön suuruutta muokataan etuliitteellä. %jokseenkin epäselvä?

\laatikko{
Tavallisimmat kerrannaisyksiköiden etuliitteet:

\begin{tabular}{c|c}
\begin{tabular}{c|c|c}
Nimi & Kerroin & Tunnus \\
\hline
deka & $10^{1}$ & da 	\\
hehto & $10^{2}$ & h 	\\
kilo & $10^{3}$ & k 	\\
mega & $10^{6}$ & M 	\\
giga & $10^{9}$ & G		\\
tera & $10^{12}$ & T 	
\end{tabular}
\begin{tabular}{c|c|c}
Nimi & Kerroin & Tunnus \\
\hline
desi & $10^{-1}$ & d 	\\
sentti & $10^{-2}$ & c 	\\
milli & $10^{-3}$ & m	\\
mikro & $10^{-6}$ & $\upmu$ \\
nano & $10^{-9}$ & n 	\\
piko & $10^{-12}$ & p
\end{tabular}
\end{tabular}
}

\begin{esimerkki}

hyvin pienistä massoista puhuttaessa voidaan puhua esimerkiksi nanogrammoista.

Esimerkiksi $1 \textrm{ ng} = 1 \cdot 10^{-9} \textrm{ g} = 0,000000009 \textrm{ g} $.

ESIMERKKI RNA:ta näin monta nanogrammaa millilitrassa... ja vielä lisää...

\end{esimerkki}

Tietotekniikassa datan määrää mitataan tavuina, mutta siellä yksi kilotavu (kt) ei tarkoita tasan $1\,000$ tavua,  vaan $1\,024$ tavua ($2^{10}$). Vastaavasti yksi megatavu on $1024$ kt (eli $1\,048\,576 = 2^{20}$ tavua), yksi gigatavu $1\,024$ Mt jne.. Vastaavat englannikieliset lyhenteet ovat kB ja MB. (Byte = tavu)

% esimerkki: dm^2 =(dm)^2=d^2 m^2 TÄMÄ ON TÄRKEÄÄ!!!!!!!!!  dm^2 tarkoittaa oikeasti (dm)^2

\subsection*{Yksiköiden yhdistäminen}

Jos lausekkeessa on useita samanlaisia yksiköitä, näitä voidaan usein yhdistää. Käytännössä yksiköitä kohdellaan kuin lukuja, ja niitä voi kertoa ja jakaa keskenään. Suureen määritelmän luvun ja yksikön kertolaskukäsityksestä seuraa, että sovelluksissa ja sievennystehtävissä yksiköillä voidaan laskea kuin luvuilla. Tutut potenssisäännöt, rationaalilukujen laskusäännöt ja kertolaskun vaihdannaisuuslaki pätevät –- lopullisessa sievennetyssä muodossa vain merkataan luvun ja yksikön väliin väli kertolaskun sijaan.

\begin{esimerkki}
2 metriä kertaa 3 metriä = $2$ m $\cdot$ $3$ m $=2 \cdot 3 \cdot$ m $\cdot$ m $=6 $ m$^2$ eli kuusi neliömetriä.
\end{esimerkki}

\begin{esimerkki}
Matkaan kulunut aika voidaan laskea kaavalla $t=s/v$, missä $s$ on kuljettu matka ja $v$ on nopeus Lasketaan, kuinka kauan kestää 300 metrin matka nopeudella 7 m/s.

Sijoitetaan yksikköineen $s=300$ m ja $v= 7$ m/s ja käytetään rationaalilukujen laskusääntöjä, jolloin saadaan:
\begin{equation*}
t=\frac{100 \textrm{ m}}{ 7 \textrm{ m/s}} = \frac{100}{7} \cdot \frac{\textrm{m}}{\frac{\textrm{m}}{\textrm{s}}} 
= \frac{100}{7} \cdot \frac{\textrm{m}}{1} \cdot \frac{\textrm{s}}{\textrm{m}}
= \frac{100}{7} \cdot \frac{\cancel{m}\textrm{s}}{\cancel{m}}
=\frac{100}{7} \textrm{ s} \approx 14 \textrm{ s}.
\end{equation*}
\end{esimerkki}

%Lisää vaiheita! (selvennetään lukujen ja yksiökiden erotusta ja rationaalilukujen laskusääntöjä) Oisko pitänyt laittaa vaiheittain tuo vielä, että voi merkata välivaiheissa tehdyt jutut... olisi hyvä.

Huomaa, että eri suureita voi kertoa ja jakaa keskenään, mutta yhteen- ja vähennyslasku ei ole määritelty. (Mitä tarkoittaisikaan "kolme metriä plus kaksi grammaa"?)

%tiheysesimerkki

\begin{esimerkki}
$3 \textrm{ m} \cdot 4 \textrm{ m} = 3\cdot 4 \textrm{ m}\cdot \textrm{ m} = 12 \textrm{ m}^2$ \\
$\frac{90 \textrm{ m}}{10 \frac{\textrm{m}}{\textrm{s}}} %= \frac{90}{10} \cdot \frac{\textrm{m}}{\frac{\textrm{m}}{\textrm{s}}}
=9 \textrm{ m}:\frac{\textrm{m}}{\textrm{s}}=9 \textrm{ m} \cdot \frac{\textrm{s}}{\textrm{m}} = 9 \frac{\textrm{ms}}{\textrm{m}}=9 \textrm{ s}$
\end{esimerkki}

\subsection*{Yksikkömuunnokset}

\laatikko{
Yksi tunti on $60$ minuuttia. Yksi minuutti on $60$ sekuntia.
\begin{itemize}
  \item $1~\text{h} = 60~\text{min}$
  \item $1~\text{min} = 60~\text{s}$
  \item $1~\text{h} = 60~\text{min} = 60 \cdot 60~\text{s} = 3600~\text{s}$ 
\end{itemize}
}

% FIXME \todo{taulukko amerikkalaisista yksiköistä}

\begin{esimerkki}
Kuinka monta minuuttia on $1,25$ h?

\textbf{Ratkaisu. }
$1,25 \text{ h} = 1,25 \cdot 60 \text{ min} = 75 \text{ min}$. $1,25$ h on siis $75$ minuuttia. Huomaa, että voit laskuissasi esittää desimaaliluvun $1,25$ yhdistettynä lukuna $1 \frac{25}{100}$ eli $1 \frac{1}{4}$, mikä saattaa helpottaa laskemista.
\end{esimerkki}

\begin{esimerkki}
Erään punaisen verisolun tilavuus on $9,0 \cdot 10^{-12} \textrm{l}$. Ilmoita tilavuus

a) pikolitroina

b) nanolitroina

\begin{esimratk}
$9,0 \cdot 10^{-12} \textrm{ l} = 9,0 \textrm{ pl} = 0,009 \textrm{nl}$.
\end{esimratk}
\end{esimerkki}

\begin{esimerkki}
Usain Boltin huippunopeudeksi 100 metrin juoksukilpailussa mitattiin 12,2 m/s. Janin polkupyörän huippunopeus alamäessä oli 42,5 km/h. Kumman huippunopeus oli suurempi?

\textbf{Ratkaisu.}
Muunnetaan yksiköt vastaamaan toisiaan. Janin polkupyörän huippunopeus oli $42,5 \textrm{ km/h} = 42500 \textrm{ m/h} = 42500 : 3600 \textrm{m/s} \approx 11,8 \textrm{m/s}$.
Bolt oli siis pyöräilevää Jania nopeampi.
\end{esimerkki}

SI-järjestelmä ei ole ainoa käytetty mittayksikköjärjestelmä. Esimerkiksi nk. brittiläisessä mittayksikköjärjestelmässä pituutta voidaan mitata esimerkiksi tuumissa. Yksi tuuma (merkitään $1^{\prime \prime}$) vastaa 2,54 senttimetriä. Massan perusyksikkönä brittiläisessä mittayksikköjärjestelmässä käytetään paunaa.

\begin{tabular}{c|c|c|c}
Suure & Yksikkö & Yksikön tunnus & SI-järjestelmässä\\
\hline
pituus & tuuma & $^{\prime \prime}$ (tai in) & 2,54 cm \\
massa & pauna & lb & 453,59237 g \\
\end{tabular}

Tuumien tai paunojen avulla ilmaistut suureet voidaan muuntaa SI-järjestelmään yllä olevassa taulukossa näkyvien suhdelukujen avulla.

\begin{esimerkki}
Kuinka monta senttimetriä on 3,5 tuumaa? Entä kuinka monta tuumaa on 7,2 senttimetriä?

\textbf{Ratkaisu. } Yksi tuuma vastaa 2,54 senttimetriä, joten kaksi tuumaa vastaa 5,08 senttimetriä jne.

\begin{kuva}
	lukusuora.pohja(0,9,9)
	lukusuora.piste(0, "$0^{\prime \prime} = 0 $ cm")
	lukusuora.piste(2.54,"$1^{\prime \prime} = 2,54$ cm")
	lukusuora.piste(5.08,"$2^{\prime \prime} = 5,08$ cm")
	lukusuora.piste(7.62,"$3^{\prime \prime} = 7,62$ cm")
	with vari("red"):
	  lukusuora.piste(7.2)
	  lukusuora.piste(8.9)
\end{kuva}

Tuumat saadaan siis senttimetreiksi kertomalla luvulla 2,54. Näin ollen $3,5^{\prime \prime} = 3,5 \cdot 2,54 \textrm{ cm} \approx 8,9 \textrm{ cm}$.

Senttimetrien muuntaminen tuumiksi onnistuu vastaavasti \emph{jakamalla} luvulla 2,54. Näin ollen $7,2 \textrm{ cm} \approx 2,8^{\prime \prime}$, koska $7,2 : 2,54 \approx 2,8$.

\end{esimerkki}

\subsection*{Pyöristäminen}

Mikäli urheiluliikkeessä lumilaudan pituudeksi ilmoitetaan tarkan mittauksen jälkeen 167,9337~cm, tämä tuskin on asiakkaalle kovin hyödyllistä tietoa. Epätarkempi arvo 168~cm antaa kaiken oleellisen informaation ja on mukavampi lukea.

Pyöristämisen ajatus on korvata luku sitä lähellä olevalla luvulla, jonka esitysmuoto on lyhyempi. Voidaan pyöristää esimerkiksi tasakymmenien, kokonaisten tai vaikkapa tuhannesosien tarkkuuteen.

Pyöristys tehdään aina lähimpään oikeaa tarkkuutta olevaan lukuun. Siis esimerkiksi kokonaisluvuksi pyöristettäessä $2,8 \approx 3$, koska 3 on lähin kokonaisluku. Lisäksi on sovittu, että puolikkaat (kuten 2,5) pyöristetään ylöspäin.

Se, pyöristetäänkö ylös vai alaspäin (eli suurempaan vai pienempään lukuun) riippuu siis haluttua tarkkuutta seuraavasta numerosta: pienet 0, 1, 2, 3, 4 pyöristetään alaspäin, suuret 5, 6, 7, 8, 9 ylöspäin.

\begin{esimerkki}
Pyöristetään luku 15,0768 sadasosien tarkkuuteen. Katkaistaan luku sadasosien jälkeen ja katsotaan seuraavaa desimaalia:\\
$15,0768 = 15,07|68 \approx 15,08$.\\
Pyöristettiin ylöspäin, koska seuraava desimaali oli 6.
\end{esimerkki}

%+1, hinta viiden sentin tarkkuudella
%+1, entäs, jos 0,3446 ja sadaosien tarkkuudella, niin eikö se kuitenkin pyöristy vitoseen, josta taasen ylöspäin...?? )

\subsubsection*{Merkitsevät numerot}

Mikä on tarkin mittaus, 23~cm, 230~mm vai 0,00023~km? Kaikki kolme tarkoittavat täsmälleen samaa, joten niitä tulisi pitää
yhtä tarkkoina. Luvun esityksessä esiintyvät kokoluokkaa ilmaisevat nollat eivät ole \emph{merkitseviä numeroita}, vain 2 ja 3 ovat.

\laatikko{\termi{merkitsevät numerot}{Merkitseviä numeroita} ovat kaikki luvussa esiintyvät numerot, paitsi nollat kokonaislukujen lopussa ja desimaalilukujen alussa.}

%lisää esimerkkejä!
\begin{center}
\begin{tabular}{r|l}
Luku & Merkitsevät numerot \\
\hline
123 & 3 \\
12~000 & 2 (tai enemmän)\\
12,34 & 4 \\
0,00123 & 3
\end{tabular}
\end{center}

%miten tämä liittyy merkitseviin numeroihin?
Jos esimerkiksi pöydän paksuudeksi on mitattu millin tuhannesosien tarkkuudella 2~cm, voidaan pituus ilmoittaa muodossa 2,0000~cm, jolloin tarkkuus tulee näkyviin.

%Samasta syystä kellonaikoja ilmoitettaessa klo 19 käsitetään siten, että kyse on tasatunnista. Klo 19.00 ei lisäinformaatiota?

Kokonaislukujen kohdalla on toisinaan epäselvyyttä merkitsevien numeroiden määrässä. Kasvimaalla asuvaa 100 citykania on tuskin laskettu ihan tarkasti, mutta 100~m juoksuradan todellinen pituus ei varmasti ole todellisuudessa esimerkiksi 113~m.

\subsubsection*{Vastausten pyöristäminen sovelluksissa}

Pääsääntö on, että vastaukset pyöristetään aina epätarkimman lähtöarvon mukaan. Yhteen- ja vähennyslaskuissa epätarkkuutta mitataan desimaalien lukumäärällä.

\begin{esimerkki} Jos 175~cm pituisen ihmisen nousee seisomaan 2,15~cm korkuisen laudan päälle, olisi varsin optimistista ilmoittaa kokonaiskorkeudeksi 177,15~cm. Kyseisen ihmisen pituus kun todellisuudessa on mitä tahansa arvojen
174,5~cm ja 175,5~cm väliltä. Lasketaan siis\\
\indent 175~cm + 2,15~cm = 177,15~cm $\approx$ 177~cm. 
Epätarkempi lähtöarvo oli mitattu senttien tarkkuudella, joten pyöristettiin tasasentteihin.
\end{esimerkki}

Kerto- ja jakolaskussa tarkkuutta arvioidaan merkitsevien numeroiden mukaan. Jos esimerkiksi pitkän pöydän pituus karkeasti mitattuna 5,9~m ja pöydän leveydeksi saadaan tarkalla mittauksella 1,7861~m, ei ole perusteltua olettaa pöydän pinta-alan olevan todella
\[ 5,9~\textrm{m} \cdot 1,7861~\textrm{m} = 10,53799~\textrm{m}^2. \] 
Pyöristys tehdään epätarkimman lähtöarvon mukaisesti kahteen merkitsevään numeroon:
\[ 5,9~\textrm{m} \cdot 1,7861~\textrm{m} = 10,53799~\textrm{m}^2 \approx 11~\textrm{m}^2.\] 

Tarkkuus ei ole aina hyvästä lukujen esittämisessä. Esimerkiksi
\[ \pi = 3,141592653589793238462643383279 \ldots \]
mutta käytännön laskuihin riittää usein $3,14$.

%esimerkkinä johonkin pii!

\subsubsection*{Tarkka arvo}

Pitkässä matematiikassa ei ilman erityistä tarkoitusta käytetä likiarvoja vaan niin sanottuja tarkkoja arvoja. Käytännössä tämä tarkoittaa arvojen esittämistä mahdollisimman yksinkertaisessa ja sievennetyssä muodossa niin, että irrationaalilukuja kuten $\pi$ ja $\sqrt{2}$ ei esitetä desimaalimuodossa. Käyttämällä tarkkoja arvoja ei menetetä tietoa arvosta kuten likiarvoilla, joissa joudutaan tyytymään tiettyyn tarkkuuteen luvun esityksessä.

\begin{esimerkki}
Lasku $3\sqrt{8}\pi \cdot \frac{1}{9}$ voidaan sieventää muotoon $\frac{2\sqrt{2}\pi}{3}$, joka on tarkka arvo, sillä tästä sieventäminen ei enää onnistu. Voidaan kyllä laskea vastaukseksi desimaaliluvun $\frac{2\sqrt{2}\pi}{3}=2,9619219587722441646772539933737957990764144595837...$, mutta siitä otettu likiarvo saavuttaa vain rajatun tarkkuuden. 
\end{esimerkki}

%MIKSI sieventäminen ei enää onnistunut?

%tehtäviin metriä sekunnissa vs kilometriä tunnissa!


%Lisää kymmenpotenssimuodoista! ja niiden avulla sieventämisestä :) valonnopeus yms.!

%yksikkötarkastelutehtäviä? esimerkkejä, hrjoitustehtäviä!
