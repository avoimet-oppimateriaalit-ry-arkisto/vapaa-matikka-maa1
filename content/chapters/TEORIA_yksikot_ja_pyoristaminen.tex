\subsection*{Suureet ja yksiköt}

Monissa sovelluksissa (ja erityisesti fysiikassa) luvut eivät esitä vain matemaattista lukuarvoa, vaan niitä käytetään yhdessä jonkin \termi{yksikkö}{yksikön} kanssa. Tällöin luku ja yksikkö yhdessä esittävät tietoa \termi{suure}{suureesta} eli jostakin mitattavasta ominaisuudesta.

%selitä vielä, mitä yksikkö tekee?

\begin{esimerkki}

Aika on suure, jota voidaan mitata käyttämällä yksikkönä esimerkiksi sekunteja tai vuosia.

Pituus on suure, ja sen arvo voidaan mitata ja esittää käyttäen yksikkönä esimerkiksi metrejä ja tuumia. (Myös matkaa ja etäisyyttä mitataan samoissa yksiköissä; näillä suureilla on sovellusriippuvainen, erityisesti geometriassa ja fysiikassa esille tuleva nyanssiero.)

Energia on suure, jota mitataan esimerkiksi yksiköissä joule tai kalori.
\end{esimerkki}

Jotkin suureet ovat \termi{skalaarisuure}{skalaarisuureita} ja toiset \termi{vektorisuure}{vektorisuureita}. Skalaarisuureet ovat sellaisia mitattavia ominaisuuksia, joiden kuvaamiseen riittää yksi luku. Vektorisuureet ovat ominaisuuksia, joilla on sekä suuruus että suunta, eikä yksi luku riitä kertomaan kaikkea. (Vektorilaskentaan tutustutaan tarkemmin MAA5-kurssilla sekä fysiikan mekaniikan kursseilla.)

\begin{esimerkki}

Lämpötilan esittämiseen riittää yksi luku, eli lämpötila on skalaarisuure. Käytetystä yksiköstä riippuen kyseinen luku saattaa olla joko negatiivinen, nolla tai positiivinen.

Nopeus on vektorisuure, koska nopeuteen liittyy sen suuruuden lisäksi myös suunta. Esimerkiksi puhuttaessa tuulen nopeudesta kerrotaan myös, mistä suunnasta tuuli puhaltaa.

%kuva nopeusvektorinuolesta (tuulennopeus)

\end{esimerkki}

Matemaattisesti suureita käsitellään aivan kuin luku ja yksikkö olisivat kerrotut keskenään:

\laatikko[Suure]{luku $\cdot$ yksikkö}

Tehtävissä yksiköistä käytetään standardilyhenteitä, eikä yksiköiden nimiä tarvitse kirjoittaa kokonaan. Käytäntö on jättää kertolaskun sijasta luvun ja yksikön väliin tyhjää. (Näin sanoo myös suomen kielioppi.) Tähän on muutama poikkeus: kulman suuruutta ilmaisevat asteet, minuutit ja sekunnit, joita merkataan symboleilla $^\circ$, ' ja '', sekä jotkin pituusyksiköt kuten tuuma, jota merkataan ''. Yksiköt kirjoitetaan konekirjoitetussa tekstissä pystyyn, ei \textit{kursiivilla}.

%Sivunlaitahuomio:: jos yksikkö on nimetty henkilön nimen mukaan, niin yksikkö kirjoitettuna lyhenteellään alkaa isolla kirjaimella, esim. N, J tai Hz. Auki kirjoitettuna kuitenkin "kolme newtonia", kymmenen kilojoulea" ja "5,7 hertziä".

%onko tuuma varmasti merkattu oikein? (kaksoislainausmerkkejä on useita erilaisia)

Seuraava taulukko selventää oikeinkirjoitusta:

\begin{center}
\begin{tabular}{c|c}
Oikein & Väärin \\
\hline
3 m & 3m 	\\
100 $\frac{\text{km}}{\text{h}}$ & 100$\frac{km}{h}$	\\
2 \% & 2\% 	\\
3$^\circ$ & 3 $^\circ$\\
3,16$^\circ$ 5' 13'' & 3,16 $^\circ$ 5 ' 13 '' 	\\
6 $^\circ$C & 6$^\circ$C 	\\
\end{tabular}
\end{center}

%Hienosti ladottuna luvun ja yksikön väliin EI tule välilyöntiä, vaan ohuke (thin space), jonka latexissa saa komennolla \,

\subsection*{SI-järjestelmä}

SI-järjestelmä eli kansainvälinen yksikköjärjestelmä (ransk. Système international d'unités) on maailman käytetyin mittayksikköjärjestelmä. Siihen kuuluvat seuraavat \termi{perussuure}{perussuureet} ja vastaavat perusyksiköt:

\laatikko[SI-järjestelmän perussuureet ja yksiköt]{
\begin{center}
\begin{tabular}{c|c|c|c}
Suure & Suureen tunnus & Yksikkö & Yksikön tunnus\\
\hline
pituus & $l$ & metri &	m\\
massa & $m$ &kilogramma & kg 	\\
aika & $t$ & sekunti & s \\
sähkövirta & $I$ & ampeeri & A \\
(termodynaaminen) lämpötila & $T$ & kelvin & K \\
ainemäärä & $n$ & mooli & mol \\
valovoima & $I$ & kandela & cd
\end{tabular}
\end{center}}

Toisin kuin yksiköt, suureiden kirjainlyhenteet kirjoitetaan \textit{kursiivilla}. %ja tämäkin lopulta sivun laitaan pois leipätekstistä... :)

Näitä suureita käsitellään tarkemmin fysiikassa ja kemiassa, mutta matematiikan kursseilla tulee tuntea ainakin pituus, massa ja aika. (Huomaa, että arkikielessä käytetään usein sanaa paino, vaikka tarkoitetaan massaa.) Tämän lisäksi matematiikan tehtävissä esiintyy usein näistä yhdistämällä saatuja \termi{johdannaissuure}{johdannaissuureita} kuten nopeus, pinta-ala, tilavuus ja tiheys.

\laatikko[Tärkeitä johdannaissuureita]{

\begin{itemize}
\item Nopeus (tunnus $v$, engl. velocity) on johdannaissuure, joka saadaan kuljetun matkan ja matkan kulkemiseen käytetyn ajan osamäärästä: $\text{nopeus}=\frac{\text{matka}}{\text{aika}}$. Vastaava nopeuden yksikkö saadaan jakamalla matkan yksikkö ajan yksiköllä, käytettiin mitä yksiköitä hyvänsä: $\frac{\text{m}}{\text{s}}$ ja $\frac{\text{km}}{\text{h}}$.

\item Pinta-ala (tunnus usein $A$, engl. area) saadaan kertomalla pituus toisella pituudella. Vastaavasti pinta-alan yksiköt saadaan pituuden yksiköistä. Jos esimerkiksi pituutta on mitattu metreissä, niin vastaava pinta-alan yksikkö on $m\cdot m=m^2$ eli neliömetri.

\item Tilavuus (tunnus usein $V$, engl. Volume) on edelleen johdettavissa kolmen pituuden tulona. Jos pituutta on mitattu metreissä, niin vastaava tilavuuden yksikkö on $m\cdot m \cdot m=m^3$ eli kuutiometri.

\item Tiheys (tunnus kreikkalainen pieni roo, $\rho$) on johdannaissuure, joka määritellään massan ja tilavuuden osamääränä: $\text{tiheys}=\frac{\text{massa}}{\text{tilavuus}}$. Vastaavasti tiheyden yksiköt saadaan samalla jakolaskulla massan ja tilavuuden yksiköistä: jos kappaleen tilavuus on mitattu kuutiometreissä ja massa kilogrammoina, esitetään kappaleen tiheys yksiköissä $\frac{\text{kg}}{\text{m}^3}$.
\end{itemize}
}

Kun johdannaisyksiköissä esiintyy jakolaskua, yksikön voi kirjoittaa useassa eri muodossa potenssisääntöjen avulla.

\begin{esimerkki}

Nopeuden yksikkö metriä sekunnissa voidaan kirjoittaa murtolausekkeena $\frac{\text{m}}{\text{s}}$, samalla rivillä m/s tai negatiivisen eksponentin avulla ms$^{-1}$.

\end{esimerkki}

Joillekin johdannaisyksiköille on annettu oma nimensä ja lyhenteensä, eikä yksikössä esiinny alkuperäisiä perusyksiköitä. 

\begin{esimerkki}

Tuhatta kilogrammaa kutsutaan tonniksi.

Yksi litra vastaa yhtä kuutiodesimetriä, eli $1\,\text{l}=1\,\text{dm}^3$. (Ks. etuliitteet edellä.)

Energian yksikkö joule ilmaistuna vain SI-järjestelmän perusyksiköitä käyttämällä: $\text{J}=\frac{\text{m}^2\text{kg}}{\text{s}^2}$.

\end{esimerkki}

\subsection*{Etuliitteet}

Monesti suureita kuvataan \termi{kerrannaisyksikkö}{kerrannaisyksiköiden} avulla. Tällöin yksikön suuruutta muokataan etuliitteellä.

\laatikko[Tavallisimmat kerrannaisyksiköiden etuliitteet]{
\begin{center}
%\begin{tabular}{c|c}
\begin{tabular}{c|c|c|c}
Nimi & Kerroin & Tunnus & Arvo suomeksi \\
\hline
deka & $10^{1}$ & da  & kymmenen	\\
hehto & $10^{2}$ & h & sata 	\\
kilo & $10^{3}$ & k & tuhat	\\
mega & $10^{6}$ & M & miljoona 	\\
giga & $10^{9}$ & G & miljardi		\\
tera & $10^{12}$ & T & biljoona \\
peta & $10^{15}$ & P & <ei vakiintunutta nimeä> \\
\end{tabular}

\begin{tabular}{c|c|c|c}
Nimi & Kerroin & Tunnus & Arvo suomeksi \\
\hline
desi & $10^{-1}$ & d & kymmenesosa	\\
sentti & $10^{-2}$ & c & sadasosa\\
milli & $10^{-3}$ & m	& tuhannesosa\\
mikro & $10^{-6}$ & $\upmu$ & miljoonasosa\\
nano & $10^{-9}$ & n 	& miljardisosa\\
piko & $10^{-12}$ & p & biljoonasosa\\
femto & $10^{-15}$ & f & tuhatbiljoonasosa
\end{tabular}
%\end{tabular}
\end{center}
}

%toi taulukko pitää tasata! 

Etuliitteitä käytetään siis muuttamaan yksikköä pienemmäksi tai suuremmaksi. Tarkoituksena on yleensä valita sovelluskohtaisesti sopivankokoinen yksikkö niin, että käytetyn luvun ei tarvitse olla valtavan suuri tai pieni. Mitä suuremman yksikön valitsee, sitä pienemmäksi sen kanssa käytettävä luku tulee -- ja toisinpäin.

\begin{esimerkki}
Kilogramma tarkoittaa tuhatta grammaa. Kilogramma on valittu SI-järjestelmässä massan perusyksiköksi gramman sijaan käytännöllisyyssyistä, sillä yksi gramma on moniin sovelluksiin liian pieni.
\end{esimerkki}

\begin{esimerkki}
Ilmaisussa 0,000046 metriä luku on niin pieni, että on käytännöllisempää ja usein helppotajuisempaa kirjoittaa 46 mikrometriä eli 46\,$\upmu$m.
\end{esimerkki}

\begin{esimerkki}
Hyvin pienistä massoista puhuttaessa voidaan puhua esimerkiksi nanogrammoista. Esimerkiksi $1\,\textrm{ng} = 1 \cdot 10^{-9}\,\textrm{g} = 0,000000009\,\textrm{g} $.
\end{esimerkki}

%tähän vai minne kymmenpotensismuotojen ja suurten lukujen lukusanoista? long ja short scale

\laatikko[Huomautus tietotekniikasta]{
Datan tai informaation määrää mitataan bitteinä ja tavuina – yksi tavu on kahdeksan bittiä. Tietotekniikassa datan määrä on diskreetti, eli bittejä voi olla vain jokin luonnollisen luvun osoittama määrä. Bittejä ei siis voi jakaa mielivaltaisen pieniin palasiin toisin kuin esimerkiksi sekunteja ja metrejä. Suomenkielessä bitin lyhenne on b ja tavun t. Usein käytettävät englanninkieliset lyhenteet bitille (engl. bit) ja tavulle (engl. byte) ovat vastaavasti b ja B.

Tietokoneidden toiminnan kaksikantaisuuden (ks. liite lukujärejstelmistä) johdosta yksi kilotavu (kt) ei tarkoita tasan $1\,000$ tavua vaan $1\,024$ tavua ($2^{10}$). Vastaavasti yksi megatavu (Mt) on $1024$ kt (eli $1\,048\,576 = 2^{20}$ tavua), yksi gigatavu $1\,024$ Mt ja niin edelleen.} % (Joskus harvoinmaining MiB-, GiB-ehdotuksista?

%esimerkki

\subsection*{Yksiköiden yhdistäminen}

%hakee vielä vähän loogista järjestystä :) T: Joonas

Jos lausekkeessa on useita samanlaisia yksiköitä, näitä voidaan usein yhdistää. Käytännössä yksiköitä kohdellaan kuin lukuja, ja niitä voi kertoa ja jakaa keskenään. Suureen määritelmän luvun ja yksikön kertolaskukäsityksestä seuraa, että sovelluksissa ja sievennystehtävissä yksiköillä voidaan laskea kuin luvuilla. Tutut potenssisäännöt, rationaalilukujen laskusäännöt ja kertolaskun vaihdannaisuuslaki pätevät –- lopullisessa sievennetyssä muodossa vain merkataan luvun ja yksikön väliin väli kertolaskun sijaan. Kerrannaisyksiköiden etuliitteet voi aina muuttaa tarvittaessa kymmenpotenssimuotoon.

\begin{esimerkki}
2 metriä kertaa 3 metriä = $2\,\text{m} \cdot 3\,\text{m} = 2 \cdot \text{m} \cdot 3 \cdot \text{m}= 2 \cdot 3 \cdot \text{m} \cdot \text{m} =6\,\text{m}^2$ eli kuusi neliömetriä.
\end{esimerkki}

%\begin{esimerkki}
%
%Vetyatomin halkaisija on noin 0,1 nanometriä. Kuinka monta vetyatomia tarvitaan peräkkäin, jotta saataisiin yhden senttimetrin pituinen atomijono?
%
%	\begin{esimratk}
%	
%	\end{esimratk}
%
%\end{esimerkki}

%\begin{esimerkki}
%
%Paperiarkkipinossa, jonka korkeus on 10 senttimetriä, on yksi riisi eli... arkkia. Mikä on yhden paperiarkin paksuus? Kuinka monta arkkia on pinossa, jonka korkeus on metri? Tehtävässä ei huomioida paperiarkkien puristumista muiden papereiden painon vuoksi.
%
%	\begin{esimratk}
%	
%	\end{esimratk}
%\end{esimerkki}


\begin{esimerkki}
Keskinopeus $v$ voidaan laskea kaavasta $\frac{\text{s}}{\text{t}}$, missä $s$ on kuljettu matka ja $t$ matkaan käytetty aika. Tästä voidaan johtaa kaava, jolla voidaan laskea katkaan kulkemiseen kulunut aika: $t=s/v$, missä $s$ on kuljettu matka ja $v$ on nopeus. (Johto tedään Yhtälö-luvussa.) Lasketaan, kuinka kauan kestää 300 metrin matka nopeudella 7 m/s.

Sijoitetaan kaavaan yksikköineen $s=300$ m ja $v= 7$ m/s ja käytetään rationaalilukujen laskusääntöjä, jolloin saadaan:
\[t=\frac{100\,\textrm{m}}{7\,\textrm{m/s}} = \frac{100}{7} \cdot \frac{\textrm{m}}{\frac{\textrm{m}}{\textrm{s}}} 
= \frac{100}{7} \cdot \frac{\textrm{m}}{1} \cdot \frac{\textrm{s}}{\textrm{m}}
= \frac{100}{7} \cdot \frac{\cancel{m}\textrm{s}}{\cancel{m}}
=\frac{100}{7}\,\textrm{s} \approx 14\,\textrm{s}.\]
\end{esimerkki}

%Oisko pitänyt laittaa vaiheittain tuo vielä, että voi merkata välivaiheissa tehdyt jutut... olisi hyvä. Saisi vielä kerran kerrattua sanallisesti, mitä rationaalilukujen ominaisuuksia käytettiin ja mitä potenssisääntöjä käytettiin... :) T: Joonas Mäkinen

%tiheysesimerkki

%jokin esimerkki, missä potensseja...

%\begin{esimerkki}
%Maapallon massa noin 
%\end{esimerkki}

Huomaa, että eri suureita voi kertoa ja jakaa keskenään, mutta yhteen- ja vähennyslasku ei ole määritelty. (Mitä tarkoittaisikaan "kolme metriä plus kaksi grammaa"?)

\subsection*{Yksikkömuunnokset}

Kerrannaisyksiköiden tapauksessa tulee vain huomata, että etuliitteiden ajatellaan olevan matemaattisesti vain kertoimia. Kaikki ilmaisut voidaan aina halutessa esittää luvun ja yksikön tulona ilman etuliitteitä.

\begin{esimerkki}
$270\,\text{mm}=270\cdot\text{mm}=270\cdot10^{-3}\,\text{m}=\frac{270}{1000}\,\text{m}=0,27\,\text{m}$
\end{esimerkki}

\begin{esimerkki}
Erään punaisen verisolun tilavuus on $9,0 \cdot 10^{-12}\,\textrm{l}$. Ilmoita tilavuus

a) pikolitroina

b) nanolitroina

\begin{esimratk}
$9,0 \cdot 10^{-12}\,\textrm{l} = 9,0\,\textrm{pl} = 0,009\,\textrm{nl}$.
\end{esimratk}
\end{esimerkki}

\laatikko[Erittäin tärkeää kerrannaisyksiköiden potensseista!]{Edellä esitetty taulukko etuliitteistä pätee vain ensimmäisen asteen yksiköihin! Desimetri on kymmenesosa metristä, mutta neliödesimetri \emph{ei} ole kymmenesosa neliömetristä.

Avain tämän ymmärtämiseen on tietää seuraava kirjoitustavan sopimus:

\begin{center}
dm$^2$ tarkoittaa oikeasti $(\text{dm})^2$!
\end{center}

Potensissääntöjä käyttämällä huomataan: $(\text{dm})^2=\text{d}^2\text{m}^2=(10^{-1})^2\, \text{m}^2=10^{-2}\,\text{m}^2=\frac{1}{100}\, \text{m}^2$. Eli yksi neliödesimetri onkin \textit{sadaosa} neliömetristä!}

\begin{esimerkki}

Kuinka mones osa kuutiodesimetri on kuutiometristä?

	\begin{esimratk}
	Puretaan auki käyttäen mainittua kirjoituskonventiota, potenssisääntöjä ja desi-etuliitteen määritelmää: $\text{dm}^3=\text{d}^3\,\text{m}^3=(10^{-1})^3\,\text{m}^3=10^-3\,\text{m}^3=\frac{1}{1000}\,\text{m}^3$
	
	Nähdään, että yksi kuutiodesimetri on tuhannesosa kuutiometristä.
	\end{esimratk}

\end{esimerkki}


%\begin{esimerkki}

%Kuinka monta desilitraa ... 

%\end{esimerkki}

%toine ja ehkä kolmaskin esimerkki

%ESIMERKKI! Vertailu: Jenkeissä autoista kerrotaan, kuinka monta mailia gallonalla – Suomessa, kuinka monta litraa satasella

Matkan, pinta-alan ja tilavuuden yksikkömuunnoksia käsitellään tarkemmin ja lisää geometrian MAA3-kurssilla. 

Ajan yksiköissä on jäänteitä 60-järjestelmästä. Yksikköjä ei jaetakaan kymmeneen pienempään osaan vaan kuuteenkymmeneen:

\laatikko{
Yksi tunti on $60$ minuuttia. Yksi minuutti on $60$ sekuntia.
\begin{itemize}
  \item $1~\text{h} = 60~\text{min}$
  \item $1~\text{min} = 60~\text{s}$
  \item $1~\text{h} = 60~\text{min} = 60 \cdot 60~\text{s} = 3600~\text{s}$ 
\end{itemize}
}

%\begin{esimerkki}
%
%Kuinka monta sekuntia on vuodessa?
%
%	\begin{esimratk}
%
%	\end{esimratk}
%
%\end{esimerkki}

\begin{esimerkki}
Kuinka monta minuuttia on $1,25$\,h?

\begin{esimratk}
$1,25\,\text{h} = 1,25 \cdot 60\,\text{min} = 75\,\text{min}$. $1,25$ tuntia on siis $75$ minuuttia. Huomaa, että voit laskuissasi esittää desimaaliluvun $1,25$ yhdistettynä lukuna $1 \frac{25}{100}$ eli $1 \frac{1}{4}$, mikä saattaa helpottaa laskemista.
\end{esimratk}
\end{esimerkki}

\begin{esimerkki}
Usain Boltin huippunopeudeksi 100 metrin juoksukilpailussa mitattiin 12,2 m/s. Janin polkupyörän huippunopeus alamäessä oli 42,5 km/h. Kumman huippunopeus oli suurempi?

\textbf{Ratkaisu.}
Muunnetaan yksiköt vastaamaan toisiaan, jolloin luvuista tulee keskenään vertailukelpoisia. Janin polkupyörän huippunopeus oli $42,5\, \textrm{km/h} = 42\,500\,\textrm{m/h} = \frac{42\,500}{3\,600}\,\textrm{m/s} \approx 11,8 \textrm{m/s}$.
Bolt oli siis pyöräilevää Jania nopeampi.
\end{esimerkki}

SI-järjestelmä ei ole ainoa käytetty mittayksikköjärjestelmä. Esimerkiksi nk. brittiläisessä mittayksikköjärjestelmässä pituutta voidaan mitata esimerkiksi tuumissa. Yksi tuuma (merkitään $1^{\prime \prime}$) vastaa 2,54 senttimetriä. Massan perusyksikkönä brittiläisessä mittayksikköjärjestelmässä käytetään paunaa.

\begin{tabular}{c|c|c|c}
Suure & Yksikkö & Yksikön tunnus & SI-järjestelmässä\\
\hline
pituus & tuuma & $^{\prime \prime}$ (tai in) & 2,54\,cm \\
massa & pauna & lb & 453,59237\,g \\
\end{tabular}

Tuumien tai paunojen avulla ilmaistut suureet voidaan muuntaa SI-järjestelmään yllä olevassa taulukossa näkyvien suhdelukujen avulla.

\begin{esimerkki}
Kuinka monta senttimetriä on 3,5 tuumaa? Entä kuinka monta tuumaa on 7,2 senttimetriä?

	\begin{esimratk}
Yksi tuuma vastaa 2,54 senttimetriä, joten kaksi tuumaa vastaa 5,08 senttimetriä jne.
	
\begin{kuva}
	lukusuora.pohja(0,9,9)
	lukusuora.piste(0, "$0^{\prime \prime} = 0 $ cm")
	lukusuora.piste(2.54,"$1^{\prime \prime} = 2,54$ cm")
	lukusuora.piste(5.08,"$2^{\prime \prime} = 5,08$ cm")
	lukusuora.piste(7.62,"$3^{\prime \prime} = 7,62$ cm")
	with vari("red"):
	  lukusuora.piste(7.2)
	  lukusuora.piste(8.9)
\end{kuva}

Tuumat saadaan siis senttimetreiksi kertomalla luvulla 2,54. Näin ollen $3,5^{\prime \prime} = 3,5 \cdot 2,54 \textrm{ cm} \approx 8,9 \textrm{ cm}$.

Senttimetrien muuntaminen tuumiksi onnistuu vastaavasti \emph{jakamalla} luvulla 2,54. Näin ollen $7,2 \textrm{ cm} \approx 2,8^{\prime \prime}$, koska $7,2 : 2,54 \approx 2,8$.
	\end{esimratk}
\end{esimerkki}

\subsection*{Pyöristäminen}

Mikäli urheiluliikkeessä lumilaudan pituudeksi ilmoitetaan tarkan mittauksen jälkeen 167,9337~cm, tämä tuskin on asiakkaalle kovin hyödyllistä tietoa. Epätarkempi arvo 168~cm antaa kaiken oleellisen informaation ja on mukavampi lukea.

Pyöristämisen ajatus on korvata luku sitä lähellä olevalla luvulla, jonka esitysmuoto on lyhyempi. Voidaan pyöristää esimerkiksi tasakymmenien, kokonaisten tai vaikkapa tuhannesosien tarkkuuteen.

Pyöristys tehdään aina lähimpään oikeaa tarkkuutta olevaan lukuun. Siis esimerkiksi kokonaisluvuksi pyöristettäessä $2,8 \approx 3$, koska 3 on lähin kokonaisluku. Lisäksi on sovittu, että puolikkaat (kuten 2,5) pyöristetään ylöspäin.

Se, pyöristetäänkö ylös vai alaspäin (eli suurempaan vai pienempään lukuun) riippuu siis haluttua tarkkuutta seuraavasta numerosta: pienet 0, 1, 2, 3, 4 pyöristetään alaspäin, suuret 5, 6, 7, 8, 9 ylöspäin.

\begin{esimerkki}
Pyöristetään luku 15,0768 sadasosien tarkkuuteen. Katkaistaan luku sadasosien jälkeen ja katsotaan seuraavaa desimaalia:\\
$15,0768 = 15,07|68 \approx 15,08$.\\
Pyöristettiin ylöspäin, koska seuraava desimaali oli 6.
\end{esimerkki}

%+1, hinta viiden sentin tarkkuudella
%+1, entäs, jos 0,3446 ja sadaosien tarkkuudella, niin eikö se kuitenkin pyöristy vitoseen, josta taasen ylöspäin...?? )

\subsubsection*{Merkitsevät numerot}

Mikä on tarkin mittaus, 23~cm, 230~mm vai 0,00023~km? Kaikki kolme tarkoittavat täsmälleen samaa, joten niitä tulisi pitää
yhtä tarkkoina. Luvun esityksessä esiintyvät kokoluokkaa ilmaisevat nollat eivät ole \emph{merkitseviä numeroita}, vain 2 ja 3 ovat.

\laatikko{\termi{merkitsevät numerot}{Merkitseviä numeroita} ovat kaikki luvussa esiintyvät numerot, paitsi nollat kokonaislukujen lopussa ja desimaalilukujen alussa.}

%lisää esimerkkejä!
\begin{center}
\begin{tabular}{r|l}
Luku & Merkitsevät numerot \\
\hline
123 & 3 \\
12~000 & 2 (tai enemmän)\\
12,34 & 4 \\
0,00123 & 3
\end{tabular}
\end{center}

%miten tämä liittyy merkitseviin numeroihin?
Jos esimerkiksi pöydän paksuudeksi on mitattu millin tuhannesosien tarkkuudella 2~cm, voidaan pituus ilmoittaa muodossa 2,0000~cm, jolloin tarkkuus tulee näkyviin.

%Samasta syystä kellonaikoja ilmoitettaessa klo 19 käsitetään siten, että kyse on tasatunnista. Klo 19.00 ei lisäinformaatiota?

Kokonaislukujen kohdalla on toisinaan epäselvyyttä merkitsevien numeroiden määrässä. Kasvimaalla asuvaa 100 citykania on tuskin laskettu ihan tarkasti, mutta 100~m juoksuradan todellinen pituus ei varmasti ole todellisuudessa esimerkiksi 113~m.

\subsubsection*{Vastausten pyöristäminen sovelluksissa}

Pääsääntö on, että vastaukset pyöristetään aina epätarkimman lähtöarvon mukaan. Yhteen- ja vähennyslaskuissa epätarkkuutta mitataan desimaalien lukumäärällä.

\begin{esimerkki} Jos 175~cm pituisen ihmisen nousee seisomaan 2,15~cm korkuisen laudan päälle, olisi varsin optimistista ilmoittaa kokonaiskorkeudeksi 177,15~cm. Kyseisen ihmisen pituus kun todellisuudessa on mitä tahansa arvojen
174,5~cm ja 175,5~cm väliltä. Lasketaan siis \indent 175~cm + 2,15~cm = 177,15~cm $\approx$ 177~cm. 
Epätarkempi lähtöarvo oli mitattu senttien tarkkuudella, joten pyöristettiin tasasentteihin.
\end{esimerkki}

Kerto- ja jakolaskussa tarkkuutta arvioidaan merkitsevien numeroiden mukaan. Jos esimerkiksi pitkän pöydän pituus karkeasti mitattuna 5,9~m ja pöydän leveydeksi saadaan tarkalla mittauksella 1,7861~m, ei ole perusteltua olettaa pöydän pinta-alan olevan todella
\[ 5,9~\textrm{m} \cdot 1,7861~\textrm{m} = 10,53799~\textrm{m}^2. \] 
Pyöristys tehdään epätarkimman lähtöarvon mukaisesti kahteen merkitsevään numeroon:
\[ 5,9~\textrm{m} \cdot 1,7861~\textrm{m} = 10,53799~\textrm{m}^2 \approx 11~\textrm{m}^2.\] 

Tarkkuus ei ole aina hyvästä lukujen esittämisessä. Esimerkiksi
\[ \pi = 3,141592653589793238462643383279 \ldots \]
mutta käytännön laskuihin riittää usein $3,14$.

%esimerkkinä johonkin pii!

\subsubsection*{Tarkka arvo}

Pitkässä matematiikassa ei ilman erityistä tarkoitusta käytetä likiarvoja vaan niin sanottuja tarkkoja arvoja. Käytännössä tämä tarkoittaa arvojen esittämistä mahdollisimman yksinkertaisessa ja sievennetyssä muodossa niin, että irrationaalilukuja kuten $\pi$ ja $\sqrt{2}$ ei esitetä desimaalimuodossa. Käyttämällä tarkkoja arvoja ei menetetä tietoa arvosta kuten likiarvoilla, joissa joudutaan tyytymään tiettyyn tarkkuuteen luvun esityksessä.

\begin{esimerkki}
Lasku $3\sqrt{8}\pi \cdot \frac{1}{9}$ voidaan sieventää muotoon $\frac{2\sqrt{2}\pi}{3}$, joka on tarkka arvo, sillä tästä sieventäminen ei enää onnistu. Voidaan kyllä laskea vastaukseksi desimaaliluvun $\frac{2\sqrt{2}\pi}{3}=2,9619219587722441646772539933737957990764144595837...$, mutta siitä otettu likiarvo saavuttaa vain rajatun tarkkuuden. 
\end{esimerkki}

%MIKSI sieventäminen ei enää onnistunut?
%tehtäviin metriä sekunnissa vs kilometriä tunnissa!
%Lisää kymmenpotenssimuodoista! ja niiden avulla sieventämisestä :) valonnopeus yms.!
%yksikkötarkastelutehtäviä? esimerkkejä, harjoitustehtäviä! kerrotaanko [nopeus]=m/s -notaatiosta?
