\subsection*{Suureet ja yksiköt}

Monissa sovelluksissa (ja erityisesti fysiikassa) luvut eivät esitä vain matemaattista lukuarvoa, vaan niitä käytetään yhdessä jonkin \termi{yksikkö}{yksikön} kanssa.
Tällöin luku yhdessä yksikön kanssa esittää tietoa \termi{suure}{suureesta} eli jostakin mitattavasta ominaisuudesta.
Esimerkiksi pituus on suure, jota voidaan mitata metreissä tai tuumissa.

\laatikko[Suure]{luku $\cdot$ yksikkö}

Matemaattisesti suureita käsitellään aivan kuin luku ja yksikkö olisivat kerrotut keskenään, mutta käytäntö on jättää kertolaskun sijasta luvun ja yksikön väliin tyhjää. (Näin sanoo myös suomen kielioppi.) Tähän on muutama poikkeus: kulman suuruutta ilmaisevat asteet, minuutit ja sekunnit, joita merkataan symboleilla $\deg$, ' ja '', ja jotkin pituusyksiköt kuten tuuma, jota merkataan ''.

Seuraava taulukko selventää oikeinkirjoitusta:

\begin{tabular}{c|c}
Oikein & Väärin \\
\hline
3 m & 3m 	\\
100 $\frac{\text{km}}{\text{h}}$ & 100$\frac{km}{h}$	\\
2 \% & 2\% 	\\
3$\deg$ & 3 $\deg$\\
3,16$\deg$ 5' 13'' & 3,16 $\deg$ 5 ' 13 '' 	\\
6 $\deg$ C & 6$\deg$ C 	\\
\end{tabular}

\subsection*{Yksiköiden yhdistäminen}

Jos lausekkeessa on useita samanlaisia yksiköitä, näitä voidaan usein yhdistää ja sieventää. Käytännössä yksiköitä kohdellaan kuin lukuja, ja niitä voi kertoa ja jakaa keskenään.
Luvun ja yksikön kertolaskukäsityksestä seuraa, että sovelluksissa ja sievennystehtävissä yksiköillä voidaan laskea kuin luvuilla.
Tutut potenssisäännöt, rationaalilukujen laskusäännöt ja kertolaskun vaihdannaisuuslaki pätevät.

\begin{esimerkki}
2 metriä kertaa 3 metriä = $2$ m $\cdot$ $3$ m $=2 \cdot 3 \cdot$ m $\cdot$ m $=6 $ m$^2$ eli kuusi neliömetriä.
\end{esimerkki}

\begin{esimerkki}
Keskinopeus lasketaan kaavalla $v=s/t$, missä $v$ on nopeus, $s$ on kuljettu matka ja $t$ matkaan kulunut aika.
Lasketaan, kuinka kauan kestää 300 metrin matka nopeudella 7 m/s:

Koska keskinopeus voidaan laskea kaavalla $v=s/t$, matkaan kulunut aika voidaan laskea kaavalla $t=s/v$.
Sijoitetaan yksiköineen $s=300$ m ja $v= 7$ m/s ja käytetään ratinaalilukujen laskusääntöjä, jolloin saadaan:
\begin{equation*}
t=\frac{100 \textrm{ m}}{ 7 \textrm{ m/s}} = \frac{100}{7} \cdot \frac{\textrm{m}}{\frac{\textrm{m}}{\textrm{s}}} 
= \frac{100}{7} \cdot \frac{\textrm{m}}{1} \cdot \frac{\textrm{s}}{\textrm{m}} = \frac{100}{7} \textrm{ s} \approx 14 \textrm{ s}.
\end{equation*}
\end{esimerkki}

%\begin{esimerkki}
%$3 m \cdot 4 m = 3\cdot 4 m\cdot m = 12 m^2$ \\
%$\frac{90 m}{10 \frac{m}{s}} = \frac{90}{10} \frac{m}{\frac{m}{s}}=9 m:\frac{m}{s}=9 m \cdot \frac{s}{m} = 9 \frac{ms}{m}=9 s$
%\end{esimerkki}

\subsection*{SI-järjestelmä}


\begin{tabular}{c|c|c}
Suure & Yksikkö & Yksikön tunnus\\
\hline
pituus & metri &	m\\
massa & kilogramma & kg 	\\

\end{tabular}



Standardin mukaan yksiköiden tunnukset kirjoitetaan pystyyn, kursiivilla.


\subsection*{Etuliitteet}


\laatikko{
Tavallisimmat kerrannaisyksiköiden etuliitteet:

\begin{tabular}{c|c}
\begin{tabular}{c|c|c}
Nimi & Kerroin & Tunnus \\
\hline
deka & $10^{1}$ & da 	\\
hehto & $10^{2}$ & h 	\\
kilo & $10^{3}$ & k 	\\
mega & $10^{6}$ & M 	\\
giga & $10^{9}$ & G		\\
tera & $10^{12}$ & T 	
\end{tabular}
\begin{tabular}{c|c|c}
Nimi & Kerroin & Tunnus \\
\hline
desi & $10^{-1}$ & d 	\\
sentti & $10^{-2}$ & c 	\\
milli & $10^{-3}$ & m	\\
mikro & $10^{-6}$ & $\upmu$ \\
nano & $10^{-9}$ & n 	\\
piko & $10^{-12}$ & p
\end{tabular}
\end{tabular}
}

Tietotekniikassa datan määrää mitataan tavuina, mutta siellä yksi kilotavu (kt) ei tarkoita tasan $1000$ tavua,  vaan $1024$ tavua ($2^{10}$). Vastaavasti yksi megatavu on $1024$ kt (eli $1048576 = 2^{20}$ tavua), yksi gigatavu $1024$ Mt jne.

\subsection*{Yksikkömuunnokset}

\laatikko{
Yksi tunti on $60$ minuuttia. Yksi minuutti on $60$ sekuntia.
\begin{itemize}
  \item $1~\text{h} = 60~\text{min}$
  \item $1~\text{min} = 60~\text{s}$
  \item $1~\text{h} = 60~\text{min} = 60 \cdot 60~\text{s} = 3600~\text{s}$ 
\end{itemize}
}

% FIXME \todo{taulukko amerikkalaisista yksiköistä}

\begin{esimerkki}
Kuinka monta minuuttia on $1,25$ h? $1,25 \text{h} = 1,25 \cdot 60 \text{min} = 75 \text{min}$. $1,35$ h on siis $75$ minuuttia. Huomaa, että voit laskuissasi esittää desimaaliluvun $1,25$ yhdistettynä lukuna $1 \frac{25}{100}$ eli $1 \frac{1}{4}$, mikä saattaa helpottaa laskemista.
\end{esimerkki}

%\begin{esimerkki}
%\todo{esimerkki jossa tarvitsee ottaa huomioon kerrannaisyksikkö, esim. nanometri tai whatever.}
%\end{esimerkki}

%\begin{esimerkki}
% FIXME \todo{esimerkki cm muuttamisesta tuumaksi ja toisinpäin}
% FIXME tämä pitäisi tekaista lukusuorana: \missingfigure{havainnollistava kuva, jossa on rinnastettu suoralla sentit ja tuumat}
%\end{esimerkki}

\subsection*{Pyöristäminen}

Mikäli urheiluliikkeessä lumilaudan pituudeksi ilmoitetaan tarkan mittauksen jälkeen 167,9337~cm, tämä tuskin on asiakkaalle kovin hyödyllistä tietoa. Epätarkempi arvo 168~cm antaa kaiken oleellisen informaation ja on mukavampi lukea.

Pyöristämisen ajatus on korvata luku sitä lähellä olevalla luvulla, jonka esitysmuoto on lyhyempi. Voidaan pyöristää
esimerkiksi tasakymmenien, kokonaisten tai vaikkapa tuhannesosien
tarkkuuteen.

Pyöristys tehdään aina lähimpään oikeaa tarkkuutta olevaan lukuun. Siis esimerkiksi kokonaisluvuksi pyöristettäessä $2,8 \approx 3$, koska 3 on lähin kokonaisluku. Lisäksi on sovittu, että
puolikkaat (kuten 2,5) pyöristetään ylöspäin.

Se, pyöristetäänkö ylös vai alaspäin (eli suurempaan vai
pienempään lukuun) riippuu siis haluttua tarkkuutta
seuraavasta numerosta: pienet
0, 1, 2, 3, 4 pyöristetään alaspäin, suuret 5, 6, 7, 8, 9 ylöspäin.

\begin{esimerkki}
Pyöristetään luku 15,0768 sadasosien tarkkuuteen. Katkaistaan
luku sadasosien jälkeen ja katsotaan seuraavaa desimaalia:\\
$15,0768 = 15,07|68 \approx 15,08$.\\
Pyöristettiin ylöspäin, koska seuraava desimaali oli 6.
\end{esimerkki}

\subsubsection*{Merkitsevät numerot}

Mikä on tarkin mittaus, 23~cm, 230~mm vai 0,00023~km? Kaikki kolme tarkoittavat täsmälleen samaa, joten niitä tulisi pitää
yhtä tarkkoina. Luvun esityksessä esiintyvät kokoluokkaa ilmaisevat nollat eivät ole \emph{merkitseviä numeroita}, vain
2 ja 3 ovat.

\laatikko{Merkitseviä numeroita ovat kaikki luvussa esiintyvät numerot, paitsi nollat kokonaislukujen lopussa ja desimaalilukujen alussa.}

Jos esimerkiksi pöydän paksuudeksi on mitattu millin tuhannesosien
tarkkuudella 2~cm, voidaan pituus ilmoittaa muodossa 2,0000~cm, jolloin tarkkuus tulee näkyviin. 

Kokonaislukujen kohdalla on toisinaan epäselvyyttä merkitsevien numeroiden määrässä. Kasvimaalla asuvaa 100 citykania on tuskin laskettu ihan tarkasti,
mutta 100~m juoksuradan todellinen pituus ei varmasti ole todellisuudessa
esimerkiksi 113~m.

\begin{center}
\begin{tabular}{r|l}
Luku & Merkitsevät numerot \\
\hline
123 & 3 \\
12~000 & 2 (tai enemmän)\\
12,34 & 4 \\
0,00123 & 3
\end{tabular}
\end{center}

\subsubsection*{Vastausten pyöristäminen käytännön laskuissa}

Pääsääntö on, että vastaukset pyöristetään aina epätarkimman
lähtöarvon mukaan. Yhteen- ja vähennyslaskuissa epätarkkuutta
mitataan desimaalien lukumäärällä.

Jos esimerkiksi 175~cm pituisen ihmisen
nousee seisomaan 2,15~cm korkuisen laudan päälle, olisi varsin
optimistista ilmoittaa kokonaiskorkeudeksi 177,15~cm. Kyseisen ihmisen pituus kun todellisuudessa on mitä tahansa arvojen
174,5~cm ja 175,5~cm väliltä. Lasketaan siis\\
\indent 175~cm + 2,15~cm = 177,15~cm $\approx$ 177~cm. 
Epätarkempi lähtöarvo oli mitattu senttien tarkkuudella, joten pyöristettiin tasasentteihin.

Kerto- ja jakolaskussa tarkkuutta arvioidaan merkitsevien numeroiden mukaan. Jos esimerkiksi pitkän pöydän pituus karkeasti
mitattuna 5,9~m ja pöydän leveydeksi saadaan tarkalla mittauksella
1,7861~m, ei ole perusteltua olettaa pöydän pinta-alan olevan todella
\[ 5,9~\textrm{m} \cdot 1,7861~\textrm{m} = 10,53799~\textrm{m}^2. \] 
Pyöristys tehdään epätarkimman
lähtöarvon mukaisesti kahteen merkitsevään numeroon:
\[ 5,9~\textrm{m} \cdot 1,7861~\textrm{m} = 10,53799~\textrm{m}^2 \approx 11 \textrm{m}^2.\] 
%Tarkkuus ei ole aina hyvästä lukujen esittämisessä. Esimerkiksi
%\[ \pi = 3,141592653589793238462643383279 \ldots \]
%mutta käytännön laskuihin riittää usein $3,14$.

