\begin{tehtavasivu}


\subsubsection*{Opi perusteet}

\begin{tehtava}
Onko $-2$ yhtälön $3x-4 = 7-2x$ ratkaisu?
\begin{vastaus}
Ei ole. $3(-2)-4 = -10 \ne 11=7-2(-2)$
\end{vastaus}
\end{tehtava}

\begin{tehtava}
Mitä yhtälölle $ax+b = 0$ tapahtuu, jos kerroin $a$ saa arvon nolla?
Onko yhtälöllä ratkaisuja?
\begin{vastaus}
Jos $b = 0$, yhtälö toteutuu kaikilla $x$:n arvoilla. Jos $b \neq 0$, yhtälö
ei toteudu millään $x$:n arvolla. Periaatteessa kyseessä ei kuitenkaan
enää ole ensimmäisen asteen yhtälö, mikäli $a = 0$
\end{vastaus}
\end{tehtava}

\begin{tehtava}
%
Ratkaise:
\begin{alakohdat}
\alakohta{$x + 4 = 5$}
\alakohta{$1 - x = -3$}
\alakohta{$7x = 35$}
\alakohta{$-2x = 4$}
\alakohta{$10 - 2x = x$}
\alakohta{$9x + 4 = 6 - x$}
\alakohta{$\frac{2x}{5} = 4$}
\alakohta{$\frac{x}{3} + 1 = \frac{5}{6} - x$}
\end{alakohdat}
\begin{vastaus}
\begin{alakohdat}
\alakohta{$x=1$}
\alakohta{$x=4$}
\alakohta{$x=35/7$}
\alakohta{$x=-2$}
\alakohta{$x=10/3$}
\alakohta{$x=1/5$}
\alakohta{$x=10$}
\alakohta{$x=-1/8$}
\end{alakohdat}
\end{vastaus}
\end{tehtava}


\begin{tehtava}
Ratkaise:
\begin{alakohdat}
\alakohta{$4x + 3 = -3x + 4$}
\alakohta{$100x = 101x - 2$}
\alakohta{$\frac{5}{6} x + \frac{4}{5} = x - 1$}
\alakohta{$2\cdot(x+4)-x = x+4$}
\alakohta{$5\cdot(x-8) + \frac{3}{2}\cdot(x-7) = 3x$}
\end{alakohdat}
\begin{vastaus}
\begin{alakohdat}
\alakohta{$x = \frac{1}{7}$}
\alakohta{$x = 2$}
\alakohta{$x = \frac{54}{5}$}
\alakohta{Ei ratkaisua}
\alakohta{$x = \frac{101}{7}$}
\end{alakohdat}
\end{vastaus}
\end{tehtava}

\subsubsection*{Hallitse kokonaisuus}

\begin{tehtava}
%Laatinut Jaakko Viertiö 2013-11-10
Päteekö yhtälö, kun $a=7$?
	\begin{alakohdat}
		\alakohta{$a^5-13a^2=a-3+16166$}
		\alakohta{$\sqrt{a}=\frac{529}{200}$}
	\end{alakohdat}
    \begin{vastaus}
	\begin{alakohdat}
		\alakohta{Pätee.}
		\alakohta{Ei päde.}
	\end{alakohdat}
    \end{vastaus}
\end{tehtava}


\begin{tehtava}
Ratkaise yhtälö. Tarkista sijoittamalla.
\begin{alakohdat}
    \alakohta{$-2x+5=2(5+x)$}
    \alakohta{$4(x-1) - 3x = 15-4x$}
    \alakohta{$6 - (2x+3) = 3-2x$}
    \alakohta{$x - 100\,000 = -0,25x$}
    \alakohta{$10^6 x - 3 \cdot 10^9 = 0$}
    \alakohta{$5(1-x) = -5x+3$}
\end{alakohdat}
%
\begin{vastaus}
\begin{alakohdat}
    \alakohta{$x=-\frac 5 4$}
    \alakohta{$x=\frac{19}{5}$}
    \alakohta{Yhtälö toteutuu kaikilla reaaliluvuilla.}
    \alakohta{$x=80\,000$}
    \alakohta{$x=3000$}
    \alakohta{Yhtälö ei toteudu millään reaaliluvulla.}
\end{alakohdat}
%
\end{vastaus}
\end{tehtava}

\subsubsection*{Lisätehtäviä}

\begin{tehtava}
Fysiikassa ja geometriassa kaavoissa esiintyy muitakin muuttujia kuin x.
Esim. kuljettu matka on nopeus kertaa aika eli s=vt, ( jossa
s=spatium latinaksi ja englanniksi v=velocity ja t=time)
Ratkaise kysytty tuntematon yhtälöstä.

\begin{alakohdat}
% 
\alakohta{$F=ma$, $m=?$ (Voima on massa kerrottuna kiihtyvyydellä.)}
\alakohta{$p=\frac{F}{A}$, $F=?$ (Paine on voima jaettuna alalla.)}
\alakohta{$A=\pi r^2$, $r=?$ (Ympyrän pinta-ala on pii kerrottuna säteen neliöllä.)}
\alakohta{$V=\frac{1}{3} \pi r^2 h$, $h=?$ (Kartion tilavuus on piin kolmasosa
kerrottuna säteen neliöllä ja kartion korkeudella.}
\end{alakohdat}
\begin{vastaus}
\begin{alakohdat}
\alakohta{$m=\frac{F}{a}$}
\alakohta{$F=p A$}
\alakohta{$r=\sqrt{\frac{A}{\pi}}$}
\alakohta{$h=\frac{3V}{ \pi r^2 }$} % Vastaus tarkistettu, Riku Laine 2013-11-09
\end{alakohdat}
\end{vastaus}
\end{tehtava}

\begin{tehtava}
Ratkaise yhtälö. Tarkista sijoittamalla.
\begin{alakohdat}
    \alakohta{$\dfrac{8x}{3} = 12$}
    \alakohta{$3 - \dfrac{x}{4} = x$}
    \alakohta{$-\dfrac{7x}{8} = x+2$}
    \alakohta{$x+7 = 1 - \dfrac{x-1}{2}$}
    \alakohta{$\dfrac{2x-1}{3} - 1 = \dfrac{4x}{9}$}
    \alakohta{$\dfrac{x+5}{3} = x+3-\dfrac{2x+1}{4}$}
\end{alakohdat}
%
\begin{vastaus}
\begin{alakohdat}
    \alakohta{$x=4\frac 1 2$}
    \alakohta{$x=2\frac{2}{5}$}
    \alakohta{$x=-1\frac{1}{15}$}
    \alakohta{$x = -3\frac 2 3$}
    \alakohta{$x=6$}
    \alakohta{$x=-6\frac{1}{2}$}
\end{alakohdat}
\end{vastaus}
\end{tehtava}

\begin{tehtava}
Ratkaise:
\begin{alakohdat}
    \alakohta{$\frac{3x+6}{9-4x} = -5$}
    \alakohta{$\frac{9x^2-6}{16-3x} = -3x$}
\end{alakohdat}

\begin{vastaus}
\begin{alakohdat}
     \alakohta{$x=3$}
     \alakohta{$x=\frac{1}{8}$}
\end{alakohdat}
\end{vastaus}
\end{tehtava}


\begin{tehtava}
Huvipuistossa yksittäinen laitelippu maksaa 7 euroa, ja sillä pääsee yhteen laitteeseen. Rannekkeella pääsee käymään päivän aikana niin monessa laitteessa kuin haluaa, ja se maksaa 37 euroa. Kuinka monessa laitteessa on käytävä, jotta rannekkeen ostaminen kannattaa?
\begin{vastaus}
$6$ laitteessa
\end{vastaus}
\end{tehtava}

\begin{tehtava}
Kännykkäliittymän kuukausittainen perusmaksu on 2,90 euroa. Lisäksi jokainen puheminuutti ja tekstiviesti maksaa 0,69 senttiä. Pekan kännykkälasku kuukauden
ajalta oli 27,05 euroa.

\begin{alakohdat}
	\alakohta{Kuinka monta puheminuuttia tai tekstiviestiä Pekka käytti kuukauden aikana yhteensä?}
	\alakohta{Pekka lähetti kaksi tekstiviestiä jokaista viittä puheminuuttia kohden. Kuinka monta tekstiviestiä Pekka lähetti?}
\end{alakohdat}

	\begin{vastaus}
		\begin{alakohdat}
			\alakohta{350 puheminuuttia tai tekstiviestiä}
			\alakohta{100 tekstiviestiä}
		\end{alakohdat}
	\end{vastaus}
\end{tehtava}

\begin{tehtava}
 Taksimatkan perusmaksu on arkisin 5,90 euroa. Taksimatkan hinta oli 28,00 euroa. Mikä oli taksimatkan pituus, kun
\begin{alakohdat}
     \alakohta{kun Petteri matkusti yksin, jolloin taksa oli 1,52 euroa kilometriltä?}
     \alakohta{kun Arttu otti taksin kuuden ystävänsä kanssa, jolloin taksa oli 2,13 euroa kilometriltä?}
\end{alakohdat}
	\begin{vastaus}
		\begin{alakohdat}
			\alakohta{14,5 kilometriä}
			\alakohta{10,4 kilometriä}
		\end{alakohdat}
	\end{vastaus}
\end{tehtava}


\begin{tehtava}
Sadevesikeräin näyttää vesipatsaan korkeuden millimetreinä. Eräänä aamuna
keräimessä oli 5~mm vettä. Seuraavana aamuna samaan aikaan keräimessä oli 23~mm vettä. Muodosta yhtälö ja selvitä, kuinka paljon vettä oli keskimäärin satanut kuluneen vuorokauden aikana tunnissa.
	\begin{vastaus}
	$0,75$ mm/tunti
	\end{vastaus}
\end{tehtava}

\begin{tehtava}
%Laatinut Jaakko Viertiö 2013-11-10
Määritä ne neljä peräkkäistä paritonta kokonaislukua, joiden summa on 72.
	\begin{vastaus}
	 $15, 17, 19, 21$
	\end{vastaus}
\end{tehtava}


\begin{tehtava}
Määritä luvulle $a$ sellainen arvo, 
  \begin{alakohdat}
     \alakohta{että yhtälön $5x-8-3ax=4-x$ ratkaisu on $-1$.}
     \alakohta{että yhtälön $5x+a = \frac{6x}{3} + \frac{7}{2}$ ratkaisu on $2$.}
  \end{alakohdat}
\begin{vastaus}
   \begin{alakohdat}
        \alakohta{$a=6$}
        \alakohta{$a=-\frac{2}{5}$}
      \end{alakohdat}
\end{vastaus}
\end{tehtava}

\begin{tehtava}
Kylpyhuoneessa on kolme hanaa. Hana A täyttää kylpyammeen 60 minuutissa, hana B 30 minuutissa ja hana C 15 minuutissa. Kuinka kauan kylpyammeen täyttymisessä kestää, jos kaikki hanat ovat yhtäaikaa auki?
\begin{vastaus}
$8$ min $34$ s
\end{vastaus}
\end{tehtava}


\end{tehtavasivu}
