\label{reaaliluvut}

Juurten ja murtopotenssien kohdalla on huomattava, että ainoastaan joidenkin juurten ja murtopotenssien
arvot ovat rationaalilukuja. Itse asiassa juurilla on seuraava mielenkiintoinen ominaisuus, jota ei tällä kurssilla todisteta:

\laatikko{
Jos $m,n\in\zz$ ja $\sqrt[m]{n} \notin \zz$, niin $\sqrt[m]{n} \notin \qq$.

Toisin sanottuna, jos kokonaisluvusta otettu kokonaislukujuuri ei ole kokonaisluku, niin se on aina irrationaaliluku.
}

Tämä tarkoittaa esimerkiksi sitä, että koska mikään kokonaisluku toiseen potenssiin korotettuna ei ole $2$, niin
$sqrt{2}$ on väistämättä irrationaaliluku. Irrationaalilukuja tulee siis matematiikassa vastaan varsin usein.

Irrationaalilukujen hahmottamista helpottaa se, että ne voidaan sijoittaa lukusuoralle rationaalilukujen tapaan.

\begin{kuva}
lukusuora.pohja(1.3, 3.3, 11)
for i in range(13, 34):
	lukusuora.kohta(i / 10., "\\footnotesize " + str(i // 10) + "," + str(i % 10), nimi_ylos = False)
with vari("red"):
	lukusuora.kohta(pi, r"$\pi$")
	lukusuora.kohta(sqrt(2), r"$\sqrt{2}$")
\end{kuva}

Reaalilukujen joukko muodostuu rationaalilukujen ja irrationaalilukujen joukoista. Havainnollisesti voidaan sanoa, että
reaalilukujen joukko sisältää kaikki lukusuoran luvut.

Rationaaliluvut eivät täytä koko koko reaalilukusuoraa, mutta niitä on niin tiheässä, että jokaisen
kahden eri reaaliluvun välissä on rationaalilukuja. Jokaiselle irrationaaliluvulle voidaan löytää
mielivaltaisen lähellä olevia rationaalilukuarvioita. Tätä kutsutaan irrationaalilukujen approksimoinniksi
rationaaliluvuilla. Esimerkiksi lukua $\pi$ voidaan halutusta tarkkuudesta riippuen approksimoida rationaaliluvuilla
\[
3; \quad 3,1; \quad 3,14; \quad 3,142; \quad 3,1416; \quad 3,14159; \quad 3,141593; \quad \ldots 
\]

tai lukua $\sqrt{2}$ rationaaliluvuilla
\[
1; \quad \frac{3}{2}; \quad \frac{5}{4}; \quad \frac{11}{8}; \quad \frac{23}{16}; \quad \frac{45}{32}; \quad \ldots 
\]

\begin{kuva}
lukusuora.pohja(1, 1.5, 11)
lukusuora.kohta(1, r"\footnotesize $1$", nimi_ylos = False)
lukusuora.kohta(3./2, r"\footnotesize $\frac{3}{2}$", nimi_ylos = False)
lukusuora.kohta(5./4, r"\footnotesize $\frac{5}{4}$", nimi_ylos = False)
lukusuora.kohta(11./8, r"\footnotesize $\frac{11}{8}$", nimi_ylos = False)
lukusuora.kohta(23./16, r"\footnotesize $\frac{23}{16}$", nimi_ylos = False)
lukusuora.kohta(45./32, r"\footnotesize $\frac{45}{32}$", nimi_ylos = False)
with vari("red"):
	lukusuora.kohta(sqrt(2), r"$\sqrt{2}$")
\end{kuva}

\laatikko{
Kaikki rationaalilukuja koskevat laskusäännöt pätevät myös reaaliluvuille.
}

Väite on uskottava, sillä reaalilukuja voi aina approksimoida mielivaltaisen tarkasti rationaaliluvuilla.
Väitteen tarkempi perusteleminen ei tämän kurssin työkaluilla vielä onnistu.

Siinä missä rationaalilukujen desimaaliesitykset ovat päättyviä tai jaksollisia, ovat irrationaalilukujen
desimaaliesitykset päättymättömiä ja jaksottomia. Monissa tapauksissa on hyvin vaikeaa osoittaa luku
irrationaaliseksi. Tällöin vaaditaan keinoja, jotka eivät kuulu tämän kurssin laajuuteen.

Luvun
\[\sqrt{2} \approx 1,414213562373095048801688724209\ldots\]
desimaaliesityksessä on kyllä toistuvia kohtia, esimerkiksi numeropari $88$ esiintyy kahdesti. Siinä ei
kuitenkaan ole jaksoa, jonka luvut toistuisivat yhä uudestaan
samassa järjestyksessä, kuten luvussa $3,80612312312312\ldots$.
