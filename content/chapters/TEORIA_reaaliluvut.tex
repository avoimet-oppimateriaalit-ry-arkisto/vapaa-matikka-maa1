Kaikki käyttämämme luvut eivät ole rationaalilukuja. Esimerkiksi $\sqrt{2}$ ei ole esitettävissä murtolukuna. Se tulee kuitenkin vastaan esimerksi suorakulmaisessa kolmiossa, jonka kateetit ovat pituudeltaan 1.

%Tähä tulee kolmio

\begin{kuva}
	skaalaa(4)

	A=(0,0)
	B=(0,1)
	C=(1,0)
	D=(0.1,0.1)
	E=(0,0.1)
	F=(0.1,0)

	geom.jana(B,A,"$1$")
	geom.jana(C,B,r"$\sqrt{2}$")
	geom.jana(A,C,"$1$")
	geom.jana(E,D)
	geom.jana(D,F)
\end{kuva}


$1$, $1$ ja $\sqrt{2}$

Luku $\sqrt{2}$ on \termi{irrationaaliluku}{irrationaaliluku}. Toinen peruskoulusta tuttu irrationaaliluku on ympyrän kehän pituuden suhde halkaisijaan: $\pi$. Irrationaaliluvut voidaan sijoittaa
lukusuoralle rationaalilukujen tapaan.

\begin{kuva}
lukusuora.pohja(1.3, 3.3, 11)
for i in range(13, 34):
	lukusuora.kohta(i / 10., "\\footnotesize " + str(i // 10) + "," + str(i % 10), nimi_ylos = False)
with vari("red"):
	lukusuora.kohta(pi, r"$\pi$")
	lukusuora.kohta(sqrt(2), r"$\sqrt{2}$")
\end{kuva}



Rationaalilukuja ja irrationaalilukuja kutsutaan yhdessä \termi{reaaliluvut}{reaaliluvuiksi}. Reaalilukujen joukkoja merkitään $\rr$.
Havainnollisesti reaalilukujen joukko sisältää kaikki lukusuoran luvut.

Rationaaliluvut eivät täytä koko koko reaalilukusuoraa, mutta niitä on niin tiheässä, että jokaisen kahden eri reaaliluvun välissä on rationaalilukuja. Jokaiselle irrationaaliluvulle voidaan löytää mielivaltaisen lähellä olevia rationaalilukuarvioita. Tätä kutsutaan irrationaalilukujen approksimoinniksi rationaaliluvuilla. Esimerkiksi lukua $\pi$ voidaan halutusta tarkkuudesta riippuen approksimoida rationaaliluvuilla
\[
3; \quad 3,1; \quad 3,14; \quad 3,142; \quad 3,1416; \quad 3,14159; \quad 3,141593; \quad \ldots 
\]

tai lukua $\sqrt{2}$ rationaaliluvuilla
\[
1; \quad \frac{3}{2}; \quad \frac{5}{4}; \quad \frac{11}{8}; \quad \frac{23}{16}; \quad \frac{45}{32}; \quad \ldots 
\]

\begin{kuva}
lukusuora.pohja(1, 1.5, 11)
lukusuora.kohta(1, r"\footnotesize $1$", nimi_ylos = False)
lukusuora.kohta(3./2, r"\footnotesize $\frac{3}{2}$", nimi_ylos = False)
lukusuora.kohta(5./4, r"\footnotesize $\frac{5}{4}$", nimi_ylos = False)
lukusuora.kohta(11./8, r"\footnotesize $\frac{11}{8}$", nimi_ylos = False)
lukusuora.kohta(23./16, r"\footnotesize $\frac{23}{16}$", nimi_ylos = False)
lukusuora.kohta(45./32, r"\footnotesize $\frac{45}{32}$", nimi_ylos = False)
with vari("red"):
	lukusuora.kohta(sqrt(2), r"$\sqrt{2}$")
\end{kuva}

\laatikko{
Kaikki rationaalilukuja koskevat laskusäännöt pätevät myös reaaliluvuille.
}

Väite on uskottava, sillä reaalilukuja voi aina approksimoida mielivaltaisen tarkasti rationaaliluvuilla. Väitteen tarkempi perusteleminen ei tämän kurssin työkaluilla vielä onnistu.

Siinä missä rationaalilukujen desimaaliesitykset ovat päättyviä tai jaksollisia, ovat irrationaalilukujen desimaaliesitykset päättymättömiä ja
jaksottomia. Monissa tapauksissa on hyvin vaikeaa osoittaa luku irrationaaliseksi. Tällöin vaaditaan keinoja, jotka eivät kuulu tämän kurssin laajuuteen.

Luvun
\[\sqrt{2} \approx 1,414213562373095048801688724209\ldots\]
desimaaliesityksessä on kyllä toistuvia kohtia, esimerkiksi numeropari $88$ esiintyy kahdesti. Siinä ei kuitenkaan ole jaksoa, jonka luvut toistuisivat yhä uudestaan
samassa järjestyksessä, kuten luvussa $3,80612312312312\ldots$.

Reaalilukujen ominaisuuksista kerrotaan lisää liitteessä \ref{aksioomat}.


Reaalilukujen myötä kaikki lukiokursseissa esiintyvät lukujoukot on nyt esitelty. Ne on lueteltu seuraavassa:
\begin{center}\begin{tabular}{l|c|l}
Joukko & Symboli & Mitä ne ovat\\
\hline
Luonnolliset luvut & $\nn$ &
Luvut $0$, $1$, $2$, $3$, $\ldots$ \\
Kokonaisluvut & $\zz$ & Luvut $\ldots$ $-2$, $-1$, $0$, $1$, $2$ $\ldots$ \\ 
Rationaaliluvut & $\qq$ & Luvut, jotka voidaan esittää
murtolukuna \\
Reaaliluvut & $\rr$ & Kaikki lukusuoran luvut \\
& & eli kaikki desimaaliluvut
\end{tabular} \end{center} 

\begin{tikzpicture}[line cap=round,line join=round,>=triangle 45,x=0.5cm,y=0.5cm]
\clip(-7.4,-8.8) rectangle (16.8,8.6);
\draw [rotate around={0.5:(2.2,0)}] (2.2,0) ellipse (1.1cm and 0.9cm);
\draw [rotate around={-0.8:(2.5,0)}] (2.5,0) ellipse (2cm and 1.6cm);
\draw [rotate around={-0.8:(2.5,0)}] (2.5,0) ellipse (2.9cm and 2.6cm);
\draw (2,1.5) node[anchor=north west] {$\nn$};
\draw (4.0,2.7) node[anchor=north west] {$\zz$};
\draw (5.5,3.9) node[anchor=north west] {$\qq$};
%\draw (6.6,-4.6) node[anchor=north west] {{\scriptsize Irrationaaliluvut}};
\draw (9.5,6.4) node[anchor=north west] {$\rr$};
\draw [rotate around={0.5:(4.4,0)}] (4.4,0) ellipse (5cm and 4.2cm);
%\draw [rotate around={18.2:(7.9,-5.4)}] (7.9,-5.4) ellipse (2.7cm and 0.6cm);
\draw (0.8,1.6) node[anchor=north west] {$1$};
\draw (1,-0.4) node[anchor=north west] {$5$};
\draw (2.4,-0.2) node[anchor=north west] {$101$};
\draw (4.8,0.7) node[anchor=north west] {$-5$};
\draw (1.2,-1.7) node[anchor=north west] {$0$};
\draw (1.2,3.1) node[anchor=north west] {$-14$};
%\draw (4.1,-1) node[anchor=north west] {$75$};
\draw (4.2,-2.8) node[anchor=north west] {$-\frac{1}{3}$};
\draw (6.6,1.4) node[anchor=north west] {$2\frac{1}{2}$};
\draw (-1.6,0.9) node[anchor=north west] {$-3$};
%\draw (0.4,-3.1) node[anchor=north west] {$-4$};
\draw (2.4,4.7) node[anchor=north west] {$2,6$};
\draw (-1.3,4.1) node[anchor=north west] {$\frac{5}{7}$};
\draw (-2.7,-1) node[anchor=north west] {$0,1$};
\draw (5.5,-5.9) node[anchor=north west] {$\pi$};
\draw (9.1,-3) node[anchor=north west] {$\sqrt[]{2}$};
%\draw (6,-4.7) node[anchor=north west] {$-\frac{\pi}{2}$};
%\draw (9.8,1.6) node[anchor=north west] {$\frac{5}{2}$};
\draw (10.5,3) node[anchor=north west] {$\frac{1}{\sqrt{2}}$};
%\draw (4.4,7.3) node[anchor=north west] {$3$};
\draw (-0.1,-5.3) node[anchor=north west] {$\sqrt{15}$};
%\draw (-4.9,1.5) node[anchor=north west] {$-5$};
\draw (11.8,-0.9) node[anchor=north west] {$-\frac{\pi}{2}$};
\draw (-0.6,6.8) node[anchor=north west] {$0,10110111011110\ldots$};
%\draw (-3.7,-3) node[anchor=north west] {$-3$};
\end{tikzpicture}

Lukualueita voidaan laajentaa lisää vielä tästäkin, esimerkiksi \termi{kompleksiluvut}{kompleksiluvuiksi}, jotka voidaan esittää tason pisteinä. Kompleksilukuja tarvitaan muun muassa insinöörialoilla yliopistoissa ja ammattikorkeakouluissa. Esimerkiksi vaihtosähköpiirien analyysissä, signaalinkäsittelyssä ja säätötekniikassa käytetään runsaasti kompleksilukuja. Kompleksiluvut ovat tärkeitä myös matematiikan tutkimuksessa itsessään.
