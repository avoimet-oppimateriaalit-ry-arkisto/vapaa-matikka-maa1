\begin{tehtavasivu}
\subsubsection*{Opi perusasiat}
\begin{tehtava}
Ratkaise: \\
a) $ x^2 = 4 $ \qquad
b) $ x^3 = 27 $ \qquad
c) $ x^5 = -1 $ \qquad
d) $ x^2 - 3 = 0 $ \qquad
e) $ x^3 + 125 = 0 $
\begin{vastaus}
a) $ x = \pm 2 $ \qquad
b) $ x = 3 $ \qquad
c) $ x = -1 $ \qquad
d) $ x = \pm\sqrt{3} $ \qquad
e) $ x = -5 $ 
\end{vastaus}
\end{tehtava}

\begin{tehtava}
Ratkaise \\
a) $ 5x^2 = 25 $ \qquad
b) $ (2x)^3 = 8 $ \qquad
c) $ x^4 = \frac{1}{4} $ \qquad
d) $ (3x)^2 = 36 $ \qquad
e) $ (4x)^2 + 16 = 0 $.
\begin{vastaus}
a) $ x = \pm\sqrt{5} $ \qquad
b) $ x = 1 $ \qquad
c) $ x = \pm\frac{1}{\sqrt{2}} $ \qquad
d) $ x = \pm 2 $ \qquad
e) Ei ratkaisua. 
\end{vastaus}
\end{tehtava}

\begin{tehtava}
Ratkaise potenssiyhtälöt
\begin{alakohdat}
\alakohta{$x^3 = 81$}
\alakohta{$x^5 = 10$}
\alakohta{$x^2 = 4$}
\alakohta{$x^4 = 1$.}
\end{alakohdat}
\begin{vastaus}
\begin{alakohdat}
\alakohta{$x= 3 \sqrt[3]{3}$}
\alakohta{$x= \sqrt[5]{10}$}
\alakohta{$x= \pm 2$}
\alakohta{$x= \pm 1$}
\end{alakohdat}
\end{vastaus}
\end{tehtava}

\begin{tehtava}
Ratkaise potenssiyhtälöt
\begin{alakohdat}
\alakohta{$x^4 - 8 = 0$}
\alakohta{$2x^3 + 7 = 0$}
\alakohta{$\frac{x^2}{4} - \frac{5}{2} = 1$}
\alakohta{$1,51 x^4 - 1,2 = 7,5$.}
\end{alakohdat}
\begin{vastaus}
\begin{alakohdat}
\alakohta{$x = \pm\sqrt[4]{8}$}
\alakohta{$x= -\sqrt[3]{\frac{7}{2}}$}
\alakohta{$x= \pm\sqrt{14}$}
\alakohta{$x= \pm\sqrt[4]{\frac{870}{151}}$}
\end{alakohdat}
\end{vastaus}
\end{tehtava}

\begin{tehtava}%Laati Henri Ruoho 10-22-2013
Valtteri sijoitti 3100 euroa pääomatilille. Yhdeksän vuoden kuluttua tilillä oli 5000 euroa. Mikä oli tilin vuotuinen korkoprosentti, kun se oli pysynyt samana vuodesta toiseen? 
\begin{vastaus}
$5,4\%$
\end{vastaus}
\end{tehtava}
\begin{tehtava}%Laati Henri Ruoho 10-22-2013
Valtteri haluaa sijoittamansa pääoman kaksinkertaistuvan seuraavassa kymmenessä vuodessa. Kuinka suurta korkoprosenttia hän esittää pankinjohtajalle?
\begin{vastaus}
$7,1\%$
\end{vastaus}
\end{tehtava}


\begin{tehtava}%Laati Henri Ruoho 10-22-2013
Torimyyjä tarvitsee kappoja eli kuution muotoisia mitta-astioita. Pienen kapan vetoisuus on 2 litraa ja ison kapan vetoisuus 5 litraa. Määritä astioiden mitat.
\begin{vastaus}
Pienen kapan sivun pituus $12,6$ cm, ison kapan $17,1$ cm. 
\end{vastaus}
\end{tehtava}

\begin{tehtava}%Laati Henri Ruoho 10-22-2013
Suomi sitoutui vähentämään kasvihuonepäästöjään vuoden 2005 alusta $20\%$ vuoteen 2020 mennessä. Kuinka paljon päästöjä oli tarkoitus vähentää vuosittain? Anna vastaus yhden desimaalin tarkkuudella.
\begin{vastaus}
$1,5\%$
\end{vastaus}
\end{tehtava}

\subsubsection*{Hallitse kokonaisuus}
\begin{tehtava}%Laati Henri Ruoho 10-11-2013
Syksyllä 2012 maapallon väkiluvun arveltiin olevan noin 7 miljardia. Väkiluku oli kaksinkertaistunut arviolta 38:ssa vuodessa. Tutki laskimella, milloin seuraava miljardi saavutettaisiin väestönkasvun jatkuessa tasaisesti.
\begin{vastaus}
Noin vuonna 2020.
\end{vastaus}
\end{tehtava}


\begin{tehtava}
Muinainen hallitsija Tauno Alpakka rakennuttaa itselleen kuution muotoista palatsia.  Palatsin tilavuuden tulee olla $5~000~\mathrm{m}^3$. 
\begin{alakohdat}
\alakohta{Kuinka korkea palatsista tulee?}
\alakohta{Palatsin ulkopuoli päällystetään $10~\mathrm{cm}$:n paksuisella kultakerroksella.  Kuinka monta kiloa kultaa tarvitaan? (Kullan tiheys on $19,23 \cdot 10^3~\mathrm{ kg}/\mathrm{m}^3$.) }
\alakohta{Kuinka monta kiloa kultaa tarvitaan, jos myös palatsin sisäpuoli päällystetään?}
\end{alakohdat}
\begin{vastaus}
\begin{alakohdat}
\alakohta{$10 \sqrt[3]{5} \approx 17~\mathrm{m}$}
\alakohta{$961500 \sqrt[3]{25} \approx 2~800~000~\mathrm{kg}$}
\alakohta{$2115300 \sqrt[3]{25} \approx 6~200~000~\mathrm{kg}$}
\end{alakohdat}
\end{vastaus}
\end{tehtava}

\begin{tehtava}%Laati Henri Ruoho 10-11-2013
Kuvaruudun pituuksien suhde on $16:9$. Määritä ruudun mitat sentin tarkkuudella, kun sen pinta ala on \(0,40\ \mathrm{m}^2\). (Muista, että suorakulmion pinta-ala on sen erisuuntaisten sivujen pituuksien tulo.)
\begin{vastaus}
$84\ cm$ ja $47\ cm$
\end{vastaus}
\end{tehtava}

\begin{tehtava}%Laati Henri Ruoho 10-11-2013
Tilan eläimelle rakennetaan uutta aitausta. Käytettävissä on 100 metriä sähköaitaan tarvittavaa metallijohtoa. Kuinka suuri pinta-ala kotieläimillä on käytettävissään, jos aitaus on muodoltaan
\begin{alakohdat}
\alakohta{neliö?}
\alakohta{ympyrä?}
\end{alakohdat}
(Muista, että neliön pinta-ala on sen sivun pituuden toinen potenssi ja ympyrän pinta-ala on \(\pi r^2\), missä \(r\) on ympyrän säde. Saman ympyrän kehän pituus on \(2\pi r\).)
\begin{vastaus}
\begin{alakohdat}
\alakohta{$625 \ m^2$}
\alakohta{$796 \ m^2$}
\end{alakohdat}
\end{vastaus}
\end{tehtava}

\end{tehtavasivu}
