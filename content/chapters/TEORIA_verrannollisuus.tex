Kaksi muuttujaa voivat riippua toisistaan monin eri tavoin.
Tavallisia riippuvuuden tyyppejä ovat suoraan ja kääntäen verrannollisuus.

\laatikko{
	Kaksi muuttujaa $x$ ja $y$ ovat suoraan verrannolliset, jos toinen saadaan
	toisesta kertomalla se jollakin vakiolla, eli $y = kx$. Vakiota $k$
	kutsutaan \termi{verrannollisuuskerroin}{verrannollisuuskertoimeksi}.
}

Jos suureet $x$ ja $y$ ovat suoraan verrannolliset, niin tätä merkitään $x\sim y$ tai $x\propto y$.

Suoraan verrannollisuus voidaan tunnistaa esimerkiksi laskemalla muuttujien
suhde ja toteamalla, että se on muuttujista riippumaton vakio.

\begin{esimerkki}
Banaanien kilohinta on $2,00$ euroa. Seuraavassa taulukossa on
banaanien paino\footnote{Fysikaalisesti kyse on massasta, mutta
arkikielessä käytetään sanaa paino.}, jonka saa ostettua tietyllä rahamäärällä:
\begin{center} 
\begin{tabular}{|l|r|r|}
\hline
Hinta (euroa) & Paino (kg) & Hinta/paino (euroa/kg) \\
\hline
$1,00$ & $0,50$ & $2,00$ \\
$2,00$ & $1,00$ & $2,00$ \\
$3,00$ & $1,50$ & $2,00$ \\
$4,00$ & $2,00$ & $2,00$ \\
\hline
\end{tabular}
\end{center}
Hinnan ja painon suhde on vakio, $2,00$ euroa/kg, joten ostettujen
banaanien paino ja niihin käytetty rahamäärä ovat suoraan verrannolliset.
\end{esimerkki}

Suoraan verrannollisuutta voidaan kuvata myös niin, että jos
toinen muuttujista kaksinkertaistuu, niin toinenkin kaksinkertaistuu.
Samoin jos toisen muuttujan arvo puolittuu, toisenkin arvo puolittuu.

Jos suoraan verrannollisista muuttujista piirretään kuvaaja, pisteet
asettuvat suoralle:

%\missingfigure{Kuva, johon piirretty vaaka-akselille käytetty rahamäärä
%ja pystyakselille ostettujen banaanien paino.}

\begin{center}
\begin{kuvaajapohja}{1.5}{0}{5}{0}{3}
\kuvaajapiste{1}{0.5}
\kuvaajapiste{2}{1}
\kuvaajapiste{3}{1.5}
\kuvaajapiste{4}{2}
\kuvaaja{0.5*x}{}{black}
\node at (8.2,0) {euroa};
\node at (0,4.8) {kg};
\end{kuvaajapohja}
\end{center}

Suoraan verrannollisia muuttujia ovat myös esimerkiksi
\begin{alakohdat}
    \alakohta{aika ja kuljettu matka, kun liikutaan vakionopeudella, tai}
    \alakohta{kappaleen massa ja painovoiman kappaleeseen aiheuttama voima.}
\end{alakohdat}

Suoraan verrannollisuutta monimutkaisempi riippuvuus on kääntäen
verrannollisuus.

\laatikko{
Muuttujat $x$ ja $y$ ovat kääntäen verrannolliset, jos toinen saadaan toisesta
jakamalla jokin vakio sillä, eli $y = \frac{a}{x}$.
}

Kääntäen verrannollisuus voidaan tunnistaa esimerkiksi
kertomalla muuttujien arvoja keskenään ja huomaamalla,
että tulo on muuttujista riippumaton vakio.

\begin{esimerkki}
	Nopeus ja matkaan tarvittava aika ovat kääntäen verrannolliset.
	Jos kuljettavana matkana on $10$ km, voidaan nopeudet ja matka-ajat
	kirjoittaa taulukoksi:
	\begin{center} 
		\begin{tabular}{|l|r|r|}
			\hline
			Nopeus (km/h) & Matka-aika (h) & Nopeus$\cdot$matka-aika (km) \\
			\hline
			$5$ & $2$ & $10$ \\
			$10$ & $1$ & $10$ \\
			$12$ & $0,8$ & $10$ \\
			\hline
		\end{tabular}
	\end{center}
	Tulo on aina $10$ km, joten nopeus ja matka-aika ovat kääntäen verrannolliset.
\end{esimerkki}

Kääntäen verrannollisuutta voidaan kuvata myös niin, että jos
toinen muuttujista kaksinkertaistuu, toinen puolittuu.

Jos kääntäen verrannollisista muuttujista piirretään kuvaaja, pisteet
muodostavat laskevan käyrän:

%\missingfigure{Kuva, johon on piirretty matka-aika ja nopeus (yllä olevan
%taulukon mukaisesti).}

% FIXME sellainen käyrätemplate, joka toimii myös silloin, jos toiseen akseliin
% FIXME tarvitaan eri jakoväli kuin toiseen.
% FIXME Myös akseleiden nimeäminen olisi kiva olla templatessa.
\begin{center}
	\begin{kuvaajapohja}{0.8}{0}{13}{0}{5}
		\kuvaajapiste{5}{2}
		\kuvaajapiste{10}{1}
		\kuvaajapiste{12}{0.8}
		\kuvaaja{10/x}{}{black}
		\node at (10.6,0) {h};
		\node at (0,4.3) {km/h};
	\end{kuvaajapohja}
\end{center}

Kääntäen verrannollisia muuttujia ovat myös esimerkiksi
\begin{alakohdat}
    \alakohta{Kaivamistyön suorittamiseen kuluva aika ja työntekijöiden lukumäärä.}
     \alakohta{Äänen intensiteetti ja äänilähteen etäisyys.}
     \alakohta{Kappaleiden välinen vetovoima ja niiden etäisyys}
     \alakohta{Irtomakeisten kilohinta ja viiden euron irtokarkkipussin massa}
\end{alakohdat}
