\begin{tehtavasivu}

\begin{tehtava}
Ovatko seuraavat luvut rationaalilukuja vai irrationaalilukuja? Kunkin desimaalit
noudattavat yksinkertaista sääntöä.
\begin{alakohdat}
\alakohta{$0,123456789101112131415 \ldots$}
\alakohta{$2,415115115115115115115 \ldots$}
\alakohta{$1,010010001000010000010 \ldots$}
\end{alakohdat}
\begin{vastaus}
a) irrationaaliluku \ b) rationaaliluku (jakso on 151) \ c) irrationaaliluku
\end{vastaus}
\end{tehtava}

\begin{tehtava}
Mikä on pienin lukua -3 suurempi luku \\
a) kokonaislukujen \ b) luonnollisten lukujen \ c) reaalilukujen joukossa?
\begin{vastaus}
a) -2 \ b) 0 \ c) Sellaista ei ole. Jos nimittäin $a > -3$, niin keskiarvo
$\frac{-3+a}{2}$ on vielä lähempänä lukua $-3$. 
\end{vastaus}
\end{tehtava}

\begin{tehtava}
$\boldsymbol{[\star]}$ Osoita, että \\
a) jokaisen kahden rationaaliluvun välissä on rationaaliluku \\
b) jokaisen kahden rationaaliluvun välillä on irrationaaliluku \\
c) jokaisen kahden irrationaaliluvun välissä on rationaaliluku \\
d) jokaisen kahden irrationaaliluvun välissä on irrationaaliluku. \\
e) Perustele edellisten kohtien avulla, että minkä tahansa kahden luvun
välissä on äärettömän monta rationaali- ja irrationaalilukua.
\begin{vastaus}
Vihjeet: a) keskiarvo \ b) $\sqrt{2}$ on irrationaaliluku. Käytä
painotettua keskiarvoa. \ c) pyöristäminen \ d) hyödynnä b-kohtaa
e) keskiarvot
\end{vastaus}
\end{tehtava}

\end{tehtavasivu}
