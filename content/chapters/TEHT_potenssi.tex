\begin{tehtavasivu}

\paragraph*{Opi perusteet}

    \begin{tehtava}
    Laske.
        \begin{alakohdatrivi}
        \alakohta{$2^4$ }
       	\alakohta{$(-2)^4$}
        \alakohta{$-2^4$}
 		\end{alakohdatrivi}
        \begin{vastaus}
        \begin{alakohdatrivi}
            \alakohta{$16$}
            \alakohta{$16$}
            \alakohta{$-16$}
        \end{alakohdatrivi}
        \end{vastaus}
    \end{tehtava}

    \begin{tehtava}
        Esitä potenssin avulla. 
        \begin{alakohdat}
        \alakohta{$a\cdot a\cdot a$}
        \alakohta{$a\cdot a\cdot a\cdot b\cdot b\cdot b\cdot b$}
        \alakohta{$a\cdot b\cdot a\cdot b\cdot a\cdot b\cdot a$}
        \end{alakohdat}
        
        \begin{vastaus}
        \begin{alakohdatrivi}
            \alakohta{$a^3$}
            \alakohta{$a^3b^4$}
            \alakohta{$a^4b^3$}
        \end{alakohdatrivi}
        \end{vastaus}
    \end{tehtava}

    \begin{tehtava}
    Sievennä.
        \begin{alakohdatrivi}
        	\alakohta{$a^2\cdot a^3$ }
       		\alakohta{$(a^2)^3$}
        	\alakohta{$\frac{a^7}{a^5}$}
        	\alakohta{$a^0$}
 		\end{alakohdatrivi}
        \begin{vastaus}
        \begin{alakohdatrivi}
            \alakohta{$a^5$}
            \alakohta{$a^6$}
            \alakohta{$a^2$}
            \alakohta{$1$, ei määritelty kun $a=0$}
        \end{alakohdatrivi}
        \end{vastaus}
    \end{tehtava}
    
        \begin{tehtava}
    Sievennä muotoon, jossa ei ole sulkuja.
        \begin{alakohdatrivi}
        	\alakohta{$(ab)^2$ }
       		\alakohta{$\left( \frac{a}{b} \right)^3$}
 		\end{alakohdatrivi}
        \begin{vastaus}
        \begin{alakohdatrivi}
            \alakohta{$a^2b^2$}
            \alakohta{$\frac{a^3}{b^3}$}
        \end{alakohdatrivi}
        \end{vastaus}
    \end{tehtava}

        \begin{tehtava}
  		Kirjoita murtolukuna
        \begin{alakohdatrivi}
        	\alakohta{$3^{-1}$}
      		\alakohta{$5^{-2}$}
        	\alakohta{$\left(\dfrac{3}{4}\right)^{-1}$}
 		\end{alakohdatrivi}
       \begin{vastaus}
        \begin{alakohdatrivi}
            \alakohta{$\frac{1}{3}$}
            \alakohta{$\frac{1}{25}$}
            \alakohta{$\dfrac{4}{3}$}
        \end{alakohdatrivi}
        \end{vastaus}
    \end{tehtava}
    
    
\begin{tehtava}
%Tehtävän laatinut Johanna Rämö 9.11.2013.
%Ratkaisun tehnyt Johanna Rämö 9.11.2013.
        \begin{alakohdat}
        \alakohta{Onko $x=4$ yhtälön $x^3=64$ ratkaisu?}
        \alakohta{Onko $x=2$ yhtälön $x^5=30$ ratkaisu?}
        \alakohta{Onko $x=-1$ yhtälön $x^4-+3x+2=0$ ratkaisu?}
        \end{alakohdat}
        
        \begin{vastaus}
        \begin{alakohdatrivi}
            \alakohta{On, sillä $4^3=4 \cdot 4 \cdot 4=64$.}
            \alakohta{Ei, sillä $2^5=2 \cdot 2 \cdot 2 \cdot 2 \cdot 2=32$.}
            \alakohta{On, sillä $(-1)^4-+3(-1)+2=1-3+2=0$.}
        \end{alakohdatrivi}
        \end{vastaus}
\end{tehtava}

    \begin{tehtava}
%Tehtävän laatinut Johanna Rämö 9.11.2013.
%Ratkaisun tehnyt Johanna Rämö 9.11.2013.
        Pudotat pallon kädestäsi lattialle. Pallo pomppaa ensin metrin korkeudelle ja sen jälkeen jokaisen pompun korkeus on aina puolet edellisestä korkeudesta. Kuinka korkea on pallon 5. pomppu? Kuinka korkea on pallon 13. pomppu?     
        \begin{vastaus}
        Viidennen pompun korkeus on $(1/2)^4=1/16 \approx 0,06$ metriä. Pallon 13. pompun korkeus on $(1/2)^{12} \approx 0{,}0002$ metriä eli $0,2$ millimetriä.
        \end{vastaus}
\end{tehtava}


\paragraph*{Hallitse kokonaisuus}

\begin{tehtava}
  		Sievennä.
        \begin{alakohdatrivi}
        	\alakohta{$a^3\cdot a^2\cdot a^5$}
       		\alakohta{$(a^2a^3)^4$}
        	\alakohta{$\frac{k^3}{k^{-5}}$}
 		\end{alakohdatrivi}
        \begin{vastaus}
        \begin{alakohdatrivi}
            \alakohta{$a^{10}$}
            \alakohta{$a^6$}
            \alakohta{$a^{20}$}
        \end{alakohdatrivi}
        \end{vastaus}
\end{tehtava}

\begin{tehtava}
  		Laske.
        \begin{alakohdatrivi}
            \alakohta{$-7^{2}$}
            \alakohta{$(-7)^2$}
        	\alakohta{$7^{-2}$}
        	\alakohta{$(-7)^{-2}$}
 		\end{alakohdatrivi}
        \begin{vastaus}
        \begin{alakohdatrivi}
            \alakohta{$-49$}
            \alakohta{$49$}
            \alakohta{$\frac{1}{7^2}=\frac{1}{49}$}
            \alakohta{$\frac{1}{(-7)^2}=\frac{1}{49}$}
        \end{alakohdatrivi}
        \end{vastaus}
\end{tehtava}

\begin{tehtava}
  		Sievennä.
        \begin{alakohdatrivi}
        	\alakohta{$b^3(ab^0)^2$}
       		\alakohta{$(ab^3)^0$}
        	\alakohta{$(aa^4)^3a^2$ }
 		\end{alakohdatrivi}
        \begin{vastaus}
        \begin{alakohdatrivi}
            \alakohta{$a^2b^3$}
            \alakohta{$1$ }
            \alakohta{$a^{17}$}
        \end{alakohdatrivi}
        \end{vastaus}
\end{tehtava}

\begin{tehtava}
  		Sievennä.
        \begin{alakohdatrivi}
        	\alakohta{$(a^2)^{-2}$}
       		\alakohta{$(ab^{-1})^3$}
        	\alakohta{$\frac{k^3}{k^{-5}}$}
 		\end{alakohdatrivi}
        \begin{vastaus}
        \begin{alakohdatrivi}
            \alakohta{$\frac{1}{a^4}$}
            \alakohta{$\frac{a^3}{b^3}$}
            \alakohta{$k^8$}
        \end{alakohdatrivi}
        \end{vastaus}
\end{tehtava}

\begin{tehtava}
  		Sievennä.
        \begin{alakohdat}
       		\alakohta{$\left(\frac{a^2b^{-2}}{a^2b}\right)^{-3}$}
        	\alakohta{$\frac{5a^2}{-15a}$}
			\alakohta{$\left(-(\frac{10^3}{100b})^2 b^{-1} \right )^2$}
 		\end{alakohdat}
        \begin{vastaus}
        \begin{alakohdatrivi}
            \alakohta{$b^9$ }
            \alakohta{$-\frac{1}{3}a = -\frac{a}{3}$}
            \alakohta{$\frac{10~000}{b^6}$}
        \end{alakohdatrivi}
        \end{vastaus}
\end{tehtava}
 
\begin{tehtava}
  		$\star$ Sievennä lauseke
$$\left[ \frac{(a^2b^{-2}c)^{-3}:\left(c^2\cdot (ab^{-2})^0 \cdot a^{-4}\right)}
{\left((c^{-1}\cdot a^3)^{-1}:a^2\right)^3(ab^2c^{-3})^3} \right]^2.$$
\begin{vastaus}
$c^2$
\end{vastaus}
\end{tehtava} 
 
\begin{tehtava}
$\star$ Tetraatio on lyhennysmerkintä ''potenssitornille'',
jossa esiintyy vain yhtä lukua. Se määritellään seuraavasti.
\[^na = \underbrace{{a^{a^{a^{\mathstrut^{.^{.^{.^{a}}}}}}}}}_{n\textrm{ kpl }}. \]
Laske \quad a) $^42$  \quad b) $^35$. \\ c) Ratkaise yhtälö $^x2= 16$.
\begin{vastaus}
a) $^42 = 2^{2^{2^2}}=2^{2^4}=2^{16}=65\ 536$ \
b) $^35 = 5^{5^5} = 5^{25} \approx 2,98 \cdot 10^{17}$. \\
c) Kokeilemalla $x =3$.
\end{vastaus}
\end{tehtava}

\paragraph*{Lisää tehtäviä}

    \begin{tehtava}
%Tehtävän laatinut Johanna Rämö 9.11.2013.
%Ratkaisun tehnyt Johanna Rämö 9.11.2013.
        Ympyrän mallisen pöytäliinan halkaisija on $1{,}2$ m. Mikä on pöytäliinan pinta-ala?
       
        \begin{vastaus}
        Pinta-ala on $\pi \cdot (0{,}6)^2 \approx 1,1$ $cm^2$.
        \end{vastaus}
\end{tehtava}

\begin{tehtava}
        Laske. \quad
        a) $\displaystyle \frac{2^3}{2^2}$ \quad \
        b) $\displaystyle \frac{2^4}{2^2}$ \quad \
        c) $\displaystyle \frac{2^3}{2^1}$ \quad \
        d) $\displaystyle \frac{2^3}{2^0}$ \quad \
        e) $\displaystyle \frac{2^3}{2^4}$ \quad \
        f) $\displaystyle \frac{2^3}{2^5}$
        
        \begin{vastaus}
            a) $2$ \qquad
            b) $4$ \qquad
            c) $4$ \qquad
            d) $8$ \qquad
            e) $\frac{1}{2}$ \qquad
            f) $\frac{1}{4}$
        \end{vastaus}
    \end{tehtava}

%%%%%%%%%%%%%%%%%%%%%%%%%%%%%%%%%%

 \begin{tehtava}
        %Sievennä. \quad
        a) $(1\cdot a)^3$ \qquad
        b) $(a\cdot 2)^2$ \qquad
        c) $(-2abc)^3$ \qquad
        d) $(3a)^4$

        \begin{vastaus}
            a) $a^3$ \qquad
            b) $4a^2$ \qquad
            c) $-8a^3b^3c^3$ \qquad
            d) $91a^4$
        \end{vastaus}
    \end{tehtava}
       
    \begin{tehtava}
        %Sievennä \quad
        a) $a^2\cdot a^3$ \qquad
        b) $a^3a^2$ \qquad
        c) $a^2 a$ \qquad
        d) $a a^2 a$ \qquad
        e) $a^2a^1a^3$
        
        \begin{vastaus}
            a) $a^5$ \qquad
            b) $a^5$ \qquad
            c) $a^3$ \qquad
            d) $a^4$ \qquad
            e) $a^6$
        \end{vastaus}
    \end{tehtava}
    
    \begin{tehtava}
        %Sievennä \quad
        a) $a^0$ \qquad
        b) $a^0a^0$ \qquad
        c) $a a^1$ \qquad
        d) $aa^0$ \qquad
        e) $a^0a^1$
        
        \begin{vastaus}
            a) $1$ \quad ($a\neq0$, koska $0^0$ ei ole määritelty) \qquad
            b) $1$ \qquad
            c) $a$ \qquad
            d) $a^2$ \qquad
            e) $a$
        \end{vastaus}
    \end{tehtava}
    
    \begin{tehtava}
        %Sievennä \quad
        a) $a^1 a a^2$ \qquad
        b) $aaaa$ \qquad
        c) $a^3ba^2$ \qquad
        d) $aba^0ba^1$
        
        \begin{vastaus}
            a) $ a^4$ \qquad
            b) $a^4$ \qquad
            c) $a^5b$ \qquad
            d) $a^2b^2$
        \end{vastaus}
    \end{tehtava}
    % teht 5
       

    \begin{tehtava}
        %Sievennä \quad
        a) $-a\cdot(-a)$ \qquad
        b) $-a\cdot(-a)\cdot(-b)$ \qquad
        c) $-a^2\cdot(-a^2)$
    
        \begin{vastaus}
            a) $a^2$ \qquad
            b) $-a^2b$ \qquad
            c) $a^4$
        \end{vastaus}
    \end{tehtava}

    \begin{tehtava}
        %Sievennä \quad
        a) $-a^3\cdot(-a^2)$ \qquad
        b) $a\cdot(-a)\cdot(-b)$ \qquad
        c) $a^2\cdot(-a^2)$
        
        \begin{vastaus}
            a) $a^5$ \qquad
            b) $a^2b$ \qquad
            c) $-a^4$
        \end{vastaus}
    \end{tehtava}

    %teht. 10
    \begin{tehtava}
        %Sievennä \quad
        a) $0^3\cdot0^3\cdot0^3$ \qquad
        b) $3^1$ \qquad
        c) $2^{2+3}$ \qquad
        d) $2^{6-4}$ \qquad
        e) $5^0$

        \begin{vastaus}
            a) $0$ \qquad
            b) $3$ \qquad
            c) $32$ \qquad
            d) $4$ \qquad
            e) $1$
        \end{vastaus}
    \end{tehtava}
    \begin{tehtava}
        %Sievennä \quad
        a) $(a^3)^1$ \qquad
        b) $(a^6)^2$ \qquad
        c) $(a^2)^4$ \qquad 
        d) $(a^1)^3$ \qquad
        e) $(a^0)^5$

        \begin{vastaus}
            a) $a^3$ \qquad
            b) $a^{12}$ \qquad
            c) $a^8$ \qquad
            d) $a^3$ \qquad
            e) $1$
        \end{vastaus}
    \end{tehtava}
    
       \begin{tehtava}
        %Sievennä \quad
        a) $(1^3\cdot 2^2)^2$ \qquad
        b) $(1^2\cdot 2^3)^2$ \qquad
        c) $(2^2a^4)^2$ \qquad
        d) $b(3b)^3$

        \begin{vastaus}
            a) $16$ \qquad
            b) $64$ \qquad
            c) $16a^8$ \qquad
            d) $27b^4$
        \end{vastaus}
    \end{tehtava}
    
    \begin{tehtava}
        %Sievennä \quad
        a) $(a^3b^2)^2$ \qquad
        b) $a(a^2b^3)^4$ \qquad
        c) $(b^2a^4)^5$ \qquad
        d) $b(2ab^2)^3$
        
        \begin{vastaus}
            a) $a^6b^4$ \qquad
            b) $a^9b^{12}$ \qquad
            c) $a^{20}b^{10}$ \qquad
            d) $8a^3b^7$
        \end{vastaus}
    \end{tehtava}
      
    \begin{tehtava}
        %Sievennä \quad
        a) $\frac{a^3}{a^2}$ \qquad
        b) $\frac{a^4}{a^2}$ \qquad
        c) $\frac{a^3}{a^1}$ \qquad
        d) $\frac{a^3}{a^0}$ \qquad
        e) $\frac{a^3}{a^4}$ \qquad
        f) $\frac{a^3}{a^5}$
        
        \begin{vastaus}
            a) $a$ \qquad
            b) $a^2$ \qquad
            c) $a^2$ \qquad
            d) $a^3$ \qquad
            e) $a^{-1} = \frac{1}{a}$ \qquad
            f) $a^{-2} = \frac{1}{a^2}$
        \end{vastaus}
    \end{tehtava}
    
    %teht. 20
    \begin{tehtava}
        %Sievennä \quad
        a) $\frac{a^2b^2}{ab}$ \qquad
        b) $\frac{a^2b}{a^2}$ \qquad
        c) $\frac{a^3}{a^3}$ \qquad
        d) $\frac{1}{a^0}$ \qquad
        e) $\frac{ab^3}{-b^4}$
        
        \begin{vastaus}
            a) $ab$ \qquad
            b) $b$ \qquad
            c) $1$ \qquad
            d) $1$ \qquad
            e) $-\frac{a}{b}$
        \end{vastaus}
    \end{tehtava}
    
   
    
    \begin{tehtava}
         Sievennä ja kirjoita potenssiksi, jonka eksponentti on positiivinen.\\
        a) $a^{-3}$ \qquad
        b) $\frac{a}{a^3}$ \qquad
        c) $a^{-2}\cdot a^5$ \qquad
        d) $\frac{b}{a^4}b^{-4}$ \qquad
        e) $\frac{a^3}{a^{-5}}$
        
        \begin{vastaus}
            a) $\frac{1}{a^3}$ \qquad
            b) $\frac{1}{a^2}$ \qquad
            c) $a^3$ \qquad
            d) $\frac{}{a^4b^3}$ \qquad
            e) $a^8$
        \end{vastaus}
    \end{tehtava}
    
    
    
    \begin{tehtava}
        Esitä ilman sulkuja ja sievennä. \\
        a) $(\frac{1}{2})^2$ \qquad
        b) $(\frac{1}{3})^3$ \qquad
        c) $(\frac{a}{b})^4$ \qquad
        d) $(\frac{a^2}{b^3})^2$ \qquad
        e) $\left(\frac{a^2}{ab^2}\right)^2$
        
        \begin{vastaus}
            a) $\frac{1}{4}$ \qquad
            b) $\frac{1}{27}$ \qquad
            c) $\frac{a^4}{b^4}$ \qquad
            d) $\frac{a^4}{b^6}$ \qquad
            e) $\frac{a^2}{b^4}$
        \end{vastaus}
    \end{tehtava}
    % FIXME alla olevat on muotoiltu erilailla (yksi alikohta per rivi).
% Mahtunee samalle riville.
% Lisäksi tässä on duplikaatteja (esim. aaaa:n sievennys on sekä yllä että alla). 
   	Sievennä.
    \begin{tehtava}%perteht
        %Sievennä
		\begin{alakohdat}
        	\alakohta{$2^3 $ }
        	\alakohta{$aaaa$ }
        	\alakohta{$a^3a^2$ }
        	\alakohta{$0^4$}
		\end{alakohdat}        
        \begin{vastaus}
        \begin{alakohdat}
            \alakohta{$8$ }
            \alakohta{$a^4$ }
            \alakohta{$a^5$ }
            \alakohta{$0$}
        \end{alakohdat}
        \end{vastaus}
    \end{tehtava}

    \begin{tehtava}%perteht
        %Sievennä
        \begin{alakohdat}
        	\alakohta{$a^2a^5 $ }
        	\alakohta{$\frac{a^5}{a^3}$ }
        	\alakohta{$(a^3)^2$ }
        	\alakohta{$12^0$}
		\end{alakohdat}        
        \begin{vastaus}
        \begin{alakohdat}
            \alakohta{$a^7$ }
            \alakohta{$a^2$ }
            \alakohta{$a^6$ }
            \alakohta{$1$}
        \end{alakohdat}
        \end{vastaus}
    \end{tehtava}    
    
    %soveltavia tehtäviä
        
    \begin{tehtava}%sovteht
        %Sievennä
        \begin{alakohdat}
        	\alakohta{$a^2(-a^4) $ }
        	\alakohta{$(ab^2)^0$ }
        	\alakohta{$(3a)^3$ }
        	\alakohta{$(a^5b^3)^3$}
		\end{alakohdat}        
        \begin{vastaus}
        \begin{alakohdat}
            \alakohta{$-a^6$ }
            \alakohta{$1$ }
            \alakohta{$27a^3$ }
            \alakohta{$a^{15}b^9$}
        \end{alakohdat}
        \end{vastaus}
    \end{tehtava} 
    
    \begin{tehtava}%sovteht
        %Sievennä
        \begin{alakohdat}
        	\alakohta{$\frac{2^7}{2^9}$ }
        	\alakohta{$\frac{a^3}{a}$ }
        	\alakohta{$\left(\frac{1}{3}\right)^2$ }
        	\alakohta{$\left(\frac{a^{-2}}{ab^4}\right)^4$}
		\end{alakohdat}        
        \begin{vastaus}
        \begin{alakohdat}
            \alakohta{$\frac{1}{4}$ }
            \alakohta{$a^2$ }
            \alakohta{$\frac{1}{9} $ }
            \alakohta{$ \left(\frac{1}{a^{12}b^{16}}\right)$ tai $a^{-12}b^{-16}$}
        \end{alakohdat}
        \end{vastaus}
    \end{tehtava}     


    \begin{tehtava}
        Laske. \quad
        a) $2^3\cdot2^3$ \qquad
        b) $4^3$ \qquad
        c) $(2^2)^3$ \qquad
        d) $2^{2+2+2}$

        \begin{vastaus}
            a) $64$ \qquad
            b) $64$ \qquad
            c) $64$ \qquad
            d) $64$
        \end{vastaus}
    \end{tehtava}

\begin{tehtava}
        %Sievennä. \quad
        a) $a^3\cdot b^2\cdot a^5$ \qquad 
        b) $(-ab^3)^2$ \qquad 
        c) $(a^5a^4)^3$ \qquad 
        d) $10^{2^3}$

        \begin{vastaus}
            a) $a^8b^2$ \qquad
            b) $a^2b^6$ \qquad
            c) $a^{15}b^{12}$ \qquad
            d) $10^8 = 100~000~000$
        \end{vastaus}
    \end{tehtava}

    \begin{tehtava}
        %Sievennä. \quad
        a) $(-a)\cdot(-a)$ \qquad
        b) $(-a)\cdot(-a)\cdot(-b)^3$ \qquad
        c) $(-a^2)\cdot(-a)^2$

        \begin{vastaus}
            a) $a^2$ \qquad
            b) $-a^2b^3$ \qquad
            c) $-a^4$
        \end{vastaus}
    \end{tehtava}

 \begin{tehtava}
        %Sievennä. \quad
        a) $(\frac{1}{2})\cdot(\frac{1}{2})$ \qquad
        b) $(-\frac{ab^2}{a^2b})^3$ \qquad
        c) $(-a^4b^4)^2$ \qquad
        d) $\left((\frac{a}{b})^4\right)^2$
        
        \begin{vastaus}
            a) $\frac{1}{4}$ \qquad
            b) $-\frac{b^3}{a^3}$ \qquad
            c) $a^8b^8$ \qquad
            d) $\frac{a^8}{b^8}$
        \end{vastaus}
    \end{tehtava}

\begin{tehtava}
	Laske
	\begin{alakohdat}
		\alakohta{$\left(\frac{4}{8}\right)^{1543} 2^{1546}$}
		\alakohta{$\left(\frac{28}{15}\right)^{214} \left(\frac{45}{98}\right)^{109} \left(\frac{5}{8}\right)^{105}$.}
	\end{alakohdat}
	
	\begin{vastaus}
		\begin{alakohdat}
			\alakohta{$8$}
			\alakohta{$\left(\frac{2\cdot 3}{7}\right)^4 = \frac{1296}{2401}$}
		\end{alakohdat}
	\end{vastaus}
\end{tehtava}

\begin{tehtava}
	Millä kokonaisluvun $n$ arvoilla
	\begin{alakohdatrivi}
		\alakohta{$2^n$}
		\alakohta{$(-3)^n$}
		\alakohta{$(-1)^{n-1}$}
		\alakohta{$(-1)^{n-1} (-2)^n$}
	\end{alakohdatrivi}
	on positiivinen?
	
	\begin{vastaus}
		\begin{alakohdat}
			\alakohta{kaikilla kokonaisluvuilla}
			\alakohta{parillisilla kokonaisluvuilla}
			\alakohta{parittomilla kokonaisluvuilla}
			\alakohta{ei millään kokonaisluvulla}
		\end{alakohdat}
	\end{vastaus}
\end{tehtava}
\end{tehtavasivu}
