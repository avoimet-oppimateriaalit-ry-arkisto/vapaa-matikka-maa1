\subsubsection*{Luvut ja laskutoimitukset}

\begin{tehtava}
	Laske
	\begin{alakohdat}
		\alakohta{$-8\cdot2\cdot(-3)$}
		\alakohta{$(-(-4)/2)/4$}
		\alakohta{$(2+3:6+1:2)\cdot2$}
		\alakohta{$-(-2)\cdot7-4\cdot3\cdot2+1$}
	\end{alakohdat}

\begin{vastaus}
	\begin{alakohdat}
		\alakohta{$48$}
		\alakohta{$\frac{1}{2}$}
		\alakohta{$6$}
		\alakohta{$-9$}
	\end{alakohdat}

\end{vastaus}
\end{tehtava}



\begin{tehtava}
	Laske laskujärjestyssääntöjä noudattaen ja jaa tulos alkutekijöihin. \\
	{$\{[(32+22:2)\cdot(5-2)]-15:(6:2)\}\cdot2-\frac{16:(5-3)}{2\cdot2+1}$}
	

\begin{vastaus}
	$2\cdot7²\cdot5$

\end{vastaus}
\end{tehtava}

\begin{tehtava}
	Laske.
	\begin{alakohdat}
		\alakohta{$3\frac{2}{15}:1\frac{7}{3}$}
		\alakohta{$\frac{6}{18}-\frac{4}{3}\cdot\frac{4+2}{2}$}
		\alakohta{$a\cdot\frac{a}{b}\cdot\frac{1}{a}+\frac{a}{2b}$}
		\alakohta{$\frac{de}{a}:\frac{db}{a}-\frac{e³b²}{be²} $}
	\end{alakohdat}

\begin{vastaus}
	\begin{alakohdat}
		\alakohta{$\frac{47}{50}$}
		\alakohta{$-\frac{11}{3}$}
		\alakohta{$\frac{3a}{2b}$}
		\alakohta{$0$}
	\end{alakohdat}
\end{vastaus}
\end{tehtava}





\begin{tehtava}
	Esitä desimaalilukuna.
	\begin{alakohdat}
		\alakohta{$\frac{23}{50}$}
		\alakohta{$-\frac{7}{11}$}
	\end{alakohdat}
	

\begin{vastaus}
	\begin{alakohdat}
		\alakohta{$0,46$}
		\alakohta{$0,636363...$}
	\end{alakohdat}
\end{vastaus}
\end{tehtava}

\begin{tehtava}
	Esitä murtolukumuodossa.
	\begin{alakohdat}
		\alakohta{$73,02$}
		\alakohta{$17,12341234...$}
		\alakohta{$0,85333...$}
		\alakohta{$-7,1232323$}
	\end{alakohdat}
	

\begin{vastaus}
	\begin{alakohdat}
		\alakohta{$\frac{3651}{50}$}
		\alakohta{$\frac{171217}{9999}$}
		\alakohta{$\frac{64}{75}$}
		\alakohta{$-\frac{3526}{495}$}
	\end{alakohdat}
\end{vastaus}
\end{tehtava}

\begin{tehtava}
	\begin{alakohdat}
		\alakohta{Muuta $1,25\,h$ minuuteiksi.}
		\alakohta{Laske ja ilmoita vastaus senttimetreinä: $1,23\,m+320\,mm$ .}
		\alakohta{Laske $300\,\frac{kgm²}{s²}:30\,\frac{kgm}{s²}$ .}
		
	\end{alakohdat}
	

\begin{vastaus}
	\begin{alakohdat}
		\alakohta{$75\,min$}
		\alakohta{$155\,cm$}
		\alakohta{$10\,m$}
	\end{alakohdat}
\end{vastaus}
\end{tehtava}

\begin{tehtava}
	Pyöristä 3 merkitsevän numeron tarkkuudella ja ilmoita kymmenpotenssimuodossa $a\cdot10^{n}$, missä $1 \leq a \geq 10$.
	\begin{alakohdat}
		\alakohta{$2,0007$}
		\alakohta{$0,00003125$}
		\alakohta{$79981633851,2$}
		
	\end{alakohdat}
	

\begin{vastaus}
	\begin{alakohdat}
		\alakohta{$2,00\cdot10¹$}
		\alakohta{$3,13\cdot10^{-5}$}
		\alakohta{$8,00\cdot10^{10}$}
	\end{alakohdat}
\end{vastaus}
\end{tehtava}

\begin{tehtava}
	Sievennä.
	\begin{alakohdat}
		\alakohta{$\sqrt{108}$}
		\alakohta{$\sqrt[3]{-64}$}
	
		\alakohta{$\frac{\sqrt{a²b}}{a}$}
		\alakohta{$\frac{abc}{\sqrt{a}\sqrt{b}\sqrt{c}}$}
		
	\end{alakohdat}
	

\begin{vastaus}
	\begin{alakohdat}
		\alakohta{$6\sqrt{3}$}
		\alakohta{$-4$}
		\alakohta{$\sqrt{b}$}
		\alakohta{$\sqrt{abc}$}
	\end{alakohdat}
\end{vastaus}
\end{tehtava}

\begin{tehtava}
	Sievennä.
	\begin{alakohdat}
		\alakohta{$\sqrt{2}+\sqrt{8}$}
		\alakohta{$\sqrt{27}-\sqrt{12}$}
		\alakohta{$\frac{\sqrt[3]{2}\cdot\sqrt[3]{15}}{\sqrt[3]{30}}$}
		
	\end{alakohdat}
	

\begin{vastaus}
	\begin{alakohdat}
		\alakohta{$3\sqrt{2}$}
		\alakohta{$\sqrt{3}$}
		\alakohta{$1$}
	\end{alakohdat}
\end{vastaus}
\end{tehtava}



\begin{tehtava}
	Muuuta murtopotenssimuotoon.
	\begin{alakohdat}
		\alakohta{$\sqrt[7]{\frac{1}{a⁵}}$}
		\alakohta{$\sqrt{\sqrt{2}}$}
	
	\end{alakohdat}
	

\begin{vastaus}
	\begin{alakohdat}
		\alakohta{$a^{-\frac{5}{7}}$}
		\alakohta{$2^{\frac{1}{4}}$}
	\end{alakohdat}
\end{vastaus}
\end{tehtava}




\subsubsection*{Yhtälöt}

	\begin{tehtava}
	Ratkaise yhtälö.
	\begin{alakohdat}
		\alakohta{$3x-6=18$}
		\alakohta{$\frac{3x}{2}+\frac{2x}{3}=13$}
		\alakohta{$f³+1=-7$}
		\alakohta{$x²+2(3x+1)=x²+3(2x+1)$}
		\alakohta{$\frac{5x+20}{20}=\frac{3}{4}x-\frac{1}{2}x+x⁰$}
	
	\end{alakohdat}
	

\begin{vastaus}
	\begin{alakohdat}
		\alakohta{$x=8$}
		\alakohta{$x=6$}
		\alakohta{$f=-2$}
		\alakohta{Ei ratkaisua.}
		\alakohta{Kaikki muuttujan x arvot.}
		
	\end{alakohdat}
\end{vastaus}
\end{tehtava}

\begin{tehtava}
	Ympyrän pinta-ala $A$ määritellään $A=\pi r²$, missä r on ympyrän säde. Tiedetään, että ympyrän pinta-ala on $\pi^{2}$. Ratkaise ympyrän säde.
	

\begin{vastaus}
	{$r=\sqrt{\pi}$}
\end{vastaus}
\end{tehtava}


\begin{tehtava} 
	\begin{alakohdat}
		\alakohta{Luolassa asuu 99 ihmistä. Aikuisia on $50\%$ vähemmän kuin lapsia. Montako aikuista luolassa asuu?}
		\alakohta{Bändi laskee cd-levynsä hintaa $20\%$. Pian heitä alkaa kuitenkin kaduttaa. Monellako prosentilla muutettua hintaa olisi korotettava, jotta hinta olisi ennallaan?}
	
	\end{alakohdat}
	

\begin{vastaus}
	\begin{alakohdat}
		\alakohta{Luolassa asuu 33 aikuista.}
		\alakohta{$25\%$}
	\end{alakohdat}
\end{vastaus}
\end{tehtava}

\begin{tehtava}
	Pölysuodatin pidättää $65\%$ pölyhiukkasista.
	
	\alakohta{Kuinka monta hiukkasta miljoonasta pääsee viiden peräkkäisen suodattimen läpi?}
	\alakohta{Kuinka monta hiukkasta jää neljänteen suodattimeen?}

\begin{vastaus}
	\alakohta{$5\,300$}
	\alakohta{$28\,000$}
\end{vastaus}
\end{tehtava}

\begin{tehtava}
	Pihtiputaan 84-henkinen sinfoniaorkesteri harjoittelee salissa. Puolet orkesterimuusikoista on jousisoittajia, lyömäsoittajia orkesterissa on yhdeksän. Loput orkesterista on puhallinsoittajia. Kaksi kolmasosaa puhallinsoittajista soittaa vaskipuhallinta. Montako puupuhallinsoittajaa orkesterissa on? Kaikki puhallinsoittimet ovat joko puu- tai vaskipuhaltimia. 
	

\begin{vastaus}
	$11$
\end{vastaus}
\end{tehtava}

\begin{tehtava}
	Riippuvainen matemaatikko joutuu vieroitukseen. Tällä hetkellä hän ratkoo 1000 tehtävää päivässä, mikä on sairaalloista. Vieroituksen aikana hän ratkoo aina $\frac{4}{5}$ edellisenä päivänä ratkotusta tehtävämäärästä. Tavoitteena on, että hän ratkoisi vieroituksen jälkeen vain 10 tehtävää päivässä. Kuinka monta kokonaista päivää vieroitus kestää?
	

\begin{vastaus}
	$20$
\end{vastaus}
\end{tehtava}

\begin{tehtava}
	Ratkaise yhtälö.
	\begin{alakohdat}
		\alakohta{$\frac{x+2}{4}=\frac{x-1}{7}$}
		\alakohta{$\frac{3x+1}{4x+2}=3$}
	
	\end{alakohdat}
	

\begin{vastaus}
	\begin{alakohdat}
		\alakohta{$x=-6$}
		\alakohta{$x=-\frac{5}{9}$}
		
	\end{alakohdat}
\end{vastaus}
\end{tehtava}

\begin{tehtava}
	Siiri haastaa Oskarin Pokemon-otteluun. Oskarin Charmanderin hännän liekin lämpötila on kääntäen verrannollinen Siirin Hoppipin maksimipomppukorkeuteen. Kun liekin lämpötila on 45 Fahrenheit-astetta, Hoppip hyppää 37\,m korkeuteen. Kuinka korkealle Hoppip voi hypätä, kun Charmanderin liekin lämpötila kasvaa 431 Fahrenheit-astetta?
	

\begin{vastaus}
	$3,5m$
\end{vastaus}
\end{tehtava}



\subsubsection*{Funktiot}

\begin{tehtava}
	Mikä on funktion $f(x)=\sqrt{x}+\frac{2}{\sqrt{4-x²}}$ määrittelyjoukko?
	

\begin{vastaus}
	{$0\leq x<2$}
		
\end{vastaus}
\end{tehtava}

\begin{tehtava}
	Olkoon $f(x)=2x+9$ ja $g(x)=12x-1$. Millä $x$:n arvolla pätee $f(x)=g(x)$? 
	

\begin{vastaus}
	{x:n arvolla 1}
		
\end{vastaus}
\end{tehtava}

\begin{tehtava}
Määritä kuvaajasta funktion $g$ nollakohdat. Mikä on funktion $g$ pienin arvo? Millä muuttujan $x$ arvolla $g(x)=-2$? Määritä $g(-\frac{1}{2})$.
\begin{kuva}
    kuvaaja.pohja(-4, 2, -4, 3)
    kuvaaja.piirra("x**2+2*x-2", nimi = "$g$")
\end{kuva}

\begin{vastaus}
	{$g(x)=0$, kun $x=-2\frac{3}{4}$ tai $x=\frac{3}{4}$; $-3$; $g(x)=-2$, kun $x=-2$ tai $x=0$; $g(-\frac{1}{2})=-2\frac{3}{4}$}
		
\end{vastaus}
\end{tehtava}

\begin{tehtava}
	Olkoon $f(x)=-4x^{-3}$ ja $g(x)=\frac{1}{2}(-x)^{2}$. Määritä
	
	\begin{alakohdat}
		\alakohta{$f(2)$}
		\alakohta{$f(0)$}
		\alakohta{$g(-4)$}
		\alakohta{$f(\frac{1}{2})\cdot g(\frac{1}{2})$}
		\alakohta{$f(g(2))$}
	
	\end{alakohdat}
	

\begin{vastaus}
	\begin{alakohdat}
		\alakohta{$-\frac{1}{2}$}
		\alakohta{Ei määritelty (Nollalla ei voi jakaa!)}
		\alakohta{$8$}
		\alakohta{$-4$}
		\alakohta{$-\frac{1}{2}$}
	\end{alakohdat}
\end{vastaus}
\end{tehtava}


	
	\begin{tehtava}
	Hyvä haltijatar taikoo karkkeja jättikulhoon syntymäpäivälahjaksi Tuhkimolle. Karkkien määrä kasvaa 3,1-kertaiseksi joka tunti. Kun kello 12.15, karkkeja on 3\,000. Muodosta funktio, joka ilmaisee karkkien lukumäärän t tunnin kuluttua. Kulhoon mahtuu 1\,000\,000 karkkia. Laske karkkien määrä
	
	\begin{alakohdat}
		\alakohta{$3,5$ tunnin kuluttua.}
		\alakohta{$5$ tuntia sitten.}
		\alakohta{Tuhkimon syntymäpäivät alkavat kello 18.00. Saako hyvä haltijatar lahjan ajoissa valmiiksi (eli kulhon täyteen)?}
	
	\end{alakohdat}
	Ilmoita vastaukset 10 karkin tarkkuudella.
	

\begin{vastaus}
	\begin{alakohdat}
		\alakohta{$157\,360$}
		\alakohta{$10$}
		\alakohta{Saa!}
	\end{alakohdat}
\end{vastaus}
\end{tehtava}

\begin{tehtava}
	Hiiri syö Erikin juustosta $10\%$ joka päivä. Nyt juustoa on 509\,g. Paljonko juustoa 
	
	\begin{alakohdat}
		\alakohta{on 3 päivän kuluttua?}
		\alakohta{oli 5 päivää sitten?}
	
	\end{alakohdat}
	

\begin{vastaus}
	\begin{alakohdat}
		\alakohta{$371\,g$}
		\alakohta{$862\,g$}
	\end{alakohdat}
\end{vastaus}
\end{tehtava}




