\subsection*{Luvut ja laskutoimitukset}
\begin{multicols}{2}
\begin{tehtava}
	Laske
	\alakohdat{
		§ $-8\cdot2\cdot(-3)$
		§ $(-(-4)/2)/4$
		§ $(2+3:6+1:2)\cdot2$
		§ $-(-2)\cdot7-4\cdot3\cdot2+1$
	}
	\begin{vastaus}
	\alakohdat{
		§ $48$
		§ $\frac{1}{2}$
		§ $6$
		§ $-9$
	}
	\end{vastaus}
\end{tehtava}

\begin{tehtava}
	Laske ja jaa alkutekijöihin
	$(32+22:2)-3(5-2)\cdot\left(2-\dfrac{16:4}{1+2\cdot4}\right)+1$
\begin{vastaus}
	$30=2\cdot3\cdot5$
%muokannut jaakko viertiö 22.3.2014

\end{vastaus}
\end{tehtava}

\begin{tehtava}
	Laske.
	\alakohdatm{
		§ $3\frac{2}{15}:1\frac{7}{3}$
		§ $\frac{6}{18}-\frac{4}{3}\cdot\frac{4+2}{2}$
		§ $a\cdot\frac{a}{b}\cdot\frac{1}{a}+\frac{a}{2b}$
		§ $\frac{de}{a}:\frac{db}{a}-\frac{e^2b^2}{be^2} $
	}

\begin{vastaus}
	\alakohdat{
		§ $\frac{47}{50}$
		§ $-\frac{11}{3}$
		§ $\frac{3a}{2b}$
		§ $\frac{e}{b}-b$
	}
\end{vastaus}
\end{tehtava}

\begin{tehtava}
	Esitä desimaalilukuna.
	\alakohdat{
		§ $\frac{23}{50}$
		§ $-\frac{7}{11}$
	}

\begin{vastaus}
	\alakohdat{
		§ $0,46$
		§ $0,636363\ldots$
	}
\end{vastaus}
\end{tehtava}

\begin{tehtava}
	Esitä murtolukumuodossa.
	\alakohdat{
		§ $73,02$
		§ $17,12341234\ldots$ %muuta yläviivamuotoon
		§ $0,85333\ldots$
		§ $-7,1232323$
	}

\begin{vastaus}
	\alakohdat{
		§ $\frac{3\,651}{50}$
		§ $\frac{171\,217}{9\,999}$
		§ $\frac{64}{75}$
		§ $-\frac{3\,526}{495}$
	}
\end{vastaus}
\end{tehtava}

	\begin{tehtava}
Muuta sekaluvut murtoluvuiksi.
	 \alakohdatm{
	  § $2\frac{3}{4}$
	  § $\frac{4}{13}$
	  § $-7\frac{2}{11}$
	  § $4$
	 }
	 \begin{vastaus}
	 \alakohdatm{
	  § $\frac{11}{4}$
	  § $\frac{4}{13}$
	  § $-\frac{79}{11}$
	  § $-\frac{4}{1}$
	 }
	 \end{vastaus}
	\end{tehtava}

\begin{tehtava}
Sievennä.
	\alakohdat{
	§ $\frac{a^2 b^2}{a}$, $a \neq 0$
	%	§  $2 \sqrt{5}-6\sqrt{2}-\sqrt{5}+5\sqrt{2}$
	§ $ab(a+2a)$
	§ $(a^3 b^2 c)^2$
	§ $\frac{ab+ca}{a} \cdot (b-c)$
	§ $\frac{7^{10\,002}}{7^{10\,000}}$
	}
	\begin{vastaus}
	\alakohdat{
	§ $a b^2$
	%§
	§ $3a^2 b$
	§ $a^6b^4c^2$
	§ $b^2-c^2$
	§ $49$
	}
	\end{vastaus}
\end{tehtava}

\begin{tehtava}
	\alakohdat{
		§ Muuta $1,25\,\text{h}$ minuuteiksi.
		§ Laske ja ilmoita vastaus senttimetreinä: $1,23\,\text{m}+320\,\text{mm}$.
		§ Laske $300\,\frac{\text{kg\,m}^2}{\text{s}^2}:30\,\frac{\text{kg\,m}}{\text{s}^2}$ .		
	}
	\begin{vastaus}
	\alakohdat{
		§ $75\,\text{min}$
		§ $155\,\text{cm}$
		§ $10\,\text{m}$
	}
	\end{vastaus}
\end{tehtava}

\begin{tehtava}
	Pyöristä $3$ merkitsevän numeron tarkkuudella ja ilmoita kymmenpotenssimuodossa $a\cdot10^{n}$, missä $1 \leq a \leq 10$.
	\alakohdat{
		§ $2,0007$
		§ $0,00003125$
		§ $79\,981\,633\,851,2$		
	}
	\begin{vastaus}
	\alakohdat{
		§ $2,00\cdot10^1$
		§ $3,13\cdot10^{-5}$
		§ $8,00\cdot10^{10}$
	}
	\end{vastaus}
\end{tehtava}

\begin{tehtava}
	Sievennä.
	\alakohdatm{
		§ $\sqrt{108}$
		§ $\sqrt[3]{-64}$
		§ $\frac{\sqrt{a^2b}}{a}$
		§ $\frac{abc}{\sqrt{a}\sqrt{b}\sqrt{c}}$	
	}
	
\begin{vastaus}
	\alakohdatm{
		§ $6\sqrt{3}$
		§ $-4$
		§ $\sqrt{b}$
		§ $\sqrt{abc}$
	}
\end{vastaus}
\end{tehtava}

\begin{tehtava}
Laske.
	\alakohdatm{
	 § $\sqrt{9+16}$
	 § $\sqrt{9} + \sqrt{16}$
	 § $\sqrt[3]{\sqrt{(9 + 16)^3}}$
	}
	\begin{vastaus}
	 \alakohdatm{
	  § $5$
	  § $7$
	  § $5$
	 }
	\end{vastaus}
\end{tehtava}

\begin{tehtava}
	Sievennä.
	\alakohdat{
		§ $\sqrt{2}+\sqrt{8}$
		§ $\sqrt{27}-\sqrt{12}$
		§ $\frac{\sqrt[3]{2}\cdot\sqrt[3]{15}}{\sqrt[3]{30}}$
		§ $a-(ab)^2+a^2 b^2$
		§ $(\sqrt[3]{a})^5 \cdot a^\frac{8}{6}$.
	}
	\begin{vastaus}
	\alakohdat{
		§ $3\sqrt{2}$
		§ $\sqrt{3}$
		§ $1$
		§ $a$
		§ $a^3$
	}
	\end{vastaus}
\end{tehtava}

\begin{tehtava}
	Muuta murtopotenssimuotoon.
	\alakohdatm{
		§ $\sqrt[7]{\frac{1}{a^5}}$
		§ $\sqrt{\sqrt{2}}$
	}
	
\begin{vastaus}
	\alakohdatm{
		§ $a^{-\frac{5}{7}}$
		§ $2^{\frac{1}{4}}$
	}
\end{vastaus}
\end{tehtava}

\end{multicols}

\subsection*{Yhtälöt}
\begin{multicols}{2}

\begin{tehtava}
	Ratkaise yhtälö.
	\alakohdat{
		§ $11x=77$
		§ $8x+174=50x$
		§ $3x-6=18$
		§ $\frac{3x}{2}+\frac{2x}{3}=13$
		§ $f^3+1=-7$
		§ $x^2+2(3x+1)=x^2+3(2x+1)$
		§ $\frac{5x+20}{20}=\frac{3}{4}x-\frac{1}{2}x+x^0$
		§ $\frac{5x}{4}-1=\frac{4}{5}x$
		}
	\begin{vastaus}
	\alakohdat{
		§ $7$
	  	§ $\frac{29}{7}$
		§ $x=8$
		§ $x=6$
		§ $f=-2$
		§ Ei ratkaisua
		§ Kaikki muuttujan $x$ arvot
		§ $\frac{20}{9}$
	}
	\end{vastaus}
\end{tehtava}

\begin{tehtava}
$1,17$ metriä pitkä lauta katkaistaan niin että toinen pala on $32$ senttimetriä pitempi kuin toinen. Minkä pituinen on lyhyempi pala?
	\begin{vastaus}
	$42,5$ senttimetriä
	\end{vastaus}
\end{tehtava}

\begin{tehtava}
	Ympyrän pinta-ala $A$ määritellään $A=\pi r^2$, missä $r$ on ympyrän säde. Tiedetään, että ympyrän pinta-ala on $\pi^2$. Ratkaise ympyrän säde.
	\begin{vastaus}
	$r=\sqrt{\pi}$
	\end{vastaus}
\end{tehtava}
 
\begin{tehtava}
Kuinka monta prosenttia
	\alakohdat{
	§ suurempi $30$ euroa on verrattuna $10$ euroon?
	§ lyhyempi $6$ metriä on verrattuna $30$ metriin?
	§ $1\,800c$ on $2\,400c$:stä?
	}
	\begin{vastaus}
	 \alakohdat{
	  § $200$\,\% suurempi
	  § $80$\,\% pienempi
	  § $75$\,\%
	  }
	\end{vastaus}
	\end{tehtava}

\begin{tehtava}
Bändi nostaa musiikkiraitansa hintaa $20\,\%$. Pian heitä alkaa kuitenkin kaduttaa. Monellako prosentilla muutettua hintaa olisi laskettava, jotta hinta olisi ennallaan?
		\begin{vastaus}
		 $17\,\%$
		\end{vastaus}
\end{tehtava}

\begin{tehtava}
Olkoon $f(t) = 35 \cdot 2^t$ bakteerien lukumäärä soluviljelmässä ajanhetkellä $t$ (tuntia). Monenko kokonaisen tunnin kuluttua bakteereita on yli $1\,000$?
	\begin{vastaus}
	$6$ tunnin kuluttua
	\end{vastaus}
\end{tehtava}

\begin{tehtava}
	Pölysuodatin pidättää $65\,\%$ pölyhiukkasista.
		\alakohdat{
	§ Kuinka monta hiukkasta miljoonasta pääsee viiden peräkkäisen suodattimen läpi?
	§ Kuinka monta hiukkasta jää neljänteen suodattimeen?
	}
\begin{vastaus}
	\alakohdat{
	§ $5\,300$
	§ $28\,000$
		}
\end{vastaus}
\end{tehtava}

\begin{tehtava}
	Pihtiputaan $84$-henkinen sinfoniaorkesteri harjoittelee salissa. Puolet orkesterimuusikoista on jousisoittajia, lyömäsoittajia orkesterissa on yhdeksän. Loput orkesterista on puhallinsoittajia. Kaksi kolmasosaa puhallinsoittajista soittaa vaskipuhallinta. Montako puupuhallinsoittajaa orkesterissa on? Kaikki puhallinsoittimet ovat joko puu- tai vaskipuhaltimia.
	\begin{vastaus}
	$11$
	\end{vastaus}
\end{tehtava}

\begin{tehtava}
	Riippuvainen matemaatikko joutuu vieroitukseen. Tällä hetkellä hän ratkoo $1\,000$ tehtävää päivässä, mikä on sairaalloista. Vieroituksen aikana hän ratkoo aina $\frac{4}{5}$ edellisenä päivänä ratkotusta tehtävämäärästä. Tavoitteena on, että hän ratkoisi vieroituksen jälkeen vain $10$ tehtävää päivässä. Kuinka monta kokonaista päivää vieroitus kestää?
	\begin{vastaus}
	$21$
	\end{vastaus}
\end{tehtava}

\begin{tehtava}
	Ratkaise yhtälö.
	\alakohdat{
		§ $\frac{x+2}{4}=\frac{x-1}{7}$
		§ $\frac{3x+1}{4x+2}=3$
	}
	\begin{vastaus}
	\alakohdat{
		§ $x=-6$
		§ $x=-\frac{5}{9}$
	}
	\end{vastaus}
\end{tehtava}

\begin{tehtava}
Siiri haastaa Oskarin Pokémon-otteluun. Oskarin Charmanderin hännän liekin lämpötila on kääntäen verrannollinen Siirin Hoppipin maksimipomppukorkeuteen. Kun liekin lämpötila on $45$ Fahrenheit-astetta, Hoppip hyppää $37$\,m korkeuteen. Kuinka korkealle Hoppip voi hypätä, kun Charmanderin liekin lämpötila kasvaa $431$ Fahrenheit-astetta? %kirjoitetaan isolla?
	\begin{vastaus}
	$3,5$\,m
	\end{vastaus}
\end{tehtava}

\end{multicols}

\subsection*{Funktiot}
\begin{multicols}{2}

\begin{tehtava}
Arvioi kuvaajasta funktion $g$ nollakohdat. Mikä on funktion $g$ pienin arvo? Millä muuttujan $x$ arvolla $g(x)=-2$? Määritä $g\left(-\frac{1}{2}\right)$.
\begin{kuva}
    kuvaaja.pohja(-4, 2, -4, 3)
    kuvaaja.piirra("x**2+2*x-2", nimi = "$g$")
\end{kuva}

	\begin{vastaus}
	$g(x)=0$, kun $x=-2\frac{3}{4}$ tai $x=\frac{3}{4}$; $-3$; $g(x)=-2$, kun $x=-2$ tai $x=0$; $g\left(-\frac{1}{2}\right)=-2\frac{3}{4}$		
	\end{vastaus}
\end{tehtava}

\begin{tehtava}
	Olkoon $f(x)=-4x^{-3}$ ja $g(x)=\frac{1}{2}(-x)^{2}$. Määritä.	
	\alakohdatm{
		§ $f(2)$
		§ $f(0)$
		§ $g(-4)$
		§ $f\left(\frac{1}{2}\right)\cdot g\left(\frac{1}{2}\right)$
		§ $f\left(g(2)\right)$	
	}
	\begin{vastaus}
	\alakohdat{
		§ $-\frac{1}{2}$
		§ Ei määritelty, koska nollalla ei voi jakaa
		§ $8$
		§ $-4$
		§ $-\frac{1}{2}$
	}
	\end{vastaus}
\end{tehtava}
%muokannut jaakko viertiö 22.3.2014

\begin{tehtava}
	Mikä on funktion $f(x)=\sqrt{x}+\frac{2}{\sqrt{4-x^2}}$ määrittelyehto?
\begin{vastaus}
	$0\leq x<2$	
	\end{vastaus}
\end{tehtava}

\begin{tehtava}
	Olkoon $f(x)=2x+9$ ja $g(x)=12x-1$. Millä $x$:n arvolla pätee $f(x)=g(x)$? 
	\begin{vastaus}
$x$:n arvolla $1$
	\end{vastaus}
\end{tehtava}

\begin{tehtava}
Hyvä haltijatar taikoo karkkeja jättikulhoon syntymäpäivälahjaksi Tuhkimolle. Karkkien määrä kasvaa $3,1$-kertaiseksi joka tunti. Kun kello $12.15$, karkkeja on $3\,000$. Muodosta funktio, joka ilmaisee karkkien lukumäärän $t$ tunnin kuluttua. Kulhoon mahtuu $1\,000\,000$ karkkia. Laske karkkien määrä
	\alakohdat{
		§ $3,5$ tunnin kuluttua.
		§ $5$ tuntia sitten.
		§ Tuhkimon syntymäpäivät alkavat kello $18.00$. Saako hyvä haltijatar lahjan ajoissa valmiiksi (eli kulhon täyteen)?
	}
	Ilmoita vastaukset $10$ karkin tarkkuudella.
	\begin{vastaus}
	\alakohdat{
		§ $157\,360$
		§ $10$
		§ Saa!
	}
	\end{vastaus}
\end{tehtava}

\begin{tehtava}
	Hiiri syö Erikin juustosta $10\,\%$ joka päivä. Nyt juustoa on $509$\,g. Paljonko juustoa
	\alakohdat{
		§ on $3$ päivän kuluttua?
		§ oli $5$ päivää sitten?
	}
	
	\begin{vastaus}
	\alakohdat{
		§ $371$\,g
		§ $862$\,g
	}
	\end{vastaus}
\end{tehtava}
\end{multicols}
\newpage