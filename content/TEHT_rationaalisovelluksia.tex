\begin{tehtavasivu}

\begin{tehtava} 
        Laatikossa on palloja, joista kolmasosa on mustia, neljäsosa valkoisia ja viidesosa harmaita. Loput palloista ovat punaisia. Kuinka suuri osuus palloista on punaisia?
        \begin{vastaus}
            $1-(\frac{1}{3}+\frac{1}{4}+\frac{1}{5})
            = \frac{60}{60}-\frac{20}{60}-\frac{15}{60}-\frac{12}{60}
            = \frac{60}{60}-\frac{47}{60}
            = \frac{13}{60}$
        \end{vastaus}
    \end{tehtava}
    
\begin{tehtava} 
Eräässä pitkän matematiikan ensimmäisen kurssin ryhmässä on $16$ opiskelijaa. Heistä $8$ on tyttöjä ja tytöistä neljänneksellä on siniset silmät. Kuinka suuri osa luokan oppilaista on sinisilmäisiä tyttöjä?
        \begin{vastaus}
			Yksi kahdeksasosa
        \end{vastaus}
\end{tehtava}
%Laatinut Henri Ruoho 9.11.2013

%ryhmätehtävä, jossa ... poikia, tyttöjä ja muita :)

\begin{tehtava}
        Mira, Pontus, Jarkko-Kaaleppi ja Milla leipoivat lanttuvompattipiirakkaa. Pontus kuitenkin söi piirakasta kolmanneksen ennen muita, ja loput piirakasta jaettiin muiden kesken tasan. Kuinka suuren osan muut saivat?
        \begin{vastaus}
            Muut saivat piirakasta kuudesosan.
        \end{vastaus}
\end{tehtava}
    
\begin{tehtava}
    Huvipuiston sisäänpääsylippu maksaa $20$ euroa, ja lapset pääsevät sisään puoleen hintaan.
	\alakohdat{
		§ Kuinka paljon kolmen lapsen yksinhuoltajaperheelle maksaa päästä sisään?
		§ Kuinka paljon sisäänpääsy maksaa perheelle avajaispäivänä, kun silloin sisään pääsee neljänneksen halvemmalla?
    }
    \begin{vastaus}
		\alakohdatm{
			§ $50$ euroa 
			§ $50$ euroa 
			§ $37,50$ euroa
		} 
    \end{vastaus}
\end{tehtava}  

\begin{tehtava}
	Eräässä kaupassa on käynnissä loppuunmyynti, ja kaikki tuotteet myydään puoleen hintaan. Lisäksi kanta-asiakkaat saavat aina viidenneksen alennusta ostoksistaan. Paljonko kanta-asiakas maksaa nyt tuotteesta, joka normaalisti maksaisi $40$ euroa?
    \begin{vastaus}
		$40\cdot \frac{1}{2} \cdot \frac{4}{5}=40\cdot \frac{4}{10}= 16$. 
	\end{vastaus}
\end{tehtava}
    
\begin{tehtava}
	Kokonaisesta kakusta syödään maanantaina iltapäivällä puolet, ja jäljelle jääneestä palasta syödään tiistaina iltapäivällä taas puolet. Jos kakun jakamista ja syömistä jatketaan samalla tavalla koko viikko, kuinka suuri osa alkuperäisestä kakusta on jäljellä seuraavana maanantaiaamuna?
	\begin{vastaus}
		Toisena päivänä aamulla kakkua on jäljellä puolet, kolmantena päivänä aamulla
		$1-\left(\frac{1}{2} + \frac{1}{4}\right) = \frac{1}{4}$, 
		neljäntenä päivänä
		$1-\left(\frac{1}{2} + \frac{1}{4} + \frac{1}{8}\right)
		= \frac{1}{8}$, jne.
		Siis seitsemän päivän jälkeen kakkua on jäljellä
		$1-\left(\frac{1}{2} + \frac{1}{4} + \frac{1}{8} +
		\frac{1}{16} + \frac{1}{32} + \frac{1}{64} + \frac{1}{128}\right)
		= \frac{1}{128}$.  
	\end{vastaus}
\end{tehtava}

\begin{tehtava}
	$\star$ Vanhalla matemaatikolla on kolme lasta. Eräänä päivänä hän antaa lapsilleen laatikollisen vuosien varrella ongelmanratkaisukilpailuista voitettuja palkintoja. Hän kertoo antavansa vanhimmalle lapselleen puolet saamistaan arvoesineistä, keskimmäiselle neljäsosan ja nuorimmalle kuudesosan. Laatikossa on kuitenkin vain $11$ palkintoa. Miten  palkinnot jaetaan ja kuinka monta arvoesinettä matemaatikko pitää itsellään?
	\begin{vastaus}
		Vanhin sai $6$ esinettä, keskimmäinen $3$ esinettä ja nuorin $2$ esinettä. Vanha matemaatikko pitää yhden palkinnon itsellään, sillä $\frac{1}{2} + \frac{1}{4} + \frac{1}{6} = \frac{11}{12}$. (Tämä vastaus on oikein. Jos ihmetyttää, kannattaa lukea tarkkaan, mitä tehtävässä oikein sanotaan.)
	\end{vastaus}
\end{tehtava}

\end{tehtavasivu}